
\pagestyle{headings}
\fontsize{7}{8}\selectfont
%\setlength{\baselineskip}{5pt}
\pagecolor{white} 

\onecolumn
\chapter{Reportes De Investigación } 
\renewcommand\thesection{RI\ \nplpadding{3}\numprint{\arabic{section}}} 
\setcounter{section}{0}
\chaptertoc
\twocolumn
\balance



\section{REFLETINDO A FORMAÇÃO CONTINUADA DO PROFESSOR QUE ENSINA MATEMÁTICA
NOS ANOS INICIAIS DO ENSINO FUNDAMENTAL ATRAVÉS DE ATITUDES COLABORATIVAS}

\begin{datos}
Renata Camacho Bezerra, Maria Raquel Miotto Morelatti. \\
Universidade Estadual do Oeste do Paraná (UNIOESTE), Campus de Foz do Iguaçu/PR,\\
Universidade Estadual Paulista (UNESP), Campus de Presidente Prudente/SP,  \\ 
Brasil,\\
\hfill  renata.bezerra@unioeste.br; mraquel@fct.unesp.br  
\end{datos}

A formação matemática dos professores vem ganhando grande destaque
e isto ocorre por diversos motivos, dentre eles, porque avaliações
externas mostram que os alunos apresentam baixo índice de aproveitamento
em Matemática. Diante disso, é necessário pensarmos alternativas,
uma possibilidade é investir na formação docente. Neste trabalho apresentamos
uma pesquisa de doutorado que busca compreender de que forma os processos
formativos vinculados à participação de professores que ensinam matemática
nos anos iniciais do ensino fundamental num grupo de estudo/pesquisa,
caracterizado por um ambiente de mutualidade, de troca, de partilha
e de trabalho colaborativo refletem no processo de ensino da Matemática.


\section{\uppercase{ Fortalezas y debilidades de Facebook y Twitter como
entornos educativos en el contexto del curso EIF-203 Estructuras Discretas
para Informática en la Universidad Nacional de Costa Rica}}

\begin{datos}
Enrique Vílchez Quesada. \\
Universidad Nacional de Costa Rica,\\
Costa Rica,\\
\hfill  enrique.vilchez.quesada@una.cr  
\end{datos}

Durante el I semestre del año 2013 se implementaron un conjunto de
experiencias de enseñanza y aprendizaje en el curso EIF-203 Estructuras
discretas para Informática utilizando como principales medios de interacción
social las redes Facebook y Twitter. Lo anterior, formó parte de las
responsabilidades asociadas a un proyecto de investigación inscrito
en la Escuela de Informática de la Universidad Nacional de Costa Rica
(UNA). El objetivo de esta implementación metodológica consistió en
determinar el impacto del uso de las redes sociales anteriormente
citadas, para desarrollar procesos de enseñanza y aprendizaje en un
contexto educativo formal. 


\section{COMUNIDAD DE CONOCIMIENTO MATEMÁTICO: EL CASO DE JÓVENES SORDOS }

\begin{datos}
Claudia L. Méndez Bello, Francisco Cordero Osorio.\\
Centro de Investigación y de Estudios Avanzados del Instituto Politécnico Nacional,\\
México,\\
\hfill  clmendezb@cinvestav.mx; fcordero@cinvestav.mx  
\end{datos}

Esta investigación, de corte socioepistemológico, se centra en la
problemática educativa del sordo. Asumimos la postura de que una educación
para el sordo debe ser planteada desde el sordo. Partiendo de reconocer
al sordo no como consumidor sino como constructor de conocimiento
donde la sordera no es una limitante sino una condición que permea
dichas construcciones. La tarea es identificar estas argumentaciones
generadas desde el sordo que le son propias, a manera de generar un
marco de referencia que dé cuenta de los usos del conocimiento matemático,
con miras a una matemática funcional en la escuela para el sordo.


\section{LA TRANSFORMADA RÁPIDA DE FOURIER COMO ELEMENTO PARA MEDIR SEÑALES
EN TIEMPO REAL}

. . Pensamiento matemático avanzado, Nivel superior (19-22 años),
empírico/experimental.\begin{datos}
Martín Sauza Toledo, Juvenal Rodríguez Reséndiz,\\
Adiel Basurto Guerrero y Juan Reséndiz Ríos.\\
Universidad Tecnológica de Tula-Tepeji,\\
México,\\
\hfill  mtoledo@uttt.edu.mx; juvenal@uaq.edu.mx,\\
\hfill abasurto@uttt.edu.mx; jresendiz@uttt.edu.mx
\end{datos}

En este trabajo de investigación se presenta la implementación de
la Transformada Rápida de Fourier como herramienta para procesar la
señal en Labview proveniente de un motor de inducción trifásico, el
diseño de una tarjeta basada en dsPic la cual se usa como convertidor
de analógico a digital. La implementación de la Transformada Rápida
de Fourier por sus siglas en ingles (FFT) es importante, ya que en
la actualidad esta herramienta tan poderosa me permite procesar datos
en tiempo real. 


\section{LA COMBINATORIA EN LIBROS DE TEXTO DE MATEMÁTICA DE EDUCACIÓN SECUNDARIA
EN ESPAÑA}

\begin{datos}
Jonathan Espinoza González, Rafael Roa Guzmán.\\
Universidad Nacional de Costa Rica, Universidad de Granada,\\
Costa Rica, España,\\
\hfill  espinozaj25@gmail.com, rroa@ugr.es
\end{datos}

Existe una problemática sobre el tema de Combinatoria caracterizada
por la dificultad de este objeto de estudio, su inclusión implícita
en el currículo y por ser un contenido que suele enseñarse de forma
aislada de los demás temas descritos en el currículo. Debido a la
problemática descrita escogimos el tema de Combinatoria concentrándonos
en su tratamiento en una muestra de libros de texto de Matemática
utilizados en la Educación Secundaria en España. El problema de investigación
abordado es la caracterización del significado institucional del objeto
matemático “Combinatoria” presente en la institución de los libros
de texto de Matemática utilizados en la Educación Secundaria en España. 


\section{ANALISIS DIDACTICO A UNA TRAYECTORIA EPISTEMICA DE ENSEÑANZA AL METODO
DE INTEGRACION POR PARTES}

\begin{datos}
Enrique Mateus Nieves.\\
Universidad Distrital Francisco José de Caldas,\\
Bogotá-Colombia,\\
\hfill  jeman124@gmail.com
\end{datos}

Se observa una tendencia en la enseñanza de los conceptos implicados
en la integración por partes: seguir un desarrollo casi exclusivamente
de rutinización algebraica. Se conocen las técnicas algorítmicas,
sin una contextualización adecuada del proceso de integración. El
enseñar separadamente los algoritmos de problemas contextualizados,
responde al interés de la presente investigación doctoral en aras
de buscar respuestas que proporcionen razones para entender ¿por qué
los estudiantes se sienten abrumados por tantos requerimientos formalistas
de las matemáticas en la formación superior? Quizá por esto el estudiante
no reconoce el doble valor que tienen las matemáticas: como ciencia
y como herramienta.


\section{LOGROS DE LOS ESTUDIANTES DE EDUCACIÓN BÁSICA EN TEMAS DE ESTADÍSTICA
Y PROBABILIDAD }

\begin{datos}
Augusta Osorio Gonzales, Elizabeth Advíncula Clemente.\\
Pontificia Universidad Católica del Perú,\\
Perú,\\
\hfill  \url{arosorio@pucp.edu.pe}; eadvincula@pucp.edu.pe
\end{datos}

En este reporte presentaremos nuestra investigación, la cual buscó
conocer los logros de aprendizaje de los estudiantes, del nivel educativo
básico, en temas de Estadística y Probabilidad. La intención era establecer
si los conocimientos básicos en esta área están siendo alcanzados
en el momento esperado, según los Mapas de progreso del Aprendizaje
de los Estándares de Aprendizaje Nacionales de la Educación Básica
Regular del Perú. La identificación de los principales logros y deficiencias
de los estudiantes nos provee de información relevante para establecer
cuáles son los conocimientos que necesitan ser reforzados y con ello
orientar mejor nuestra acción pedagógica. 


\section{UNA SECUENCIA DE MODELACIÓN PARA INTRODUCCIÓN SIGNIFICATIVA DE LA
FUNCIÓN CUADRÁTICA}

\begin{datos}
Octavio Augusto Briceño Silva, Gabriela Buendía Ábalos.\\
Instituto Politécnico Nacional Centro de Investigaciones en Ciencia Aplicada y Tecnología Avanzada. CICATA,\\
México,\\
\hfill  Octavioco11@gmail.com; buendiag@hotmail.com 
\end{datos}

Esta investigación gira alrededor de secuencias didácticas, donde
la práctica de modelación se introduce en forma intencional y de esta
manera llegue a que los estudiantes obtengan un aprendizaje significativo
sobre la función cuadrática y sus aspectos variacionales. La aplicación
a estudiantes de comienzos del bachillerato (12-13 años) permite que
ellos interactúen entre sí, con el medio y el profesor, para que surjan
argumentos que aporten al proceso enseñanza aprendizaje de la función
cuadrática. La investigación es fortalecida por las contribuciones
que se obtienen al usar la metodología de “experimentos de diseño”,
donde el marco teórico que sustenta el trabajo es la socioepistemología.


\section{LA MOVILIZACIÓN DE OBJETOS CULTURALES DESDE LAS MEMORIAS DE LA PRÁCTICA
DE CONSTRUCCIÓN DE LA VIVIENDA TRADICIONAL EMBERA CHAMI: posibilidades
para pensar el (por)venir de la educación (matemática) indígena}

\begin{datos}
Carolina Higuita Ramirez, Diana Victoria Jaramillo Quiceno.\\
Universidad de Antioquia,\\
Colombia,\\
\hfill  cahira0605@gmail.com; diana\_{}jaramillo@hotmail.com 
\end{datos}

Presentamos el tejido de la investigación “La movilización de objetos
culturales desde las memorias de la práctica de construcción de la
vivienda tradicional Embera-Chami: posibilidades para pensar el (por)venir
de la educación (matemática) indígena”. Fueron objetivos de este estudio:
analizar la movilización de objetos culturales desde las memorias
de la práctica de la construcción de la vivienda tradicional Embera-Chami;
y, problematizar esa movilización de objetos culturales para pensar
el (por)venir de la educación (matemática) indígena. Los planteamientos
de la comunidad indígena Embera-Chamí en diálogo con D’Ambrosio (2011),
Miguel (2010), Thompson (2011) y Walsh (2005) se constituyeron en
fuentes teóricas del estudio.


\section{MODELANDO LO CUADRÁTICO Y LO EXPONENCIAL DESDE EL ENTORNO HACIA LA
ESCUELA}

\begin{datos}
Daniela González, Noelia Orellana, Patricio Rodríguez, Leonora Díaz .\\
Universidad Católica Silva Henríquez,\\
Santiago de Chile,\\
\hfill  digonzalezc@miucsh.cl; noelia.orellana.s@gmail.com, \\
\hfill  prrodriguez@miucsh.cl; leonoradm@gmail.com
\end{datos}

La práctica de modelación inserta en el currículum nacional, se ha
desvinculado del contexto escolar generando un profundo divorcio entre
lo cotidiano y lo académico. Por ello se vio la necesidad de realizar
un estudio empírico-experimental con fines didácticos, tomando como
modelos la función cuadrática y exponencial, generando una metodología
alternativa a la del currículum nacional chileno. Para ello se aplicaron
reactivos que reflejen un entramado que una tanto lo académico con
lo experimental, como son la caída libre de objetos y el rebote de
una pelota. Los participantes pertenecen al segundo ciclo de enseñanza
media de establecimientos particulares subvencionados.


\section{EFICACIA EN LA SOLUCIÓN DE PROBLEMAS VERBALES DE MATEMÁTICAS: MÉTODOS
HEURÍSTICO Y CONFERENCIA EXPOSITIVA}

\begin{datos}
Lina Soraya Llanos Vargas.\\
Universidad Interamericana de Puerto Rico - Recinto de San Germán,\\
Puerto Rico,\\
\hfill  lisolla@gmail.com 
\end{datos}

La solución de problemas verbales de matemáticas, constituye el árbol
temático en el que se sustenta esta investigación. Es el primer estudio
formal en el ámbito universitario, que examina la diferencia en la
eficacia en la solución de problemas verbales de matemáticas en estudiantes
de Álgebra, cuando son expuestos a estrategias de enseñanza del método
heurístico y la conferencia expositiva. Los hallazgos del estudio,
contribuyen a conocer y establecer las relaciones existentes entre
esas variables, al comparar los resultados obtenidos en un grupo que
experimenta actividades de enseñanza del método heurístico con un
grupo que experimenta la conferencia expositiva. Los profesores de
matemáticas que enseñan Fundamentos de Álgebra, podrían utilizar esta
información, para realizar cambios e innovaciones curriculares que
se ajusten a las verdaderas necesidades e intereses de los estudiantes;
se plantean en este estudio las características de un tratamiento
adecuado de problemas de ecuaciones lineales, que permiten un aprendizaje
significativo de los temas, además, es uno de los caminos alternativos
de solución, no necesariamente enmarcados en procesos algebraicos. 


\section{O PENSAMENTO CRÍTICO NA RESOLUÇÃO DE PROBLEMAS NAS ENGENHARIAS}

\begin{datos}
Maria Alice Veiga Ferreira de Souza, Sotério Ferreira de Souza.\\
Instituto Federal do Espírito Santo, Universidade Federal do Espírito Santo,\\
Brasil,\\
\hfill  alicevfs@hotmail.com; mariaalice@ifes.edu.br/soterio.souza@hotmail.com 
\end{datos}

Sistemas lineares estão presentes em aplicações das Engenharias e,
por isso, devem fazer sentido para esses estudantes. A insuficiência
de significados denunciada em pesquisas científicas justificam investimentos
que busquem potencializar esse aprendizado. Assim, verificou-se o
desempenho de estudantes de Engenharia em uma atividade problematizadora
que utilizasse sistemas lineares e que promovesse o pensamento crítico.
Para explicar a pesquisa, selecionou-se um dos trabalhos. Tratou-se
de uma pesquisa descritiva e qualitativa de um estudo de caso. As
análises dos protocolos dos estudantes revelaram que a atividade proporcionou
significados sobre sistemas lineares e favoreceu o desenvolvimento
do pensamento crítico pelos estudantes.


\section{ARITMÉTICAS DE UNA REGIÓN}

\begin{datos}
Armando Aroca Araújo.\\
Universidad del Atlántico,\\
Colombia,\\
\hfill  armandoaroca@mail.uniatlantico.edu.co 
\end{datos}

Una aritmética existe debido a una actividad humana que la necesita
y sólo tiene sentido en el intercambio de los productos que ésta produce
o moviliza, en este sentido, hay muchas aritméticas. Solo la aritmética
escolar tiene como objetivo que otras personas la aprendan masivamente
e incorporen en su forma de vivir. Generando tensiones que se condensan
al pertenecer a un contexto sociocultural específico y la comprensión
interlógica de las aritméticas. Reducir las tensiones implica la enculturación
del currículo matemático, y quienes deben ejercer esta transformación
son los profesores de matemáticas. A esta reducción de las tensiones
llamémosla supra aritmética.


\section{UN CRITERIO PARA COMPARAR VISUALIZACIONES CONTENIDAS O ASOCIADAS
A REGISTROS DE REPRESENTACIÓN SEMIÓTICA DE DUVAL PARA LA TRIGONOMETRÍA}

\begin{datos}
Oscar Jesús San Martín.\\
Sicre Unidad 261 (Hermosillo) de la Universidad Pedagógica Nacional,\\
México,\\
\hfill  sicreo@outlook.es 
\end{datos}

Se presenta un avance de una investigación teórica fundamentada en
la teoría de los registros de representación semiótica de Raymond
Duval. Se demuestra que un nuevo registro de representación semiótica
elaborado por el autor para la trigonometría, satisface las tres actividades
cognoscitivas fundamentales de la semiosis: formación de una representación;
tratamiento y conversión de la representación. Principalmente se aborda
el problema de cómo comparar visualizaciones contenidas o asociadas
a registros de representación semiótica y de definir lo que debe entenderse
cuando se afirma que una visualización en una representación es mejor
que otra. Se presentan las conclusiones obtenidas. 


\section{LA GEOMETRÍA Y EL ARTE COMO MÉTODOS DE ENSEÑANZA DEL OBJETO FRACCIÓN}

\begin{datos}
Antonio Di Teodoro, María Fernanda Romero T.\\
GIEMCIEA Grupo de Investigación en Educación y Matemática del Colegio Integral El Ávila,\\
Venezuela,\\
\hfill  aditeodoro@usb.ve; mfromeroavila@gmail.com
\end{datos}

El presente trabajo resume una propuesta pedagógica para la enseñanza
del objeto fracción y la iniciación a parte de su álgebra para niños
de primaria basada en la geometría y el arte. Esta propuesta se sustenta
en las teorías constructivistas, retoma métodos de enseñanza, conceptos
matemáticos conocidos y utilizados desde hace varias décadas en educación
matemática desde una perspectiva distinta. Fue aplicada a niños de
doce años de edad. Las actividades pedagógicas en aula, parten de
la elaboración de un vitral inspirado en los inicios del arte cubista
para explicar de forma concreta los contenidos. 


\section{\uppercase{ Causa del la deficiencia de la enseñanza de la Matemática
a Nivel Primario en Panamá}}

\begin{datos}
Cesiah Alemán, Ricardo López.\\
Universidad Tecnológica de Panamá,\\
Panamá,\\
\hfill  cesiah.aleman@utp.ac.pa; Ricardo.lopez@utp.ac.pa 
\end{datos}

Con esta investigación queremos comprobar que el alto índice de fracasos
está íntimamente ligado con la mala formación cognitiva del maestro
en el área de la matemática Los temas a nivel de primaria que se imparten
en los colegios oficiales en Panamá son: la potenciación, la radicación
con índice 2 o 3, operaciones con fracciones, transformación de decimal
a fracción y viceversa, perímetro, área, paralelogramos, circunferencia,
estadística y probabilidades, razones y proporciones y tanto por ciento.


\section{EVALUACIÓN Y DESARROLLO DEL CONOCIMIENTO DE FUTUROS PROFESORES ESPAÑOLES
SOBRE MUESTREO}

\begin{datos}
Emilse Gómez-Torres, Carmen Batanero, \\
José Miguel Contreras, María M. Gea.\\ 
Universidad Nacional de Colombia, Universidad de Granada,\\
Colombia, España,\\
\hfill  egomezt@unal.edu.co; batanero@ugr.es,\\
\hfill jmcontreras@ugr.es; mmgea@ugr.es
\end{datos}

Se analizan respuestas de futuros profesores de educación primaria
a una tarea de estimación en un contexto de muestreo con la técnica
de captura-recaptura. Los resultados reflejan conocimiento sobre elementos
básicos de muestreo insuficiente para la práctica docente, que requiere
la comprensión de estos elementos para implementar el enfoque frecuencial
de la probabilidad con una aproximación experimental. Se describe
también una actividad dirigida por el formador de profesores para
desarrollar este conocimiento, mediante una discusión de posibles
soluciones a la tarea y el apoyo de simulación se reconocen respuestas
correctas e incorrectas y los razonamientos que conllevan a ellas.


\section{INSTRUMENTACIÓN DE LA SIMETRÍA AXIAL MEDIADA POR EL GEOGEBRA}

\begin{datos}
Daysi Julissa García Cuéllar, Jesús Victoria Flores Salazar. \\
Pontificia Universidad Católica del Perú, Instituto de Investigación sobre Enseñanza de las Matemáticas- IREM,\\
\hfill Perú, \\
\hfill garcia.daysi@pucp.pe; jvflores@pucp.pe
\end{datos}

Este estudio nace de la problemática que existe en la enseñanza de
la Geometría y su poca profundización en esta área de la matemática,
tal como lo son las transformaciones geométricas en el plano. Nuestra
investigación se centra en el estudio de la instrumentación de la
simetría axial mediada por el software Geogebra. Tomamos como marco
teórico y metodológico el Enfoque instrumental de Rabardel (1995)
y la Ingeniería Didáctica de Artigue (1995) respectivamente. La experimentación
se realizó con estudiantes entre 12 y 13 años, quienes consiguieron
conjeturar propiedades de la simetría axial. Afirmamos que están instrumentadas
en este objeto matemático.


\section{DIALECTICA HERRAMIENTA-OBJETO: EL CASO DE LA SEMEJANZA DE TRIÁNGULOS
CON EL USO DEL GEOGEBRA}

\begin{datos}
Luis Alberto Masgo Lara, Jesús Victoria Flores Salazar. \\
Pontificia Universidad Católica del Perú, Instituto de Investigación sobre Enseñanza de las Matemáticas- IREM,\\
\hfill Perú, \\
\hfill a20035014@pucp.edu.pe; jvflores@pucp.pe
\end{datos}

A partir de nuestra experiencia en la enseñanza de Geometría, surge
el interés por investigar el proceso de aprendizaje de semejanza de
triángulos mediadas por el Geogebra con estudiantes de secundaria
(14 y 15 años). Tomamos como fundamento teórico la dialéctica Herramienta-Objeto
de Douady (1986) y aspectos de la Teoría de la Registros de Representación
Semiótica de Duval (1995). Usamos como método aspecto de la Ingeniería
Didáctica de Artigue (1995). El análisis de la actividad del presente
reporte evidenció que el uso del Geogebra permite conjeturar el criterio
de semejanza de triángulos lado-lado-lado.


\section{LAS CONCEPCIONES DOCENTES DE NIVEL PRIMARIO EN TORNO A UN CONTENIDO
DEL DISEÑO CURRICULAR VIGENTE: LAS FRACCIONES PARA INICIAR EL APRENDIZAJE
DE LOS NÚMEROS RACIONALES EN LA ESCUELA PRIMARIA%
\footnote{Investigación realizada para la elaboración de la Tesis doctoral dirigida
por la Dra. María Elena Candioti%
}}

\begin{datos}
Nancy Ross. \\
Universidad Nacional de Entre Ríos,\\
\hfill Argentina, \\
\hfill nancyross@gesell.com.ar; macandioti@arnet.com.ar
\end{datos}

Nuestra investigación aborda las concepciones docentes al planificar
y diseñar estrategias para superar dificultades en la enseñanza de
las fracciones. Las informaciones recogidas en las salidas de campos
fueron analizadas desde la perspectiva fenomenológica sustentada por
Puig. En la investigación realizada hemos detectado que no hay un
conocimiento explícito de cuál es la finalidad de construir este campo
numérico, es decir, los docentes participantes no han expresado que
entienden por fracciones, para qué sirve este conocimiento y qué posibilitan
las fracciones en la resoluciones de problemas en los que hay necesidad
de trabajar con mediciones, razones, porcentajes, probabilidades o
proporciones.


\section{INSTRUMENTALIZACIÓN DEL LADO RECTO DE LA ELIPSE INFLUENCIADA POR
EL GEOGEBRA }

\begin{datos}
José Carlos León Ríos, Jesús Victoria Flores Salazar. \\
Pontificia Universidad Católica del Perú/Instituto de Investigación sobre Enseñanza de las Matemáticas- IREM,\\
\hfill Perú, \\
\hfill jleonr@ulima.edu.pe; jvflores@pucp.pe
\end{datos}

Presentamos el proceso de instrumentalización de la noción de lado
recto de la elipse, que corresponde a una de las actividades que comprende
nuestra investigación. Consideramos como marco teórico y metodológico
el Enfoque Instrumental de Rabardel (1995) y la Ingeniería Didáctica
de Artigue (1995), respectivamente. Identificamos algunas restricciones
de existencia y de acción que condicionan las acciones de los estudiantes
con algunas herramientas del Geogebra y otras que se inscriben en
la fase de personalización según Trouche (2004). Las propiedades de
los elementos existentes contribuyeron a que el proceso de Génesis
Instrumental esté dirigido al artefacto elipse.


\section{ANÁLISIS DE CONCEPCIONES SOBRE CIENCIA Y SU ENSEÑANZA EN PROFESORES
DE MATEMÁTICA }

\begin{datos}
Cecilia Crespo Crespo, Patricia Lestón. \\
Instituto Superior del Profesorado “Dr. Joaquín V. González”,\\
\hfill Buenos Aires. Argentina, \\
\hfill crccrespo@gmail.com; patricialeston@gmail.com
\end{datos}

Este trabajo presenta los resultados de una investigación llevada
a cabo con profesores de matemática que se encuentran realizando estudios
de postítulo en el área de matemática educativa. Se ha indagado a
través de cuestionarios de selección múltiple y entrevistas semiestructuradas
acerca de sus concepciones de ciencia y del proceso de aprendizaje
de las ciencias. Asimismo se comparan los resultados con las respuestas
obtenidas a las mismas preguntas dadas por profesores de física que
fueron reportados en una investigación previamente reportada (Ortalda,
2013).


\section{LA FRACCIÓN COMO VÍA DE EXPRESIÓN DE UNA RAZÓN Y DE UN COCIENTE.
ANÁLISIS DE UNA EXPERIENCIA DIDÁCTICA}

\begin{datos}
Daniela Ramos Banda, David Block Sevilla. \\
Departamento Centro de Investigación y de estudios avanzados del Instituto Politécnico Nacional - DIE-CINVESTAV,\\
\hfill México, \\
\hfill danielabanda7@gmail.com; davidblock54@gmail.com
\end{datos}

Esta experiencia de Ingeniería didáctica pretendió beneficiar el estudio
de las fracciones considerando los razonamientos que los alumnos pueden
hacer al trabajar con razones. Constó de diez clases en un grupo de
sexto grado de primaria del Distrito Federal. En las dos primeras
sub secuencias los alumnos llegan a conclusiones del tipo “n unidades
entre m es igual a n/m de unidad, mientras en la tercera expresan
una razón enunciada con dos números enteros (n de cada m) con una
fracción (n/m de). Se analizan fortalezas y debilidades del diseño
a la luz de los procedimientos de los alumnos. 


\section{ERRORES EN SITUACIONES DE VALIDACIÓN EN GEOMETRÍA ANALÍTICA: ANÁLISIS
DE UNA CATEGORÍA EMERGENTE EN UN ESTUDIO CON ALUMNOS UNIVERSITARIOS}

\begin{datos}
Carolina Boubée, Ana María Graciela Rey, Patricia Sastre Vázquez. \\
Facultad de Agronomía. UNCPBA,\\
\hfill Argentina, \\
\hfill cboubee@faa.unicen.edu.ar; grey@faa.unicen.edu.ar; \\ \hfill psastre@faa.unicen.edu.ar
\end{datos}

Del análisis de los resultados de un trabajo anterior, sobre tipos
de razonamientos en situaciones de validación referidas a Secciones
Cónicas, surgió la categoría denominada Interpretación simbólica de
un objeto algebraico, que incluye aquellas producciones que evidencian
la pérdida del carácter icónico de la expresión algebraica de una
cónica, reduciendo toda la expresión a símbolo, debido al peso otorgado
al símbolo + o – de las ecuaciones de la elipse y de la hipérbola,
respectivamente. Se presenta un análisis de este error, el cual excede
el marco algebraico denotando una falta de significado en el marco
de la Geometría Analítica. 


\section{GÉNERO Y DESARROLLO DEL TALENTO EN MATEMÁTICAS }

\begin{datos}
Rosa María Farfán Márquez, María Guadalupe Simón Ramos. \\
Centro de Investigación y de Estudios Avanzados del Instituto Politécnico Nacional,\\
\hfill México, \\
 \hfill rfarfan@cinvestav.mx; gsimon@cinvestav.mx
\end{datos}

Los resultados de un análisis desde la perspectiva de género indican
que las interacciones sociales y con el conocimiento por parte de
las mujeres, contribuyen al desarrollo de su auto-percepción de habilidad,
influenciando las decisiones de las niñas y adolescentes talentosas
acerca de sus logros, actuación escolar y aspiraciones educativas.
Analizaremos el rol del contexto escolar, familiar, social y del mismo
conocimiento en la formación de la auto-percepción de talento matemático
de niñas y adolescentes y en la construcción de conocimiento matemático
bajo la teoría Socioepistemológica.


\section{LA ARGUMENTACIÓN SUSTANCIAL EN EL ESTUDIO DE LA REPRESENTACIÓN GRÁFICA }

\begin{datos}
Alma Alicia Benítez Pérez, Martha Leticia García Rodríguez, Alicia López Betancourt. \\
Instituto Politécnico Nacional, Universidad de Durango,\\
\hfill México, \\
 \hfill albenper@gmail.com; martha.garcia@gmail.com;\\ \hfill abetalopez@gmail.com 
\end{datos}

La presente investigación tuvo como propósito analizar las estrategias
que el alumno empleó, en tareas que impulsan la exploración e interpretación
de la representación gráfica, lo que contribuyó a enriquecer la justificación
de sus afirmaciones con argumentos de tipo sustancial en un ambiente
de actitudes abiertas, reflexivas y críticas. La investigación se
ubicó en un paradigma de investigación cualitativo de corte etnográfico,
la observación del estudio se llevó a cabo con alumnos del nivel medio
superior. Los hallazgos muestran que los argumentos sustanciales resultaron
ricos en el desarrollo de recursos, pues originó la refutación entre
los equipos, elemento fundamental en el proceso de la argumentación.


\section{LA DECONSTRUCCIÓN COMO HERRAMIENTA EN LA MODELACIÓN DEL ÁREA DE LA
PESCA Y LA ACUICULTURA}

\begin{datos}
José Trinidad Ulloa Ibarra, Jaime Arrieta Vera, Gessure Abisaí Espino Flores. \\
Universidad Autónoma de Nayarit, Universidad Autónoma de Guerrero,\\
\hfill México, \\
 \hfill jtulloa@hotmail.com; jaime.arrieta@gmail.com; \\
\hfill abisai\_{}8282@hotmail.com 
\end{datos}

El presente trabajo es una contribución al proyecto global “Las prácticas
de modelación y la construcción de lo exponencial en comunidades de
la pesca” que bajo el marco de la socioepistemología hemos venido
desarrollando desde hace algunos años. En él analizamos las interacciones
de quiénes se encuentran en el campo de trabajo y requieren de la
utilización de modelos matemáticos para realizar su labor profesional;
sin embargo y con la finalidad de contribuir a la vinculación de los
conceptos teóricos que se estudian en la escuela se propone a la deconstrucción
como una herramienta útil que permite ese vínculo.


\section{SISTEMAS DE ECUACIONES LINEALESCON DOS VARIABLES: UNA VISIÓN DESDE
LA MATEMÁTICA EN EL CONTEXTO DE LAS CIENCIAS }

\begin{datos}
Verónica Neira Fernández , Jesus Victoria Flores Salazar. \\
Pontificia Universidad Católica del Perú/Instituto de Investigación sobre Enseñanza de las Matemáticas- IREM,\\
\hfill Perú, \\
 \hfill vneira@pucp.pe; jvflores@pucp.pe 
\end{datos}

Esta investigación tiene por objetivo analizar las dificultades que
los estudiantes tienen para traducir, del lenguaje verbal al matemático
y viceversa problemas contextualizados presentes en el libro texto
que utilizan cuando estudian sistemas de ecuaciones lineales con dos
variables. Además, elaboramos una propuesta que permita facilitar
la traducción de estos problemas contextualizados. Utilizamos como
marco teórico la Matemática en el Contexto de las Ciencias (MCC) de
Camarena (1999) y como metodología recurriremos a algunos aspectos
del Diseño de Programas de Estudio de las Ciencias básicas en Ingeniería
(DIPCING). 


\section{LIVROS DE MATEMÁTICA DO SÉCULO XVII E O ENSINO ATUAL DE MATEMÁTICA}

\begin{datos}
Arlete de Jesus Brito. \\
UNESP RIO CLARO,\\
\hfill Brasil, \\
 \hfill arlete@rc.unesp.br 
\end{datos}

Nessa apresentação exporemos investigação sobre as mudanças no discurso
sobre a matemática e sobre seu ensino, no século XVII. Para isso analisamos
livros textos que foram utilizados no Ginásio Acadêmico de Hamburgo,
relacionando-os a outros textos da época. Utilizamos Foucault (1972)
como referencial para nossas análises. Concluímos que a matemática
ensinada em ginásios acadêmicos protestantes teve um importante papel
naquelas mudanças que nortearam o ensino de matemática encontrado
até os dias atuais, na educação escolar.


\section{LOS NÚMEROS NEGATIVOS ¿CONSTITUYEN UN OBSTÁCULO EPISTEMOLÓGICO PERSISTENTE? }

\begin{datos}
Aurora Gallardo Cabello, José Luis Mejía Rodríguez. \\
Centro de Investigación y de Estudios Avanzados del IPN,\\
\hfill México, \\
 \hfill agallardo@cinvestav.mx; jose.luc.am@hotmail.com 
\end{datos}

Vía el método histórico-crítico, realizamos un análisis de las producciones
de estudiantes de secundaria al enfrentarse a dos problemas históricos
de enunciado verbal. Los resultados muestran que los obstáculos enfrentados
por los alumnos pueden ser de origen epistemológico. En el problema
de Bháskara (Colebrooke, 1817) al verificar una de las soluciones,
aparece un número negativo, el cual es descartado por los estudiantes.
En el problema de Chuquet (Marre, 1881), los alumnos ajustan los datos
solamente a una de las ecuaciones del sistema, ignorando la otra ecuación,
evitando así la solución negativa y dando origen a multiplicidad de
soluciones. 


\section{EL USO DE UN JUEGO DE ENTRENAMIENTO PARA LA ENSEÑANZA DE LOS CONCEPTOS
DE MEDIA Y VARIANZA}

\begin{datos}
José Marcos Lopes, Jaime Edmundo Apaza Rodriguez. \\
Universidade Estadual Paulista “Júlio de Mesquita Filho” – UNESP,\\
\hfill Brasil, \\
 \hfill jmlopes@mat.feis.unesp.br; jaime@mat.feis.unesp.br 
\end{datos}

En este artículo socializamos una experiência de enseñanza en sala
de clases, donde se usó un juego de dados para reforzar el aprendizaje
obtenido al estudiar los conceptos de media y varianza. El juego es
original y utiliza simultaneamente esos conceptos de Estadística Descriptiva.
Formulamos también algunos problemas envolviendo situaciones del juego
y donde las soluciones, obtenidas por los propios alumnos, tienen
por objetivo reforzar los conceptos matemáticos presentes en esas
definiciones. A partir de lo relatado por alumnos y profesores que
aplicaron el juego, podemos asegurar que la actividad realizada contribuyó
efectivamente para la fijación de esos conceptos.


\section{ACERCAMIENTO A LA NEGATIVIDAD EN NÚMEROS RACIONALES POR ESTUDIANTES
DE SECUNDARIA Y PROFESORES EN FORMACIÓN}

\begin{datos}
Aurora Gallardo, Gil Saavedra. \\
CENTRO DE INVESTIGACIÓN Y DE ESTUDIOS AVANZADOS,\\
\hfill México, \\
 \hfill agallardo@cinvestav.mx; gsaavedra@cinvestav.mx 
\end{datos}

El presente estudio aporta elementos teóricos sobre el estudio de
la negatividad en los números racionales y en las distintas formas
de representación de estos números encontradas en la matemática escolar,
por ejemplo como razones, porcentajes, medidas, etc. Los resultados
obtenidos hasta el momento, revelan que tanto estudiantes de secundaria
como profesores en formación tienden a centrar la atención en los
valores absolutos de los números, suponen por ejemplo que $0,2<-1,6$.
Además, al comprobar resultados en un sistema de ecuaciones simultáneas,
existe la tendencia a expresar en forma decimal las soluciones halladas
inicialmente en forma fraccionaria.


\section{\uppercase{ Fracaso escolar o vínculo transferencial negativo} }

\begin{datos}
María Analía Contarino, Mónica Adriana Real. \\
ISFD Nº1,\\
\hfill Argentina, \\
 \hfill pspmanaliac@hotmail.com;  monireal@gmail.com 
\end{datos}

Análisis crítico de diferentes experiencias didácticas durante tres
años en escuelas secundarias bonaerenses. Se trabajó con adecuaciones
de acceso, atendiendo a la diversidad cognitiva de los alumnos. Se
analizaron las intervenciones del docente; participación y respuestas
de los alumnos, padres en el rol e interacciones entre los actores.
Esta investigación analizaron las razones socio-afectivas que mediaron
en el proceso de aprendizaje de un grupo de alumnos que no alcanzó
los contenidos propuestos. Interesó, el por qué del fracaso cognitivo
en relación a los vínculos transferenciales y contra-transferenciales
que se suscitaban. Hacemos una propuesta de solución a los problemas.


\section{CÓMO INCLUIR LA CATEGORÍA DE MODELACIÓN ESCOLAR EN LAS PRÁCTICAS
DEL PROFESOR}

\begin{datos}
María Esther Magali Méndez Guevara, Diana García Abarca. \\
Universidad Autónoma de Guerrero, Unidad Académica de Matemáticas,\\
\hfill México, \\
 \hfill dna.alo@msn.com; mguevara83@gmail.com 
\end{datos}

Reportamos la etapa inicial de nuestra investigación cuyo objetivo
es evidenciar cómo los profesores se apropian de una categoría de
modelación de manera que puedan hacer sus propios diseños de clases.
Nuestro interés es indagar sobre la modelación, vista desde una postura
socioepistemológica, y su inclusión en las prácticas docentes. Para
nuestra investigación retomaremos una categoría de que ha demostrado
ser funcional para la construcción y evolución de usos de conocimientos
matemáticos. Sin embargo, profundizamos en cómo incluir esta categoría
en las prácticas del docente al grado que los profesores generen sus
propios diseños para sus clases de matemáticas. 


\section{RUTA DIDÁCTICA PARA EL ANÁLISIS DE LOS HITOS FUNDAMENTALES QUE SURGEN
DURANTE LA TRANSICIÓN ARITMÉTICA-ALGEBRAICA}

\begin{datos}
Esmeralda Ivonne Espinoza Martínez, Aurora Gallardo Cabello. \\
Centro de Investigación y de Estudios Avanzados del Instituto Politécnico Nacional,\\
\hfill México, \\
 \hfill eespinoza@cinvestav.mx; agallardo@cinvestav.mx 
\end{datos}

La Ruta Didáctica permite analizar qué sucede con 16 hitos fundamentales,
cuando un alumno de 14 años resuelve 13 problemas históricos en entrevista
video-grabada. Los resultados muestran la ampliación del campo semántico
y funcional del signo igual; el surgimiento de formas semánticas y
sintácticas equivalentes, en el planteamiento del problema resuelto
por el estudiante; el significado de la adición y sustracción como
una sola operación se enriquece en su dominio sintáctico vía la notación
completa de los números con signo, pero en el dominio semántico es
obstruido por la equivalencia sintáctica debido al poco uso de las
formas semánticas equivalentes


\section{LA MODELACIÓN Y LA EMERGENCIA DE LA INTEGRAL}

\begin{datos}
Mayra Rosalia Tocto Erazo; María Esther Magali Méndez Guevara. \\
Universidad Autónoma de Guerrero, Unidad Académica de Matemáticas,\\
\hfill México, \\
 \hfill mayra.tocto@gmail.com; mguevara83@gmail.com  
\end{datos}

Se presenta un panorama de la noción de Modelación Matemática y de
investigaciones hechas con respecto a la construcción de la integral.
De acuerdo a la revisión hecha, adoptamos a la Modelación Matemática
como construcción del conocimiento matemático, como una actividad
que trasciende y se resignifica. Se está desarrollando una idea que
mediante diseños de modelación de movimiento que implique el uso de
las graficas, resignifique el uso de la integral para descubrir lo
que sucedió o podría suceder en una situación de movimiento y se espera
ver como la categoría de Modelación permite la construcción de la
integral.


\section{TECNOLOGÍA DE LA INFORMACIÓN Y LA COMUNICACIÓN APLICADA EN EL PROCESO
ENSEÑANZA-APRENDIZAJE DE LA GEOMETRIA PLANA}

\begin{datos}
Roberto Byas, Ramón Blanco Sánchez. \\
UASD,  UC, \\
\hfill República Dominicana, Cuba, \\
 \hfill robertobyas@hotmail.com; ramón.blanco@reduc.edu.cu  
\end{datos}

El presente estudio pretende establecer el estado del conocimiento
sobre las Tics aplicado al aprendizaje de la geometría plana. En esta
etapa del estudio se procedió a la revisión de documentos impresos
y digitales, como forma de identificar las necesidades de los docentes
en formación tecnológica y la importancia que tiene el uso pertinente
de la tecnología en el proceso enseñanza-aprendizaje de la geometría
plana. Situación que induce a la necesidad de investigar sobre la
construcción de nuevos paradigmas y estrategias innovadoras que permitan
la creación de escenarios adecuados para que los estudiantes logren
un aprendizaje significativo en la geometría plana


\section{PROBLEMAS DE SOLUCIÓN ÓPTIMA EN GEOMETRÍA PLANA, SU ASPECTO MOTIVACIONAL
CON APOYO DE LAS TIC }

\begin{datos}
Roberto Byas, Ramón Blanco Sánchez. \\
UASD,  UC, \\
\hfill República Dominicana, Cuba, \\
 \hfill robertobyas@hotmail.com; ramón.blanco@reduc.edu.cu  
\end{datos}

Cuando se habla de buscar solución óptima de un problema, se piensa
de inmediato en métodos del cálculo diferencial u otros métodos de
optimización, pero existe una variedad de problemas geométricos cuya
solución óptima puede determinarse a través de conceptos geométricos.
Se fundamenta el uso de las TIC para apoyar procesos del pensamiento
lógico, en particular inducción-deducción. Esto es importante dada
la frecuencia con que en las clases de geometría plana, el maestro
hace deducciones sin que los alumnos tengan la menor idea de por qué
el maestro realiza cada uno de los pasos de la demostración.


\section{EL LABORATORIO DE FÍSICA I PARA LA ENSEÑANZA DE LOS SISTEMAS DE ECUACIONES
LINEALES EN EL BACHILLERATO TECNOLÓGICO}

\begin{datos}
Rogelio Martínez García, Ignacio Garnica y Dovala. \\
CECyT No 4 “Lázaro Cárdenas”, DME-Cinvestav Instituto Politécnico Nacional, \\
\hfill México, \\
 \hfill rmartinezga@ipn.mx; igarnica@cinvestav.mx 
\end{datos}

Durante el tiempo prescrito institucionalmente para la enseñanza de
las funciones y sistemas de ecuaciones lineales, los docentes de Álgebra
y del Laboratorio de Física I diseñaron e implementaron una estrategia
de enseñanza complementaria a la tradicional, relativa a la experimentación
de “fuerzas concurrentes (tensiones)”, por la que los estudiantes
interactuaron con una situación concreta, obtuvieron sus propios datos
y los dotaron de sentido al relacionar el fenómeno físico con el sistema
de ecuaciones que plantearon y solucionaron. Se reportan los resultados
de indagación e investigación que incluyó dos entrevistas semiestructuradas,
enfocadas a la exigencia matemática. 


\section{RESOLUÇÃO DE PROBLEMAS DE TAXAS RELACIONADAS COM OA}

\begin{datos}
Júlio Paulo Cabral dos Reis, João Bosco Laudares, Dimas Filipe de Miranda. \\
Pontifícia Universidade Católica de Minas Gerais, \\
\hfill Brasil, \\
 \hfill  julio.paulo1986@hotmail.com; jblaudares@terra.com.br;\\
\hfill dimasfm48@yahoo.com.br 
\end{datos}

A partir da aplicação de um Objeto de Aprendizagem (OA), com resolução
de problemas de Taxas Relacionadas, avaliou-se as contribuições oferecidas
pelo mesmo. O OA foi aplicado a estudantes de Cálculo nos cursos de
Licenciatura em Matemática e Engenharia, ambos da PUCMinas Brasil.
A análise dos dados obtidos pelas observações no momento da aplicação
do OA e as anotações realizadas pelos estudantes permitiram avaliar
tais contribuições. O embasamento teórico foi constituído da Resolução
de Problemas e Informática Educativa. A análise da aplicação do OA
permitiu verificar: compreensão do conceito de taxas relacionadas
e ressignificação do conteúdo por parte dos alunos envolvidos. 


\section{INFLUENCIA DE LA METODOLOGÍA INDAGATORIA DE LA ENSEÑANZA DE LAS CIENCIAS
MEDIANTE LA INDAGACIÓN (ECBI) EN EL DESARROLLO DEL PENSAMIENTO LÓGICO
MATEMÁTICO EN NIÑOS DE 3 A 5 AÑOS” EN UNA COMUNIDAD NATIVA DE PERÚ}

\begin{datos}
Regina Moromizato Izu, Rosa Eulalia Cardoso Paredes, Patricia Quevedo. \\
Pontificia Universidad Católica del Perú, \\
\hfill Perú, \\
\hfill rmotomizaro@pucp.pe; rcardoso@pucp.pe 
\end{datos}

En este trabajo mostramos la evaluación del impacto de la metodología
indagatoria en el desarrollo de las competencias matemáticas (comunicación
matemática, razonamiento y demostración y resolución de problemas)
relacionadas con la adquisición del concepto de número en escolares
de 3 a 5 años de edad de una comunidad nativa de la Selva central
de Perú. A los niños se evaluó mediante la aplicación de pre-test
y post-test que incluyen los procesos lógicos relacionados con la
adquisición del número. Se identificó y analizó las prácticas pedagógicas
de los docentes relacionadas al aprendizaje de competencias matemáticas
a fin de detectar los desempeños que favorezcan la aplicación del
método. 


\section{UN ESTUDIO DE LOS MEDIOS SEMIÓTICOS DE OBJETIVACIÓN Y PROCESOS DE
OBJETIVACIÓN EN ESTUDIANTES DE GRADO CUARTO DE PRIMARIA}

\begin{datos}
Adriana Lasprilla Herrera, Rodolfo Vergel Causado. \\
Universidad Distrital Francisco José de Caldas, \\
\hfill Colombia, \\
 \hfill  arranala@gmail.com; Rodolfovergel@gmail.com
\end{datos}

Este trabajo se inspira en los trabajos realizados sobre la semiótica-cultural
adelantados principalmente por Radford (2008, 2010 y 2010a). En ellos
se reconoce un espacio, en donde se estudia la emergencia del pensamiento
algebraico y los procesos interpretativos de los estudiantes se investigan
a través de la objetivación del saber. Con el interés de empezar a
comprender cómo surge el pensamiento algebraico en los niños, se identifica,
describe y analiza los medios semióticos de objetivación y los procesos
de objetivación en tareas de generalización de patrones, en contextos
figurales y numéricos, en alumnos de cuarto grado de educación básica
primaria (9-10 años).


\section{LA CONTRACCIÓN SEMIÓTICA COMO PROCESO DE OBJETIVACIÓN EN EL PENSAMIENTO
VARIACIONAL}

\begin{datos}
Paola Carolina Moreno Cabeza. \\
Universidad Distrital, \\
\hfill Bogotá - Colombia, \\
 \hfill  Paca.mate@gmail.com
\end{datos}

Esta propuesta presenta los desarrollos de un trabajo de grado para
maestría realizado en el Colegio Claretiano de Bosa (Bogotá, Colombia)
Dentro del trabajo se pretende analizar el proceso de objetivación
de un grupo de estudiantes a partir de la evolución de los medios
semióticos emergentes. El foco teórico se centra en el proceso de
contracción semiótica vista como la evolución de nodos semióticos
desde la perspectiva de la teoría cultural de la objetivación propuesta
por Radford (2006). El análisis de la información se realiza desde
la conformación de nodos semióticos abordando una perspectiva multimodal
del pensamiento humano (Arzarello, 2006). 


\section{\uppercase{ El uso de la cantidad en una comunidad de conocimiento}}

\begin{datos}
Otomí Teresa Gpe. Parra Fuentes, Francisco Cordero Osorio. \\
Cinvestav-IPN, \\
\hfill México, \\
 \hfill  parra.tere@gmail.com
\end{datos}

Esta investigación trata de dar cuenta de los usos de la cantidad
en una comunidad otomí de San Pablito, Puebla, México. Específicamente
nos enfocamos en las prácticas de curación y de comercio, en donde
emplean el papel amate. En el primero se recortan figuras de papel
amate que de acuerdo a su cosmovisión toman vida durante las ceremonias,
y en la segunda el papel amate es visto como un medio de sustento
económico. Bajo estas diferencias el uso de la cantidad es analizado
con base al modelo comunidad de conocimiento matemático. 


\section{INTRODUCCIÓN TEMPRANA DE IDEAS RELACIONADAS CON INFERENCIA ESTADÍSTICA:
EL DESARROLLO DE UNA ACTIVIDAD}

\begin{datos}
Blanca Ruiz; Armando Albert. \\
Cinvestav-IPN, \\
\hfill México, \\
 \hfill  bruiz@itesm.mx; albert@itesm.mx
\end{datos}

En este reporte retomamos la propuesta de Aliaga y Gunderson sobre
la inserción de inferencia estadística al inicio de un primer curso
universitario de probabilidad y estadística tomando en cuenta el desarrollo
de las vinculaciones históricas entre la variable estadística y aleatoria,
los resultados de investigación alrededor de inferencia informal y
un marco sustentado en pre-figuraciones de los conceptos. Narraremos
el desarrollo de la puesta en escena de la actividad ‘El contrato
del basquetbolista’ por estudiantes universitarios que han llevado
el curso bajo esta propuesta en ciernes, así como los resultados obtenidos
cuando resuelven de manera individual problemas semejantes.


\section{CONOCIMIENTO PEDAGÓGICO DISCIPLINAR DE PROFESORES DE MATEMÁTICA }

\begin{datos}
ALEJANDRO PEDREROS, PIERINA ZANOCCO. \\
Pontificia Universidad Católica de Chile, Universidad Santo Tomás, \\
\hfill Chile, \\
 \hfill  alejandro.pedreros@uc.cl; pzanocco@santotomas.cl
\end{datos}

Investigación exploratoria-cuantitativa, en un paradigma interpretativo,
consideró una muestra de 380 estudiantes de Pedagogía en Educación
Básica de cuatro universidades. Pretendía identificar la relación
entre variables específicas (cursos disciplinares y de Didáctica de
Matemática, puntaje PSU, desempeño a lo largo de la carrera, formación
de sus docentes) y el conocimiento pedagógico de las Matemáticas.
El análisis estadístico demostró que los factores explicativos del
desarrollo del conocimiento pedagógico-disciplinar son: Calidad de
aprendizajes en Didáctica y relación con la práctica pedagógica; Conocimientos
disciplinares; Metodología de los académicos de la Didáctica; Valoración
de los académicos de las facultades disciplinares por enseñar en Educación. 


\section{CONSTRUCCIONES Y MECANISMOS MENTALES PARA IMPLEMENTAR Y DESARROLLAR
EL CONCEPTO DE LOS VECTORES EN TRES DIMENSIONES (3D) MEDIANTE EL APOYO
DE LA HERRAMIENTA CABRI PARA EL CÁLCULO DE VOLÚMENES.}

\begin{datos}
Luís Albeiro Zabala Jaramillo, Marcela Parraguez. \\
 Universidad De Medellín, Pontificia Universidad Católica de Valparaíso, \\
\hfill Colombia, Chile, \\
 \hfill  lzabala@udem.edu.co; marcela.parraguez@ucv.cl
\end{datos}

La propuesta presenta un reporte de aspectos Histórico-Epistemológico
(Martínez y Benoit, 2008) sustentando la construcción del conocimiento
matemático del producto vectorial. Como resultado de la indagación,
dicho concepto matemático se interpreta como elemento organizador
de sistemas simbólicos cartesianos, igualmente puede concebirse como
un concepto geométrico de volumen (Ricardo, 2012), partiendo de diferentes
figuras geométricas que se encuentran al interior del paralelepípedo.
Estas interpretaciones sustentan construcciones y mecanismos mentales
provistas en la Teoría APOE (Arnon et al, 2014) para implementar y
desarrollar el concepto de los vectores en tres dimensiones, en aprendices
del álgebra lineal, mediados con software Cabri (Artigue, 2011).


\section{METODOLOGÍA PARA LA OBTENCIÓN DE LOS QUÉS EN LA CASA DE LA CALIDAD
PARA LA DETERMINACIÓN DEL PERFIL DE EGRESO EN UNA MAESTRÍA EN MATEMÁTICAS }

\begin{datos}
Gladys Denisse Salgado Suárez, José Dionicio Zacarías Flores. \\
 FCFM, BUAP, \\
\hfill México,\\ 
 \hfill  gladys008@hotmail.com; jzacarias@fcfm.buap.mx
\end{datos}

Con el propósito de fortalecer un programa de maestría, es necesario
poner especial atención a la calidad de sus egresados y para ello
al perfil de egreso. Al considerar a un programa educativo como un
proceso, el área de control estadístico de la calidad tiene la capacidad
de ser aplicable para generar respuestas de mejora, una de estas técnicas
de calidad es el Despliegue de la Función de Calidad, que guía este
trabajo en el que se muestra la metodología para la identificación
de las características deseables que debe contemplar el perfil de
egreso con ayuda de diversas herramientas estadísticas. 


\section{EVOLUCIÓN EN EL ESQUEMA DEL CONCEPTO TRANSFORMACIÓN LINEAL. UNA MIRADA
A TRES INTERPRETACIONES DESDE LA TEORÍA APOE}

\begin{datos}
Isabel Maturana Peña, Marcela Parraguez González. \\
 Pontificia Universidad Católica de Valparaíso, \\
\hfill Chile, \\
 \hfill  isamatup@hotmail.com; marcela.parraguez@ucv.cl
\end{datos}

Basados en la teoría APOE y con un diseño metodológico de estudio
de caso, investigamos los niveles de coherencia del esquema para el
concepto Transformación Lineal (TL); entendiendo este como una articulación
entre diferentes interpretaciones, las que hemos denominado interpretación
funcional, matricial y geométrica. En este reporte, damos cuenta de
la segunda etapa en la investigación, referida al proceso de validación
del modelo multinterpretativo para el concepto TL, son las entrevistas
la fuente de nuestros datos, y es desde estas, que se logró caracterizar
los niveles - Intra, Inter y Trans- para el concepto TL. 


\section{LA HISTORIA DE LA MATEMÁTICA COMO UN RECURSO DIDÁCTICO. EL CASO DE
LA SOLUCIÓN DE LA ECUACIÓN CÚBICA EN LA OBRA DE OMAR JAYYAM}

\begin{datos}
Ismael Arcos, Mónica Lorena Micelli. \\
Facultad de Ingeniería - Universidad Autónoma del Estado de México, Instituto Superior del Profesorado “Dr. Joaquín V. González”\\
\hfill México, Buenos Aires (Argentina) \\
 \hfill  ismael\_{}arcos@msn.com; monikmathis@gmail.com
\end{datos}

El trabajo que se presenta es parte de una investigación que se encuentra
en proceso, en la cual se concibe la Historia de la Matemática como
un recurso didáctico. En este caso se aborda el asunto de la resolución
de ecuaciones cúbicas, específicamente lo encontrando en la obrade
Omar Jayyam, poeta y matemático árabe del siglo xi, quien realizó
un trabajo minucioso de esta clase de ecuaciones. El gran aporte de
la Matemática árabe que no siempre es valorizada. Con este reconocimiento
no sólo se busca recuperar aportaciones a la Matemática que han caído
en el olvido; también se pretende mostrar, mediante la exhibición
de un método geométrico para la solución de ecuaciones algebraicas,
una manera didáctica para la integración distintas ramas de la Matemática. 


\section{OS MÓDULOS DE DIDÁTICA DA MATEMÁTICA NA FORMAÇÃO DE PROFESSORES LEIGOS
NO PROJETO LOGOS II}

\begin{datos}
Cristiane Talita Gromann de Gouveia. \\
 Universidade Estadual Paulista “Júlio Mesquita Filho” – Rio Claro, \\
\hfill SP - Brasil, \\
 \hfill  thalita\_{}hehe@hotmail.com
\end{datos}

Neste resumo vamos apresentar um recorte de uma pesquisa de mestrado
em desenvolvimento que tem como objetivo geral elaborar uma história
sobre a formação de professores leigos no estado de Rondônia, que
aconteceu na década de 70 do século XX, por meio do Projeto Logos
II. Este recorte, contará com uma descrição dos módulos da disciplina
de didática da matemática, onde encontramos vestígios da teoria de
Piaget e também indicativos de que a mesma recebeu aportes teóricos
do movimento da matemática moderna (MMM), destacando a ideia do tratamento
dos conteúdos via teoria dos conjuntos.


\section{EL CONOCIMIENTO MATEMÁTICO PARA LA ENSEÑANZA DEL LÍMITE AL INFINITO
DE UNA FUNCIÓN: UN ESTUDIO DE CASOS}

\begin{datos}
José Rafael Couoh Noh, María Guadalupe Cabañas Sánchez, \\
Salvador Llinares Ciscar, Julia Valls González. \\
 Universidad Autónoma de Guerrero, Universidad de Alicante, \\
\hfill México, España, \\
 \hfill  jose\_{}rafael\_{}1988@hotmail.com; gcabanas.sanchez@gmail.com;\\
\hfill sllinares@ua.es; julia.valls@ua.es
\end{datos}

Nuestro estudio se interesa por caracterizar el conocimiento matemático
para la enseñanza del límite al infinito de una función. La investigación
se sustenta en el modelo denominado MKT de Ball, Thames y Phelps (2008)
y participan tres profesores de matemáticas. Los datos se recolectaron
mediante entrevistas semiestructuradas e involucraron aspectos relacionados
con los datos personales de los participantes, el currículum escolar,
la planificación del profesor y del investigador sobre el tópico.
El análisis de los datos se realiza en tres fases: generación de las
unidades de análisis; agrupamiento en categorías de dichas unidades
y caracterización del conocimiento del profesor.


\section{ENSEÑANZA DE FRACCIONES Y EL PENSAMIENTO MATEMÁTICO}

\begin{datos}
Ubaldo José Buelvas Solórzano. \\
 Universidad de Sucre, \\
\hfill Colombia, \\
\hfill Ubaldo959@hotmail.com
\end{datos}

El propósito de esta investigación fue identificar los métodos empleados
por docentes de Secundaria del área de matemáticas, del municipio
de Sincelejo-Colombia entre los años 2012 y 2013, en el proceso de
enseñanza de fracciones y su incidencia en el desarrollo del pensamiento
matemático. La indagación sigue resultados obtenidos por Kieren (1981),
Behr (1983), Dickson (1984) y Llinares y Sánchez (1988); quienes consideran
que la enseñanza y aprendizaje de las fracciones debe realizarse partiendo
de la relación parte –todo, fuente primaria para la adquisión de vocabulario
y de recursos pedagógicos para la enseñanza de fracciones.


\section{HEURÍSTICA DEL DESCUBRIMIENTO, UNA EXPERIENCIA CON PROFESORES }

\begin{datos}
Álvaro Sebastián Bustos Rubilar, Gonzalo Zubieta Badillo. \\
 Cinvestav, \\
\hfill Chile, México, \\
\hfill bustos.rubilar@gmail.com; gzubieta@cinvestav.mx
\end{datos}

El presente reporte muestra un estudio de casos en el cual se aplicó
una metodología basada en el cuasi-empirismo de Lakatos. Se exponen
los resultados de una primera exploración llevada a cabo con profesores.
Centrándose principalmente en el proceso que hay entre una conjetura
primitiva y una conjetura mejorada. Se expone cómo los profesores
redescubren conocimiento y perfeccionan la conjetura durante un proceso
de discusión, en el cual el rol del contraejemplo es vital para generar
controversia en las conclusiones dadas por estos, junto con la incidencia
del uso de un software de geometría dinámica durante el proceso de
discusión.


\section{ESPACIOS DE TRABAJO GEOMÉTRICO CON ACODESA: DOS EJEMPLOS}

\begin{datos}
José Luis Soto Munguía. \\
 Universidad de Sonora UNISON, \\
\hfill México, \\
\hfill jlsoto@gauss.mat.uson.mx
\end{datos}

El presente trabajo reporta una experiencia de diseño para la formación
de profesores de matemáticas de Escuelas Secundarias y de Bachillerato,
en el marco de la reforma curricular en curso en nuestro país. Nuestro
trabajo se ha centrado en el diseño de actividades didácticas para
profesores; nos basamos en la metodología ACODESA propuesta por Fernando
Hitt, construyendo lo que Kuzniak y colaboradores hay llamado Espacios
de Trabajo Geométrico. Describimos aquí algunos ejemplos sobre este
diseño y analizamos las experiencias que los profesores han tenido
con ellos. Por razones de espacio nos referiremos parcialmente a estas
actividades. 


\section{ANÁLISIS COGNITIVO Y ANÁLISIS DE INSTRUCCIÓN EN EL ESTUDIO DE LIBROS
DE TEXTO HISTÓRICOS DE MATEMÁTICAS}

\begin{datos}
Miguel Picado. \\
Universidad Nacional de Costa Rica, \\
\hfill Costa Rica, \\
\hfill miguepicado@hotmail.com
\end{datos}

Se presenta un estudio histórico sobre el tratamiento dado al Sistema
Métrico Decimal (SMD) en libros de texto de matemáticas en España
en el siglo XIX. Este se enmarca en las investigaciones en Historia
de la Educación Matemática. El análisis de los libros de texto se
llevó a cabo desde el Análisis Didáctico, enfocando los análisis cognitivo
y de instrucción. Los resultados muestran que libros de texto analizados
presentan una involución en la manera de abordar el SMD. Hubo una
disminución de ejemplos y ejercicios mostrados conforme sucedieron
las etapas; hubo un estancamiento en el aprendizaje memorístico.


\section{MODELOS DE PROFESORES FORMADORES DE PROFESORES DE MATEMÁTICA: ¿CUÁLES
SON Y EN QUÉ MEDIDA SE TRANSMITEN A LOS FUTUROS DOCENTES?  UN ESTUDIO
DE CASOS }

\begin{datos}
Mónica Olave Baggi, Javier Lezama Andalón, Verónica Molfino Vigo. \\
Instituto de Profesores Artigas, Instituto Politécnico Nacional, \\
\hfill Uruguay, México, \\
\hfill monicaolave23@gmail.com; jlezamaipn@gmail.com;\\\hfill veromolfino@gmail.com
\end{datos}

En este trabajo presentamos un estudio que busca caracterizar los
modelos docentes de un grupo de formadores de profesores de Matemática
en un Instituto de Formación Docente de Uruguay y analizar si éstos
son transmitidos a sus estudiantes, futuros profesores de Matemática
de nivel Secundario y Bachillerato. Para lograr los objetivos se exploraron
algunos aspectos relativos a los formadores de profesores - su formación,
sus prácticas de aula, su visión de la docencia, la naturaleza del
tipo de actividades que lleva adelante con sus estudiantes- y se indagó
cómo viven dichos estudiantes la experiencia de asistir a esas clases. 


\section{CURSO VIRTUAL DE ESTADÍSTICA DESCRIPTIVA EN MOODLE CON APOYO DE R
EN UNA UNIVERSIDAD COLOMBIANA}

\begin{datos}
Martin German Zambrano Castro. \\
Corporación Universitaria Minuto de Dios, \\
\hfill Colombia, \\
\hfill mzambrano@uniminuto.edu; gerzamcas@gmail.com
\end{datos}

La construcción del curso virtual de Estadística Descriptiva en Moodle
y con apoyo de R usa el diseño instruccional en el desarrollo del
curso virtual de aprendizaje en la plataforma Moodle, a través de
él, se estructura el material usado como apoyo en el aprendizaje del
estudiante, mucho de este material está desarrollado con el uso del
software estadístico R, específicamente R-Comander y el uso del paquete
TeachingDemos para apoyar el aprendizaje del estudiante en temas específicos
de la Estadística Descriptiva. La investigación se centra en programas
de formación que tienen en su pensum un curso de Estadística Descriptiva.


\section{LA IMPORTANCIA DE LAS ACTIVIDADES EXTRAMUROS EN LA CONSTRUCCIÓN DEL
CONOCIMIENTO MATEMÁTICO, ESTUDIO SOBRE UNA COSTUMBRE DIDÁCTICA: EL
CASO PARTICULAR DEL COLEGIO MADRID }

\begin{datos}
Juan Antonio González Macias, Rosa María Farfán Márquez. \\
Cinvestav-IPN, \\
\hfill México, \\
\hfill jgonzalezm@cinvestav.mx, rfarfan@cinvestav.mx
\end{datos}

En la presentación destacaremos el hecho de que las actividades extramuros
conocidas por las instituciones como “Prácticas de Campo” (PC), no
sólo ponen en juego el conocimiento teórico, sino que generan al mismo
tiempo lazos de pertenencia e identidad entre los alumnos, así como
también hacia la institución. Investigaciones en el campo de la Matemática
Educativa analizan la tesis que sostiene que el problema mayor de
la enseñanza de las matemáticas no está en la organización y jerarquización
temática del contenido, sino en la escasa claridad de la textura social
del conocimiento matemático.


\section{\uppercase{ Las prácticas de simulación lineal y la emergencia de
la integral} }

\begin{datos}
Melvis Ramírez Barragán, Jaime Arrieta Vera. \\
Universidad Autónoma de Guerrero - Unidad Académica de Matemáticas, \\
\hfill México, \\
\hfill rbmelvis1789@gmail.com; jaime.arrieta@gmail.com
\end{datos}

Esta investigación aborda la problemática de la separación de la escuela
y su entorno: las prácticas escolares donde la integral esta presente
son ajenas en ambientes no escolares. Proponemos las prácticas de
modelación/simulación como puente entre ellos. La simulación de fenómenos
es una práctica recurrente de diferentes comunidades con intención
de reproducir algún fenómeno partiendo de sus modelos, con ello posibilita
manipularlo al variar sus parámetros sin la necesidad de que ocurra.
Este trabajo se circunscribe a la emergencia de la integral al simular
fenómenos partiendo de modelos lineales diferenciales. El marco que
sustenta la investigación es la Socioepistemología (TSME).


\section{EL USO DE HERRAMIENTA TECNOLÓGICA EN UNA PROPUESTA DIDÁCTICA EN TORNO
A LA CORRELACIÓN LINEAL }

\begin{datos}
Gessure Abisaí Espino Flores, Enrique Hugues Galindo. \\
Universidad Autónoma de Nayarit, Universidad de Sonora,  \\
\hfill México, \\
\hfill gessure@uan.edu.mx; ehugues@gauss.mat.uson.mx
\end{datos}

En el presente trabajo se muestra cómo se considera debe incorporarse
una herramienta tecnológica en los procesos de enseñanza y de aprendizaje
de conceptos estadísticos o, englobándolos, en la educación estadística.
Para ilustrar tales consideraciones se usa una propuesta didáctica
diseñada para desarrollar el concepto de correlación lineal dentro
de un curso universitario de estadística, en la cual se ha incorporado
la herramienta tecnológica GeoGebra y se ha seguido la metodología
ACODESA. Además de esto, se presenta un análisis de interacciones
que surgieron al implementar dicha propuesta. 


\section{LA CORRELACIÓN A TRAVÉS DEL GEOGEBRA COMO HERRAMIENTA COGNITIVA EN
LA CLASE DE ESTADÍSTICA}

\begin{datos}
Gessure Abisaí Espino Flores$^1$, José Trinidad Ulloa Ibarra$^2$,\\ Jaime Lorenzo Arrieta Vera$^3$ . \\
 $^{1,2}$Universidad Autónoma de Nayarit, $^3$Universidad Autónoma de Guerrero,  \\
\hfill México, \\
\hfill gessure@uan.edu.mx; jtulloa@uan.edu.mx;\\ \hfill jaime.arrieta@gmail.com
\end{datos}

En el presente trabajo se muestra una actividad didáctica para el
tema de correlación bivariada, la cual fue diseñada para la utilización
del software GeoGebra, así como aquellas consideraciones teóricas
que se realizaron para el uso de éste; además de la implementación
a través de la metodología ACODESA, la cual proporciona las pautas
necesarias para la implementación de la tecnología. Además se realiza
un análisis sobre las concepciones que presentan los alumnos en el
tema de correlación bivariada debido a que la actividad fue implementada
para estudiantes de psicología a nivel superior.


\section{ASIMILACIÓN DEL CONCEPTO DE DERIVADA EN UN CONTEXTO GEOMÉTRICO }

\begin{datos}
Jorge Nájera Godínez, Esteban Mendoza Sandoval, \\  Julián Huitzi Patricio Martínez Guadalupe Cabañas-Sánchez, Catalina Navarro. \\
 Universidad Autónoma de Guerrero,  \\
\hfill México, \\
\hfill jnajera@uagro.mx; emendonza@uagro.mx; \\ \hfill jupatricio@uagro.mx; gcabanas.sanchez@gmail.com;  \\ \hfill nasacamx@yahoo.com.mx
\end{datos}

En el presente trabajo, exploramos niveles de comprensión del concepto
de derivada en estudiantes de nivel superior, considerando que la
comprensión de un concepto es lograda a través de su asimilación,
compuesta por tres acciones fundamentales: identificación, realización
y aplicación del concepto. El estudio consistió en tres actividades,
correspondiéndose a cada una de las actividades de asimilación, aplicadas
en tres sesiones de manera escrita, para su posterior análisis. Tras
el análisis, nos encontramos con que los estudiantes muestran deficiencias
en su comprensión del concepto, no pudiendo la mayoría de ellos, identificar,
realizar o aplicar el mismo.


\section{CONTRUCCIÓN DEL CONCEPTO DE PROBABILIDAD EN PRIMARIA DESDE UNA PERSPECTIVA
INTUITIVA}

\begin{datos}
Edwin Chaves Esquivel. \\
 Universidad de Costa Rica (UCR), Universidad Nacional (UNA),  \\
\hfill Costa Rica, \\
\hfill echavese@gmail.com; edwin.chaves.esquivel@una.cr 
\end{datos}

Esta propuesta ha sido elaborada desde un proyecto de investigación
desarrollado para mejorar la enseñanza y aprendizaje de la Estadística
y la Probabilidad en la educación primaria. Se fundamenta teóricamente
una propuesta didáctica encaminada a la construcción del concepto
de laplaciano de Probabilidad durante la primaria (6½ a 12 años).
Por medio de las creencias e intuiciones sobre el azar con que los
niños llegan al sistema educativo, se propone pasa a paso ir generando
ciertas habilidades que les permitan identificar y emplear nociones
básicas para la construcción de concepto de probabilidad. 


\section{LA NOCIÓN DE COMPARACIÓN EN MATEMÁTICAS COMO UN ELEMENTO ARTICULADOR
EN LA DIDÁCTICA}

\begin{datos}
Juan Alberto Acosta Hernández, Carlos Rondero Guerrero, Anna Tarasenko. \\
 Universidad Autónoma del Estado de Hidalgo,  \\
\hfill México, \\
\hfill acostah@uaeh.edu.mx: ronderocar@gmail.com, \\
\hfill anataras@uaeh.edu.mx 
\end{datos}

Se realiza un rescate epistemológico de la noción de comparación desde
una perspectiva sistémica, lo que permite sostener que detrás de algunos
de sus significados se ubica una idea germinal. Además se considera
que existe una desarticulación entre los diferentes objetos matemáticos
que son estudiados a lo largo de la trayectoria escolar de un estudiante
asociados a la noción de comparación. Por otra parte, en la enseñanza
institucionalizada, los significados de la comparación no se perciben
en su estatus metamatemático, y su concepción en cada nivel educativo,
es abordada de manera desarticulada y el discurso escolar no propicia
su articulación. 


\section{LA RESIGNIFICACIÓN Y CONSTRUCCIÓN DE LA FUNCIÓN CUADRÁTICA A PARTIR
DE LA MODELACIÓN-GRAFICACIÓN }

\begin{datos}
Fredy de la Cruz Urbina, Hipólito Hernández Pérez. \\
 Universidad Autónoma de Chiapas, \\
\hfill México, \\
\hfill frecu@hotmail.com, polito\_{}hernandez@hotmail.com 
\end{datos}

Construir el diseño y modelo algebraico que representa a un fenómeno
o situación, se ha vuelto complejo para el alumno. Por ello abordamos
el estudio de la función cuadrática en el nivel medio superior, desde
una mirada Socioepistemológica, para averiguar que prácticas sociales
permiten la resignificación y construcción de la concepción de “función
cuadrática” a partir de la modelación-graficación y con el uso de
la Ingeniería Didáctica diseñar secuencias didácticas basadas en el
movimiento con sensores, llenado y vaciado de recipientes; creemos
que estas actividades experimentales ayudarán a construir y resignificar
lo cuadrático en la estructura mental del alumno. 


\section{USO DE LA METODOLOGÍA ACODESA EN UN EXPERIMENTO DE ENSEÑANZA CON
FUTUROS DOCENTES DE EDUCACIÓN PRIMARIA}

\begin{datos}
Gabriela Valverde Soto. \\
 Universidad de Costa Rica, \\
\hfill Costa Rica, \\
\hfill gabriela.valverde@ucr.ac.cr  
\end{datos}

El propósito del reporte es describir brevemente una investigación
de diseño que ha sido desarrollada en el contexto de la formación
de docentes en España. Esta contempla dos objetivos generales: (1)
estudiar el proceso de elaboración, implementación y análisis de una
“secuencia de trabajo en el aula” sobre la razón y la proporcionalidad,
y (2) investigar cómo contribuye esta secuencia en la competencia
matemática de dichos estudiantes; la cual se procuró promover mediante
la resolución de problemas y uso de la metodología de trabajo colaborativo
ACODESA. Este reporte muestra resultados de la investigación respecto
a la dinámica de trabajo colaborativo elegida para implementar el
diseño instruccional. 


\section{ESTRATEGIAS DIDÁCTICAS DE ATENCIÓN A LA DIVERSIDAD UTILIZANDO LAS
TIC´S }

\begin{datos}
Hilda Icela Garzón Barrientos, Alicia Rocendo Ramírez, Guadalupe Cabañas-Sánchez. \\
 Universidad Autónoma de Guerrero, \\
\hfill México, \\
\hfill gabhix@yahoo.com.mx; ton.ali24.07@gmail.com; \\ \hfill gcabanas.sanchez@gmail.com  
\end{datos}

El presente estudio se enfoca en el desarrollo de aplicaciones didácticas
en el proceso de enseñanza de niños con hiperactividad mediante las
nuevas tecnologías, en el interés de reducir su impulsividad e inquietud
motriz y centrar su atención, que es la fuente principal de sus problemas.
Se sustenta en un enfoque interdisciplinario el cual plantea preguntas
clave, cuyas respuestas deben ser atendidas por expertos. Estas preguntas,
giran en torno a los paradigmas psicoafectivo, evolutivo, cognitivo,
biológico, educacional y sociocultural. Es en el cognitivo en que
interviene este trabajo, a partir del diseño de actividades sustentadas
en el software educativo Scratch.


\section{ASIMILACIÓN DEL CONCEPTO VARIACIÓN CUADRATICA EN ESTUDIANTES DE NOVENO
GRADO }

\begin{datos}
Anairis de la Cruz Benito, Melby Cetina Vázquez, \\
Miriam Ramos Hernández, Guadalupe Cabañas-Sánchez, Catalina Navarro Sandoval.\\
 Universidad Autónoma de Guerrero, \\
\hfill México, \\
\hfill iris1790@gmail.com; melby\_{}gcv@hotmail.com; miriam.rever99@gmail.com; \\ 
\hfill gcabanas.sanchez@gmail.com; nasacamx@yahoo.com.mx   
\end{datos}

En el presente trabajo se reportan los niveles de comprensión alcanzados
por estudiantes de noveno grado (14-15 años) de una secundaria del
estado de Guerrero, México, sobre el concepto variación cuadrática.
Esto de acuerdo a las tres acciones definidas en la metodología de
la enseñanza de la matemática de la escuela cubana: identificar, realizar
y aplicar un concepto. Para ello se exploró un diseño que contempló
dichas acciones inmersas en situaciones contextualizadas. Los resultados
obtenidos proyectaron que la mayoría de los estudiantes alcanzaron
los niveles de identificación y aplicación, presentando dificultades
para alcanzar el nivel de realización.


\section{ÁREA: CONCEPTO Y DEFINICIÓN ARTICULADOS POR LA TSD}

\begin{datos}
Mihály André Martínez Miraval, Francisco Ugarte Guerra. \\
Universidad Peruana de Ciencias Aplicadas, Pontificia Universidad Católica del Perú, \\
\hfill Perú, \\
\hfill mihaly.martinez@upc.edu.pe, fugarte@pucp.edu.pe   
\end{datos}

Nuestra investigación tiene por objetivo fundamentar el uso de la
Teoría de Situaciones Didácticas (TSD) de Brousseau y de la Ingeniería
Didáctica (ID) de Artigue para definir el área como una suma infinita.
Para ello mostraremos cómo se pueden utilizar la TSD y la ID para
construir una secuencia didáctica, mediada por el GeoGebra, que permita
articular la concepción que tienen los estudiantes universitarios
del concepto de área con la definición de área como una suma de infinitos
términos.

\setcounter{section}{71}


\section{FORMACIÓN DEL CONCEPTO MATEMÁTICO ALTURAS DE UN TRIÁNGULO }

\begin{datos}
Elizabeth Antero Tepec, Omar Cienfuegos Sarabia, José Roosevelt Mojíca Rodríguez,\\
 Ma. Guadalupe  Cabañas Sánchez, Catalina Navarro Sandoval. \\
Unidad Académica de Matemáticas - UAGro, \\
\hfill México, \\
\hfill eliza\_{}atavy@hotmail.com; omar\_{}cienfuegos@hotmail.com;\\
\hfill rooseveltgm@msn.com;    
\end{datos}

En esta investigación nos interesamos porque un grupo de estudiantes
de quinto grado de una escuela primaria rural en nuestro país, México,
se formaran (o fijaran), el concepto de altura en triángulos. Desde
el punto de vista Metodológico, la formación de concepto se da por
la vía deductiva o la inductiva. La primera, implica que los estudiantes
definan formalmente el concepto, contrario a la segunda, que trata
de que describan las características invariantes (a nivel de introducción).
Por el nivel de desarrollo cognitivo de los estudiantes con los que
se llevó a cabo este trabajo, el nivel de formación es el introducción
del concepto altura con base en dos características: como: recta perpendicular
a la base, y como distancia entre el vértice y el lado opuesto. El
estudio reporta, que posterior al desarrollo de la experiencia de
aprendizaje, una mayoría de estudiantes reconoce estas dos características.


\section{UN ACERCAMIENTO DINÁMICO AL CONCEPTO DE FUNCIÓN A TRAVÉS DEL ESTUDIO
DE FENÓMENOS DE VARIACIÓN CON EL APOYO DE UNA HERRAMIENTA DIGITAL}

\begin{datos}
Hugo Rogelio Mejía Velasco, Juan Carlos Torres Flores.\\
Cinvestav–IPN, \\
\hfill México, \\
\hfill hmejia@cinvestav.mx; jctorres@cinvestav.mx
\end{datos}

Muchas investigaciones se han realizado acerca del concepto de función;
sin embargo, los problemas en la enseñanza y el aprendizaje de dicho
concepto siguen latentes. Recientemente, ha crecido la idea de que
un acercamiento covariacional proporciona un primer paso importante
hacia una comprensión más profunda de dicho concepto. En este sentido,
este trabajo propone un acercamiento dinámico al concepto de función
mediante el planteamiento de problemas en diversos contextos que involucran
situaciones de cambio o variación, y apoya tales actividades con el
empleo de una herramienta digital que permite un manejo simultáneo
de las diferentes representaciones de una función.


\section{\uppercase{  Justificación argumentativa en la prueba geométrica
informal }}

\begin{datos}
Ma. Dalia Lozano Grande, Gonzalo Zubieta Badillo.\\
Cinvestav, \\
\hfill México, \\
\hfill dlozanog@cinvestav.mx, gzubieta@cinvestav.mx
\end{datos}

La investigación consiste en el análisis de los argumentos generados
por estudiantes de Nivel Medio Superior cuando intentan resolver problemas
geométricos de prueba informal, con ayuda de un software de Geometría
Dinámica. El objetivo es identificar algunas razones por las cuales,
con demasiada frecuencia, los estudiantes no logran transitar de la
elaboración de conjeturas, a la justificación argumentativa de las
mismas y posteriormente, a la prueba deductiva. 


\section{El CONOCIMIENTO DIDÁCTICO-MATEMÁTICO DE PROFESORES COLOMBIANOS ACTIVOS
PARA LA ENSEÑANZA DE LA DERIVADA }

\begin{datos}
Ma. Walter F. Castro, Luis R. Pino-Fan, Juan D. Godino, Vicenç Font.\\
Universidad de Antioquia, Universidad de los Lagos,\\
Universidad de Granada, Universitat de Barcelona,\\
\hfill Colombia, Chile, España, \\
\hfill wfcastro82@gmail.com; luis.pino@ulagos.cl;\\
\hfill jgodino@ugr.es; vfont@ub.edu
\end{datos}

En este documento se informa sobre los resultados de una investigación
que intenta caracterizar aspectos relevantes del conocimiento didáctico-matemático
sobre la derivada de profesores colombianos activos. La metodología
adoptada fue cualitativa y de estudio de caso. Los datos reportados
en este documento se tomaron con once maestros activos de matemáticas
inscritos en la maestría en enseñanza de las matemáticas de la Universidad
Pedagógica Nacional de Colombia. Las conclusiones del estudio se agrupan
en dos: 
\begin{description}
\item [{1)}] se evidencia un conocimiento amplio sobre la derivada;
\item [{2)}] los maestros logran identificar conflictos cognitivos en las
respuestas de los estudiantes.
\end{description}

\section{PRAXEOLOGÍAS PARA REVISAR PRÁCTICAS HEREDADAS}

\begin{datos}
Norma Beatriz Di Franco, Claudia Gentile,   Wiliams Noel Uribe.\\
Universidad de Antioquia, Universidad de los Lagos,\\
Universidad Nacional de La Pampa,\\
\hfill Argentina, \\
\hfill difranconb@gmail.com; claudiagentile@cpenet.com.ar
\end{datos}

Este trabajo describe y analiza un dispositivo organizado con una
capacitadora, seis profesores acompañantes de maestros de 4to, 5to
y 6to grado de 32 escuelas primarias, en la conceptualización de los
números racionales. El análisis se centra en las relaciones entre
lo desarrollado con los profesores en el ciclo de formación, con los
maestros en la discusión de secuencias didácticas y lo que pudo concretarse
en las aulas. Intervenciones, variables didácticas y el medio como
herramienta -de la TSD- tanto como las praxeologías de la TAD nos
permiten profundizar el estudio de las prácticas docentes en tanto
dispositivos de conocimiento, de comprensión y de intervención.


\section{GEOMETRÍA DE ALGUNAS FUNCIONES ELEMENTALES DE VARIABLE COMPLEJA }

\begin{datos}
Anairis de la Cruz Benito, Catalina Navarro Sandoval, Marco Antonio Taneco Hernández.\\
Universidad Autónoma de Guerrero,\\
\hfill México, \\
\hfill iris1790@gmail.com; nasacamx@yahoo.com.mx;\\\hfill moodth@gmail.com
\end{datos}

En este trabajo nos planteamos la pregunta ¿cómo realizar la gráfica
de una función de variable compleja?, existen investigaciones que
se han ocupado de estudiar funciones reales, aquí nos centraremos
en el caso de la graficación de funciones complejas elementales. Considerando
a éstas como transformaciones del plano R2 en sí mismo, descubrimos
la geometría que presentan algunas de ellas cuando son aplicadas a
ciertos subconjuntos del plano R2. Los resultados obtenidos son similares
a las funciones reales para las funciones lineal y cuadrática, excepto
para funciones cúbicas que transforman rectas en gráficas que se interceptan
a sí mismas.


\section{CONOCIMIENTO PEDAGÓGICO DISCIPLINAR DE PROFESORES DE MATEMÁTICA}

\begin{datos}
ZANOCCO PIERINA, RIPAMONTI CONSTANZA.\\
Universidad Santo Tomás,\\
\hfill Chile, \\
\hfill pzanocco@santotomas.cl mripamonti@santotomas.cl
\end{datos}

Investigación exploratoria-cuantitativa, consideró trece estudiantes
del curso Didáctica de la Matemática, pretendía demostrar que las
estrategias Estudio de Clases y Estudio de Casos potencian habilidades
de pensamiento crítico que preparan a los profesores en formación
para tomar decisiones fundamentadas frente a situaciones pedagógicas
de enseñanza de la Matemática, El análisis estadístico demostró que
dichas estrategias producen efectos significativos como la formulación
de juicios fundamentados en referentes teóricos frente a situaciones
pedagógicas analizadas; formulación de conjeturas respecto de problemáticas
asociadas a una clase planificada utilizando referentes teóricos y
evaluar justificadamente secuencias de aprendizaje en sus propias
planificaciones y las de otros. 


\section{UNA PROPUESTA DIDÁCTICA PARA LA COMPRENSIÓN DE LAS CUATRO OPERACIONES
BÁSICAS CON FRACCIONES}

\begin{datos}
Karen Rosario Calderón Ignacio, Maribel Vicario Mejía.\\
Universidad Autónoma de Guerrero,\\
\hfill México, \\
\hfill cair.k@hotmail.com; mvicario\_{}maribel@hotmail.com
\end{datos}

El presente es un trabajo de investigación incipiente, misma que pretende
abordar la problemática referida al tema de fracción, esto debido
a las dificultades evidenciadas en la enseñanza y aprendizaje del
tópico, cabe mencionar que existe una ausencia de problemas que permitan
en los estudiantes explorar el trabajo con los cuatro algoritmos para
operar con las fracciones. Por ello nos planteamos como objetivo de
investigación realizar y validar una propuesta de actividades que
permita la comprensión de las cuatro operaciones básicas con fracciones.
El estudio se sustenta en la teoría de situaciones didácticas y como
metodología, la ingeniería didáctica. 


\section{LA PROPORCIONALIDAD COMO UN EJE DE ARTICULACIÓN ENTRE LA SECUNDARIA
Y EL BACHILLERATO}

\begin{datos}
Jesús Israel Monroy Muñoz, Carlos Rondero Guerrero,\\
Juan Alberto Acosta Hernández.\\
Universidad Autónoma del Estado de Hidalgo,\\
\hfill México, \\
\hfill imunoz\_{}emc2@hotmail.com
\end{datos}

El propósito de esta investigación es la revisión de planes y programas
de estudio del ciclo escolar 2013 – 2014 en educación básica, para
conocer la forma de articulación en el tema de proporcionalidad. Además
se decidió hacer una revisión histórica y epistemológica del este
concepto con el objetivo de mostrar su importancia conceptual en la
didáctica de las matemáticas, así como sus diferentes formas de articulación
con otros conceptos que se trabajan tanto en la matemática básica
como en la matemática avanzada. De esta manera se muestra como la
proporcionalidad puede ser un eje articulador.


\section{NIVELES DE DESEMPEÑO EN EL APRENDIZAJE DEL CÁLCULO}

\begin{datos}
Carmen Luisa Méndez Fabret, Juan Raúl Delgado Rubí.\\
Universidad de las Ciencias Informáticas (UCI), ISP “José A. Echeverría” (CUJAE),\\
\hfill Cuba, \\
\hfill clmfabret@gmail.com rdelgado@cemat.cujae.edu.cu
\end{datos}

Se define la categoría desempeño en el aprendizaje de la matemática
y se presenta una taxonomía de niveles de asimilación para un efectivo
desempeño que se ajusta a las características de la enseñanza-aprendizaje
de las matemáticas en el nivel universitario y tiene en cuenta los
principales procesos presentes en el aprendizaje del Cálculo. Tanto
la caracterización de desempeño en el aprendizaje matemático como
la taxonomía de niveles de asimilación presentadas, contribuyen a
evidenciar indicadores y dimensiones que posibilitan realizar mediciones
y valorar los alcances y la calidad del proceso de aprendizaje de
la matemática en la universidad. 


\section{\uppercase{ a noção de derivada de uma função com recurso ao ''software''
geogebra }}

\begin{datos}
Pedro Mateus, Marlene Alves Dias.\\
UNIBAN – Universidade Bandeirante Anhanguera,\\
\hfill Brasil, \\
\hfill pzulu1010@yahoo.com.br, alvesdias@ig.com.br
\end{datos}

Neste trabalho consideramos as ideias relacionadas às noções de derivadas
de funções reais de uma variável e da integral de Riemann que no contexto
moçambicano apresenta sérios problemas de ensino e aprendizagem nas
classes terminais do secundário e nos primeiros anos da universidade.
Nosso objetivo é procurar formas de trabalho com os estudantes que
possam melhorar e potencializar o ensino e aprendizagem das noções
acima visando uma aprendizagem baseada no significado e na compreensão.
Assim, após uma pesquisa da relação institucional existente, fizemos
algumas modificações nas mesmas para utilizar o Geogebra como elemento
de reflexão para atingir o objetivo visado. 


\section{LA PRÁCTICA DOCENTE EN LA ARITMÉTICA: UNA MIRADA ETNOGRÁFICA }

\begin{datos}
Francisco Emmanuel González Ángeles, Felipe Gaytán Alcalá.\\
Facultad de Humanidades y Ciencias Sociales - Universidad La Salle,\\
\hfill México, \\
\hfill fga\_{}1994@hotmail.com; felipe.gaytan@ulsa.mx
\end{datos}

El objeto de estudio de esta investigación fue identificar qué acciones
docentes son necesarias para mediar el aprendizaje de las relaciones
aditivas entre diferentes cardinalidades con alumnos de segundo grado
de educación primaria. El propósito de investigación fue el de identificar
el proceso y los ambientes de aprendizaje didácticos fundamentales
e idóneos para lograr este andamiaje cognitivo necesario para que
un niño transite de un pensamiento concreto a uno abstracto en el
marco de la solución de problemas. Se utilizó el enfoque y los materiales
propuestos por la SEP para ejercer la enseñanza con estos contenidos
curriculares. 


\section{CONSULTORÍA PARA EL DESEMPEÑO EN LA VIRTUALIDAD DEL DOCENTE DE MATEMÁTICA}

\begin{datos}
Ivonne Burguet Lago, Luisa Beatriz García de la Vega.\\
Universidad de las Ciencias Informáticas, Universidad de las Ciencias Pedagógicas,\\
\hfill Cuba, \\
\hfill iburguet@uci.cu, luisagv@ucpejv.rimed.cu
\end{datos}

La investigación que en este informe se presenta, tiene la finalidad
de contribuir a elevar el nivel científico de los docentes de matemática
en la Universidad de las Ciencias Informáticas, para lo cual propone:
una consultoría para el desempeño pedagógico en la virtualidad del
docente de matemática, cuya novedad radica en utilizar a la consultoría
como forma organizativa de la superación profesional y al mismo tiempo
con la intencionalidad de favorecer el desempeño en la virtualidad
de ese docente, a partir del aprovechamiento óptimo de la infraestructura
tecnológica adecuada para modelos de formación de posgrado.


\section{MATEMÁTICA EDUCATIVA Y EQUIDAD: UN ESTUDIO SOCIOEPISTEMOLÓGICO DEL
TALENTO EN MATEMÁTICAS }

\begin{datos}
Erika Canché Góngora, Rosa Ma. Farfán Márquez.\\
Cinvestav-IPN,\\
\hfill México, \\
\hfill emcanche@cinvestav.mx; rfarfan@cinvestav.mx
\end{datos}

La tesis central de esta investigación se finca en la construcción
de un modelo teórico que postula una alternativa al concepto tradicional
de “talento”, específicamente en el ámbito de las matemáticas, rompiendo
con un paradigma clave en las políticas públicas en materia de equidad
educativa. El talento en matemáticas, es abordado desde la perspectiva
de la socioepistemología, que considera al conocimiento como una construcción
social de naturaleza multidimensional. 


\section{DESARROLLOCONCEPTUAL SOBRE RECTAS Y PUNTOS NOTABLES DEL TRIÁNGULO
EN LIBROS DE TEXTO DE NIVEL BÁSICO}

\begin{datos}
Luz Esmeralda Reyes García, Flor Monserrat Rodríguez Vásquez luzes.\\
Universidad Autónoma de Guerrero,\\
\hfill México, \\
\hfill luzes\_{}rega@hotmail.com; flor.rodriguez@uagro.mx
\end{datos}

Presentamos un estudio referente al desarrollo conceptual del tema
rectas y puntos notables del triángulo en libros de texto de nivel
básico en México. Utilizamos la metodología de análisis de textos
y analizamos definiciones del concepto, los problemas o ejercicios
referentes al tema, representaciones que usan para enseñar el tema,
actividades donde se apliquen esos conceptos, entre otras cuestiones.
La revisión indica que el tema se inicia en nivel primaria con la
recta altura, dando su definición y trazos de ella en triángulos acutángulos
y obtusángulos, continuando en secundaria el trabajo con las cuatro
rectas (bisectriz, mediatriz, altura y mediana). 


\section{CONHECIMENTOS DE PROFESSORES DE MATEMÁTICA DO BRASIL SOBRE PROBABILIDADE }

\begin{datos}
José Ivanildo Felisberto de Carvalho, Ruy Cesar Pietropaolo,Tânia M. M. Campos.\\
Universidade Anhanguera de São Paulo,\\
\hfill Brasil, \\
\hfill ivanfcar@hotmail.com; rpietropaolo@gmail.com;\\
\hfill taniammcampos@hotmail.com
\end{datos}

Apresentamos uma discussão sobre resultados iniciais de uma pesquisa
que investiga os conhecimentos necessários aos professores de matemática
para o ensino de probabilidade no Ensino Fundamental (11 a 14 anos).
A investigação acontece no âmbito do Projeto Observatório da Educação
no Brasil Discutimos com base na análise de um questionário respondido
por 38 professores que integra a fase diagnóstica deste projeto. Podemos
afirmar que este grupo de professores ainda não apresenta os conhecimentos
necessários para ensinar as ideias que sustentam o conceito de probabilidade
nos anos finais do Ensino Fundamental, a saber: aleatoriedade, espaço
amostral, quantificação de probabilidade e risco. \newpage


\section{NIVELES DE COMPRENSIÓN DEL CONCEPTO CIRCUNFERENCIA MEDIADO POR EL
GEOGEBRA}

\begin{datos}
Enrique Arturo Valerio Santos Napán, Cecilia Gaita Iparraguirre.\\
Pontificia Universidad Católica del Perú, Instituto de Investigación sobre Enseñanza de las Matemáticas-IREM,\\
\hfill Perú, \\
\hfill esantos@pucp.pe; cgaita@pucp.edu.pe
\end{datos}

El presente reporte se enmarca dentro de un trabajo de investigación
que tiene por objetivo determinar los niveles de comprensión de la
circunferencia alcanzados por estudiantes de 2\textdegree{} de secundaria,
según el Modelo Van Hiele y teniendo en cuenta el tipo de justificación
que brindan a los procedimientos matemáticos desarrollados. Se presentan
actividades que relacionan los elementos de la circunferencia y que
han sido diseñadas considerando al software Geogebra como mediador.
Se plantea identificar el efecto que ha tenido el recurso tecnológico
en relación a la adquisición de un nivel de aprendizaje superior,
según el modelo adoptado. 


\section{AS ESTRATÉGIAS UTILIZADAS PELOS ALUNOS DA EDUCAÇÃO BÁSICA AO RESPONDEREM
QUESTÕES SOBRE NÚMEROS RACIONAIS EM AVALIAÇÕES EXTERNAS NO BRASIL}

\begin{datos}
Rosivaldo Severino dos Santos, Tânia Maria Mendonça Campos.\\
UNIBAN, SP,\\
\hfill Brasil, \\
\hfill rosivaldo100@ig.com.br; taniammcampos@hotmail.com
\end{datos}

Neste trabalho apresentamos as estratégias utilizadas por alunos da
Rede Estadual de São Paulo/BR ao responderem questões sobre números
racionais, particularizando o SARESP/Sistema de Avaliação de Rendimento
Escolar do Estado de São Paulo. Tomamos como aporte teórico a Teoria
dos Campos Conceituais, segundo a qual, o conhecimento de determinado
conceito não deve ser considerado isoladamente, mas sim como inserido
dentro de um campo conceitual, relacionando-se com outros conhecimentos.
Para ele os conceitos são interdependentes, ou seja, existe uma dependência
recíproca. Quanto às estratégias utilizadas pelos alunos, observamos
que os mesmos se utilizam de diferentes estratégias para responder
aos itens propostos.


\section{PRISMA. PROGRAMA PARA DE MEJORAMIENTO DE LA ENSEÑANZA Y EL APRENDIZAJE
DE LAS MATEMÁTICAS EN BARRANQUILLA}

\begin{datos}
Escudero Rafael, Arteta Judith, Pacheco Anuar, Cervantes Guillermo, Rojas Carlos, Martínez Rafael, Jiménez Germán, Jiménez Myrna, Monroy Andrea.\\
Universidad del Norte - División de Ciencias Básicas, Fundación Promigas,\\
\hfill Colombia, \\
\hfill rescuder@uninorte.edu.co; gcervant@uninorte.edu.co;
\\ \hfill vjudith@uninorte.edu.co; anuar.pacheco@promigas.com;\\\hfill  crojas@uninorte.edu.co; rmartine@uninorte.edu.co;\\ \hfill gjimenez@uninorte.edu.co; mjimenez@uninorte.edu.co;\\ \hfill amonroy@uninorte.edu.co
\end{datos}

El Programa PRISMA busca reconocer, identificar y potenciar las competencias
de los profesores de matemáticas de primaria para impactar en el mejoramiento
del aprendizaje y competencias de los estudiantes. Se está utilizando
el enfoque de Investigación-Acción para lograr resultados preliminares
como: establecimiento y actualización de la línea de base de cada
institución; aplicación y análisis de una prueba diagnóstica a los
estudiantes; diseño innovaciones didácticas. El alcance de la primera
etapa involucró la atención y acompañamiento a 15 escuelas de la ciudad
de Barranquilla en los grados tercero, cuarto y quinto de educación
básica. 


\section{CONCEPCIONES Y CONOCIMIENTO DIDÁCTICO MOVILIZADOS POR LOS PROFESORES
AL PLANIFICAR ACTIVIDADES PARA EL DESARROLLO DEL PENSAMIENTO ADITIVO }

\begin{datos}
Juan Barboza Rodriguez, Emis Brun Caro, Estefany Herrera Martínez.\\
Universidad de Sucre - grupo de investigación PROPED,\\
\hfill Colombia, \\
\hfill Juan.barboza@unisucre.edu.co; emisbrun@hotmail.com;\\\hfill hmestephy@hotmail.es
\end{datos}

Desde la investigación formativa desarrollada con estudiantes de licenciatura
en matemáticas de la universidad de Sucre (Colombia), se pretende
caracterizar las concepciones y el conocimiento didáctico del contenido
de los profesores al momento de planificar una clase que promueva
el desarrollo del pensamiento aditivo. Un aporte, es ofrecer un modelo
metodológico de indagación basado en el plan de clase. Los resultados
parciales indican que las concepciones y el conocimiento didáctico
movilizado por los profesores se basa en los conocimientos, actividades
aprendidas al ser estudiantes, el uso del texto guía, la experiencia
individual y poco uso de referentes pedagógicos y didácticos.


\section{LOS SIGNIFICADOS DE LA PROBABILIDAD EN LOS PROFESORES DE MATEMÁTICA
EN FORMACIÓN: UN ANÁLISIS DESDE LA TEORÍA DE LOS MODELOS MENTALES }

\begin{datos}
Amable Moreno, José María Cardeñoso,  Francisco González-García.\\
Univ. Nac. de Cuyo, Univ. de Cádiz, Univ. de Granada,\\
\hfill Argentina, España,\\
\hfill morenoamable6@gmail.com; cardenoso.josemaria@uca.es; \\\hfill pagoga@ugr.es
\end{datos} 

En este trabajo analizamos los significados de la probabilidad que
tienen los estudiantes para profesor de matemáticas, de la provincia
de Mendoza, Argentina; a partir del marco teórico proporcionado por
Cardeñoso (2001), para determinar las tendencias de pensamiento probabilístico;
y la teoría de los modelos mentales (Johnson-Laird, 1994). Se aplicó
un cuestionario a 583 estudiantes y el análisis de las respuestas
se realizó a partir de la aplicación de diversas técnicas estadísticas,
como el test de Pearson, test de Friedman, test de Wilcoxon, el análisis
de clusters y análisis discriminante. Los resultados evidencian una
variedad de significados 


\section{INTERSECCIÓN DE LA MATEMÁTICA EDUCATIVA Y LAS CIENCIAS SOCIALES:
EL CASO DE LOS SEGUROS DE VIDA }

\begin{datos}
María Rosa Rodr\'iguez,   Jesús Alberto Zeballos.\\
Facultad de Ciencias Económicas – Universidad Nacional de Tucumán,\\
\hfill Argentina, \\
\hfill mrrodriguez@face.unt.edu.ar; jesusalbertozeballos@gmail.com 
\end{datos}

La enseñanza de la Economía implica un problema pedagógico-didáctico:
¿Qué y cuánta Matemática resulta imprescindible en las teorías económicas?
Esto exige una profunda reflexión sobre la formación matemática de
los profesores de Economía y de investigaciones en Educación Matemática.
Este trabajo muestra, de manera didáctica, la medición de la incertidumbre
que involucran los seguros de vida. El asegurador determina el precio
de la cobertura que brinda un seguro por medio de las Tablas de Mortalidad,
estudiadas por la Matemática Actuarial. Este tema evidencia la intersección
de problemas económicos y poblacionales, utilizando conceptos matemáticos
aplicables a la Educación Matemática en Economía.


\section{LA SUMA Y LA RESTA DE NUMEROS NATURALES, SU LENGUAJE Y REGISTROS
DE REPRESENTACIÓN, EN LA ESCUELA PRIMARIA }

\begin{datos}
Lorena Trejo Guerrero, Marta Elena Valdemoros Álvarez.\\
Cinvestav-IPN,\\
\hfill México DF, \\
\hfill ltrejog@cinvestav.mx, mvaldemo@cinvestav.mx 
\end{datos}

El trabajo muestra el resultado obtenido al aplicar una situación
de enseñanza en la cual se utilizaron la suma y la resta (como operaciones
inversas) en relación a la construcción del número natural, en la
escuela primaria, para analizar y reconocer las situaciones en las
que las operaciones cobran sentido matemático y poder escoger el procedimiento
más sencillo en la resolución de un problema, mediante la inclusión
de una gran variedad de registros de representación. La aplicación
y reconocimiento de las propiedades de ambas operaciones nos permiten
analizar el uso del lenguaje y su relación con los registros de representación. 


\section{USO DE LA ESTADÍSTICA EN LAS TESIS DE LA LICENCIATURA EN PSICOLOGÍA
EDUCATIVA DE LA UNIVERSIDAD PEDAGÓGICA NACIONAL}

\begin{datos}
Cuauhtémoc Gerardo Pérez López, Alba Yanalte Álvarez Mejía.\\
Universidad Pedagógica Nacional,\\
\hfill México, \\
\hfill cgperez@upn.mx; yanicuau@yahoo.com.mx; 
\end{datos}

Se revisan 65 tesis elaboradas por egresados de Psicología Educativa
en la UPN de México. La finalidad fue estudiar los modos fundamentales
de razonamiento estadístico, dentro del proceso de investigación,
llevado a cabo por estos estudiantes en sus trabajos. Se asume que
la comprensión de la estadística se alcanza al aplicarla al resolver
un problema en el campo de investigación. El análisis se concentra
en el tercer componente del modelo de razonamiento estadístico de
Wild y Pfannkuch (1999) y Pfannkuch y Wild (2004). Se muestran resultados
de algunos indicadores de dichos razonamientos puestos de manifiesto
en las tesis de licenciatura.


\section{FORMACIÓN MATEMÁTICA CONTINUA DE PROFESORES DE PRIMARIA}

\begin{datos}
Alfonso Jiménez Espinosa.\\
Universidad Pedagógica y Tecnológica de Colombia (UPTC),\\
\hfill Colombia, \\
\hfill alfonso.jimenez@uptc.edu.co
\end{datos}

La investigación, centrada en la formación continua de setenta mil
maestros de primaria, tiene como objetivo analizar la formación matemática
y la transformación de prácticas, con fundamento en formación situada
y reflexión en, y sobre la acción hacia la (re)significación de saberes
y prácticas. Se usaron bases de datos del Ministerio y cuestionarios
de pregunta abierta; para la reflexión y (re)significación de prácticas
se hace observación participante, entrevistas y análisis de narrativas.
Se percibe que el nivel de formación no guarda relación directa con
los logros de los niños, pero que este tipo de formación sí sensibiliza
a los profesores.


\section{\uppercase{ Usos de las gráficas en las derivadas, en estudiantes
de pedagogía en matemáticas en universidades de Chile}}

\begin{datos}
Claudio Enrique Opazo Arellano, Francisco Cordero Osorio.\\
Cinvestav,\\
\hfill México D.F, \\
\hfill Opazoferrari\_{}claudio@hotmail.com; Fcordero@cinvestav.mx
\end{datos}

El trabajo se enmarca en la teoría Socioepistemologíca, la cual centra
la atención en la construcción social del conocimiento. Con base en
ello, se observa de manera particular, la enseñanza y el aprendizaje
del Cálculo en la formación inicial de profesores en Universidades
de Chile. Ello con objeto de mirar los Usos de las gráficas en las
derivadas. Lo cual es parte del programa que se desarrolla en el grupo
de Matemática Educativa que adhiere a la teoría antes mencionada.
Destacamos al finalizar, la búsqueda permanente en torno al rediseño
al discurso matemático escolar (RdME), lo cual es fruto de observar
una centración en el objeto matemático. 


\section{COMPARACIÓN NUMÉRICA Y VALOR POSICIONAL }

\begin{datos}
Susana Andrade Neyra, Marta Elena Valdemoros Álvarez.\\
Centro de Investigación y de Estudios Avanzados del IPN,\\
\hfill México, \\
\hfill sandrade@cinvestav.mx; mvaldemo@cinvestav.mx
\end{datos}

Analizamos la diferencia en la comprensión del valor posicional de
un grupo de estudiantes de primer grado de primaria, derivada de una
enseñanza experimental, a través de la comparación de magnitudes numéricas.
Los resultados obtenidos muestran que después de la enseñanza experimental
los estudiantes lograron comprender numerales de dos y tres cifras,
y consideraron el valor posicional al comparar magnitudes numéricas.


\section{UNA DIDÁCTICA DE LA MATEMÁTICA PARA LA FORMACIÓN EN DIVERSIDAD}

\begin{datos}
Eliécer Aldana Bermúdez, Jorge Hernán López Meza.\\
Universidad del Quindío,\\
\hfill Colombia, \\
\hfill eliecerab@uniquindio.edu.co- jhlopez@uniquindio.edu.co
\end{datos}

Este reporte de investigación en proceso tiene como objeto de investigación
el aprendizaje y como objeto matemático el concepto de perímetro y
de área en estudiantes con défici cognitivo y Síndrome Down. El objetivo
es mostrar cómo el problema que tiene esta población para el aprendizaje
de las matemáticas puede ser minimizado mediante la intervención del
profesor. Para ello, se ha utilizado como marco teórico las situaciones
didácticas de Brousseau y como metodología la Ingeniería didáctica.
Los resultados permiten concluir que una didáctica de la matemática
adaptada a esta población en condición de inclusión admite que puedan
también aprender matemáticas. 

\setcounter{section}{100}


\section{FORMACIÓN DE PROFESORES DE MATEMÁTICAS DE MEDIO SUPERIOR El caso
del Colegio de Ciencias y Humanidades}

\begin{datos}
Víctor Manuel Pérez Torres, Marco Antonio Santillán Vázquez.\\
CCH-UNAM,\\
\hfill México, \\
\hfill victorpt07@gmail.com; santillanmarco11@gmail.com
\end{datos}

Presentamos resultados de ocho años de experiencia en formación de
profesores de nivel medio superior. Exponemos con qué tipo de docentes
tratamos y porqué, las orientaciones teóricas del trabajo y los temas
seleccionados: Resolución de problemas, tecnologías digitales y temas
selectos de didáctica. Abordamos preguntas centrales como: ¿Qué significa
formar profesores de matemáticas de nivel medio superior en México?
¿Cómo formar adecuadamente a estos profesores? ¿Cómo evaluar los programas
de formación docente en matemáticas? ¿Cómo organizar procesos de apropiación
de las tecnologías digitales para docentes novatos?


\section{OBSTÁCULOS COGNITIVOS EN EL APRENDIZAJE DE LAS MATEMÁTICAS: EL CASO
DEL CONCEPTO DE LÍMITE}

\begin{datos}
Ana Cecilia Medina Mariño, Clara Emilse Rojas Morales.\\
Universidad Pedagógica y Tecnológica de Colombia,\\
\hfill Colombia, \\
\hfill ana.medina@uptc.edu.co; clara.rojas@uptc.edu.co
\end{datos}

Se presenta el proceso y resultados de una investigación sobre las
concepciones manifestadas por estudiantes de la Licenciatura en Matemáticas
y Estadística de la Uptc Duitama relativas al concepto de límite y
se infieren factores que obstaculizan o favorecen su comprensión.
Se desarrolló bajo un enfoque sistémico, mediante análisis histórico
– epistemológico, didáctico y cognitivo. Se encontró que los estudiantes
revelan concepciones espontáneas mezcladas con las inducidas por la
enseñanza y modelos mentales no pertinentes que causan conflictos
cognitivos. En los textos predomina la concepción analítica- estática
y en los estudiantes y profesores prevalece la concepción algebraica
(finitista- estática). 


\section{\uppercase{ Análisis DE SENSIBILIDAD PARAMÉTRICA para SISTEMAS LINEALES
SEMI-INFINITOS A TRAVÉS del método de relajación EXTENDIDO}}

\begin{datos}
Javier Barrera Ángeles$^ 1$, Enrique González Gutiérrez$^2$, Magda Muñoz Martínez$^3$.\\
$^1$Universidad Autónoma del Estado de Hidalgo, México, $^{2,3}$Universidad Politécnica de Tulancingo Hidalgo,\\
\hfill México, \\
\hfill gutierrez.gonzalez.e@gmail.com; Magda.munoz@upt.edu.mx;\\\hfill  Jbarrera12@hotmail.com
\end{datos}

El presente trabajo tiene como objetivo presentar evidencias acerca
del rol que juegan los parámetros a través del estudio de sensibilidad
para sistemas lineales semi-infinitos (SLSI). El estudio de SLSI inicio
en 1920. En trabajos recientes, se propone el método de relajación
extendido (MRE). Se parte de la idea de extender el método de relajación
(MR). Sin embargo, cuando el problema trata con SLSI es conveniente
utilizar el método de relajación extendido (MRE). Finalmente se presentan
resultados obtenidos sobre la evidencia computacional del análisis
de sensibilidad de $\lambda,\,\beta$ y $M$ que aparecen en MRE.
Una conclusión importante es que $\lambda$ desempeña el rol principal
del algoritmo.


\section{ASPECTOS METODOLÓGICOS PARA EL ANÁLISIS DEL DESARROLLO PROFESIONAL
EN MATEMÁTICAS: EL CASO DE UNA MAESTRA DE PRIMARIA}

\begin{datos}
Edelmira Badillo, Núria Planas, Isabel Moreno.\\
Departament de Didàctica de les Matemàtiques i de les CCEE. Universitat Autónoma de Barcelona,\\
\hfill España, \\
\hfill Edelmira.Badillo@uab.cat; Núria.Planas@uab.cat; \\\hfill  madre.moreno@ciasalvador.org
\end{datos}

Este estudio del desarrollo profesional de una maestra está ubicado
en la agenda de investigación de un proyecto en curso. Entendemos
por desarrollo profesional un proceso de crecimiento en los aspectos
que capacitan al profesor para mejorar la enseñanza y el aprendizaje
de sus estudiantes. Nos hemos centrado en la construcción y el posterior
análisis de cuatro aspectos clave: conocimiento didáctico matemático,
reflexión sobre la práctica, colaboración en comunidad y empoderamiento
en la acción. Resumimos los tipos de datos y los relacionamos entre
ellos y con los cuatro aspectos. Damos cuenta de un resultado sobre
evidencias de empoderamiento en la acción. 


\section{INSTRUMENTO DE VISUALIZACIÓN PARA EL ANÁLISIS COMPARATIVO DE PRÁCTICAS
MATEMÁTICAS DE AULA}

\begin{datos}
Edelmira Badillo$^1$, Lourdes Figueiras$^1$, Vicenç Font$^2$.\\
$^1$Departament de Didàctica de les Matemàtiques i de les CCEE - Universitat Autónoma de Barcelona, $^2$Departament de Didàctica de les ciències Experimentals i de la Matemàtica -  Universitat de Barcelona.\\
\hfill España, \\
\hfill Edelmira.Badillo@uab.cat; Lourdes.Figueiras@uab.cat;\\\hfill vfont@ub.edu 
\end{datos}

En este informe se presentan datos sobre la aplicación de una metodología
de análisis de procesos de enseñanza y aprendizaje que genera un instrumento
que permite resaltar y visualizar los elementos esenciales de la actividad
matemática en el desarrollo temporal de una clase. Esta metodología
de análisis se ha aplicado al estudio de los elementos comunes y las
diferencias entre tres clases realizadas por tres profesoras diferentes
en una misma institución, año y nivel escolar cuando enseñan la mediatriz.
Los resultados de este análisis didáctico permiten inferir, además,
aspectos del conocimiento matemático activado por las profesoras en
su práctica profesional de aula.


\section{USO DE LA CALCULADORA CIENTÍFICA EN EL APRENDIZAJE DE LOS NÚMEROS
ENTEROS }

\begin{datos}
Patricia N. Bernardi, Elvira G. Rincón Flores, Leopoldo Zúñiga Silva.\\
Universidad Siglo 21 - Sede Río Cuarto, Instituto Tecnológico y de Estudios Superiores de Monterrey - Campus Monterrey,\\
\hfill Argentina, México, \\
\hfill pbernardi@uesiglo21.edu.ar; elvira.rincon@itesm.mx; \\\hfill lzs@tecvirtual.mx
\end{datos}

El detonante de este estudio fue la dificultad observada en el aprendizaje
de las operaciones con números negativos. Esta investigación permitió
demostrar que el uso de la calculadora científica en el proceso de
enseñanza-aprendizaje de los números enteros facilita la comprensión
de las operaciones con números negativos en alumnos de segundo año
de nivel medio. Se desestimó el falso preconcepto de que su uso reemplaza
a la inteligencia humana. La calculadora fue un complemento para la
actividad mental del alumno. Su uso favoreció el aprendizaje a través
de la comprensión del algoritmo involucrado aunque no fue usada en
las evaluaciones. 


\section{RAZÓN DE CAMBIO E IDENTIFICACIÓN DEL MOVIMIENTO }

\begin{datos}
Mario Armando Giordano Moreno, Orlando Moctezuma Cruz,\\ Ignacio Garnica y Dovala.\\
Centro de Estudios Científicos y Tecnológicos No. 4 “Lázaro Cárdenas del Río” - Departamento de Matemática Educativa, Cinvestav. Instituto Politécnico Nacional,\\
\hfill  México, \\
\hfill mgiordano@prodigy.net.mx; orlymx2000@yahoo.com.mx;\\\hfill igarnica@cinvestav.mx 
\end{datos}

Se enfoca la interacción entre estudiantes de Física (3er. semestre)
y estudiantes de Cálculo (5\textdegree{} semestre) al hacer sentido
de experiencias en el laboratorio de Física. La experiencia consistió
en la realización de los experimentos de movimiento rectilíneo uniforme
y movimiento rectilíneo uniformemente variado. Los reportes de los
estudiantes de Cálculo indican que su participación permitió introducir
en la discusión la noción de razón de cambio para definir y determinar
la velocidad. Los reportes de los estudiantes de Física indican que
llegaron a diferenciar entre ambos tipos de movimiento, empleando
la velocidad registrada en cada experimento como criterio de diferenciación.


\section{NIVELES DE COMPRENSIÓN Y DE RAZONAMIENTO LOGRADO POR LOS ESTUDIANTES
DE PRIMER NIVEL EN EL APRENDIZAJE DE LA GEOMETRÍA CUANDO SE APLICAN
ACTIVIDADES LÚDICAS}

\begin{datos}
Sandra Liliana Botello Suárez, Elvira G. Rincón Flores, Leopoldo Zúñiga Silva.\\
Tecnológico de Monterrey - Campus Monterey,\\
\hfill  México, \\
\hfill sandralilianabotello@itesm.mx; elvira.rincon@itesm.mx;\\\hfill lzs@itesm.mx 
\end{datos}

En los procesos de enseñanza- aprendizaje, concretamente de la Geometría,
se han observado algunas irregularidades a través de los años ya que
tradicionalmente se imparte de manera axiomática independientemente
del grado del estudiante. Por lo que surgió la iniciativa de utilizar
el modelo de razonamiento geométrico de Van Hiele y el aprendizaje
lúdico para valorar el nivel de comprensión y razonamiento que alcanzan
los alumnos de primer nivel de secundaria. Se encontró que el desempeño
de los estudiantes mejoró ya que en un inicio se clasificaron en un
nivel y después del estudio se ubicaron en el siguiente.


\section{SISTEMAS DE ECUACIONES LINEALES Y LA ENSEÑANZA DE TENSIONES EN EL
LABORATORIO DE FÍSICA EN EL BACHILLERATO TECNOLÓGICO}

\begin{datos}
Pedro Javier Ubaldo Salinas, Ana María Ojeda Salazar.\\
Instituto Politécnico Nacional CECyT 4. DME Cinvestav,\\
\hfill  México, \\
\hfill pubaldos@ipn.mx; amojeda@cinvestav.mx 
\end{datos}

En el marco de un proyecto interinstitucional sobre docencia e investigación
en matemática educativa, se indaga con estudiantes de tercer semestre
en el laboratorio de Física su comprensión de las tensiones en un
cuerpo suspendido en equilibrio y la resolución del sistema de ecuaciones
lineales que lo modela, luego de la enseñanza del tema en el curso
de Física I (DEMS, 2009). Aunque los alumnos expresaron una noción
general de las condiciones de equilibrio, el uso irreflexivo de algoritmos
causa bajo desempeño, esto motiva una investigación acerca de cómo
hacer efectiva para el estudiante la interrelación entre Unidades
de Aprendizaje.


\section{¿QUÉ ES ENSEÑAR MATEMÁTICAS? REPRESENTACIONES SOCIALES DE PROFESORES
DE SECUNDARIA DEL ESTADO DE GUERRERO }

\begin{datos}
Ámbar Carranza Santoyo, Magdalena Rivera Abrajan.\\
Universidad Autónoma de Guerrero, Unidad Académica de Matemáticas,\\
\hfill  México, \\
\hfill ambar\_{}cs@hotmail.com; magrivab@hotmail.com 
\end{datos}

El objetivo de nuestro estudio es mostrar las representaciones sociales
que tiene un grupo de profesores de secundaria acerca de su práctica
de Enseñar Matemáticas. Tomamos la teoría de las representaciones
sociales que proviene de la psicología social como postura teórica-
metodológica. Para la recolección de datos se empleó un cuestionario
de preguntas abiertas y posteriormente entrevistas semiestructuradas
a profundidad. En el análisis de las respuestas pudimos identificar
algunos elementos: del campo de representación sobre lo que es enseñar
matemáticas; de la información acerca de la práctica del profesor;
y actitudes positivas ante la tarea de enseñar matemáticas.


\section{O ENSINO DA MATEMÁTICA COM MODELAGEM DE FENÔMENOS FÍSICOS}

\begin{datos}
Daniel Guimarães Silva.\\
Pontifícia Universidade Católica de Minas Gerais – PUC Minas ,\\
\hfill  Brasil, \\
\hfill danielgsd@hotmail.com 
\end{datos}

Este artigo descreve resultados de uma Pesquisa de Mestrado Profissional
em Ensino de Ciências e Matemática. A investigação desenvolvida é
de Modelagem Matemática como alternativa pedagógica para o estudo
da Matemática integrada à Física, utilizando as Tecnologias de Informação
e Comunicação. Foram aplicadas atividades para os alunos do ensino
médio integrado ao curso técnico de Educação Profissional de uma instituição
de ensino. O objeto de estudo foram algumas Funções Matemáticas, que
pela Modelagem Matemática, foram abordadas sob o contexto de fenômenos
físicos analisados em laboratório de Física, com alguns recursos tecnológicos
do próprio laboratório e de softwares como o GeoGebra. 


\section{EFECTOS DE UNA ESTRATEGIA CON UN ENFOQUE DE LABORATORIO EN EL APRENDIZAJE
DE LA FÍSICA}

\begin{datos}
Víctor Hugo Meriño Córdoba, Carmen Martínez, Jesús Timaure.\\
Universidad Nacional Experimental "Rafael María Baralt",\\
\hfill  Venezuela, \\
\hfill victor0463@gmail.com; cimartinezunermb@gmail.com\\\hfill  jesustimaure@gmail.com 
\end{datos}El propósito de la presente investigación fue determinar los efectos
de la aplicación de la estrategia con un enfoque de laboratorio en
los aprendizajes de los alumnos de física I, de la Universidad Nacional
Experimental “Rafael María Baralt”. El estudio se fundamentó en la
teoría centrada en los procesos de Gagné, la teoría humanista de Rogers
y los postulados con un enfoque de laboratorio de González, Meriño
y Aguirre. Entre los resultados obtenidos se reporta que el grupo
experimental: tuvo mayores aprendizajes de la física y tienen pocas
dificultades para: descubrir conceptos, descubrir reglas y principios
y resolver problemas.


\section{TALLERES SOBRE LA ENSEÑANZA Y APRENDIZAJE DE LA ESTADÍSTICA EN EDUCACIÓN
SECUNDARIA DIRIGIDOS A PROFESORES DE MATEMÁTICA }

\begin{datos}
Rodolfo David Fallas Soto.\\
Universidad de Costa Rica,\\
\hfill  Costa Rica, \\
\hfill rdfallass@gmail.com
\end{datos}

En este escrito se presenta la validación de una unidad didáctica
previamente diseñada y orientada al tema de Estadística para estudiantes
de segundo año de secundaria en Costa Rica. Esta unidad se basa en
los fundamentos teóricos de Didáctica de la Matemática. La validación
se realiza a partir del trabajo en conjunto con profesores en servicio.
Se trabajan en módulos de talleres para profundizar en cada contenido
propuesto por el Ministerio de Educación Pública y así obtener un
documento que sea de apoyo para los docentes del país en el aprendizaje
de este tema.

%%%%%%%%%%%%%%%%%%%%%%%%%%%%%%%%%%%%%%%%%%%%%%%5


\section{CONSTRUCCIONES MENTALES PARA EL APRENDIZAJE DE CONCEPTOS BÁSICOS
DEL ÁLGEBRA LINEAL}

\begin{datos}
Marcela Parraguez González Pontificia.\\
Universidad Católica de Valparaíso,\\
\hfill  Chile, \\
\hfill marcela.parraguez@ucv.cl
\end{datos}

En el marco del proyecto FONDECYT Nº 1140801 titulado: CONSTRUCCIONES
y MECANISMOS MENTALES PARA EL USO DE LOS CONCEPTOS BÁSICOS DEL ÁLGEBRA
LINEAL se propuso investigar desde una postura cognitiva el proceso
de enseñanza-aprendizaje de los conceptos básicos del álgebra lineal,
en estudiantes universitarios; utilizando como marco teórico la TEORÍA
APOE, desarrollada por Dubinsky y sus colaboradores. En esta primera
fase de la investigación reportamos cómo los estudiantes universitarios
construyen, los conceptos de espacio vectorial, combinación lineal
y Transformación lineal. 


\section{\uppercase{ ANÁLISIS DEL Currículo MATEMÁTICO PARA UNA MEJOR TRANSICIÓN
DEL NIVEL MEDIO AL NIVEL SUPERIOR} }

\begin{datos}
Yolanda Serres Voisin.\\
Universidad Central de Venezuela,\\
\hfill  Venezuela, \\
\hfill yolanda.serres.voisin@gmail.com
\end{datos}

El objetivo del trabajo es analizar el currículo de matemática de
la Facultad de Ingeniería de la Universidad Central de Venezuela,
específicamente en el inicio de la carrera. El análisis abarcará todas
las dimensiones curriculares señaladas por Rico (1997): 1) La sociológica;
2) La cognitiva. 3) La conceptual. 4) La de formación docente. Se
entiende por currículo al conjunto de prácticas educativas destinadas
al logro de los aprendizajes, de manera que más que estudiar los elementos
del currículo se estudiará las prácticas docentes y el significado
que estas tienen en el contexto particular en que se realizan. La
metodología del estudio es etnográfica. Las dimensiones curriculares
se estudiarán utilizando: .- Entrevistas y reuniones de socialización
con los docentes.- Elaboración de mapas conceptuales por bloques de
contenido. .- Análisis de los materiales instruccionales. Como resultado
se espera reconstruir el currículo matemático inicial de la carrera.


\section{TALLERES SOBRE LA ENSEÑANZA Y APRENDIZAJE DE LA ESTADÍSTICA EN EDUCACIÓN
SECUNDARIA DIRIGIDOS A PROFESORES DE MATEMÁTICA}

\begin{datos}
Rodolfo David Fallas Soto.\\
Universidad de Costa Rica,\\
\hfill  Costa Rica, \\
\hfill rdfallass@gmail.com
\end{datos}

En este escrito se presenta la validación de una unidad didáctica
previamente diseñada y orientada al tema de Estadística para estudiantes
de segundo año de secundaria en Costa Rica. Esta unidad se basa en
los fundamentos teóricos de Didáctica de la Matemática. La validación
se realiza a partir del trabajo en conjunto con profesores en servicio.
Se trabajan en módulos de talleres para profundizar en cada contenido
propuesto por el Ministerio de Educación Pública y así obtener un
documento que sea de apoyo para los docentes del país en el aprendizaje
de este tema.


\section{\uppercase{ Oficina de jogos para professores de Matemática da Educação
de jovens e adultos (EJA)}}

\begin{datos}
Dosilia Espirito Santo Barreto$^1$, Maria Helena Palma de Oliveira$^2$.\\
$^{1,2}$Universidade Anhanguera de São Paulo,\\
\hfill   Brasil, \\
\hfill dosiliamat@gmail.com; mhelenapalma@gmail.com
\end{datos}

Apresenta-se uma oficina sobre jogos e sobre o campo conceitual multiplicativo
realizada com os professores de Matemática que lecionam para a Educação
de Jovens e Adultos no Ensino Fundamental, da rede municipal de Guarulhos,
São Paulo. Buscou-se promover momento de reflexão, análise e socialização
sobre as práticas e o uso dos jogos para o ensino do campo multiplicativo.
Foram trabalhados e avaliados quatro jogos: bingo de tabuada, jogo
dos produtos, pirâmide matemágica e memória de multiplicação. A maioria
dos professores não utiliza jogos em suas práticas, consideraram a
oficina produtiva e pretendem passar a utilizá-los no ensino de Matemática. 


\section{DIFICULTADES EN LA APLICACIÓN DE LOS CONCEPTOS DE RAZÓN Y PROPORCIÓN
DE LOS BACHILLERES DE 15 A 16 AÑOS; VISTOS A TRAVÉS DE LA ABSTRACCIÓN
REFLEXIVA DE PIAGET}

\begin{datos}
Omar Cecilio Martínez$^1$, Hugo R. Mejía Velasco$^2$.\\
$^{1,2}$Centro de Investigación y Estudios Avanzados del Instituto Politécnico Nacional,\\
\hfill   México, \\
\hfill ocecilio@cinvestav.mx; hmejia@cinvestav.mx
\end{datos}

La siguiente investigación está dirigida a los problemas de la comprensión
y aplicación de razón y proporción con el uso de tecnología digital
en jóvenes pre-universitarios. El marco teórico en que se sustenta
esta investigación es la Abstracción Reflexiva desarrollada por Piaget
en 1977, que explica la formación de nuevas estructuras de aprendizaje.
Se desarrolló una exploración de los conocimientos que conservan los
jóvenes bachilleres sobre estos conceptos y la problemática que encierra
la constitución y reconstrucción cognitiva de esquemas relacionados
con el uso de artefactos tecnológicos. Los resultados obtenidos corroboran
parte de la hipótesis planteada.


\section{EL CONTEXTO AULICO EN LA ENSEÑANZA APRENDIZAJE DEL PERÍMETRO}

\begin{datos}
GRACIELA HERNÁNDEZ TEXOCOTITLA.\\
INSTITUTO SUPERIOR DE CIENCIAS DE LA EDUCACIÓN DEL ESTADO DE MÉXICO (ISCEEM),\\
\hfill   México, \\
\hfill texocotitla@yahoo.com
\end{datos}

El reporte de investigación da a conocer el análisis del proceso de
enseñanza aprendizaje del concepto de perímetro, en el nivel primaria,
en los grupos de segundo, cuarto y sexto grado, considerando las diversas
actividades que se desarrollan en el aula. La investigación es de
corte cualitativo interpretativo, por lo que se recupera la voz de
los actores, a fin de dar cuenta de las actividades en un tiempo y
un lugar determinados, examinar la enseñanza en sí misma, la actividad
instruccional y las diversas maneras en que se organiza y se desarrolla
en lo cotidiano.


\section{APOYANDO EL APRENDIZAJE DE LAS FUNCIONES VECTORIALES DESDE MÚLTIPLES
REPRESENTACIONES}

\begin{datos}
Martha Leticia García Rodríguez, Alma Alicia Benítez Pérez, Alicia López Betancourt.\\
ESIME y CECyT 11 del Instituto Politécnico Nacional, Universidad Juárez del Estado de Durango,\\
\hfill   México, \\
\hfill martha.garcia@gmail.com; abenitez@ipn.mx;\\\hfill abetalopez@gmail.com
\end{datos}

En esta propuesta se analiza el razonamiento de un grupo de estudiantes
que trabajan en actividades que incorporan simulaciones en comparación
con los de una clase tradicional, en el tema de funciones vectoriales.
El razonamiento se estudió identificando las complejas tareas cognitivas
que efectúan al interactuar con diferentes representaciones. El trabajo
se realizó en dos sesiones en tres etapas: i) revisión de conceptos;
ii) trabajo en actividades extra-clase iii) discusión grupal. Se encontró
evidencia para señalar que el trabajo con la simulación favoreció
la comprensión de los atributos que se debían incluir en la gráfica
y ayudó a relacionar representaciones.


\section{MEDIACIÓN SEMIÓTICA DEL PROFESOR Y CONSTRUCCIÓN DE SIGNIFICADO EN
EL AULA: UNA PERSPECTIVA PEIRCEANA}

\begin{datos}
Patricia Perry$^1$, Leonor Camargo$^1$, Carmen Samper$^1$, Adalira Sáenz-Ludlow$^2$, Óscar Molina$^1$.\\
$^1$Universidad Pedagógica Nacional, \\$^2$University of North Carolina at Charlotte\\
\hfill   $^1$ Colombia, $^2$Estados Unidos de América, \\
\hfill pperryc@yahoo.com.mx; lcamargo@pedagogica.edu.co;\\
\hfill csamper@pedagogica.edu.co; sae@uncc.edu;\\\hfill ojmolina@pedagogica.edu.co
\end{datos}

La investigación que tenemos en curso se propone documentar el proceso
de construcción de significado en un aula de geometría, de nivel universitario,
con la mediación semiótica del profesor. Recurrimos a elementos de
la teoría del SIGNO triádico de C.S. Peirce, en versión de Sáenz-Ludlow
y Zellweger. En esta ponencia presentamos el marco de referencia,
en el que introducimos un nuevo constructo (objeto dinámico didáctico)
en términos del cual precisamos lo que entendemos por mediación semiótica
del profesor, y damos un ejemplo del tipo de análisis que estamos
haciendo para seguir el rastro de la actividad semiótica en el aula. 


\section{PLAN DE INDUCCIÓN PARA EL DISEÑO DE JUEGOS COMO TÉCNICA DE INSTRUCCIÓN
EN LA MATEMÁTICA DIRIGIDO A ESTUDIANTES DE EDUCACIÓN INTEGRAL DEL
INSTITUTO PEDAGÓGICO DE CARACAS}

\begin{datos}
Patricia Bohórquez, Kelly Yepes.\\
Universidad Pedagógica Experimental Libertador, Instituto Pedagógico De Caracas,\\
\hfill   Venezuela, \\
\hfill pabv\_{}24@hotmail.com; keyepes@gmail.com
\end{datos}

Este trabajo, tiene como objetivo la ejecución de un plan de inducción
en juegos como técnica de instrucción en la matemática, dirigido a
los estudiantes de Educación Integral, del Instituto Pedagógico de
Caracas. La investigación, es de campo, fundamentada en el paradigma
socio-crítico y de tipo investigación-acción, su diseño es cualitativo
emergente-flexible-holístico. La metodología se iniciará con un diagnostico,
diseño del plan de inducción y seguidamente ejecución. Las técnicas
e instrumentos que se utilizaran para la recolección de información,
serán: entrevista-cuestionario; observación participante-notas de
campo. Con ello se espera dar respuesta a las interrogantes planteadas,
así mismo alcanzar los objetivos.


\section{BONDADES DEL APRENDIZAJE BASADO EN PROBLEMAS }

\textbf{\begin{datos}
Santa del Carmen Herrera Sánchez, Carlos Enrique Recio Urdaneta,\\ 
Mario Saucedo Fernández, Juan José Díaz Perera.\\
Universidad Autónoma del Carmen,\\ México. \\
\hfill sherrera@pampano.unacar.mx; crecio@pampano.unacar.mx; \\\hfill msaucedo@pampano.unacar.mx; jjdiaz@pampano.unacar.mx
\end{datos}}

La investigación demuestra la bondad del método de Aprendizaje Basado
en Problemas (ABP), como un medio para desarrollar la competencia
solución problema en los estudiantes. Se realizó un estudio cuasiexperimental
con dos grupos de la facultad de ciencias educativas, usando preprueba
y posprueba; se dio tratamiento al grupo experimental durante un semestre
en el curso de razonamiento lógico que se imparte en la Universidad
Autónoma del Carmen, se procesaron y analizaron los datos con la finalidad
de poder comparar las calificaciones de cada uno de los alumnos por
grupo y los promedios grupales. 


\section{HABILIDADES DIGITALES PARA TODOS. ANÁLISIS DE PRÁCTICAS DOCENTES }

\begin{datos}

José Pedraza Tovar.

Instituto Superior de Ciencias de la Educación del Estado de México,

México,

jpedrazat28@yahoo.com.mx 

\end{datos}

La investigación desarrollada es de corte interpretativo cualitativo.
Realizada en una escuela del Estado de México, centrada en el docente,
desde las posturas de autoridad (Supervisor Escolar), Directiva (Director
y Subdirector Escolar) al nivel del que imparte las clases, principalmente
de matemáticas. Análisis de prácticas docentes es la línea de investigación,
ya que son ellos la parte operativa de la Reforma Curricular y en
las mismas prácticas, además de sus creencias se involucran tensiones
por una innovación, saberes, formación y profesionalización, situación
laboral, uso de las TIC, infraestructura, capacitación que se requiere.
Concluyendo: distanciamiento entre Reforma y Práctica docente.


\section{CONCEPCIONES DE LOS DOCENTES EN FORMACIÓN SOBRE DIDÁCTICA}

\begin{datos}

Alberto Jesús Iriarte Pupo. 

Universidad de Sucre,

Colombia,

albertoiriarte4@yahoo.es

\end{datos}

Esta investigación tiene como objetivo principal identificar cuáles
son las concepciones de los estudiantes de licenciatura en matemática
sobre didáctica, analizando construcciones conceptuales, desde la
aplicación de una propuesta de enseñanza basada en métodos y técnicas
participativas. La metodología de investigación está enfocada a un
estudio descriptivo interpretativo; la intervención docente se basa
en la técnica de lluvia de ideas, seminario alemán y exposición de
matriz comparativa, permitiendo un aprendizaje significativo y desarrollando
habilidades comunicativas y criticas frente al punto de vista diferencial. 


\section{LAS INECUACIONES: UNA MIRADA DESDE EL ESPACIO DE TRABAJO MATEMÁTICO}

\begin{datos}

Silvia Soledad López Fernández, Elizabeth Montoya Delgadillo.

Pontificia Universidad Católica de Valparaíso,

Chile,

Soledadlopezf7@gmail.com, emontoya@ucv.cl\textbf{ }

\end{datos}

El trabajo que se reporta, da cuenta de una investigación que tiene
por objetivo analizar el ETM de referencia y robustecer el ETM personal
de estudiantes de primer año de universidad en torno al objeto matemático
de inecuación, bajo un enfoque cognitivo. Para ello utilizamos la
teoría de espacio de trabajo matemático de Alain Kuzniak, la cual
señala que, mediante la articulación de los planos (cognitivo y epistemológico),
a través de las génesis (semiótica, instrumental y discursiva) se
propicia el conocimiento matemático. Se realizó un estudio de casos
como diseño metodológico, se diseña y aplica una situación de aprendizaje
que evidencia elementos que favorecen el ETM personal del aprendiz. 


\section{\uppercase{ Una propuesta para la caracterización y configuración
de un recurso pedagógico en el contexto de la práctica de un profesor.} }

\begin{datos}

Gilbert Andrés Cruz , Mayra Alexandra Mosquera.

Universidad del Valle,

Santiago de Cali - Colombia,

andrescruz1008@hotmail.com; mayi8724@hotmail.com 

\end{datos}

En el marco de la Didáctica de las Matemáticas surgen investigaciones
e incluso paradigmas, en donde se estudian aspectos relacionados con
la enseñanza y aprendizaje de las Matemáticas, de esta forma se tienen
los desarrollos teóricos de Artigue M. (1995), Trouche \& Gueudet
(2010) entre otros. En este sentido, surge la noción de recurso pedagógico
propuesta por Garzón \& Vega (2011), la cual es una “noción joven”
y, como tal, varía y se desarrolla con gran dinamismo y vitalidad.
Teniendo como referente los desarrollos teóricos mencionados, interesa
en este documento caracterizar la configuración de un recurso pedagógico
en el contexto de la práctica de un profesor, partiendo de documentos
que dan cuenta de la sistematización de experiencias, los cuales pueden
ser adaptados, transformados y discutidos entre pares y replanteados
en material escrito. Es importante considerar que la transformación
de las prácticas profesionales de los docentes es un proceso complejo
de comprender, sin embargo autores como Artigue M. (2011), reconocen
avances a nivel investigativo que apuntan al estudio de las limitaciones
de una práctica y a la comprensión de las razones de éxito de la misma.
La práctica de enseñanza que lleva a cabo un maestro ha sido foco
de diferentes investigaciones, las cuales han estudiado la forma de
mejorar la enseñanza de las Matemáticas. Considerando necesario elaborar
situaciones de enseñanza, que generen una oportunidad para que los
estudiantes den sentido a un concepto. Una forma en la que el maestro
puede lograr esto es diseñando o adaptando una situación, cuyo problema
tenga una solución óptima que implique el concepto en cuestión y que
la configuración del medio posibilite y potencie su uso. Siendo esencial
adoptar una perspectiva de los recursos, la cual hace que trascienda
su materialidad. Desarrollos teóricos como el de Guin \& Trouche (2007)
en relación con la noción de Recurso Pedagógico, lo conciben como
un artefacto que está a disposición del profesor y es susceptible
de evolución. Desde este enfoque se asocian dos nociones centrales
al Recurso Pedagógico: Instrumento y Comunidad de Práctica. La primera
noción emerge según los planteamientos de Rabardel (1995) y la segunda
a partir de las interpretaciones, concepciones y adaptaciones que
debe discutir un maestro con sus pares, en un lugar que facilite estos
procesos. Lo anterior se tiene desde los planteamientos de Wenger
(2001). La noción de contrato didáctico es pieza esencial en el estudio
y comprensión de la práctica de enseñanza, pues determina las reglas
de juego, las estrategias, las responsabilidades y expectativas de
los protagonistas en la situación didáctica. Además, Perrin-Glorian
\& Hersant (2003), realizan algunas precisiones en relación a la noción
de medio, planteando que ésta puede permitir analizar una secuencia
ordinaria; dicho análisis permite evidenciar por una parte, aquello
que está a cargo del alumno y las oportunidades de aprendizaje que
se le confían y, por otra parte, las ayudas que da el profesor al
alumno para propiciar la utilización de sus conocimientos previos
y la construcción de conocimientos nuevos. Algunas consideraciones
curriculares y didácticas diferentes a las planteadas en los párrafos
anteriores permitieron orientar y cualificar el criterio de adaptación
y transformación de las situaciones que se implementaron. En la investigación
se asumieron cinco fases en el desarrollo de la metodología: fundamentación
teórica del proyecto; búsqueda selección y análisis del documento;
adaptación y transformación de la situación seleccionada; implementación
de la situación adaptada y transformada; análisis de resultados. Es
importante mencionar que para esta investigación interesaron los documentos
que se elaboraron bajo las modalidades de comunicaciones breves y
talleres que se presentan en el marco de los encuentros organizados
por la Asociación Colombiana de Matemática Educativa (ASOCOLME) (2005-2011).
Se privilegió la sistematización de los documentos que abordan temas
relacionados con la generalización matemática haciendo uso de patrones
geométricos; así, se identificaron ciertos elementos específicos a
la luz de los referentes teóricos adoptados. Para la recolección de
información se construyó una rejilla de análisis, la cual permite
la caracterización del documento seleccionado y transformación de
las situaciones. La rejilla fue conformada por las unidades de análisis
asociadas al recurso y a la situación como tal. Siendo necesario explicitar
que la rejilla de análisis fue utilizada en dos niveles; en el primero
se tomó el documento seleccionado y se realizó una caracterización.
En el segundo se caracterizó el proceso de adaptación y transformación
de las situaciones seleccionadas. Una vez hecha la adaptación y transformación
de las situaciones se procedió a realizar la implementación, la cual
se hizo con estudiantes de grado 9\textdegree{} en el marco de una
institución de carácter privado. Con la información recolectada, se
procedió a realizar el análisis correspondiente de la puesta en acto
de las situaciones, estableciendo relaciones entre lo observado y
las unidades de análisis, a través de una rejilla similar a la descrita
anteriormente. Teniendo como referente la noción de recurso pedagógico
asumida en la realización de la investigación y a partir de la experiencia
sistematizada, es importante reconocer que la práctica de enseñanza
adelantada por los maestros de Matemáticas requiere de un trabajo
en comunidad, en el cual se den a conocer y a evaluar las producciones
de estos, evidenciando interés por su desarrollo profesional. Unidades
de análisis como las determinadas brindan una alternativa para cualificar
los criterios de selección y transformación de situaciones matemáticas. 


\section{UNIDADES DE ANÁLISIS PARA IDENTIFICAR LA RELACIÓN “CONOCIMIENTO HISTÓRICO
- CONOCIMIENTO DIDÁCTICO DEL CONTENIDO”}

\begin{datos}

Lyda Constanza Mora, William Jiménez, Edgar Alberto Guacaneme.

Universidad Pedagógica Nacional,

Colombia,

lmendieta@pedagogica.edu.co; williamajg@hotmail.com; 

guacaneme@pedagogica.edu.co

\end{datos}

Educación del Profesor de Matemáticas, Universitario - Formación inicial
de profesores Matemáticas, Investigación cualitativa en Educación.

Resumen En el marco de la investigación “El conocimiento histórico
en la constitución de una visión sobre la naturaleza de la Aritmética
y el Álgebra en maestros de Matemáticas en formación” financiada por
el Centro de Investigaciones de la Universidad Pedagógica Nacional,
se plantea como situación central el estudio de la relación Historia
de las Matemáticas - Conocimiento Didáctico del Contenido, para lo
cual ha sido necesario establecer, a partir de la revisión de literatura
especializada y de las construcciones del grupo de investigación,
unidades de análisis que permitan hallar evidencias de tal relación.
Tales unidades se presentan en este documento.


\section{LA DERIVADA DESDE LA TEORÍA DE LOS MODOS DE PENSAMIENTO, SUSTENTADA
EN LA EPISTEMOLOGÍA DE CAUCHY }

\begin{datos}

Irma Pinto Rojas, Marcela Parraguez González.

Pontificia Universidad Católica de Valparaíso,

Chile,

ipinto@ucn.cl; marcela.parraguez@ucv.cl 

\end{datos}

La investigación en curso presenta la comprensión del concepto derivada,
desde una variación del marco teórico –Los Modos de Pensamiento-,
perspectiva teórica que muestra los niveles de abstracción en conceptos
ligados al álgebra lineal, sin embargo, se propone la posibilidad
de extenderlo al estudio del cálculo, se plantea entonces, el modo
Sintético-Geométrico-Gráfico (SGG), el modo Analítico-Operacional
(AO) y el modo Analítico-Estructural (AE). Se construyen los elementos
que facilitan el tránsito en estos modos y utilizando un estudio de
casos, como marco metodológico para evidenciar que los informantes
muestran dificultades en la articulación de los modos, específicamente
SGG y AE.


\section{\uppercase{ El aula de matemáticas a través de la diversidad: aportes
del enfoque sociocultural en educación matemática a la práctica docente}}

\begin{datos}

Camilo Fuentes, Claudia Acosta, Aura Acero, Lorena Díaz,

Liceth Casallas, Juan Cruz, John Parra.

Universidad Distrital Francisco José de Caldas,

Colombia,

cristianfuentes558@hotmail.com; pat\_9212@hotmail.com;

aura-nana@hotmail.com; lore-2820@hotmail.com;

licethcasallas1208@gmail.com; jdiegomaru@hotmail.com; 

jhonnoside@hotmail.com. 

\end{datos}

El semillero de Etnomatemática está conformado por un grupo de estudiantes
de la Universidad Distrital, en este espacio sus integrantes han estudiado
y reflexionado sobre los aportes del enfoque sociocultural en educación
matemática a las prácticas pedagógicas del profesor de matemáticas
en el contexto Colombiano, en este documento se muestra una experiencia
investigativa en la cual se indaga sobre los aportes del enfoque sociocultural
a las prácticas docentes en la ciudad de Bogotá, para ello se implementaron
a varios docentes encuesta tipo Likert y entrevistas semiestructuradas
en los cuales se buscaban sus concepciones y aportes de este enfoque
en sus prácticas pedagógicas.


\section{REPRESENTACIONES EXTERNAS ASOCIADAS A LOS NÚMEROS RACIONALES EN TEXTOS
DE PRIMER AÑO DE SECUNDARIA}

\begin{datos}

Yaneth josefina, Ríos García

Universidad del Zulia - Facultad de Humanidades y Educación - Centro
de Estudios Matemáticos y Físicos, 

Venezuela - Estado Zulia,

yanriosgarcia@gmail.com 

\end{datos}

El propósito de este estudio es caracterizar las operaciones cognitivas
(creación de signos, transformación sintáctica invariante y variante,
y traducción entre sistemas de representación) asociadas a las representaciones
externas en lo referido a los Números Racionales, puestas en juego
en el libro de texto de primer año de secundaria. Es un estudio de
corte documental. Algunos resultados: el texto es innovador en cuanto
a la creación de los signos a partir de contextos venezolanos; abundan
transformaciones sintácticas invariantes en las representaciones aritméticas;
y se promueve el uso de diferentes sistemas de representación de un
mismo concepto de forma aislada.


\section{COMPRENSIÓN DE LOS NUMEROS COMPLEJOS DESDE LOS MODOS DE PENSAMIENTO }

\begin{datos}

Valeria Randolph Veas, Marcela Parraguez González.

Pontificia Universidad Católica de Valparaíso,

Chile,

valerandolphveas@gmail.com; marcela.parraguez@ucv.cl 

\end{datos}

Pensamiento matemático avanzado, Medio superior, Estudio de casos

La investigación busca evidenciar las diferentes formas de pensar
que estudiantes de enseñanza media (16-18 años) manifiestan al trabajar
el concepto de número complejo, e indagar en las articulaciones de
estas maneras de comprender el concepto. Se utiliza como marco teórico
los modos de pensamiento que sitúa tres modos de pensar: Sintético-Geométrico
(línea dirigida en el plano complejo), Analítico-Aritmético (expresión
algebraica a+bi) y Analítico-Estructural (estructura algebraica de
cuerpo). Bajo el estudio de casos como diseño metodológico, se aplica
un cuestionario que evidencia elementos que facilitan el tránsito
de un modo de pensamiento a otro, para la comprensión profunda del
concepto. 


\section{EL ANÁLISIS EPISTÉMICO COMO ACTIVIDAD PARA EL DESARROLLO DEL CONOCIMIENTO
DIDÁCTICO-MATEMÁTICO EN LA FORMACIÓN DE FUTUROS PROFESORES }

\begin{datos}

Mauro A. Rivas.

Universidad de Los Andes,

Venezuela,

rmauro@ula.ve

\end{datos}

En este documento se informa sobre los resultados de la realización
de un análisis epistémico, llevado a efecto por un grupo de futuros
profesores, sobre la resolución de un problema de proporcionalidad.
Asimismo, se observa la relación entre esa tarea de análisis y el
desarrollo del conocimiento didáctico-matemático de los sujetos. Se
considera el análisis epistémico como la actividad en la que se reconoce
los objetos/significados matemáticos activados en la resolución de
un problema. Los resultados indican que el análisis epistémico conlleva
una reflexión en torno a la actividad de resolución de un problema,
que contribuye al desarrollo del conocimiento didáctico-matemático. 


\section{LAS RELACIONES SOCIALES EN LA CONFIGURACIÓN DE LA IDENTIDAD PROFESIONAL
DESDE LA VOZ DE FUTUROS LICENCIADOS EN MATEMÁTICAS ÁREA MATEMÁTICA
EDUCATIVA DE ACAPULCO GUERRERO}

\begin{datos}

Magdalena Rivera Abrajan, Gustavo Martínez Sierra.

Unidad académica de Matemáticas Universidad Autónoma de Guerrero,
Centro de Investigación en Ciencia Aplicada y Tecnología Avanzada
del Instituto Politécnico Nacional,

México,

magrivab@hotmail.com , gmartinezsierra@gmail.com

\end{datos}

Este trabajo explora las relaciones sociales que se establecen en
el contexto universitario y que configuran la identidad profesional
de estudiantes de la licenciatura en Matemáticas del área de Matemática
educativa de la Universidad Autónoma de Guerrero. Las evidencias se
obtuvieron de entrevistas en grupos focales realizadas a 14 estudiantes
del octavo semestre a través de la metodología narrativa, por medio
del análisis tematizado se localizó el tema de relaciones de poder
en la universidad. Los estudiantes reconocen las relaciones de apoyo
con algunos de sus compañeros y profesores, discriminación en la universidad,
lo que les permite formarse una imagen positiva de su profesión, construyendo
actitudes, valores que acentúan las diferencias y configura rasgos
de auto-percepción entre los grupos.


\section{LA FORMACIÓN INICIAL DE LA NORMAL PARA LA ENSEÑANZA DEL NÚMERO NATURAL.}

\begin{datos}

María Teresa Carballo Riva Palacio, Marta Elena Valdemoros Álvarez.

DME, Cinvestav IPN. 

México ,

carivpa@yahoo.com.mx; mvaldemo@convestav.mx

\end{datos}

Este reporte plantea algunos resultados obtenidos en un estudio exploratorio
efectuado a un grupo de normalistas, para identificar el proceso evolutivo
en la construcción de estructuras cognitivas, los procesos de interpretación
y los usos didácticos otorgados al número natural. Bajo un enfoque
cualitativo se registraron y analizaron las interacciones sociales
entre el investigador, los docentes formadores de formadores, los
normalistas y el alumno de primaria para dar sentido a este conocimiento
matemático. El análisis de datos es de orden epistemológico, cognitivo,
de comunicación social y diseño didáctico, con base a los elementos
teóricos estudiados por el normalista. 


\section{O ESTUDO DA ANÁLISE COMBINATÓRIA A PARTIR DE PROBLEMAS QUE ESTIMULAM
O USO DE DIFERENTES REGISTROS DE REPRESENTAÇÃO E A COMUNICAÇÃO EM
MATEMÁTICA}

\begin{datos}

Tereza Raquel Couto de Lima, Davidson Paulo Azevedo Oliveira.

Instituto Federal de Minas Gerais – IFMG, Campus Ouro Preto,

Brasil ,

tereza.lima@ifmg.edu.br; davidson.oliveira@ifmg.edu.br

\end{datos}

Afim de suavizar as dificuldades encontradas pelos estudantes elaboramos
um conjunto de atividades no qual o uso de fórmulas não é privilegiado.
Os alunos deveriam ler, interpretar e resolver problemas, além de
criar enunciados para situações envolvendo Análise Combinatória. O
trabalho está embasado nas teorias dos registros de representação
semiótica e da Leitura, Interpretação e Construção de Enunciados.
A pesquisa foi desenvolvida com alunos do Ensino Médio de uma escola
pública de Minas Gerais, Brasil. Pudemos verificar que o estudo de
análise combinatória nessa perspectiva contribui para o desenvolvimento
de habilidades, pois os estudantes tornam-se mais autônomos e seguros,
sendo capazes de propor soluções por diferentes caminhos. 


\section{¿CÓMO PROMOVER LA CREATIVIDAD MATEMÁTICA? EL PAPEL DE LAS COMUNIDADES
DE INTERÉS Y DEL DISEÑO DE C-UNIDADES}

\begin{datos}

Berta Barquero, Vicenç Font, Mario Barajas, 

Andrea Richter.

Universitat de Barcelona (UB),

España,

bbarquero@ub.edu; andrea.richter@ub.edu; 

mbarajas@ub.edu; vfont@ub.edu

\end{datos}

Este trabajo se centra en las primeras fases de investigación en el
marco del proyecto europeo MC2 (MathematicalCreativitySquared) que
se propone el objetivo central de indagar en cómo promover la creatividad
a través del diseño creativo de unidades didácticas (c-unidades).
En este diseño, en manos de las denominadas comunidades de interés
(CdI), confluirán diferentes formas de entender qué es la creatividad
matemática. Nos centraremos aquí en indagar sobre las preconcepciones
que surgen cuando interrogamos a nuestra CdI sobre la creatividad
matemática y sobre los primeros criterios a considerar para promover
el pensamiento matemático creativo en el diseño de c-unidades.


\section{LA EDUCACIÓN FÍSICA COMO ALTERNATIVA PEDAGÓGICA PARA EL MEJORAMIENTO
DE LAS COMPETENCIAS MATEMÁTICAS EN EL NIVEL PRIMARIA}

\begin{datos}

Evelia Reséndiz Balderas, Horacio García Mata.

Universidad Autónoma de Tamaulipas,

México,

erbalderas@uat.edu.mx ; lachomata\_@hotmail.com

\end{datos}

Esta investigación pretende construir otra visión de la educación
física y las matemáticas en la escuela, esta última asignatura ha
sido elegida debido a las grandes deficiencias que existen hoy en
día en los alumnos para su comprensión. La idea central es revitalizar
ambas asignaturas a través de la vinculación curricular y hacer de
sus prácticas una opción importante para el mejoramiento del aprendizaje
de los niños. Esta investigación pretende orientar ambas asignaturas
a una intervención educativa basada en la vinculación de las experiencias
motrices y cognitivas de los niños, es por eso que el producto final
es elaborar una propuesta de Vinculación Curricular.


\section{UN EJEMPLO DE RESIGNIFICACIÓN DE LA ESTABILIDAD DESDE EL USO DE LAS
GRÁFICAS. UN ANÁLISIS DESDE EL COTIDIANO}

\begin{datos}

David Zaldívar Rojas, Francisco Cordero.

Cinvestav-IPN,

México,

jzaldivar@cinvestav.mx; fcordero@cinvestav.mx 

\end{datos}

El objetivo de este reporte es evidenciar la conveniencia de integrar
al análisis en las investigaciones que atienden al uso del conocimiento
en socioepistemología a una categoría del cotidiano del ciudadano
para avanzar hacia un rediseño del discurso Matemático Escolar. Lo
anterior a partir de concretar los elementos inmersos en un ejemplo:
la resignificación de una propiedad de estabilidad a partir de la
modelación-graficación en un escenario particular. 


\section{EL CONCEPTO DE FUNCIÓN LINEAL EN EL BACHILLERATO TECNOLÓGICO MEXICANO}

\begin{datos}

Rebeca Flores García.

CICATA – IPN,

México,

rebefg@gmail.com

\end{datos}

El estudio de la función lineal en la enseñanza de las matemáticas
en el nivel medio superior desempeña un papel importante en el aprendizaje
de los estudiantes, no sólo por estar relacionado con temas de otras
asignaturas, sino porque permite representar situaciones reales. Esta
investigación pretende articularse a lo que en la disciplina de Matemática
Educativa se está produciendo mediante diversas vertientes de investigación
relacionadas con el concepto de función lineal, específicamente se
pretende profundizar en el currículum escrito, el cual incluye a todos
aquellos documentos, programas y libros de texto proporcionados al
profesor para desarrollar su quehacer docente. 


\section{REFINAMIENTO DE UNA DESCOMPOSICIÓN GENÉTICA PARA EL CONCEPTO DE INDUCCIÓN
MATEMÁTICA}

\begin{datos}

Isabel García Martínez, Marcela Parraguez González.

Pontificia Universidad Católica de Valparaíso,

Chile,

igarcia@ucn.cl; marcela.parraguez@ucv.cl 

\end{datos}

En este trabajo se presenta un estudio del concepto inducción matemática
en la enseñanza superior bajo el marco teórico de la teoría APOE y
estudio de casos como diseño metodológico. Basándonos en una descomposición
genética diseñada por Dubinsky y Lewin para dicho concepto, analizamos
las pr oducciones de diez estudiantes universitarios para sustentar
cuáles de las construcciones que propone la descomposición genética
muestran los estudiantes. A la luz de los resultados obtenidos, refinamos
la descomposición genética, explicitando en ella el paso base de la
inducción matemática como construcción mental proceso.


\section{UNA SECUENCIA DE FORMACION PARA MAESTROS: REFLEXIONANDO ACERCA DE
LOS PAEV ADITIVOS DE UNA ETAPA}

\begin{datos}

Angela Castro, Núria Gorgorió, Montserrat Prat.

Universitat Autônoma de Barcelona,

España,

angicastro-27@hotmail.com; nuria.gorgorio@uab.cat;

Montserrat.prat@uab.cat

\end{datos}

La secuencia de formación que presentamos tiene como objetivo conseguir
que los alumnos del Grado de Educación Primaria descubran que plantear
problemas aritméticos de enunciado verbal de suma y resta con una
operación, va más allá de enunciar situaciones que contengan los verbos
“añadir” o “juntar” para la suma, y “quitar” o “separar” para la resta.
Queremos que descubran las posibilidades, la variedad y la riqueza
que ofrecen estos problemas, con la intención de que en su práctica
como docentes se sientan competentes para incorporar estos conocimientos
a su actividad en el aula.


\section{LA METÁFORA DE LA REPRESENTACIÓN VECTORIAL }

\begin{datos}

Elizabeth Hernández Arredondo, Claudia M. Acuña Soto.

CINVESTAV-IPN,

México,

eli\_visual@hotmail.com ; claudiamargarita\_as@hotmail.com

\end{datos}

En este trabajo hacemos un estudio de la construcción de metáforas
en problemas de cinemática resueltos por universitarios, al mismo
tiempo detectamos el uso de signos asociados al vector y sus significados
en el proceso de estos problemas. Las actividades se apoyaron del
uso de lápiz y papel y la computadora en trabajo colaborativo, identificamos
el uso de la metáfora del camino simultáneamente con otras, encontrando
que el empleo exitoso de esta metáfora no es suficiente para modelar
adecuadamente los problemas de movimiento propuestos, se requiere
además de un uso del vector como un símbolo matemático y de una interpretación
espacial adecuada para la solución de los problemas.


\section{LSM EN EL AULA DE SORDOS {[}17-21{]} Y COMPRENSIÓN DE NOCIONES DEL
SISTEMA MÉTRICO DECIMAL: ESTUDIO DE CASOS}

\begin{datos}

Ignacio Garnica y Dovala, Andrea Barojas Gómez.

Departamento de Matemática Educativa, Cinvestav-IPN,

México,

igarnica@cinvestav.mx; abarojas@cinvestav.mx 

\end{datos}

Se reportan resultados del proyecto de investigación que orientó sus
preguntas a identificar niveles de competencia lingüística y comunicativa
de la Lengua de Señas Mexicana (LSM) en relación con la comprensión
de las nociones del Sistema Métrico Decimal (SMD). Se diseñó un modelo
de comunicación entre los estudiantes y los investigadores no competentes
en LSM. Se acordaron señas referentes a las nociones de cantidad magnitud
masa, que se usaron en: enseñanza, indagación e investigación en tiempo
real en el aula de educación básica de ocho jóvenes Sordos (17-21)
de una ONG. Para las señas propuestas se hizo su análisis fonológico.

Palabras clave: competencia lingüística y comunicativa, Lengua de
Señas Mexicana (LSM), Sistema Métrico Decimal (SMD), Señas propuestas, 


\section{EVALUACIÓN PARA EL APRENDIZAJE DE LA MATEMÁTICA}

\begin{datos}

Angélica Gálvez Pacheco, Crisólogo Dolores Flores, Gerardo Salgado
Beltrán.

Universidad Autónoma de Guerrero,

México,

angelicagalvezpacheco@gmail.com; cdolores2@gmail.com;

gerardosalgadobeltran@yahoo.es 

\end{datos}

Este documento da cuenta de avances de una tesis de carácter interventiva
en la docencia de la matemática. Tiene por objetivo elaborar y poner
a prueba una secuencia para el aprendizaje del concepto razón de cambio
en estudiantes de bachillerato privilegiando la evaluación para el
aprendizaje. Para la intervención se seguirá el método investigación-acción
cuyas fases son: problematización, diagnóstico, diseño, aplicación
y evaluación. La evaluación para el aprendizaje está orientada a recoger
evidencias respecto del aprendizaje de los alumnos en forma planificada
y sistemática para emitir juicios con la finalidad de mejorar la calidad
de la enseñanza y el aprendizaje. 


\section{UM ESTUDO SOBRE OS CONHECIMENTOS NECESSÁRIOS AO PROFESSOR PARA ENSINAR
NOÇÕES CONCERNENTES À PROBABILIDADE NOS ANOS INICIAIS}

\begin{datos}

Ruy César Pietropaolo, Tânia M. M. Campos, Angélica da Fontoura Garcia
Silva.

Universidade Bandeirante Anhanguera,

Brasil ,

rpietropaolo@gmail.com; taniammcampos@hotmail.com;

angelicafontoura@gmail.com

\end{datos}

A finalidade deste trabalho, desenvolvido no âmbito do Observatório
da Educação com 27 professores, é discutir sobre os conhecimentos
necessários ao professor para ensinar probabilidade nos anos iniciais.
Realizou-se um Diagnóstico com o objetivo de investigar a imagem conceitual
dos professores sobre esse tema e seu ensino. Depois dessa fase, desenvolveu-se
uma Formação, segundo princípios do “Design Experiments” de Cobb et
al. (2003), cujo objetivo foi ampliar a base de conhecimentos para
o ensino de conceitos referentes a esse tema segundo as categorias
de Ball et al (2008). 


\section{LOS NÚMEROS RACIONALES: UNA MIRADA DESDE LA TEORÍA LOS MODOS DE PENSAMIENTO
EN LA FORMACIÓN INICIAL DE PROFESORES}

\begin{datos}

Daniela Bonilla Barraza, Marcela Parraguez González.

Universidad de La Serena, Pontificia Universidad Católica de Valparaíso

Chile, 

danielabonillab@gmail.com; marcela.parraguez@ucv.cl

\end{datos}

Pensamiento matemático. Superior. Estudio de casos. 

El siguiente reporte de investigación, tiene por objetivo mostrar
evidencias, de las diferentes maneras de pensar que los profesores
en formación inicial, ponen de relieve, para dar cuenta de la comprensión
del sistema de los números racionales. El marco teórico sobre el cual
se basa este estudio es los modos de pensamiento de Anna Sierpinska,
desde este referente comprendemos el sistema de los números racionales
en tres perspectivas, (AE): como un representante de una clase de
equivalencia, (AA): como un cociente de dos números enteros (con divisor
distinto de cero), y (SG): como un punto en la recta numérica.


\section{DIFICULTADES EN CONCEPTOS MATEMÁTICOS QUE IMPLIQUEN EL USO DE FRACCIONES}

\begin{datos}

Noelia Londoño Millán, Alibeit Kakes Cruz, Jacqueline Llanes Castro.

Universidad Autónoma de Coahuila, Facultad de Ciencias Físico Matemáticas,

México,

noelialondono@uadec.edu.mx; alibeitkakes@uadec.edu.mx;

jacquelinellanes@hotmail.com

\end{datos}

Números racionales y proporcionalidad Nivel básico Estudio de casos

Luego de identificar algunas dificultades de los alumnos para realizar
operaciones con fracciones en los niveles de secundaria y medio superior,
se emprendió un proyecto con alumnos de primaria, para detectar dificultades
que se presentan cuando los alumnos de 5\textdegree{} y 6\textdegree{}
de educación básica usan las fracciones propias, impropias y equivalentes
en sus distintas representaciones y en la resolución de problemas,
el presente reporte de investigación hace referencia a algunas de
estas dificultades que tienen que ver con el uso del lenguaje, manejo
de algoritmos y ubicación en la recta numérica. 


\section{RENDIMIENTO ACADÉMICO EN LOS PRIMEROS CURSOS DE MATEMÁTICA UNIVERSITARIA }

\begin{datos}

Mario Castillo Sánchez, Ronny Gamboa Araya, Randall Hidalgo Mora.

Universidad Nacional,

Costa Rica,

mario.castillo.sanchez@una.cr; ronny.gamboa.araya@una.cr;

randall.hidalgo.mora@una.cr

\end{datos}

Esta investigación profundiza en aspectos que influyen en el rendimiento
académico de los estudiantes. Se realizó un cuestionario dirigido
a 386 estudiantes del curso Matemática General, impartido en la Universidad
Nacional de Costa Rica. Respecto al apoyo de su grupo familiar indican
que estos consideran que hay que estudiar para ser alguien en la vida
y poseen el concepto de que ellos son buenos estudiantes. En relación
con sus padres la mayoría mencionan que sus progenitores se interesan
por su desempeño académico que poseen expectativas muy altas sobre
su rendimiento. Mencionan que sus conocimientos previos no son suficientes
para obtener éxito.


\section{MATEMÁTICA E INTERDISCIPLINARIDADE: ESTABELECENDO CONEXÕES ATRAVÉS
DE PARADIDÁTICOS}

\begin{datos}

Elaine Souza de Macedo, Maria Dolores Costa Lhamas Cardoso, Mércia
de Oliveira Pontes, Micarlla Priscilla Freitas da Silva.

UFRN, 

esmtot@gmail.com; doloreslhamas@yahoo.com.br;

merciaopontes@gmail.com; micarllapriscilla@hotmail.com

\end{datos}

A pesquisa Matemática e interdisciplinaridade: estabelecendo conexões
através de Paradidáticos desenvolvida por um grupo ligado ao Observatório
da Educação – OBEDUC 2012 da Universidade Federal do Rio Grande do
Norte – UFRN, objetiva verificar a possibilidade de realização de
atividades interdisciplinares entre a Matemática e outras áreas do
currículo escolar por meio da leitura de paradidáticos de Matemática.
Através do roteiro de exploração analisado por professores, sugerimos,
inicialmente, atividades de integração entre a Língua Materna e a
Matemática e, posteriormente, atividades interdisciplinares com outras
áreas/disciplinas do currículo escolar.

\setcounter{section}{151}


\section{EL DESGASTE ESTUDIANTIL EN LA CARRERA DE ENSEÑANZA DE LA MATEMÁTICA}

\begin{datos}

Mario Castillo Sánchez, Ronny, Gamboa Araya, Randall Hidalgo Mora.

Universidad Nacional,

Costa Rica,

mario.castillo.sanchez@una.cr ; ronny.gamboa.araya@una.cr;

randall.hidalgo.mora@una.cr

\end{datos}

Campo: Educación de Adultos, Nivel: Superior (19-22 años), Tipo: Estadístico

La investigación analiza el desgaste estudiantil en los estudiantes
de la Carrera de Enseñanza de la Matemática. Este afecta al estudiante
y aparecen síntomas como insomnio, úlceras, pérdida de peso, problemas
gastrointestinales, fatiga. Se aplicó cuestionario a 104 estudiantes,
el 65\% perdió al menos un curso, el 42\% ha pensado cambiarse de
carrera, el 21\% no la recomendaría, sin embargo el 82\% se encuentra
a gusto. Importante analizar con detalle los resultados obtenidos,
pues este es un primer acercamiento, pero si no se atienden pueden
provocar situaciones de estrés que incidirían en el rendimiento académico
y en el bienestar estudiantil


\section{LA CONEXIÓN ENTRE LA DERIVADA Y LA INTEGRAL}

\begin{datos}

Crisólogo Dolores Flores, Bartolo Ponce García.

Universidad Autónoma de Guerrero, 

México,

cdolores2@gmail.com, ponce.22@hotmail.com

\end{datos}

Pensamiento y lenguaje variacional, Nivel Superior La presente investigación
ha adoptado como objeto de estudio a las relaciones entre los conceptos
esenciales del Cálculo: la integral y la derivada. Es motivada por
los escasos trabajos que se encarguen de atender las conexiones entre
estos conceptos. Por tal razón, nuestro objetivo es explorar los procesos
de conexión entre la derivada y la integral que establecen los estudiantes
universitarios. Nuestra hipótesis es que los estudiantes establecen
escasa conexión entre estos conceptos en particular los relativos
a la reversibilidad. En el plano matemático la relación entre ambos
conceptos está fundada en el Teorema Fundamental del Cálculo. 

\setcounter{section}{154}


\section{SENTIDOS E SIGNIFICADOS EM EDUCAÇÃO MATEMÁTICA: INTERPRETANDO O OBJETO
E A NEGAÇÃO DO OBJETO.}

\begin{datos}

Lúcia Cristina S. Monteiro, Michael F. Otte.

Universidade Anhanguera de São Paulo,

Brasil,

lucia.csmonteiro@uol.com.br; michaelontra@aol.com 

\end{datos}

Trata da elaboração de uma epistemologia para a Educação Matemática.
Consideraremos discussões encontradas sobre os paradoxos de Zenão,
relacionando-o ao tempo, ao espaço, ao movimento, e, às noções de
infinito e de continuidade. Aqui nesse trabalho, acrescentaremos que
as aporias de Zenão dizem respeito à produção de novos signos, sentidos
e significados. Assim, buscaremos compreender a cognição e o conhecimento
como um processo semiótico. Para acrescentar essa proposta sugeriremos
a negação do objeto matemático, como proposto por Hegel. Palavras-chave:
Educação Matemática, semiótica, objeto e negação do objeto.

Referências bibliográficas CARAÇA, B. de J. (2002). Conceitos fundamentais
da matemática. 6. ed. Lisboa: Gradiva. HEGEL, G. W. F. (2013). Ciencia
de la Lógica. 1ª ed. – Buenos Aires: Las cuarenta. HUGGETT, N. (1997).
Space from Zenon to Einstein: classic readings with a contemporary
commentary/editado e comentado por Nick Huggett. A. Bradford book.
London, England. KOYRÉ, A. (2011). Estudos de história do pensamento
filosófico; Tradução: de Maria de Lourdes Menezes. – 2.ed.-Rio de
Janeiro: Forense. OTTE, M. (1990). Arithmatic and Geometry: Some Remarks
on the concept of complemetary. In Studies in Philosophy and Education,
nº 10. pp 37-62, Kluwer Academic Publishers. Printed in the Netherlands.
OGDEN C.K. \& RICHARDS, I.A. (1989). The Meaning of Meaning: a study
of the influence of language upon thought and of the science of symbolism.
HBJ Book. New York. PEIRCE, C. S. (2010) Semiótica. Tradução José
Teixeira Coelho Neto - 4 ed. – São Paulo: Perspectiva. RUSSEL, B.
(2007). Introdução à filosofia matemática. Trad., Maria Luiza X. de
A. Borges. Rio de Janeiro: Zahar. SANTAELLA, L. (1995). A teoria Geral
dos signos: semiose e autogeração. São Paulo, SP: Ática.


\section{APRENDER A FORMAR EN CIUDADANÍA EN LA FORMACIÓN DE PROFESORES DE
MATEMÁTICAS}

\begin{datos}

Yuly Vanegas, Joaquin Giménez, Vicenç Font.

Universidad de Barcelona,

España,

ymvanegas@ub.edu; quimgimenez@ub.edu;

vfont@ub.edu

\end{datos}

Se presenta un estudio basado en el diseño (DBR) sobre una propuesta
de formación de profesores de matemáticas que pretende el desarrollo
de la competencia: aprender a formar en ciudadanía a través de las
matemáticas. Se justifican las componentes principales del diseño,
se analizan algunos resultados después de tres sucesivas implementaciones
de la misma. Encontramos que los futuros docentes mejoran en el análisis
didáctico a partir de la reflexión sobre su propia práctica e incorporan
aspectos de la ciudadanía fundamentalmente en el análisis sobre la
contextualización, el pensamiento crítico y el análisis de interacciones
en el aula.


\section{ELEMENTOS PARA LA CONSTRUCCIÓN DEL ESPACIO DE TRABAJO ALGEBRAICO$\left(ETM_{A}\right)$
.}

\begin{datos}

Ninoska Valdés Verdugo, Arturo Mena Lorca.

Pontificia Universidad Católica de Valparaíso,

Chile,

ninoskandreavaldes@gmail.com; arturo.mena@ucv.cl

\end{datos}

Pensamiento Algebraico, Nivel medio superior, Estudio de casos

El presente reporte propone la creación de un espacio de trabajo algebraico$\left(ETM_{A}\right)$.
Nos basamos en la génesis epistemológica y cognitiva que propone Kuzniak
(2011) en su Espacio de trabajo Matemático, las cuales se orientarán
en la construcción y caracterización de un $ETM_{A}$. A su vez, nos
permitirá identificar y clarificar elementos, concepciones y/o niveles
en que se encuentran estudiantes Chilenos de cuarto año medio (17-19
años). Se utilizará un estudio de casos, que contempla el diseño y
aplicación de un cuestionario que pretende aportar información sobre
elementos del $ETM_{A}$ y contribuir desde esta perspectiva a la
matemática escolar del país. 


\section{CONOCIMIENTO DIDÁCTICO DEL CONTENIDO: ESTIMACIÓN MÉTRICA }

\begin{datos}

Noemí Pizarro, Nuria Gorgorió, Lluís Albarracín.

Universitat Autònoma de Barcelona,

España,

noemipizarro@gmail.com; nuria.gorgorio@uab.cat;

lluis.albarracin@uab.cat 

\end{datos}

Presentamos un estudio sobre el conocimiento del contenido matemático
para la enseñanza que poseen los maestros sobre el concepto de estimación
de medida. A partir de una encuesta a 112 maestros de Santiago de
Chile analizamos las definiciones dadas de estimación de medida a
partir de una reelaboración de la definición que permite su análisis.
Los resultados muestran que el concepto de estimación de medida está
poco desarrollado en los maestros, por lo que es probable que el conocimiento
para la enseñanza que los maestros tienen no sea correcto ni completo. 


\section{VARIABILIDAD EN LA CORRECCIÓN DE PRUEBAS DE MATEMÁTICAS }

\begin{datos}

Elena Mengual, Núria Gorgorió, Lluís Albarracín.

Universitat Autònoma de Barcelona,

España,

@gmail.com; nuria.gorgorio@uab.cat;

lluis.albarracin@uab.cat 

\end{datos}

Evaluación de pruebas de matemáticas, educación secundaria, trabajo
empírico

En esta comunicación presentamos estudio empírico sobre la variabilidad
introducida por los correctores en una prueba de matemáticas de acceso
a la universidad. El análisis de las calificaciones muestra que el
modelo de corrección propuesto genera una disminución de la variabilidad
de las calificaciones en un porcentaje elevado de las respuestas.
Al mismo tiempo, el análisis de los casos que no disminuyen la variabilidad
muestra que los correctores no siguen fielmente los criterios dados,
por lo que se propone como tema de formación del profesorado.


\section{LAS PRÁCTICAS DE SIMULACIÓN Y LA EMERGENCIA DE LA INTEGRAL: RECURSOS
GESTUALES}

\begin{datos}

Juan Felipe Flores Robles, Jaime Arrieta Vera, Eduardo Carrasco Henríquez.

Universidad Autónoma de Guerrero, Universidad Austral,

México, Chile,

Juan.F10res@hotmail.com; jaime.arrieta@gmail.com; 

ecarrasc@gmail.com

\end{datos}

Consideramos la simulación como práctica recurrente de diversas comunidades
con intención de reproducir algún fenómeno partiendo de sus modelos.
Los recursos para simular algún fenómeno son variados: lenguaje natural,
recursos numéricos, algebraicos, gráficos, pictográficos, computacionales
y gestuales. Las herramientas para simular, también son variadas.
Este trabajo aporta evidencias de cómo estudiantes construyen la integral
como herramienta para simular utilizando recursos gestual. Para ello
analizamos las producciones de los actores de la puesta en escena
de un diseño de aprendizaje basado en la simulación del movimiento
partiendo de modelos gráficos diferenciales. El marco teórico que
soporta nuestro trabajo es la socioepistemología. 


\section{USANDO LAS MATEMÁTICAS PARA HACER HISTORIA: UN CONTEXTO PARA DESARROLLAR
LA COMPETENCIA EN INDAGACIÓN}

\begin{datos}

Gemma Sala, Berta Barquero, Vicenç Font, Joaquim Giménez.

Universitat de Barcelona,

España,

gsala@ub.edu; bbarquero@ub.edu;

vfont@ub.edu; quimgimenez@ub.edu 

\end{datos}

En este trabajo se describe el proceso de diseño e implementación
de una secuencia didáctica interdisciplinar de contexto histórico-matemático
basada en la guerra de Sucesión de 1714 en Catalunya (una nación de
España). Su implementación, con dos grupos pilotos de 12-14 años de
Educación Secundaria Obligatoria, permitirá desarrollar sus competencias
en modelización e indagación. Posteriormente, el rediseño de esta
secuencia será la base de una propuesta para los estudiantes de Formación
de Profesorado de la Universitat de Barcelona, con el objetivo de
desarrollar la competencia profesional que les permita diseñar este
tipo de secuencias didácticas para sus futuros alumnos.


\section{UN ENTORNO GEOMÉTRICO PARA LA RESIGNIFICACIÓN DE LAS RAZONES TRIGONOMÉTRICAS
EN ESTUDIANTES DE INGENIERÍA }

\begin{datos}

Diana del Carmen Torres Corrales{*}, Gisela Montiel Espinosa{*}{*},
Omar Cuevas Salazar{*}.

Instituto Tecnológico de Sonora{*}, Instituto Politécnico Nacional{*}{*},

México,

d.torres@live.com.mx; gmontiel@ipn.mx; 

ocuevas@itson.edu.mx 

\end{datos}

El fenómeno didáctico asociado a la razón trigonométrica, la aritmetización,
en estudiantes de ingeniería, originó el diseño de un entorno geométrico
para buscar su resignificación. Fundamentado en las propuestas de
Moore en el contexto del círculo (devuelve a lo trigonométrico su
naturaleza geométrica), en la epistemología basada en la actividad
que propone Montiel, la hipótesis de Molina de las 4 condiciones de
resignificación desde una perspectiva de evidencia en la acción, la
metodología de experimentos de diseño de Coob, y el uso de materiales
(manipulables y el software GeoGebra), desde un enfoque teórico de
corte social.


\section{UNA HERRAMIENTA PARA VALORAR LA PRODUCCIÓN DE LOS ESTUDIANTES ANTE
TAREAS DE INVENCIÓN DE PROBLEMAS ARITMÉTICOS VERBALES }

\begin{datos}

Johan Espinoza González.

Universidad Nacional de Costa Rica..

Costa Rica,

jespinoza@una.cr 

\end{datos}

Recientemente se observa un progreso en el desarrollo de enfoques
de instrucción que incorporan el planteamiento de problemas como una
actividad de clase (Brown y Walter, 1990; Silverman, Winograd y Strohauer,
1992; Skinner, 1991); sin embargo, se ha prestado poca atención a
la valoración de las producciones de los estudiantes ante este tipo
de tareas (Silver \& Cai, 2005). De esta forma, se muestra una herramienta
que permite valorar las producciones de los estudiantes cuando inventan
problemas aritméticos, la cual es resultado de un estudio teórico
sobre el planteamiento de problemas, su interpretación y cómo han
valorado otras investigaciones los enunciados de los estudiantes cuando
inventan problemas; así como la conceptualización, clasificación y
variables de estudio de los problemas aritméticos verbales. 


\section{EL USO DE ITERACIÓN COMO BASE PARA DESARROLLAR EL RAZONAMIENTO PROPORCIONAL}

\begin{datos}

Juan C. Ramírez, Claudia Acuña.

Centro de Investigación y Estudios Avanzados del IPN,

México,

jcrmaciel@yahoo.com.mx; claudia\_as@hotmail.com

\end{datos}

La siguiente investigación muestra cómo la iteración utilizada como
mecanismo de solución permite la interpretación de problemas de forma
unitaria y expone atributos que contribuyen al razonamiento proporcional,
en particular el caso donde hay que encontrar el valor faltante. Al
sugerir problemas de encontrar el valor faltante en diversos contextos
observamos que: cuando los estudiantes utilizan como estrategia la
iteración son capaces en un primer momento de establecer razones e
interpretar el problema adecuadamente mejorando su razonamiento proporcional,
en cambio cuando optan por utilizar únicamente recursos o reglas aritméticas
como la regla de tres, el razonamiento proporcional es oscurecido.


\section{ENSEÑANZA VIRTUAL DEL CÁLCULO. APORTES INICIALES}

\begin{datos}

Ricardo E. Valles P, Dorenis J. Mota V. 

Universidad Simón Bolívar-Sede del Litoral,

Venezuela,

revalles@usb.ve, dorenismota@usb.ve

\end{datos}

Este reporte tiene como finalidad describir los aportes que se vienen
dando en la enseñanza del cálculo con el apoyo de recursos tecnológicos,
los cuales brindan la posibilidad a los usuarios (estudiantes y docentes)
de manipular, asimilar, intercambiar, y generar nuevos materiales
y contenidos matemáticos. El estudio se desarrolla en la Universidad
Simón Bolívar Sede Litoral-Venezuela, cuyos participantes son estudiantes
de la Sección 11 de Matemática I correspondiente a Cálculo Integral,
donde la enseñanza se imparte con apoyo de recursos tecnológicos:
Aula virtual (interacción asincrónica por medio de foros de discusión)
y proyección audiovisual como apoyo a las clases presenciales.


\section{EL JUEGO COMO UNA ESTRATEGIA DIDÁCTICA PARA DESARROLLAR EL PENSAMIENTO
NUMÉRICO EN LAS CUATRO OPERACIONES BÁSICAS }

\begin{datos}

Humberto Colorado Torres, Jorge Hernán Aristizabal Zapata.

Universidad del Quindío, 

Colombia,

colorado@uniquindio.edu.co ; jhaz@uniquindio.edu.co

\end{datos}

La presente investigación permite desarrollar distintas habilidades
y relaciones para familiarizarse y reforzar las operaciones básicas
en estudiantes de grado quinto en la I.E MARIN GRANADA en Circasia
(Quindío), asumiendo que el juego ocupa un lugar primordial entre
las múltiples actividades del niño(a), El proyecto se enmarco en la
modalidad experimental, además exploratoria por cuanto se implementó
una estrategia didáctica consistente en una serie de actividades y/o
juegos en cada operación. La implementación de la estrategia permitió
reconocer el nivel de comprensión de la temática tratada y generando
mayor motivación e interés en los estudiantes en el tema propuesto. 


\section{REPRESENTACIÓN MATEMÁTICA DEL CAMBIO UNIFORME EN ESTUDIANTES DEL
NIVEL MEDIO SUPERIOR. }

\begin{datos}

Jesús Eduardo Hinojos Ramos, Juan Antonio Alanis Rodríguez, Julia
Xochilt Peralta García. 

Instituto Tecnológico de Sonora, 

Instituto Tecnológico de Estudios Superiores de Monterrey, 

Instituto Tecnológico de Sonora, 

México,

Jesus.hinojos@live.com.mx; juan.antonio.alanis@itesm.mx; 

julia.peralta@itson.mx

\end{datos}

La manera en que el cálculo es presentado de manera tradicional como
reglas y algoritmos que genera capacidad de resolver problemas tradicionales,
pero sin comprensión e incapacidad de aplicarlo en la resolución de
problemas de aplicación ha propiciado el desarrollo de diversos desarrollos
en didáctica de las matemáticas, como la implementación de secuencias
didácticas no formales en la educación superior; basando la instrucción
en la propuesta didáctica de Salinas, Alanis, Pulido, Escobedo y Garza,
se genera una evaluación para alumnos de bachillerato con el fin de
determinar su nivel de comprensión de las funciones lineales en contexto
del cálculo diferencial.


\section{CONSTRUCCIÓN DE SIGNIFICADOS DE LAS RAZONES TRIGONOMÉTRICAS EN EL
CONTEXTO GEOMÉTRICO DEL CÍRCULO}

\begin{datos}

Olivia Alexandra Scholz Marbán, Gisela Montiel Espinosa.

Instituto de Educación Media Superior,

Instituto Politécnico Nacional, 

México,

olivia.scholz@iems.edu.mx; gmontiel@ipn.mx

\end{datos}

La investigación se enmarca en la trigonometría en el escenario escolar
del nivel medio superior, específicamente en el tema de la razón trigonométrica;
se diseñó una secuencia didáctica para estudiar el proceso de significación
progresiva de los estudiantes en el desarrollo de actividades de construcción
geométrica, en el contexto del círculo. Provocando que la Trigonometría
sea el estudio de los triángulos en un sentido geométrico y la operatividad
(división) entre las longitudes del triángulo se introduce como herramienta
para cuantificar la relación entre el ángulo central y la distancia
entre dos puntos sobre la circunferencia, en un momento de institucionalización.


\section{LECTURA Y CONSTRUCCIÓN DE GRÁFICAS EN EDUCACIÓN SECUNDARIA}

\begin{datos}

Santiago Ramiro Velázquez Bustamante, Josip Slisko Ignjatov, Hermes
Nolasco Hesiquio.

Secretaría de Educación Guerrero, Benemérita Universidad Autónoma
de Puebla, Universidad Autónoma de Guerrero,

México,

sramiro@prodigy.net.mx; jslisko@fcfm.buap; nolascohh.@hotmail.com

\end{datos}

La lectura y construcción de gráficas se enmarca en el manejo de la
información. Por lo general en la escuela se escolariza el saber referente
a estos contenidos, lo que dificulta reconocer los usos y significados
de las gráficas y trabajar su lectura y construcción como una práctica
social. Proponemos analizar, diversas evidencias sobre lo que hacen
profesores y alumnos cuando abordan este contenido. Para lograrlo
se hace un estudio del estado del arte sobre la lectura y construcción
de gráficas, se analizan los diarios de campo de quince alumnos de
cinco escuelas secundarias y se entrevista a sus profesores.


\section{LA MATEMÁTICA FUNCIONAL DESDE LA COMUNIDAD DE CONOCIMIENTO MATEMÁTICO
DE INGENIEROS. EL CASO DE LA ESTABILIDAD DEL EQUILIBRIO. }

\begin{datos}

Johanna Mendoza, Francisco Cordero.

CINVESTAV,

México ejmendoza@cinvestav.mx; fcordero@cinvestav.mx 

\end{datos}

La investigación plantea que la ingeniería, como disciplina, usa la
matemática como una herramienta para la construcción de su conocimiento
disciplinar. La tarea consiste en caracterizar elementos de la funcionalidad
del conocimiento matemático desde la comunidad de ingenieros. Identificar
sus usos, tanto en su saber como en el hacer, así como la intencionalidad.
Esto conlleva crear y establecer un diálogo para identificar situaciones
en el cotidiano del ingeniero, que ofrecerán los significados, procedimientos,
argumentaciones y usos. Buscaremos las evidencias en situaciones específicas
que aludan a las prácticas del ingeniero. En ese sentido, se formulará
un epistemología que consistirá en problematizar el movimiento permanente
en cualquier tiempo. La naturaleza de las situaciones que definirán
a la comunidad de conocimiento del ingeniero corresponderá a una categoría
de modelación.


\section{CONCEPCIONES DE PROFESORES DE MATEMÁTICAS EN FORMACIÓN RESPECTO A
LOS INTERVALOS DE CONFIANZA }

\begin{datos}

Luzdari Rangel Ruiz, Gabriel Yáñez Canal.

Universidad Industrial de Santander; Grupo de investigación EDUMAT-UIS,

Colombia,

lrangelruiz@gmail.com; gyanez@uis.edu.co

\end{datos}

Los estudios revisados hasta el momento alrededor de los intervalos
de confianza son descriptivos y carecen de razones que permitan superar
y explicar la naturaleza de las concepciones y dificultades presentes
en estudiantes, profesores, expertos e investigadores. Por tanto,
se está realizando un estudio con el cual se pretende describir las
ideas y razonamientos que hacen los profesores de matemáticas en formación
para dar respuesta a los diferentes cuestionamientos en términos de
las estructuras y mecanismos mentales que un individuo desarrolla
cuando aprende un concepto estadístico. Para esto nos fundamentaremos
en los componentes teóricos planteados por la teoría APOE.


\section{AMBIENTES COMPUTACIONALES DE APOYO PARA LA ENSEÑANZA DE LAS MATEMÁTICAS
(ACAEM) CASO PARTICULAR, LOS SISTEMAS DE ECUACIONES LINEALES EN EDUCACIÓN
SUPERIOR.}

\begin{datos}

Carlos Armando Cuevas Vallejo, Yani Betancourt González, José Del
Carmen Orozco Santiago.

entro de Investigación y de Estudios Avanzados del I.P.N. (CINVESTAV),

México, 

ccuevas@cinvestav.mx; betancourt@cinvestav.mx;

jorozco@cinvestav.mx C

\end{datos}

Este trabajo de investigación se presenta bajo un proyecto de acción
práctica (Cuevas y Pluvinage, 2003), mediante un par de ambientes
computacionales de apoyo a la enseñanza de las matemáticas para el
estudio de los Sistemas de Ecuaciones Lineales. El primer recurso
denominado ALSEL, es un software educativo que apoya la resolución
de sistemas de ecuaciones lineales paso a paso bajo el método de eliminación
gaussiana y que destaca el concepto de sistema equivalente, el segundo
recurso, es un recurso Online denominado ALyTEC, que ofrece una herramienta
para generar Sistemas de Ecuaciones Lineales con características dadas
por el usuario.


\section{EL ESTADO ACTUAL DE LOS LIBROS DE TEXTO DE EDUCACIÓN PRIMARIA EN
MÉXICO: EL CASO DEL SISTEMA DE NUMERACIÓN DECIMAL }

\begin{datos}

Catalina Navarro Sandoval, Luis Augusto Campistrous Pérez.

Unidad Académica de Matemáticas de la UAGro,

México,


 nasacamx@yahoo.com.mx; celrizo@yahoo.com.mx 

\end{datos}

Lenguaje matemático, Básico, Estudio de caso En este reporte interesa
mostrar aspectos relacionados con el análisis critico de los libros
de texto de matemáticas de la ultima reforma en México, en particular
de la educación primaria sobre la presentación del sistema numérico
decimal. Para lo cual nos apoyaremos de un marco teórico metodológico,
basado en el análisis de contenidos, el cual nos permite identificar
el objeto o tema de análisis a desarrollar, establecer y definir tanto
las unidades de análisis como las categorías, mismas que nos permitirán
más tarde mostrar esos aspectos ocultos entre líneas de los libros
de texto. 


\section{CONOCIMIENTO MATEMÁTICO DE UN PROFESOR DE SECUNDARIA Y SU ACTUACIÓN
DOCENTE EN EL TÓPICO DE FRACCIONES}

\begin{datos}

Ivón García Díaz$^{1}$, Guadalupe Cabañas-Sánchez$^{1}$; Leticia
Sosa$^{2,3}$; C. Miguel Ribeiro$^{4}$

$^{1}$Universidad Autónoma de Guerrero , $^{2}$Universidad Autónoma
de Zacatecas ,

$^{3}$Centro de Investigación sobre el Espacio y las Organizaciones
- Universidad de Algarve,

$^{4}$UNESP - Rio Claro,

$^{1,2}$México, $^{3}$Portugal, $^{4}$Brasil,

(Brasil) bony\_999@hotmail.com; gcabanas@uagro.mx; 

lsosa19@hotmail.com; cmribeiro@ualg.pt

\end{datos}

Se reportan avances de una investigación en desarrollo que estudia
el ”Conocimiento Matemático para la Enseñanza” que revelan profesores
de matemáticas de secundaria acerca del concepto de fracción, particularmente,
del Conocimiento del Contenido y Enseñanza. El análisis considera
la planeación de los profesores sobre el tópico fracciones y su desarrollo
en condiciones de enseñanza. Los resultados evidencian que el profesor
sabe qué convenciones matemática utilizar para que los alumnos representen
de forma adecuada, fracciones en la recta numérica, y; qué tipo de
situaciones plantear para que sean capaces de resolver problemas relativos
a la multiplicación y división de fracciones, mediante algoritmos
usuales.


\section{UN ACERCAMIENTO FIGURAL A LA GRÁFICA, EL CASO DE ESTUDIANTES DE BACHILLERATO}

\begin{datos}

Beatriz Alejandra Veloz Díaz, Claudia Margarita Acuña Soto.

CINVESTAV,

México,

betty.vdiaz@gmail.com; claudiamargarita\_as@hotmail.com 

\end{datos}

En este trabajo se desarrolla un interés particular por las actividades
de graficación en la enseñanza del nivel medio superior, específicamente
observamos la manera cómo los estudiantes trabajan con las gráficas,
las interpretan y cómo son usadas como herramientas con fines de aprendizaje
de la geometría analítica y el cálculo. Los conceptos involucrados
son; función, función creciente, función decreciente, puntos máximos
y puntos mínimos tanto absolutos como locales, recta tangente, pendiente,
concavidad y punto de inflexión.


\section{LOS MÉTODOS DE DEMOSTRACIÓN Y LAS POSTURAS SOBRE SEMIÓTICA PARA LA
INTERPRETACIÓN Y SOLUCIÓN DE PROBLEMAS DE MATEMÁTICAS}

\begin{datos}

José de Jesús Gutiérrez Ramírez, Gloria Baca Lobera.

Universidad Autónoma Metropolitana - Xochimilco,

México,

jgramirez@correo.xoc.uam.mx, gbaca52@hotmail.com

\end{datos}

El presente trabajo contiene un conjunto de comentarios de una aplicación
de las sugerencias metodológicas que hace el Daniel Solow en su libro
Como entender y hacer demostraciones en matemáticas, publicado en
1981, acerca de un libro de texto, de próxima aparición de la Universidad
Autónoma Metropolitana, sobre análisis matemático. El desarrollo del
libro, cuyos autores suscribimos este trabajo, está basado en las
sugerencias de Daniel Solow y enmarcado en otra clasificación paralela
sugerida por Bruno D´Amore y Juan D. Godino en su artículo El enfoque
ontosemiótico como un desarrollo de la teoría antropológica en didáctica
de la matemática (2007). 


\section{CALIDAD MATEMÁTICA DE TAREAS}

\begin{datos}

Marta Adán, Vicenç Font.

Universitat de Barcelona,

España,

marta.adan@hotmail.com; vfont@ub.edu

\end{datos}

Investigamos el diseño, implementación y rediseño de tareas que permitan
la emergencia de herramientas teóricas para que los futuros profesores
realicen la valoración de la idoneidad matemática de procesos de instrucción,
uno de los componentes de la competencia en análisis didáctico de
procesos de instrucción. Para conseguirlo, utilizamos una metodología
de investigación que tienen elementos de la investigación basada en
el diseño. Los participantes son los futuros profesores de matemáticas
de secundaria del Máster de Formación del Profesorado de Secundaria
de Matemáticas de la Universitat de Barcelona, España. 


\section{EL FENÓMENO DE OPACIDAD Y LA SOCIALIZACIÓN DEL CONOCIMIENTO MATEMÁTICO}

\begin{datos}

Karla Gómez Osalde, Francisco Cordero Osorio.

Cinvestav-IPN,

México,

kmgomez@cinvestav.mx; fcordero@cinvestav.mx

\end{datos}

Dos nociones fundamentales justifican este trabajo. 
\begin{description}
\item [{1)}] Identificamos la categoría de conocimiento del cotidiano como
elemento cardinal del Rediseño del dME para plantearnos una caracterización
de la problemática fundamental ocasionada por el dME: el fenómeno
de opacidad de la vida cotidiana. 
\item [{2)}] Desarrollamos una visión socioepistemológica del proceso de
socialización del conocimiento matemático como una condición cotidiana
del humano. En la vida cotidiana nos encontramos en un proceso continuo
de socialización en el que la sociedad misma nos hace seres sociales
a través del conocimiento. Nos hacemos parte de la sociedad a partir
de la socialización de su conocimiento. 
\end{description}

\section{\uppercase{ una expresión algebraica sin letras}}

\begin{datos}

John Gómez Triana .

Secretaría de Educación Distrital,

Bogotá-Colombia,

johngomezt@gmail.com

\end{datos}

Se propone presentar un reporte de investigación que se enmarca en
la perspectiva semiótica cultural de la enseñanza y aprendizaje de
las matemáticas propuesta en la Teoría Cultural de la Objetivación.
El objetivo es reportar una evidencia de cómo el sentido de la indeterminancia,
la analiticidad y la expresión simbólica se hacen evidentes en la
actividad matemática de los estudiantes cuando resuelven tareas en
contextos algebraicos. Para tal fin se muestra un análisis multimodal
del pensamiento y de la actividad reconociendo la importancia de los
recursos cognitivos, físicos y perceptuales que los estudiantes utilizan
cuando trabajan con ideas matemáticas. 


\section{UNA APROXIMACIÓN AL PENSAMIENTO MULTIPLICATIVO DESDE LA PERSPECTIVA
SEMIÓTICO CULTURAL }

\begin{datos}

Anderson Javier Mojica Vargas.

Universidad Distrital Francisco José de Caldas,

Bogotá-Colombia,

javiermojicav@hotmail.com 

\end{datos}

Situados desde la Teoría Cultural de la Objetivación se presenta el
análisis realizado a un grupo de estudiantes colombianos de grado
sexto de educación básica cuando resuelven tareas de tipo multiplicativo,
en el cual se estudiaron los modos de significación y despliegue de
recursos semióticos dentro de un proceso de objetivación que da cuenta
de aspectos semióticos tanto corporales como lingüísticos que posibilitan
una mirada alternativa a los procesos de enseñanza aprendizaje de
la multiplicación escolar y que a su vez sugiere una aproximación
semiótica a las formas prototípicas de pensar multiplicativamente.


\section{\uppercase{ Coordinación de registros en la solución de problemas
de transformación lineal.}}

\begin{datos}

Osiel Ramírez Sandoval, Natividad Nieto Saldaña.

Universidad Autónoma de Ciudad Juárez, 

México,

osiel73150@gmail.com; nnieto@uacj.mx

\end{datos}

Se presentan los resultados de una investigación, realizada con estudiantes
de Matemáticas, basados en el análisis de una entrevista que incluye
diversas situaciones de Transformaciones Lineales. Utilizando la teoría
de registros de representación semiótica se analiza la coordinación
de registros por parte de los estudiantes y su relación con el éxito
y eficiencia al resolver las situaciones planteadas. Se incluyen descripciones
de algunos casos exitosos de coordinación, como también donde no se
logró ésta. Un estudiante tiene la habilidad de coordinar registros
exitosamente al presentársele alguna situación matemática, busca y
está en mejores condiciones de encontrar estrategias eficientes para
resolverlas.


\section{INCORPORACIÓN DE TIC EN LA ENSEÑANZA DE LAS MATEMÁTICAS: UN DIÁLOGO
ENTRE INVESTIGACIÓN EDUCATIVA Y DOCENCIA }

\begin{datos}

Amaranta Martínez De La Rosa, Liliana Suárez Téllez, Blanca Ruiz Hernández.

Instituto Politécnico Nacional y Tecnológico de Monterrey,

MÉXICO,

amaranta.14@hotmail.com; lsuarez@ipn.mx; bruiz@itesm.mx 

\end{datos}

Nuestro objetivo es determinar cuál es la naturaleza del uso de los
resultados de investigación en la docencia, en cuanto a la incorporación
de las TIC. Elegimos cuatro aspectos a tomar en cuenta: 
\begin{description}
\item [{1)}] creencias, actitudes o ideologías pedagógicas, 
\item [{2)}] conocimiento del contenido en el área específica, 
\item [{3)}] conocimiento pedagógico en las prácticas de enseñanza, y
\item [{4)}] estrategias, métodos o diferentes enfoques para adaptar las
prácticas. Nuestro análisis, a manera de estudio de casos, se centra
la interacción entre docentes e investigadores en un seminario en
línea con el tema de las representaciones y la construcción de conceptos
matemáticos con TIC. 
\end{description}

\section{INTEGRANDO EL USO DE HABILIDADES ESPACIALES Y GEOMÉTRICAS PARA EL
APRENDIZAJE SIGNIFICATIVO DEL CONCEPTO DE VOLUMEN DE SÓLIDOS CON ESTUDIANTES
DE DIBUJO TÉCNICO}

\begin{datos}

Edgar David Jaimes, Avenilde Romo Vázquez.

Centro de Investigación en Ciencia Aplicada y Tecnología Avanzada
del IPN,

México,

edjaimes@gmail.com; 

avenildita@gmail.com

\end{datos}

Pensamiento geométrico, Medio básico, Empírico/experimental

Resumen. El presente reporte plantea la necesidad de estudiar la problemática
que gira en torno al aprendizaje del concepto de volumen de cuerpos
sólidos desde el análisis del contexto del desarrollo del pensamiento
espacial de estudiantes que demuestran habilidades en cursos independientes
de la matemática escolar como es el dibujo técnico. Uno de los objetivos
es proponer una metodología de trabajo que permita potenciar dichas
habilidades en el aprendizaje de algunos conceptos geométricos, en
particular el concepto de volumen, a través de un proyecto de aula
transversal entre las áreas de dibujo y matemáticas con estudiantes
de secundaria en Colombia.


\section{RAZÓN DE CAMBIO SU ANÁLISIS Y CONSTRUCCIÓN A TRAVÉS DE LAS GRÁFICAS
GENERADAS POR EL MODELO DE LLENADO DE RECIPIENTES}

\begin{datos}

Carlos David Esqueda Gómez, Hipólito Hernández Pérez.

Universidad Autónoma de Chiapas - Red de Cimates,

México,

enconsur@hotmail.com, polito\_hernandez@hotmail.com

\end{datos}

Este trabajo está orienta en la problemática generada en el Cálculo
Diferencial, el concepto de Derivada se basa en el aprendizaje de
Cálculos algorítmicos y no en la razón de cambio. Se rediseñó una
secuencia didáctica para la representación gráfica del modelo de llenado
de recipientes de diferentes geometrías, fenómeno presente en la Ingeniería
en formación y en la práctica. Se pretende analizar, interpretar a
través de la representación gráfica del modelo de llenado de recipientes,
la construcción de la noción de razón de cambio que poseen los estudiantes
que cursan el primer semestre de la carrera de Ingeniería Civil.


\section{LA MODELIZACIÓN MATEMÁTICA Y EL DISEÑO EN INGENIERÍA}

\begin{datos}

Arnaldo Mendible$^{1}$, José Ortiz$^{2}$

$^{1}$Universidad Nacional Experimental Politécnica de la Fuerza
Armada,

$^{2}$Universidad de Carabobo,

Venezuela, 

arnmen2005@yahoo.com; ortizbuitrago@gmail.com

\end{datos}

La modelización matemática se desarrolla en contextos adecuados. El
objetivo es analizar las relaciones entre la modelización y el diseño
en ingeniería. La metodología requiere construir ambientes profesionales
de la ingeniería. Para el desarrollo de competencias de modelización,
se programan actividades para matemáticas en ingeniería. Se analizaron
las producciones de estudiantes de Ingeniería Mecánica de una Universidad
Venezolana, en un estudio de caso. Se concluyó que los estudiantes
consideraron al diseño como un fin, que depende de la representación
de ideas y de los diseños, con analogías con esas formas de representación
y se incorporaron conceptos y procedimientos al diseño.


\section{DESCOMPOSICIÓN GENÉTICA DEL TEOREMA DE ISOMORFISMO DE GRUPOS}

\begin{datos}

Arturo Mena Lorca$^{1}$, Astrid Morales Soto$^{1}$, Marcela Parraguez
González$^{1}$, Isabel Maturana$^{1}$ .

$^{1}$Pontificia Universidad Católica de Valparaíso,

Chile,

arturo.mena@ucv.cl; ammorale@ucv.cl;

marcela.parraguez@ucv.cl; isamatup@ucv.cl

\end{datos}

El teorema del isomorfismo de grupos, TIG, es un teorema importante
para la comprensión del álgebra abstracta, pero las investigaciones
señalan que los estudiantes generalmente no lo aprenden. Hacemos una
descomposición genética del TIG, separando el teorema en uno sin estructura
algebraica, TIC, y otro que agrega esa estructura, o TIG propiamente
tal. Damos evidencia de que se puede construir el TIG a partir del
TIC, y explicitamos nuestra descomposición genética refinada. 


\section{INTERPRETACIONES DE ESTUDIANTES CON RESPECTO A LA RELACIÓN DE IGUALDAD,
CUANDO SE USA LA VISUALIZACIÓN DE REPRESENTACIONES PICTÓRICAS }

\begin{datos}

Diana Carolina Sierra Caro.

Universidad Distrital Francisco José De Caldas,

Colombia,

diana\_sierra\_1987ud@hotmail.com

\end{datos}

En este trabajo se busca abordar una problemática escolar en torno
al significado de la relación de igualdad en el álgebra, para ello,
y tras identificar y comprender en qué consisten las problemáticas
con referencia a la interpretación de dicha relación, se hizo una
revisión de antecedentes en tesis de pregrado y especialización de
la Universidad Distrital Francisco José de Caldas y en algunas tesis
de pregrado de la Universidad Pedagógica Nacional de Bogotá, las cuales
permitieron establecer y analizar de qué forma se ha avanzado en la
superación de dichas problemáticas. 


\section{DISEÑO DIDÁCTICO DE UN TEMA DE ESTADÍSTICA Y SU MANEJO A DIVERSOS
NIVELES EDUCATIVOS, SEGÚN POLYA }

\begin{datos}

Guillermina Sánchez López, José Dionicio Zacarías Flores. Benemérita
Universidad Autónoma de Puebla, México. guisalop@hotmail.com , jzacarias@fcfm.buap.mx 

\end{datos}

La metodología implementada para la elaboración de esta propuesta,
se llevó a cabo en el desarrollo de las clases en las materias de
probabilidad y estadística a nivel bachillerato, probabilidad y estadística
en la ingeniería en software de la UPAM y en control estadístico de
procesos en 9\textdegree{} cuatrimestre de la UTP. En el desarrollo
de esta propuesta se toma como base el paradigma constructivista como
eje fundamental, así como las técnicas de conversación (entrevista
clínica, según Piaget) y conversación heurística a través de la participación
del alumno en actividades intencionales planificadas y sistemáticas
que produzcan la actividad mental deseada (Coll, 1998) .


\section{INFLUENCIA DE LA COORDINACIÓN DE TEORÍAS DE EDUCACIÓN MATEMÁTICA
EN EL APRENDIZAJE DE LAS MATEMÁTICAS}

\begin{datos}

Zenón Eulogio Morales Martínez.

Pontificia Universidad Católica del Perú – PUCP, Instituto de Investigación
en Enseñanza de las Matemáticas – IREM-PUCP, Institución Educativa
Particular Agroestudio,

Perú,

morales.ze@pucp.edu.pe

\end{datos}

Se analiza la influencia de la coordinación la Teoría del Aprendizaje
Significativo, Ausubel (1983), Teoría de la Transposición Didáctica,
Chevallard (1998) y la Teoría de las Representaciones Semióticas,
Duval (1995), centrándonos en el análisis de la influencia de la coordinación
y aporte de estas teorías en una clase de matemáticas. La pregunta
de investigación es: ¿Cómo influye el aporte de las principales Teorías
de la Educación Matemática en los procesos de enseñanza y aprendizaje
de la matemática, y la coordinación de estas teorías en la mejora
de estos procesos en la Educación Básica Regular?.


\section{ANALIZANDO TAREAS ESPACIALES EN EDUCACION INFANTIL}

\begin{datos}

Majorie Samuel, Yuly Vanegas{*}, Joaquin Giménez{*}.

Universidad Católica del Maule, {*}Universidad de Barcelona, 

Chile, España,

marjoriesamuel@gmail.com; ymvanegas@ub.edu; 

quimgimenez@ub.edu

\end{datos}

Formación de profesores. Educación Superior. Estudio de casos.

Se describe un estudio de caso en el que se analizan el posicionamiento
inicia de futuros docentes de educación infantil, cuando se enfrentan
al análisis reflexivo de experiencias escolares que desarrollan conocimientos
geométricos con niños de 5 años. Se realiza un análisis de los textos
producidos por los futuros profesores focalizando los procesos de
la actividad matemática que reconocen que se desarrollan en las experiencias
citadas. Se describen dificultades cuando tienen que usar el conocimiento
matemático específico para la enseñanza. Confirmamos que la competencia
docente \textquotedbl{}mirar profesionalmente\textquotedbl{} no es
innata y que se sitúan en niveles bajos de dicha competencia.


\section{LA IMPOSIBILIDAD DE RESOLVER ECUACIONES POLINOMICAS DE GRADO UTILIZANDO
METODOS ALGEBRAICOS}

\begin{datos}

Mercado Arrieta Alexander de Jesús, Paternina Marmolejo Edgar Romario,
Salgado Sánchez José Antonio.

Universidad de Sucre,

Colombia,

ajma\_boss92@hotmail.com; edgar\_uchiha28@hotmail.com;

joseantoss2011@gmail.com

\end{datos}

Los problemas como, la cuadratura del círculo, la duplicación del
cubo y la trisección del ángulo, fueron un verdadero problema para
los matemáticos de antigüedad, ya que surgen aproximadamente en el
siglo V a.c., a los que casi 2200 años después se dice que estos no
se pueden resolver utilizando regla y compás. Las razones a decir
que no se pueden resolver utilizando regla y compás, se muestra en
la teoría de Galois en la cual existen teoremas que afirman la imposibilidad
de resolver ecuaciones polinómicas de grado utilizando métodos algebraicos
los cuales muestran el porqué de estos problemas.


\section{TALLER PARA EL DISEÑO DE ACTIVIDADES DIDÁCTICAS EN MATEMÁTICAS: UN
MEDIO PARA CONOCER CONCEPCIONES Y CREENCIAS DE PROFESORES DE SECUNDARIA
MEXICANOS}

\begin{datos}

Silvia Elena Ibarra Olmos, Agustín Grijalva Monteverde, María A. Rodríguez
I. 

Universidad de Sonora,

México,

sibarra@gauss.mat.uson.mx; guty@gauss.mat.uson.mx; 

mariaa.rodriguezt@gmail.com

\end{datos}

Se presentan los resultados de un estudio exploratorio cuyo propósito
fue conocer las concepciones y creencias de profesores de matemáticas
de secundaria mexicanos sobre lo que son las actividades didácticas,
los elementos que la conforman y el papel que juegan en su quehacer
cotidiano. A partir de esta información se establecen algunas conjeturas
sobre cómo conciben estos profesores a la matemática y su enseñanza.
La información obtenida será utilizada posteriormente para el diseño
de proyectos de formación continua dirigido a maestros de ese nivel
educativo.

\setcounter{section}{193}


\section{ESTUDIO DEL USO DE LOS REGISTROS DE REPRESENTACIÓN EN TEXTOS PARA
EL APRENDIZAJE DE LAS DERIVADAS}

\begin{datos}

Juan Carlos Sandoval Peña.

Instituto de Investigación sobre la enseñanza de las matemáticas (IREM),

Perú,

sandovaljc007@gmail.com ; sandoval.j@pucp.edu.pe

\end{datos}

Son muchos aspectos que intervienen en el proceso E-A de las matemáticas,
entre ellos el texto didáctico que serán analizados con la teoría
de Registros de R. Duval (2009), el cual nos permitirá visualizar
los distintos registros utilizados, comparando la incidencia de uno
u otro registro, también podremos analizar a los autores cuando utilizan
tratamientos y conversiones. Para el desarrollar el análisis de la
investigación se utilizo como metodología las habilidades de lectura
a nivel superior de Argudín et al. (1994), que buscan una buena comprensión
del texto, desarrollo del pensamiento crítico y potenciar la lectura
en la educación superior.


\section{LA INVESTIGACIÓN EN LA FORMACIÓN INTERCULTURAL DE PROFESORES INDÍGENAS:
NUEVAS TRAYECTORIAS ACADÉMICAS EN EDUCACIÓN MATEMÁTICA}

\begin{datos}

Vanessa Sena Tomaz, Rafael Andrés Urrego Posada.

Universidad Federal de Minas Gerais,

Brasil,

vanessastomaz@gmail.com; raurregop@gmail.com 

\end{datos}

Este trabajo discute las tensiones que se producen en el desarrollo
de investigaciones realizadas por dos grupos de estudiantes-educadores
indígenas brasileños, del Curso de Formación Intercultural de Profesores
Indígenas de la Universidad Federal de Minas Gerais, en la línea de
Matemáticas. Estas investigaciones incorporan aspectos matemáticos
para fundamentar argumentos de análisis. Evidenciamos algunas diferencias
y similitudes entre la producción de investigaciones académicas convencionales
y las investigaciones de los estudiantes indígenas. La perspectiva
teórica Histórico-Cultural de la Actividad fue usada para poner en
evidencia las tensiones que se producen en el desarrollo de esas investigaciones. 


\section{DIDÁCTICA E INVESTIGACIÓN DEL PENSAMIENTO ALGEBRAICO}

\begin{datos}

Andrés González R. 

Universidad Pedagógica Experimental Libertador - Núcleo de Investigación
en Educación Matemática “Dr. Emilio Medina” (NIEM), 

Venezuela,

agorondell@yahoo.es 

\end{datos}

Tomando en cuenta los rasgos distintivos del pensamiento algebraico
se realizó este trabajo el cual es producto de la revisión del estado
del arte con respecto a su conceptualización. Se consideraron diversas
fuentes (impresas y electrónicas) nacionales y extranjeras. Se concretó
un documento el cual da cuenta de algunos aspectos que evidencian
la complejidad de este constructo; se tuvo el propósito de discutir,
dilucidar y mostrar sus rasgos relevantes desde las perspectivas histórica-epistemológica,
didáctica e investigativa; particularmente se muestra su vínculo con
otro concepto como el pensamiento relacional, a través del cual se
entrelazan colaborativamente la aritmética y el álgebra. 


\section{LOS CRITERIOS DE CONGRUENCIA Y SU USO EN DISTINTOS NIVELES ESCOLARES}

\begin{datos}

Noelia Londoño Millán, Silvia Morelos Escobar, Mirhiam Alejandra Iñigo
Berumen,

Universidad Autónoma de Coahuila - Facultad de Ciencias Físico Matemáticas,

México,

noelialondono@uadec.edu.mx; s.morelos@uadec.edu.mx;

mirhiamiñigob@hotmail.com 

\end{datos}

En este reporte de investigación se presentan los resultados parciales
de un estudio comparativo del uso de los criterios de congruencia
desde la perspectiva de alumnos de: secundaria, preparatoria, primeros
y últimos semestres de licenciatura y alumnos de maestría. La investigación
se enmarca dentro del modelo de razonamiento de Van Hiele, en lo que
respecta a la parte descriptiva y prescriptiva. El estudio permite
aseverar que los alumnos implicados en el proyecto no superan el nivel
3 de razonamiento y que tienen serias dificultades para identificar
y usar los criterios de congruencia de triángulos. 


\section{CONCEPCIONES DEL NÚMERO NATURAL ELABORADAS POR DOCENTES EN FORMACIÓN}

\begin{datos}

Patricia Lamadrid González, Marta Elena Valdemoros Álvarez.

Cinvestav – IPN, 

México, D. F.

lago.pattricia@gmail.com; mvaldemo@cinvestav.mx 

\end{datos}

En este documento expresamos de forma breve, el análisis de los instrumentos
de investigación: cuestionarios y diseño didáctico, a través de los
cuales pudimos identificar el concepto de número natural elaborado
por los docentes en formación y su relación con sus propuestas de
enseñanza y evaluación.





 
%\setcounter{section}{199}


\section{CONOCIMIENTO DE VARIABLE ALEATORIA DE PROFESORES DE MATEMÁTICAS DE
EDUCACIÓN SECUNDARIA}

\begin{datos}

Elika Sugey Maldonado Mejía, Javier Lezama Andalón.

CICATA-IPN,

México,

elika.mm@hotmail.com; jlezamaipn@gmail.com

\end{datos}

El estudio del conocimiento del profesor de matemáticas, en la actualidad,
se considera importante para mejorar la calidad de la educación. En
este sentido se plantea estudiar el conocimiento de variable aleatoria
de profesores de matemáticas de educación secundaria. Se espera con
esta investigación contribuir en la explicación de qué sabe el profesor
de matemáticas, cómo lo sabe, cómo construye este conocimiento, qué
factores intervienen en su quehacer, para tener elementos que permitan
modificar o proponer programas dirigidos a la formación del profesor
de matemáticas, en el área de estadística, del nivel básico en México.


\section{MODELACIÓN EN MATEMÁTICA ESCOLAR: EXPERIENCIAS CON ESTUDIANTES DE
INGENIERÍA EN CÁLCULO DIFERENCIAL, INTEGRAL Y ECUACIONES DIFERENCIALES}

\begin{datos}

Francisco Javier Córdoba Gómez, Pablo Felipe Ardila Rojo.

Instituto Tecnológico Metropolitano,

Colombia,

fjcordob@yahoo.es; franciscocordoba@itm.edu.co;

pabloardila@itm.edu.co 

\end{datos}

Una de las características de los cursos de matemáticas, en general,
en ingeniería es la escasa vinculación con actividades de modelación
que logren articular los contenidos matemáticos con situaciones o
fenómenos reales y cercanos a la cotidianidad de los estudiantes.
El siguiente trabajo presenta los resultados de la puesta en escena
de tres prácticas de modelación con estudiantes de ingeniería. Mediante
un trabajo experimental y práctico en el que los estudiantes participaron
de manera activa, se pone en evidencia que la modelación como práctica
favorece la interacción en el aula y la resignificación de conocimiento
matemático escolar en contextos específicos.


\section{DOMINIO DEL CONTENIDO MATEMÁTICO EN EL CONOCIMIENTO DIDÁCTICO DEL
CONTENIDO DE ESTUDIANTES PARA EDUCADORES INTEGRALES }

\begin{datos}

Ángel Vílchez Báez. 

Universidad del Zulia ,

Venezuela,

angelvilchez1501@gmail.com

\end{datos}

Uno de los elementos que constituyen el Conocimiento Didáctico del
Contenido es el Dominio del Contenido por Enseñar. Se buscó develar
el Domino de los Contenidos Matemáticos en Docentes en formación,
mención integral la Universidad del Zulia. Se utilizó la metodología
cualitativa, bajo el enfoque etnográfico y se contó con 62 informantes,
los cuales fueron organizados en equipos, cada uno de estos desarrolló
una clase de un tema de matemática. Se aplicaron registros de observación.
Los informantes mostraron dinámicas, creatividad y diferentes materiales
y recursos durante clases, se evidenció bajo dominio del los contenidos
de matemática, requieren ayuda.


\section{ELEMENTOS PARA LAS CONSTRUCCIONES MENTALES DEL FRACTAL TRIÁNGULO
DE SIERPISNKI}

\begin{datos}

Ximena Gutiérrez Figueroa, Marcela Parraguez González.

Pontificia Universidad Católica de Valparaíso,

Chile,

ximegf@gmail.com; marcela.parraguez@ucv.cl

\end{datos}

Sustentados en un estudio histórico-epistemológico y la Teoría APOE
se diseña una Descomposición Genética (DG) para describir las construcciones
y mecanismos mentales que muestran estudiantes de educación media
en relación a un fractal específico, llamado Triángulo de Sierpinski.
Los objetos fractales responden a la necesidad de las investigadoras
de indagar en un tema no contemplado en el currículum del plan común
de Matemática. Estos objetos permiten describir variadas formas presentes
en la naturaleza, que poseen la característica de ser autosemejantes
y se relacionan con conceptos de naturaleza compleja para su aprendizaje,
por los niveles de abstracción necesarios para su construcción.


\section{FORMAÇÃO CONTINUADA: CONTRIBUIÇÕES PARA A PRÁTICA DO PROFESSOR DE
MATEMÁTICA DA EDUCAÇÃO BÁSICA }

\begin{datos}

Fátima Aparecida da Silva Dias, Nielce Meneguelo Lobo da Costa.

Universidade Bandeirante de São Paulo,

Brasil,

fnnidias@gmail.com; nielce.lobo@gmail.com 

\end{datos} 

Este artigo tem o propósito de apresentar os primeiros resultados
da pesquisa de doutorado em andamento que teve por objetivo investigar
as contribuições de um programa de formação continuada para integrar
as tecnologias ao currículo de modo a subsidiar a prática docente
do professor de matemática.. A metodologia qualitativa se desenvolve
a partir dos critérios propostos por Moraes e Valente para se pesquisar
pela teoria da complexidade de Morin e tem como referencial teórico
a Formação Continuada do Professor de Matemática, Tecnologias Digitais
de Informação e Comunicação na Educação Matemática. A análise qualitativa
dos Instrumentos Complementares e Qualitativa Estatística.


\section{RELACIONES PROPORCIONALES ENTRE SEGMENTOS EN EL CONTEXTO DEL MODELO
DE VAN HIELE}

\begin{datos}

TanithCeleny Ibarra Muñoz, Edison Sucerquia Vega, Carlos Mario Jaramillo
López. 

Universidad de Antioquia,

Colombia,

tanith927@yahoo.es; esucerquia@gmail.com;

camaja59@gmail.com

\end{datos}

Pretendemos socializar los resultados del trabajo de investigación
donde se describen los razonamientos de estudiantes del grado quinto
de la Institución Educativa Antonio Roldán Betancur, con respecto
a las relaciones proporcionales entre segmentos. Para el análisis
cualitativo de la información se diseñó una entrevista de carácter
socrático en correspondencia con el modelo de van Hiele; la entrevista
implementó como componente visual geométrica las relaciones proporcionales
que se establecen entre segmentos, para su aplicación se construyeron
previamente unos descriptores, los cuales fueron refinados durante
la investigación de modo que permitieran caracterizar y determinar
el nivel de razonamiento que exhibían los estudiantes.


\section{UNA CATEGORÍA DE COTIDIANO EN UNA SITUACIÓN DE DIVULGACIÓN}

\begin{datos}

Andrés Ruiz Esparza Pérez, Francisco Cordero Osorio.

CINVESTAV, 

México,

andresruizep@gmail.com; fcordero@cinvestav.mx 

\end{datos}

Modelación matemática; medio superior; etnográfico / interpretativo

La modelación es reconocida como la estrategia por excelencia del
ser humano para generar conocimiento. Ante este status, nuestro estudio
toma como base una categoría de modelación escolar, y pretendemos
ampliar esta categoría al adaptar y reproducir estos diseños en un
escenario de divulgación. Planteamos como hipótesis que la incorporación
de la categoría de cotidiano, entendida como mantenimiento de rutinas,
en una situación específica provoca una situación de divulgación del
conocimiento matemático, y para dicha incorporación es necesario un
diálogo con la disciplina que se pretende divulgar.


\section{ESTRATEGIA DIDACTICA PARA GENERAR APRENDIZAJE SIGNIFICATIVO DE LAS
OPERACIONES BASICAS, SUMA, SUSTRACCION, MULTIPLICACION Y DIVISION
DE NUMEROS NATURALES EN LOS ESTUDIANTES DE 6\textdegree{} GRADO DE
LA INSTITUCION EDUCATIVA SAN VICENTE DE PAUL SINCELEJO }

\begin{datos}

EMILIANO RODRIGUEZ ARANGO, KAROL RODRIGUEZ CARDENAS, DANIEL MONTES
PEREIRA.

UNIVERSIDAD DE SUCRE,

COLOMBIA,

erodrigueza\_10@hoymail.com; karolrodriguezc@hotmail.com;

danielj-montesp@hotmail.com

\end{datos}

Con este proyecto se pretende comprobar la eficacia del uso de guías
y talleres con situaciones problemas contextualizadas para generar
en los estudiantes aprendizaje significativo de las operaciones básicas,
suma, sustracción, multiplicación y división, Aplicando una prueba
diagnóstica para conocer el estado conceptual e inicial de los estudiantes
de sexto grado, referente a las operaciones de adición, sustracción,
multiplicación y división de números naturales. Diseñando situaciones
problemas adecuadas para la elaboración de las guías y talleres que
se van a desarrollar. Verificando la incidencia de una lectura comprensiva
para la solución del problema. Debido a que se arrojan resultados
desfavorables.


\section{EL VALOR SOCIAL DEL CONOCIMIENTO MATEMÁTICO: UN ESTUDIO SOCIOEPISTEMOLÓGICO
DE LOS JUEGOS INFANTILES DEL PUEBLO MAPUCHE.}

\begin{datos}

Karla Sepúlveda Obreque, Anggie Hidalgo Silva, Carlos Lepicheo Velásquez,
Katheriine Mercado Avello.

Universidad Católica de Temuco, 

Chile,

ksepulveda@uct.cl; ahidalgo2010@alu.uct.cl; 

clepicheov2010@alu.uct.cl; kmercado2010@alu.uct.cl 

\end{datos}

El estudio busca aportar a una propuesta socioepistemológica de enseñanza
de las matemáticas para implementar en escuelas de la región de la
Araucanía que considere cultura y las prácticas sociales como generadoras
de conocimiento. Durante 18 meses indagamos en comunidades de la región
en busca de juegos infantiles tradicionales del lof, con el fin estudiar
las matemáticas generadas desde ellos. Recopilamos 12 juegos en los
que encontramos conocimiento referido a números, probabilidades, medición
del tiempo, geometría y desarrollo de habilidades como resolución
de problemas, representar, evaluar estrategias, argumentar y comunicar.
Hemos entendido que el pueblo mapuche sabe matemática, pero no sabe
que sabe. 


\section{PROCESOS DE INTERACCIÓN EN AMBIENTES DE APRENDIZAJE ONLINE PARA LA
FORMACIÓN POSGRADUADA DE PROFESORES DE MATEMÁTICAS}

\begin{datos}

Edison Sucerquia Vega, René Alejandro Londoño Cano, Carlos Mario Jaramillo
López.

Universidad de Antioquia,

Colombia,

esucerquia@gmail.com; rene2@une.net.co;

camaja59@gmail.com 

\end{datos}

Se pretende socializar los avances del proyecto de investigación “La
formación posgraduada de profesores de matemáticas, en un ambiente
de aprendizaje online”, en el cual se indaga por los elementos teóricos
y metodológicos que se deben tener en cuenta para desarrollar estos
procesos de formación posgraduada. El proyecto se desarrolla con maestros
en formación colombianos, estudiantes del programa de Maestría en
Enseñanza de las Matemáticas, que se desarrolla en la Universidad
de Antioquia, en ambientes online. Reconocer las diferentes interacciones
que pueden ocurrir en estos ambientes, ha permitido diseñar actividades
para el aprendizaje de las matemáticas, mediante alternativas metodológicas.


\section{UNA PROPUESTA DIDÁCTICA PARA LA ENSEÑANZA DEL DETERMINANTE}

\begin{datos}

Larissa Sbitneva, Nehemías Moreno Martínez.

Universidad Autónoma del Estado de Morelos, CINVESTAV-IPN,

México,

larissa@uaem.mx; nehemias\_moreno@live.com 

\end{datos}

Se presenta una propuesta didáctica que podría permitir a los estudiantes
la comprensión del concepto de Determinante, y su naturaleza abstracta,
del Algebra en la modalidad educativa hibrida, presencial/línea. Consideramos
algunos elementos teóricos del Enfoque Ontosemiótico para caracterizar
la actividad matemática de los estudiantes cuando resuelven problemas
de complejidad creciente, en términos de prácticas, configuraciones,
epistémicas y cognitivas, de objetos primarios y de procesos activados
en dichas prácticas. Se muestra la viabilidad de la propuesta mediante
un estudio de caso donde se compara la configuración cognitiva, obtenida
de la respuesta del estudiante, con la configuración epistémica de
la solución experta. 


\section{SIGNIFICADOS ASOCIADOS A PROCESOS DE VARIACION Y CAMBIO}

\begin{datos}

Martínez-Ortega Minerva, Mejía-Velasco Hugo R, Santillán-Vázquez Marco
A.$^{1}$

Cinvestav-IPN, $^{1}$UNAM,

México,

cetis76@hotmail.com; hmejia@cinvestav.mx; 

santillanmarco11@gmail.com 

\end{datos}

La variación es una de las ideas centrales alrededor de las que se
articula el Cálculo. Es posible construir conceptos fundamentales
a partir de fenómenos que involucren la variación, como lo es el estudio
de gráficas obtenidas por un sensor de movimiento. Una gráfica no
tiene una interpretación trivial, no calca icónicamente los fenómenos
modelados; la semiótica ayuda a interpretarlas. Al no conocer el proceso
generador de una gráfica, ésta puede llegar a carecer de sentido;
sin embargo, si una persona es la propia generadora de la gráfica,
es posible comprender el significado de conceptos como posición, velocidad,
derivada, etc. 


\section{CREENCIAS DE ESTUDIANTES DE SECUNDARIA SOBRE LAS MATEMÁTICAS: UN
DIAGNÓSTICO PREOCUPANTE Y SUS POSIBLES IMPLICACIONES }

\begin{datos}

Francisco Javier Córdoba Gómez.

Instituto Tecnológico Metropolitano,

Colombia,

fjcordob@yahoo.es; franciscocordoba@itm.edu.co 

\end{datos}

Este es un avance de investigación sobre las creencias que tienen
los estudiantes acerca de las matemáticas en los grados noveno, décimo
y undécimo en instituciones oficiales. Se presentan los resultados
parciales obtenidos con una muestra de 950 estudiantes mediante la
aplicación de un cuestionario, mejorado, adaptado y contextualizado,
que ya se había validado y aplicado en diferentes países europeos,
el Mathematics-Related Beliefs Questionnaire (MRBQ). Los resultados
muestran que las creencias de los estudiantes sobre sí mismos como
aprendices de matemáticas son muy negativas, aunque las consideran
muy importantes y necesarias. Estas creencias influyen en su desempeño
y elección profesional.


\section{INECUACIONES: UN CAMINO HACIA SU DESCOMPOSICIÓN GENÉTICA}

\begin{datos}

Ana María Narvaez, Clarisa Noemí Berman, Marcela Rodriguez.

Universidad Tecnológica Nacional - Facultad Regional Mendoza, Universidad
Nacional de Cuyo - Facultad de Ingeniería,

Mendoza - Argentina,

Ana.narvaez@frm.utn.edu.ar; bercla @gmail.com; 

iqborbollon@speedy.com.ar

\end{datos}

Las investigaciones realizadas sobre Límite Funcional desde la teoría
APOE, nos permitieron ver la necesidad de analizar la enseñanza de
prerrequisitos, como es el caso de las inecuaciones. El objetivo del
trabajo es elaborar un conjunto de actividades basadas en una metodología
que sustente una mejor enseñanza y aprendizaje, realizando una descomposición
genética preliminar. Entre las conclusiones obtenidas, se destaca
la necesidad, para adquirir la concepción esquema de inecuación, de
favorecer la resolución de actividades sobre la interpretación de
inecuación, de su resolución algebraica y gráfica. Es importante concientizar
a la comunidad docente sobre la importancia del tratamiento del tópico. 


\section{\uppercase{ ''Escenarios Didáctico'': el Caso de la Profesora Isabel} }

\begin{datos}

Eliza Minnelli Olguín Trejo, Marta Elena Valdemoros Álvarez.

Cinvestav-IPN,

México, D.F.

minnelli\_angel@yahoo.com.mx; mvaldemo@cinvestav.mx 

\end{datos}

Este trabajo da cuenta de los resultados obtenidos en el diseño de
una intervención para la enseñanza-aprendizaje del reparto con fracciones,
en primaria, a través de “Escenarios Didácticos”. Se identificaron
las dificultades que tienen los profesores al abordar temas relacionados
con las fracciones; de los datos obtenidos se tomaron criterios para
el diseño de los “Escenarios Didácticos”. Por último, la profesora
Isabel diseñó un “Escenario Didáctico”. Se observó que los problemas
cognitivos que tienen los niños son similares a los que presentan
los docentes, por ello, la importancia de identificar dichas dificultades
y diseñar propuestas que ayuden a superarlas.


\section{METROLOGIA DE LA CULTURA ARHUACA DE LA SIERRA NEVADA DE SANTA MARTA }

\begin{datos}

Eduard Jhon Pérez Osorio, Emilse Rosa Calderón de Luquez, Ever Enrique
de la Hoz Molinares, Omar Enrique Trujillo Varilla,

Universidad Popular del Cesar,

Colombia,

eduardperez@unicesar.edu.co; emisora\_1964@hotmail.com;

everdelahoz@unicesar.edu.co; omartrujillo@unicesar.edu.co

\end{datos}

La necesidad de medir es una actividad que el hombre desarrolla desde
la antigüedad, esto ha permitido construir patrones metrológicos,
que con el tiempo fueron evolucionando hasta crear las medidas estandarizadas
actuales. Pero existen comunidades como los Indígenas arhuacos, que
antes de la llegada de los españoles poseían sus propios sistemas
de medidas y que en la actualidad usan y conservan por su cosmovisión
y su cosmología. Esta investigación sobre la metrología utilizada
por los arhuacos en sus prácticas tradicionales permitirá conocer
los patrones de medidas de: (Longitud, Masa, Tiempo, capacidad), además
de establecer su significado, representación y uso actual. 


\section{LA GEOMETRIA EN LA ARQUITECTURA DE LA VIVIENDA TRADICIONAL ARHUACA}

\begin{datos}

Ever Enrique de la Hoz Molinares, José Rafael González Rocha, Luis
Guillermo Suarez Arias, Omar Enrique Trujillo Varilla.

Universidad Popular del Cesar,

Colombia,

everdelahoz@unicesar.edu.co; rocharg09@hotmail.com;

el1genio@hotmail.com; omartrujillo@unicesar.edu.co 

\end{datos}

La vivienda en los arhuacos es un lugar sagrado, está relacionada
con su parte espiritual y ancestral desde su cosmovisión, cosmología
y cosmogonía. La geometría no es enseñada en la escuela, esta es fundamental
en la construcción de sus viviendas por las relaciones, figuras y
teoremas usados en este proceso. La investigación busca establecer
como los saberes adquiridos en sus prácticas tradicionales son incorporados
en el diseño. La base de su vivienda es cuadrada por ser la figura
geométrica perfecta, consideran que la tierra está formado por una
pirámide de base cuadrada simétrica de 4 pisos hacia arriba y viceversa


\section{COSMOVISIÓN NUMÉRICA DE LA CULTURA ARHUACA DE LA SIERRA NEVADA DE
SANTA MARTA}

\begin{datos}

Eduard Jhon Pérez Osorio, Ever Enrique de la Hoz Molinares, Liliana
Patricia Barón Amarís, Omar Enrique Trujillo Varilla.

Universidad Popular del Cesar,

Colombia,

eduardperez@unicesar.edu.co; everdelahoz@unicesar.edu.co;

lilianabaron@unicesar.edu.co; omartrujillo@unicesar.edu.co

\end{datos}

El estudio del sistema de numeración de la cultura Arahuaca de la
Sierra Nevada de Santa Marta, permite mostrar los conceptos de: Orden,
número y la cosmovisión numérica, para realizar algunas operaciones
elementales y avanzadas, contar, el registro del paso del tiempo y
conocer la llegada de las estaciones, realizando aportes a la matemática
occidental. Analizamos el sistema de numeración en sus partes, composición
y organización a partir de la ley de origen. El uso actual del sistema
de numeración en las prácticas comunitarias, la forma de trasmitirlo
y desarrollarlo, además se presentan similitudes y diferencias con
el sistema decimal.


\section{CONFORMACIÓN DE ESTILOS DE APRENDIZAJE EN MATEMÁTICAS}

\begin{datos}

Jesús Evaristo Argüello Ramírez Expositor, Jesús Evaristo Argüello
Ramírez.

Instituto Superior de Ciencias de la Educación del Estado de México
(ISCEEM),

México,

arguello15@hotmail.com 

\end{datos}

La investigación realizada es cualitativa de corte etnográfico, realizada
en una escuela secundaria (Nivel medio básico), del Estado de México,
con estudiantes de los tres grados escolares, en la asignatura de
matemáticas; en ésta, defino una línea de investigación que se centra
en la manera en que el estudiante conforma sus estilos, dadas las
experiencias previas, así como de los contextos en que se desenvuelve;
está orientada en el estudio de las teorías y enfoques de aprendizaje,
con el objetivo de re-conocer e identificar las estrategias que utiliza
éste, para construir su conocimiento en colaboración con otros. 


\section{IDENTIDAD PROFESIONAL EN MATEMÁTICAS: ANÁLISIS DE SU CONFORMACIÓN
EN PROFESORES DE TELESECUNDARIA}

\begin{datos}

Erika García Torres, Ricardo Cantoral Uriza.

Cinvestav-IPN,

México,

egarciat@cinvestav.mx; rcantor@cinvestav.mx

\end{datos}

El estudio de la identidad profesional permite analizar la práctica
profesional desde el punto de vista del profesor. Se presenta una
narrativa autobiográfica de un profesor detelesecundaria en la que
emerge una multiplicidad de identidades. Negocia su identidad como
profesor general y como profesor de matemáticas, relacionando de manera
transversal los contenidos. Fomenta la construcción de significados
de diversas disciplinas mediante un mismo concepto tratado en una
secuencia didáctica.


\section{ACTITUD HACIA LAS MATEMÁTICAS: UN INDICADOR DE ÉXITO O FRACASO EN
EL CURSO DE RAZONAMIENTO LÓGICO}

\begin{datos}

Juan José Díaz Perera, Santa del Carmen Herrera Sánchez, Mario Saucedo
Fernández, Carlos Enrique Recio Urdaneta.

Universidad Autónoma del Carmen,

México,

jjdiaz@pampano.unacar.mx; sherrera@pampano.unacar.mx ;

msaucedo@pampano.unacar.mx; crecio@pampano.unacar.mx

\end{datos}

El propósito del trabajo es dar a conocer el nivel de relación entre
el rendimiento académico y la actitud hacia las matemáticas de los
estudiantes de nivel superior del curso de Razonamiento Lógico de
la Universidad Autónoma del Carmen. La muestra estuvo constituida
por 120 estudiantes. Se utilizó un instrumento para medir la actitud
hacia las matemáticas constituido por 19 ítems con una confiabilidad
de 0.9706. De acuerdo a los resultados se puede afirmar que existe
relación significativa entre el rendimiento académico y la actitud
hacia las matemáticas de los estudiantes, esto significa que los alumnos
que obtuvieron calificaciones altas tuvieron una actitud más positiva.


\section{COMPETENCIAS MATEMÁTICAS: UNA APLICACIÓN CON SENSORES EN UN AMBIENTE
COLABORATIVO}

\begin{datos}

López Betancourt Alicia, García Rodríguez Martha, Benítez Pérez Alma
Alicia.

Universidad Juárez del Estado de Durango (UJED), ESIME Zacatenco,
CECyT 11, Instituto Politécnico Nacional (IPN),

México,

ablopez@ujed.mx; martha.garcia@gmail.com;

abenitez@ipn.mx 

\end{datos}

La presente investigación aplicó los sensores de Vernier como una
herramienta para integrar los contenidos de enseñanza y permita desarrollar
competencias en los estudiantes. Se tomó el aprendizaje por proyectos.
Los estudiantes trabajaron de forma colaborativa precisaron el proyecto
a desarrollar durante un semestre. Los resultados muestran que la
aplicación de los sensores favoreció un ambiente de aprendizaje dinámico
y en el cual el protagonista fue el propio estudiante. Definieron
qué les interesaba investigar, cómo lo iban a medir con los sensores,
codificaron la información, tabularon, graficaron, encontraron relaciones
y ajuste de curvas. Aprendieron a resolver un problema de interés
al experimentarlo y relacionaron la teoría matemática. 


\section{MATEMÁTICAS ESCOLARES COMO IDEAS PODEROSAS (?). UN CONTEXTO CHILENO}

\begin{datos}

Melissa Andrade, Alex Montecino, Paola Valero.

Universidad de Aalborg,

Dinamarca,

melissa@learning.aau.dk; montecino@learning.aau.dk; paola@learning.aau.dk

\end{datos}

La matemática se ha convertido en un método de selección para la obtención
de educación superior en Chile. Quien ingresa a la Universidad, independiente
de lo que estudia, debe ser buen matemático. Invitándonos a reflexionar
sobre cómo se va dando forma a un discurso que presupone que las matemáticas
empoderarán a quien posea tal conocimiento. Una lectura política,
construida desde Foucault, nos permite ver que las matemáticas escolares
tienen un efecto de poder sobre los sujetos, en torno al saber matemático
y en la fabricación de subjetividades, transformándose en un método
de selección elitista y burocrático.


\section{O USO DA PLANILHA E CORREIO ELETRÔNICO COMO RECURSOS DIDÁTICOS NO
ENSINO DA MATEMÁTICA}

\begin{datos}

Marco Aurélio Meira Fonseca, Maria Deusa Ferreira da Silva.

Instituto Federal do Norte de Minas Gerais, Universidade Estadual
do Sudoeste da Bahia,

marco.fonseca@ifnmg.edu.br; mariadeusa@gmail.com

\end{datos}

Pesquisa-ação sobre o uso das novas tecnologias no ensino da matemática
com alunos do ensino médio

Neste artigo apresentamos um recorte da dissertação de mestrado realizada
junto ao mestrado Profmat - Polo UESB, em que relatamos a experiência
do primeiro dos autores fazendo uso das novas tecnologias, no caso
o correio eletrônico e a planilha, em suas atividades docentes, ao
tempo que traz à tona discussões sobre o impacto das novas tecnologias
na educação contemporânea, com a ampliação dos espaços escolares,
especialmente com o uso da internet. Nesse sentido, buscamos abordar
as incertezas dos professores em lidar com essa nova realidade e as
dificuldades encontradas em conciliar o uso das TIC com a construção
do conhecimento. 


\section{\uppercase{ Estilos de aprendizaje y rendimiento académico en alumnos
de Ingeniería}}

\begin{datos}

Mario A. Di Blasi Regner, Silvia Santos, Andrea M. Comerci .

Universidad Tecnológica Nacional - Facultad Regional General Pacheco,

Argentina,

mario.diblasi@gmail.com; silvia.santos@live.com.ar;

andreacomerci@yahoo.com.ar.

\end{datos}

Los Estilos de Aprendizaje (EA) son los rasgos cognitivos, afectivos
y fisiológicos, que sirven como indicadores relativamente estables,
de cómo los estudiantes perciben, interaccionan y responden a sus
ambientes de aprendizaje. El panorama de trabajos sobre rendimiento
académico y EA es muy amplio y después de analizar distintas investigaciones
se llega a la conclusión de que parece suficientemente probado que
los estudiantes aprenden con más efectividad cuando se les enseña
con sus EA predominantes. En este trabajo reportamos algunos resultados
del primer año de nuestra investigación sobre estilos de Aprendizaje,
estrategias de enseñanza y rendimiento académico.


\section{RASGOS IDENTITARIOS DE PROFESORES DE MATEMÁTICAS EN FORMACIÓN BAJO
LA MODELACIÓN COMO PRÁCTICA SOCIAL}

\begin{datos}

Miriam Carolina Ortiz Torrescano, Nancy Marquina Molina.

Universidad Autónoma de Guerrero, Unidad Académica de Matemáticas,

México,

miriam\_carolin@hotmail.com; nanmaquina@hotmail.com 

\end{datos}

El presente trabajo es parte de una investigación en proceso, donde
el objetivo principal es identificar qué rasgos identitarios concurren
con la Modelación bajo la perspectiva teórica de la Socioepistemología.
Se reporta un estudio acerca de la identidad de los futuros profesores
de matemáticas, los datos se obtuvieron en la Universidad Católica
Silva Henríquez de Santiago, Chile bajo la asignatura Actividad Matemática
con base a la modelación.

\setcounter{section}{225}


\section{QUESTIONANDO O ENSINO DE GEOMETRIA PLANA: DA DISCUSSÃO TEÓRICA PARA
A SALA DE AULA}

\begin{datos}

Alexandre Botelho Brito, Gabriel Botelho Brito, Kewla Dias Pires Brito,
Lizandra Almeida Araújo, Paula Adriana Matos Mourão.

Instituto Federal de Educação, Ciência e Tecnologia do Norte de Minas
Gerais (IFNMG),

Brasil,

alexandre.brito@ifnmg.edu.br; gabrielbotelhobrito@yahoo.com.br;

kewla.brito@ifnmg.edu.br; lizandra-almeida@hotmail.com;

paulaadrianamatos@hotmail.com 

\end{datos}

A utilização do software GeoGebra e de sequências didáticas, são alternativas
de trabalhar de forma diferenciada conteúdos matemáticos, como Geometria,
Cálculo e Álgebra. O uso destes recursos torna o aprendizado mais
simples, além de constituir uma metodologia dinâmica que contribui
para que o aluno produza o seu conhecimento de forma mais significativa.
Nesse sentido relatamos a construção e aplicação de 11 atividades
elaboradas de acordo com a engenharia didática de Michéle Artigue,
utilizando o software para o ensino-aprendizagem da Geometria. Estas
atividades foram aplicadas em uma turma inicial do curso de licenciatura
em matemática, onde podemos verificar a sua aplicabilidade.


\section{LA PRÁCTICA DEL PROFESOR DE MATEMÁTICAS. EL CASO DE LA SEMEJANZA}

\begin{datos}

Élgar Gualdrón.

Grupo de Investigación EDUMATEST-Universidad de Pamplona,

Colombia,

elgargualdron@yahoo.es 

\end{datos}

Se muestra los resultados de una investigación cualitativa que se
desarrolló desde la línea de investigación “formación de profesores”
del grupo EDUMATEST en el Departamento de Matemática de la Universidad
de Pamplona (Colombia), sobre la mejora de la práctica del profesor
en términos de su desarrollo profesional (estudio de caso único),
usando como pretexto la semejanza de figuras planas. Concretamente,
el objetivo fue caracterizar el proceso de enseñanza de la semejanza
de figuras planas en la Educación Secundaria. Algunos resultados sugieren
y confirman que la amplia experiencia profesional de un profesor no
necesariamente implica que su desempeño sea óptimo (Gualdrón, Giménez
y Gutiérrez, 2011).


\section{EVALUACIÓN SOBRE CONOCIMIENTOS DIDÁCTICOS E INVESTIGATIVOS DE FUTUROS
PROFESORES DE MATEMÁTICAS DE LA UNIVERSIDAD DE SUCRE}

\begin{datos}

Tulio R Amaya De Armas$^{1}$, Antonio Medina Rivilla$^{2}$,

$^{1}$Universidad de Sucre-Colombia, $^{2}$Universidad Nacional
de Educación a Distancia,

España,

tuama1@hotmail.com; amedina@edu.uned.es 

\end{datos}

Se reporta el análisis de la idoneidad didáctica de 90 estudiantes
del programa Licenciatura en Matemáticas de la Universidad de Sucre,
al analizar un cuestionario y las respuestas dadas a éste por estudiantes
de la básica, al hacer transformaciones de las representaciones semióticas
de una función. El estudio se fundamenta en dos teorías: la de los
registros semióticos de representación, y la del análisis de la idoneidad
didáctica de procesos de estudio de las matemáticas. Los resultados
evidencian serias dificultades con la identificaron de la función
como el contenido matemático estudiado y con la motivación de los
estudiantes al resolverlo. 


\section{ANALISIS DE LOS PROCESOS ATRIBUTIVOS QUE SE GENERAN EN ESTUDIANTES
DE SECUNDARIA: EL CASO DE CHILE}

\begin{datos}

María del Valle Leo, Mauricio Gamboa Inostroza, Francisco Pradenas.

Parra Universidad de Concepción, 

Chile,

mdelvall@udec.cl; maurigamboa@udec.cl;

francisco.pradenas@udec.cl

\end{datos}

A partir de la aplicación del Cuestionario “Sídney Attibution Scale”
(SAS), adaptado en Chile y la teoría de atribuciones, se trabaja con
12 cursos y 411 alumnos que pertenecen a la región de Concepción que
se caracterizan por su nivel de dependencia administrativa dentro
del sistema educativo. Este cuestionario permite indagar las causas
a las cuales los estudiantes atribuyen el logro en los procesos de
aprendizaje que se generan en clases de matemática, teniendo en consideración
las variables “éxito por esfuerzo”, “éxito por habilidad”, “éxito
por factores externos”, “fracaso por esfuerzo”, “fracaso por habilidad”
y “fracaso por factores externos”.

Palabras clave: atribuciones, logros, éxito, fracaso, habilidad.


\section{CONOCIMIENTO PROFESIONAL DE LOS PROFESORES DE MATEMATICAS ACERCA
DE LAS FRACCIONES EN EL DEPARTAMENTO DEL CESAR}

\begin{datos}

Lucía Martínez de Amaya, Álvaro de J. Solano Solano, {*}Diana Carolina
García Galindo.

Universidad Popular del Cesar, {*}Institución Educativa Leonidas Acuña,

Valledupar - Colombia,

luciamar1@yahoo.es; alsolano13@gmail.com;

dianacarolina24@gmail.com 

\end{datos}

En la investigación realizada, se propuso apreciar el nivel del conocimiento
profesional referido al concepto de fracción de los profesores de
matemática grado tercero a sexto de las instituciones educativas oficiales
del Departamento del Cesar para determinar su nivel de competitividad
y desempeño. La intención del trabajo fue proponer a las instituciones
formadoras de profesores, un programa de mejoramiento profesional
en cuanto calidad y sostenibilidad. Los resultados de la investigación
fueron analizados y contrastados según los planteamientos de Bromme,
como son Matemática como disciplina, matemática escolar, conocimiento
didáctico el contenido, filosofía de las matemáticas y pedagogía general.


\section{CONOCIMIENTO MATEMÁTICO CULTURAL EN GRUPOS INDÍGENAS DE CHILE Y COSTA
RICA: SU INCIDENCIA EN EL DOMINIO AFECTIVO COMO PERSPECTIVA DIDÁCTICA}

\begin{datos}

Luis Marcelo Casis Raposo$^{1}$, Ma. Elena Gavarrete Villaverde $^{2}$

$^{1}$U. Metropolitana Cs. Educación, $^{2}$U. Nacional,

$^{1}$Chile, $^{2}$Costa Rica,

marcelocasis@gmail.com; marielgavarrete@gmail.com

\end{datos}

Este documento muestra el avance del trabajo cooperativo colaborativo
que realizan los autores con comunidades indígenas de Chile y Costa
Rica. Se profundiza en aspectos vinculados a la caracterización del
Conocimiento Matemático Cultural vinculado a dos signos culturales
pertenecientes a dos grupos étnicos distintos, uno en Chile y otro
en Costa Rica: el Kultrún de los Mapuche y el Nopatkuö de los Talamanqueños.
Los autores incorporan sus respectivos enfoques didáctico-matemáticos
asociados a las etnomatemáticas y el dominio afectivo, con la finalidad
de ahondar en la perspectiva sociocultural de las matemáticas. Este
estudio pretende fomentar la utilización de elementos matemáticos
presentes en la cosmovisión de ambos grupos étnicos, con la finalidad
por una parte de contextualizar la construcción del conocimiento matemático
a partir del conocimiento ancestral, promoviendo con ello el desarrollo
de un positivo dominio afectivo en el proceso de enseñanza aprendizaje
de los estudiantes de dichas etnias y por otro promover la creatividad,
la valoración y fortalecimiento de la identidad cultural. 


\section{\uppercase{ MODELACIÓN LÚDICA En el Aprendizaje DEL CÁLCULO}}

\begin{datos}

Elvira Rincón, Lorenza Illanes.

Tecnológico de Monterrey Campus Monterrey - N.L. 

México,

elvira.rincon@itesm.mx; lillanes@itesm.mx

\end{datos}

La Modelación Lúdica dentro del estudio del Cálculo es un proceso
que requiere de irse perfeccionando como lo demuestra esta investigación
mixta a nivel superior. Se describen cinco secciones: a) Espacio Físico
de la experimentación; b) Descripción de la muestra; c) Antecedentes
del diseño experimental; d) El diseño e implementación de la actividad
y su evolución y e) Análisis cualitativo y cuantitativo. La investigación
prueba las ventajas de utilizar la Modelación Lúdica en un problema
de volúmenes de revolución y se establece una metodología aplicable
en otro tipo de problemas de Cálculo.


\section{RESIGNIFICACIÓN Y PRÁCTICAS SOCIALES}

\begin{datos}

Alberto Camacho Ríos.

Instituto Tecnológico de Chihuahua II, 

México,

camachoalberto@hotmail.com

\end{datos}

Las prácticas sociales son actividades de transformación del conocimiento
con cuyas técnicas y procedimientos se lo resignifica. Se estudia
en el escrito la práctica de la escritura de manuales para la enseñanza
de la matemática, desde la perspectiva de la resignificación en tanto
constructo de la socioepistemología. En esa actividad las normas de
la escritura se reconocían a través de un proceso de elementarización
que tenía como finalidad hacer fundamental el conocimiento. De la
elementarización surgía un elemento, el elemento es en si mismo una
resignificación del conocimiento que daba para establecer una primera
proposición o principio con el cual se iniciaba la escritura de la
obra. En el documento extenso se analizan diferentes obras elementales
bajo la perspectiva de la resignificación.


\section{ANÁLISIS DE OBJETOS Y PROCESOS MATEMÁTICOS EN UN ESTUDIO DE LA “PROPORCIONALIDAD”}

\begin{datos}

Francisco Javier Parra Bermúdez, Ramiro Ávila Godoy, Jesús Ávila Godoy.

Universidad Autónoma de Baja California, Universidad de Sonora, 

México,

fjparra@correo.física.uson.mx; ravilag@gauss.mat.uson.mx;

jag\_virgo@hotmail.com 

\end{datos}

En este reporte presentamos un ejercicio de análisis en la identificación
de objetos y procesos matemáticos intervinientes y emergentes en las
actividades de enseñanza diseñadas y desarrolladas por un profesor
para el estudio del objeto matemático proporcionalidad (OMP). Nuestros
análisis se realizan desde la perspectiva del enfoque ontosemiótico
de la cognición e instrucción matemática (EOS). La experiencia se
realizó con estudiantes de ingeniería y es un reporte parcial de un
proyecto de investigación sobre el papel de las (NTIC´s) en la construcción
de significados de los objetos matemáticos en el contexto de la mecánica.


\section{EL SEMILLERO DE INVESTIGACIÓN MATHEMA, UNA EXPERIENCIA UTÓPICA EN
LA FORMACIÓN DE DOCENTES DE MATEMÁTICAS.}

\begin{datos}

Carlos Eduardo Leon Salinas, Jefer Camilo Sáchica.

Castillo Universidad La Gran Colombia, 

Colombia,

Carlos.leon@ugc.edu.co; jefer.sachica@ugc.edu.co 

\end{datos}

En la Licenciatura en Matemáticas y Tecnologías de la Información
de la Universidad La Gran Colombia se conformó el semillero de Investigación
Mathema, el cual busca formar profesores con una metodología de trabajo
basado en el reconocimiento de la física como marco de referencia
para la construcción del conocimiento matemático. Esta propuesta busca
centrar el interés, no en el concepto, sino en las prácticas que se
involucran desde la experimentación en la construcción del conocimiento
matemático. Se escogió la física debido a la relación constituyen
que ha tenido con las matemáticas a lo largo de la historia.


\section{CONSTRUCCIÓN DE CONVERSIONES PARA LA PARÁBOLA CON SOPORTE EN LAS
SECUENCIAS DIDÁCTICAS Y APPLETS DE GEOMETRÍA DINÁMICA}

\begin{datos}

José Luis García Valdez.

Universidad de Guadalajara,

México,

gvjl@hotmail.com 

\end{datos}

La propuesta parte de la premisa de que la adquisición conceptual
de un objeto pasa a través de más de una representación semiótica,
situación que se propicia para que el estudiante se apropie del objeto
matemático “parábola”, con el diseño de secuencias didácticas que
incluyen actividades para aprendizaje, y se promueve el tránsito entre
representaciones gráficas, algebraicas y verbales. Se incluye la descripción
de las actividades a desarrollar, con hojas de trabajo a lápiz y papel
y con computadora, en forma individual y colaborativa, con applets
en Geogebra y con videos en línea, recursos ubicados en una página
web.


\section{LA MATEMÁTICA ESCOLAR EN EL CONOCIMIENTO PROFESIONAL DEL PROFESOR
DE MATEMÁTICAS}

\begin{datos}

Luis Cabrera, Ricardo Cantoral.

Centro de Investigación y de Estudios Avanzados del IPN,

México,

lmcabrera@cinvestav.mx; rcantor@cinvestav.mx

\end{datos}

Para determinar los dominios o categorías que integran al conocimiento
profesional del profesor de matemáticas, los esfuerzos se han enfocado
en comprender las tareas que el profesor debe desempeñar y los fenómenos
asociados a esto, y en función de esto determinar los conocimientos
requeridos. No obstante, al normar el discurso Matemático Escolar
a las tareas anteriores, dicho conocimiento ha estado supeditado a
la Matemática Escolar. En este sentido, resignificar la Matemática
Escolar se convierte en una fuente de conocimientos profesionales.
Este reconocimiento tendría una gran incidencia en los dominios y
categorías que lo integran, por ejemplo, exigirá repensar al currículo. 


\section{TECNOLOGÍAS EN EL APRENDIZAJE, EL APRENDIZAJE CONECTADO EN LA SOCIEDAD
DEL CONOCIMIENTO}

\begin{datos}

Carlos Enrique Recio Urdaneta, Juan José Díaz Perera, Mario Saucedo
Fernández, Sergio Jimenez Izquierdo.

Universidad Autónoma del Carmen,

México, reciocarlos714@gmail.com; jjdiaz@pampano.unacar.mx;

msaucedo@pampano.unacar.mx; sjimenez@pampano.unacar.mx

\end{datos}

La tecnología digital proporciona el potencial para nivelar el campo
de juego para el aprendizaje y multiplicar las oportunidades para
todos los jóvenes al acertar su lugar. El aprendizaje conectado es
un enfoque educativo que busca que el aprendizaje sea relevante. Se
fundamenta en el potencial de los medios digitales para ampliar el
acceso al aprendizaje que es socialmente empapado, impulsado hacia
intereses y oportunidades educativas, económicas y cívicas. Y no sólo
los jóvenes tengan acceso a una gran cantidad de información y conocimiento
en línea, sino que también pueden ser creadores y participes, comprometidos
en su propio aprendizaje autodirigido.


\section{INVESTIGAÇÃO SOBRE O ENSINO DE MATEMÁTICA EM ESCOLAS DE ENSINO TÉCNICO
NO BRASIL E EM PORTUGAL (1942 a 1978)}

\begin{datos}

Elmha Coelho Martins Moura.

Universidade Estadual Paulista “Júlio Mesquita Filho” – Rio Claro/SP,

Brasil. 

elmhac@yahoo.com.br 

\end{datos}

Esta pesquisa de doutoramento está sendo realizada no Programa de
Educação Matemática da Unesp de Rio Claro, SP/ Brasil, e tem por finalidade
analisar a trajetória do ensino da matemática no período de 1942 a
1978, em instituições de ensino técnico do Brasil e de Portugal. Pretendemos
averiguar, nesses diferentes contextos, aproximações e divergências
acerca da importância do conhecimento matemático, na formação de trabalhadores
industriários da época. É uma pesquisa no campo investigativo da História
da Educação Matemática, com referenciais na História Cultural e utiliza
fontes impressas, orais, imagéticas, museológicas e arquitetônicas,
para constituir uma possível história desse ensino.


\section{LA VARIABLE ALEATORIA DESDE LA PERSPECTIVA DE LA TEORÍA APOE }

\begin{datos}

Rodrigo Salazar Bórquez, Marcela Parraguez González.

Pontificia Universidad Católica de Valparaíso,

Chile,

salazarborquez@hotmail.com; marcela.parraguez@ucv.cl

\end{datos}

Este reporte de investigación presenta un estudio en torno a las características
y propiedades del concepto Variable aleatoria. El referente teórico
de este estudio es la teoría APOE (Acciones, Procesos, Objetos y Esquemas)
y el estudio de casos como diseño metodológico. Se construyeron instrumentos
que dan cuenta de la viabilidad del modelo diseñado, con el fin de
refinar la Descomposición Genética (DG) teórica y construir situaciones
que permitan a los estudiantes aprender y comprender el concepto de
variable aleatoria a partir de la DG. Evidenciamos problemas y obstáculos
en la construcción del concepto de variable aleatoria.


\section{FACTORES QUE CONTRIBUYEN A MEJORAR LOS RESULTADOS EN LA APROBACIÓN
EN LAS MATEMÁTICAS DEL BACHILLERATO TECNOLÓGICO DEBIDO A CURSO INTERSEMESTRAL }

\begin{datos}

Fausto Mendoza Díaz.

CECyT No 4 “Lázaro Cárdenas” IPN,

México,

mendizf@hotmail.com

\end{datos}

Se indagó sobre el notable incremento en los índices de aprobación
en las matemáticas del bachillerato tecnológico en el IPN, porcentajes
que del 0 \% llegan a alcanzar hasta el 70\%, lo cual se debe a cursos
de corto tiempo en las vacaciones de verano e invierno, es interesante
apreciar en los análisis de resultados de cuestionarios, entrevistas,
y las calificaciones de los exámenes, factores que provocan una corrección
en el déficit de atención y producen mejores resultados académicos,
esto señalado por autores como Johnson y Myklebust (1967) y Tarnopol
(1971), y como el pensamiento algebraico de los estudiantes mejora.


\section{LA MATEMÁTICA MODERNA Y SU INFLUENCIA EN LA EDUCACIÓN PRIMARIA Y
SECUNDARIA. PERÍODO (1960 – 1980) }

\begin{datos}

Mary Carmen Arrieche Aristiguieta, Mario José Arrieche Alvarado.

Universidad Pedagógica Libertador,

Maracay-Venezuela ,

maryarrieche@hotmail.com; marioarrieche@hotmail.com

\end{datos}

Esta investigación se centró en describir el fenómeno didáctico conocido
como “matemática moderna” y su influencia en los niveles de Educación
Primaria y Secundaria en el período comprendido de los años sesenta
a ochenta, así como en los currículos de formación de Profesores de
Educación Básica en Europa y en América. Para tal fin se combinó el
estudio documental y cualitativo de diversas fuentes relacionadas
con la historia de la matemática moderna e investigaciones curriculares
sobre la enseñanza de la matemática contenida en textos, artículos,
tesis doctorales, entre otros. Los resultados revelan que se enseñó
con sistemacidad y rigor.


\section{VIRTUALIZACIÓN DE LOS OBJETOS MATEMÁTICOS COMO APORTES AL CONOCIMIENTO
DIDÁCTICO DEL CONTENIDO EN PROFESORES DE EDUACACION SUPERIOR}

\begin{datos}

Félix Movilla, Hugo Parra.

Universidad Popular del Cesar, Universidad del Zulia

Colombia, Venezuela,

felixmovilla@unicesar.edu.co; hps1710@gmail.com 

\end{datos}

Este trabajo retoma los desarrollos teóricos del programa pensamiento
del profesor propuesto por Shulman articulados a los constructos derivados
de la vinculación de las TIC a procesos de enseñanza-aprendizaje.
Se muestra como la categoría del conocimiento didáctico del contenido
es complementada con la virtualización de los objetos matemáticos
con propósitos de enseñanza. Los resultados obedecen a un estudio
de naturaleza vivencialista hecho con profesores en ejercicio de cálculos
diferencial de programas de ingenierías. Las reflexiones con los profesores
permitieron la construcción de objetos virtuales para la derivada
de funciones teniendo como escenario de enseñanza el nivel de educación
superior.


\section{FACTORES ASOCIADOS A LA REPROBACIÓN DE ESTUDIANTES DE NIVEL LICENCIATURA.}

\begin{datos}

Mario Saucedo Fernández, Juan José Díaz Perera, Santa del Carmen Herrera
Sánchez, Carlos Enrique Recio Urdaneta.

Universidad Autónoma del Carmen - Ciudad del Carmen,

Campeche - México,

saferma2006@hotmail.com; jjdiaz@pampano.unacar.mx;

sherrera@pampano.unacar.mx; crecio@pampano.unacar.mx

\end{datos}

En este reporte se analizaron los factores asociados a la reprobación
en estudiantes universitarios. Se utilizó una muestra representativa
de 135 estudiantes de educación superior. Los resultados indican que
los factores personales que están relacionados con la reprobación
estudiantil es el género, la forma en que financian sus estudios;
trayectoria en la universidad: interés por la materia, no entender
las explicaciones del profesor; la actitud hacia el aprendizaje: la
cantidad de tiempo que se le dedica al estudio, el interés que se
tiene por el tema; los métodos de estudio que se tienen para enfrentar
su aprendizaje. 


\section{APLICACIONES EXTRAMATEMÁTICAS DE LOS POLINOMIOS: UNA APROXIMACIÓN
DE LA MATEMÁTICA A LA REALIDAD}

\begin{datos}

Dorenis Mota, Ricardo Valles.

Universidad Simón Bolívar-Sede Litoral,

Venezuela,

dorenismota@usb.ve; revalles@usb.ve

\end{datos}

En este artículo se presentan varios ejemplos de aplicaciones extramatemáticas
de los polinomios, las cuales se pretenden implementar en la enseñanza
de ese tópico matemático a nivel de 8vo grado de educación media básica
(estudiantes de 11 a 14 años) en una Institución pública venezolana.
El motivo de la implementación no convencional de esta enseñanza surge
como como posible alternativa de respuesta que busca confrontar al
bajo rendimiento que han presentado los estudiantes de la institución
mencionada en los dos últimos años escolares (2011-2012 y 2012-2013)
en el contenido sobre polinomios. 


\section{LAS DOS DISTANCIAS. UN ESTUDIO DIDÁCTICO SOBRE MEDICIÓN DE LONGITUDES}

\begin{datos}

Pedro Bollás García.

Universidad Pedagógica Nacional,

México D. F. 

pbollas@hotmail.com 

\end{datos}

En este documento presentamos el análisis de una situación didáctica
para la enseñanza de la medición de longitudes en alumnos de 4\textdegree{}
grado de primaria. En esta situación, diseñada en el marco de la Teoría
de las Situaciones Didácticas, los alumnos usan distintas unidades
para medir distancias, registran y discuten el resultado de la medición.
El análisis se centra en el tratamiento de la diferencia (cuando la
última unidad iterada no cabe exactamente en la longitud a medir)
y en los resultados escritos de la medición cuando se usan dos unidades
de medida, siendo una la mitad de la otra. 


\section{DESARROLLO DE LA HABILIDAD PAR AOPERAR CON FRACCIONES}

\begin{datos}

Crisólogo Dolores Flores, Francis Irene Aviles Valladares.

Universidad Autónoma de Guerrero,

México,

cdolores2@gmail.com; irene\_1787@hotmail.com

\end{datos}

En el presente documento se reporta un avance de tesis. Esta adopta
como objeto de estudio las operaciones básicas con fracciones y su
objetivo consiste en desarrollar la habilidad para realizar operaciones
básicas con fracciones por medio de la visualización. Las operaciones
con fracciones son tema de estudio desde la educación primaria, sin
embargo existen muchos reportes de investigadores que señalan las
grandes dificultades que los estudiantes presentan al trabajar con
dicho concepto. Una de las vías para posibilitar el aprendizaje de
estos temas, suponemos, puede ser a través del uso adecuado de los
procesos de visualización. La investigación se llevará a cabo a través
de la metodología investigación-acción propuesta por Kurt Lewin (1973).
Por otra parte se fundamenta en dos elementos conceptuales: la habilidad
y la visualización. Para lograr nuestro objetivo se diseñará una secuencia
de aprendizaje que permita desarrollar la habilidad mencionada en
condiciones concretas de enseñanza.


\section{ELEMENTOS DE DISEÑO PARA UNA CLASE DE MATEMÁTICAS A TRAVÉS DE LA
MODELACIÓN}

\begin{datos}

Ruth Rodríguez, Samantha Quiroz.

Tecnológico de Monterrey, CIMATE Tecnológico de Monterrey, 

México,

ruthrdz@itesm.mx; samanthaq.rivera@gmail.com

\end{datos}

El presente estudio pretende mostrar una propuesta para el diseño
de una clase de Ecuaciones Diferenciales en el contexto de Circuitos
Eléctricos basada en modelación matemática. Se reconoce a la modelación
matemática como el medio para el uso o construcción de modelos que
permitan la resolución de problemas en contextos cotidianos. A través
del uso de un esquema de dicho proceso se describen además de la manera
en que fue diseñada e implementada la clase, los resultados encontrados
así como las dificultades detectadas en la realización de las diversas
actividades propuestas.


\section{SOCIOEPISTEMOLOGÍA, MEDICINA Y MATEMÁTICAS. ELEMENTOS PARA EL ESTUDIO
DE P{*}}

\begin{datos}

Gloria Angélica Moreno Durazo, Ricardo A. Cantoral Uriza.

CINVESTAV – IPN,

México,

angelicadzo@hotmail.com; rcantor@cinvestav.mx

\end{datos}

Presentamos en este escrito, los avances de una investigación sobre
los principios que sustentan, “se encuentran detrás” de las actividades
propiamente matemáticas. Hemos identificado la intervención del que
denominamos principio estrella (P{*}), y lo ejemplificamos con el
problema de los tres cuerpos restringido donde nuestro objetivo, es
determinar el papel que juega dicho principio en un ambiente un tanto
distinto, el del trabajo médico especializado en la medicina interna. 


\section{MODELACIÓN DE ÚN MÓVIL SOBRE UNA TRAYECTORIA EN ESPIRAL, MODULANDO
LA AMPLITUD DE MODELOS SENOSOIDALES }

\begin{datos}

Francisco Jofré Vidal, Carolina Wa Kay Galarza, Jaime Arrieta Vera$^{1}$.

Universidad de Santiago de Chile, $^{1}$Universidad Autónoma de Guerrero,

Chile , $^{1}$México,

francisco.jofre @usach.cl; carolina.wakay@usach.cl;

jaime.arrieta@gmail.com 

\end{datos}

El trabajo se enfoca en la modelación del movimiento de un móvil sobre
una trayectoria espiral con velocidad constante por modelos senosoidales
modulados por exponenciales. El experimento consiste en girar una
masa atada a una cuerda, alrededor de un punto fijo, esto se filma
y a partir de los datos recogidos con Tracker, se ajustan gráficamente
con LDM. Interesa analizar argumentos, herramientas y procedimientos
de los estudiantes con la intención de caracterizar procedimientos
para la modulación de la amplitud de modelos senosoidales. La perspectiva
teórica en que sustentamos nuestro trabajo es la socioepistemología.


\section{CONCEPCIONES SOBRE LA ARGUMENTACIÓN COMO MEDIO DE VALIDACIÓN EN LA
CLASE DE GEOMETRÍA. UN ESTUDIO CON PROFESORES DE NIVEL MEDIO SUPERIOR}

\begin{datos}

María Victoria Ramos Abundio, Gema Rubí Moreno Alejandri, Efrén Marmolejo
Vega.

Universidad Autónoma de Guerrero - Unidad Académica de Matemáticas, 

México,

vick.ramath@gmail.com; alejandrigemath@gmail.com;

efrenmarmolejo@yahoo.com 

\end{datos}

Reportes de Investigación, Factores afectivos, Postgrado, Estudio
de casos

Este trabajo pretende mostrar avances de un proyecto de investigación
con respecto a concepciones sobre la argumentación como medio de validación
en la clase de geometría evidencian profesores de educación media
superior, teniendo como objetivo identificarlas y caracterizarlas.
Palabras Claves: Argumentación, validación, concepciones, profesores


\section{OBJETOS DE APRENDIZAGEM ELETRÔNICOS COM ANÁLISE COMBINATÓRIA}

\begin{datos}

Agostinho Iaqchan Ryokiti Homa, Claudia Lisete Oliveira Groenwald.

Universidade Luterana do Brasil, 

Brasil,

iaqchan@ulbra.br; claudiag@ulbra.br 

\end{datos}

Este trabalho apresenta Objetos de Aprendizagem (OA) desenvolvidos
para um e-learning com o conteúdo de análise Combinatória, para o
Ensino Médio. Os OA foram desenvolvidos, em Flash e programação actionscript,
sendo dois para o Princípio Fundamental da Contagem, um de revisão
da operação de Fatorial e um de Conjuntos Numéricos, dois para a Permutação
Simples, dois para Arranjo Simples, dois para Combinação Simples.
Os OA têm como recurso vídeos tutoriais para a resolução dos problemas
propostos nas atividades. O e-learning com os OA desenvolvidos foram
aplicados em duas disciplinas do curso de Licenciatura em Matemática
da Universidade Luterana do Brasil. Os resultados apontam que os estudantes
não apresentaram dificuldades na execução dos mesmos e preferiram
os vídeos tutoriais com áudio.


\section{SITUACIONES PROBLEMA EN LA UNIDAD DEPORTIVA DEL CUCEI DE LA UNIVERSIDAD
DE GUADALAJARA Y LA MODELACIÓN MATEMÁTICA CON TRACKER }

\begin{datos}

Rafael Pantoja Rangel, Sandra Minerva Valdivia Bautista. 

Centro Universitario de Ciencias Exactas e Ingenierías, Universidad
de Guadalajara,

México,

rafael.pantoja@red.cucei.udg.mx; minerva.valdivia@red.cucei.udg.mx

\end{datos}

En el curso de análisis numérico del departamento de Matemáticas del
CUCEI se ha incluido la modelación matemática de situaciones problema
de la vida cotidiana y para ello en la Unidad Deportiva del CUCEI,
los alumnos han desarrollado y grabado en video, actividades lúdicas
como el lanzamiento de balón, el recorrido en bicicleta o una carrera
de atletismo, entre otras, como un recurso para que el estudiante
a partir de situaciones cotidianas, obtenga datos en tiempo real con
el software Tracker, determine el polinomio que mejor se ajuste y
lo relacione con el problema en contexto.


\section{LEY DE ENFRIAMIENTO DE NEWTON: UNA PRACTICA DE MODELACION PARA EL
APRENDIZAJE SIGNIFICATIVO DE LAS ECUACIONES DIFERENCIALES}

\begin{datos}

Eduardo Tomás Torres, Jaime Arrieta, Carmelinda Benítez.

Instituto Tecnológico Superior de la Montaña,

México,

eduttt@gmail.com; jaime.arrieta@gmail.com;

karmelinda78@gmail.com

\end{datos}

El presente trabajo reporta un diseño de aprendizaje significativo
basado en la modelación del enfriamiento de un líquido, a través de
la ecuación diferencial lineal de primer orden (EDL) como modelo de
un fenómeno. Se parte de la toma de datos en la experimentación física,
se calcula la primera razón de cambio y se ajustan linealmente los
datos velocidad-temperatura con la ayuda del software LDM. La intención
de este trabajo es aportar evidencias de cómo los estudiantes construyen
una ecuación diferencial como modelo de un fenómeno aportando significado
e intención a los elementos matemáticos.


\section{MODELOS ESTIMADOS PARA EL TIEMPO DE PERMANENCIA DE ESTUDIANTES EN
ASIGNATURAS DE CÁLCULO EN LA UNIVERSIDAD FRANCISCO DE PAULA SANTANDER}

\begin{datos}

Mawency Vergel Ortega , José Joaquín Martínez Lozano, Jorge Luis Orjuela
Abril.

Universidad Francisco de Paula Santander,

Cúcuta - Colombia,

mawency@ufps.edu.co; jjmartiloz@hotmail.com;

jorgeorjuela@ufps.edu.co 

\end{datos}

La técnica de análisis de supervivencia permitió realizar estudio
con 120 estudiantes de diferentes programas de la Universidad Francisco
de Paula Santander, teniendo en cuenta eventos: abandono, sanción,
profesor. Resultados permitieron determinar que el manejo de preconceptos
en ciencias es factor que incrementa el riesgo, género masculino,
y sanciones presentan tendencia al abandono, edad mayor a 30 años
e inferior a 16 años tienen riesgo de retiro; cuatro determinantes
fueron identificados como razones de riesgo a través del modelo de
regresión de Cox: factores individuales, socioeconómicos, académicos
e institucionales y el miedo generado por el maestro.


\section{EQUIDAD DE GÉNERO Y MATEMÁTICA ESCOLAR: UN PROBLEMA QUE PODEMOS RESOLVER}

\begin{datos}

Claudia Rodríguez Muñoz.

Centro de investigación y de Estudios Avanzados del IPN,

México,

Claurom65@yahoo.com 

\end{datos}

Se trata de una investigación longitudinal sobre las estudiantes mujeres
de educación básica mexicana y la matemática escolar. Desde la Teoría
Feminista y la Teoría de las Representaciones Sociales se busca comprender
mejor la relación entre los procesos identitarios de género, la construcción
de representaciones de la matemática escolar y el logro académico.
Se emplea una metodología etnográfica interpretativa, fundamentada
en el análisis de la subjetividad de las participantes. Este reporte
muestra resultados del seguimiento realizado a las mismas estudiantes
durante tres ciclos escolares. Los datos que se aportan pueden ser
un beneficio en la formación del profesorado. 


\section{USO DE TECNOLOGÍA EN EL AULA DE MATEMÁTICAS: UN ESPACIO PARA REFLEXIONAR }

\begin{datos}

Bertha Ivonne Sánchez Luján, Javier Montoya Ponce.

Instituto Tecnológico de Cd. Jiménez, 

México,

ivonnesanchez10@yahoo.com; jmontoyaponce@yahoo.com.mx 

\end{datos}

Se presenta una investigación mediante el diseño de una aula prototipo
donde se utilizan herramientas tecnológicas y estrategias como la
proyección de videos para reforzar el conocimiento adquirido, creación
de un grupo en red social para intercambio de información y generación
de propuestas, ubicación estratégica de mobiliario y equipo, se creó
una app (Celmath) para Smartphone a través de la cual los estudiantes
pueden graficar, compartir archivos y obtener una imagen de los ejercicios
realizados en clase. El objetivo es apoyar a los estudiantes en el
proceso de aprendizaje y mejorar el rendimiento académico en Cálculo
Diferencial e Integral.


\section{LA FORMACIÓN DOCENTE EN LA ADQUISICIÓN DE LA COMPETENCIA MATEMÁTICA
Y EL PAPEL DE LA CALCULADORA}

\begin{datos}

Luis Callo, Armando Novoa, José Cuevas.

Universidad Peruana de Ciencias Aplicadas (UPC),

Perú,

luis.callo@upc.edu.pe; armando.novoa@upc.edu.pe;

jose.cuevas@upc.edu.pe

\end{datos}

El perfil de los docentes de Matemática de la UPC demanda alta preparación
en herramientas tecnológicas, gestión pedagógica, dominio de la Matemática.
El área de Matemática y Casio Académico Perú organizaron una capacitación
para el dominio de la ClassPad 330, sus potencialidades y el papel
que juega en la adquisición de la competencia matemática. Este trabajo
tiene como objetivos mostrar las experiencias y estrategias llevadas
a cabo; compartir los objetivos específicos, cronograma de trabajo,
actividades diseñadas y tres experiencias presentadas por los profesores,
para después hacer una valoración de los logros conseguidos por ellos
en la articulación de la herramienta.


\section{EXCLUSIÓN - INCLUSIÓN: DIALÉCTICA ENTRE EL DISCURSO MATEMÁTICO ESCOLAR
Y LA CONSTRUCCIÓN SOCIAL DEL CONOCIMIENTO MATEMÁTICO }

\begin{datos}

Daniela Soto.S, Ricardo cantoral.

U. Centro de Investigación y de Estudios Avanzados del IPN,

México,

dsoto@cinvestav.mx; rcantor@cinvestav.mx

\end{datos}

La investigación que presentamos pretende dar visibilidad, desde la
teoría Socioepistemológica, a la dialéctica exclusión – inclusión
en el campo de la Matemática Escolar. En particular, estudiaremos
las prácticas del profesor de matemáticas de bachillerato cuando transita,
en un ir y venir, entre la Construcción Social del Conocimiento Matemático
(CSCM) y el discurso Matemático Escolar (dME). Para ello utilizaremos
una perspectiva metodológica poco explorada en Matemática Educativa
conocida como Análisis Crítico del Discurso (ACD).


\section{CONCEPCIONES DE LA DEMOSTRACIÓN DE PROFESORES DE CÁLCULO}

\begin{datos}

Eric Hernández Sastoque.

Universidad del Magdalena, Universidad de Antioquia,

COLOMBIA,

ehernandezs@unimagdalena.edu.co 

\end{datos}

En la presente comunicación se describen algunas aproximaciones sobre
las concepciones de la demostración por profesores de cálculo. Estas
aproximaciones se enmarcan en una investigación doctoral que está
en curso y trata sobre la negociación de significados de la demostración
por profesores de cálculo. El estudio se encuadra en un paradigma
cualitativo y se gesta bajo el enfoque fenomenológico-hermenéutico.
El escenario de investigación es un programa de formación continua,
cuyos participantes son profesores de una Facultad de Ingeniería de
una universidad colombiana. 


\section{MATETÍTERES EN ACCIÓN: CLASIFICACIÓN DE CUADRILÁTEROS}

\begin{datos}

Adilene García Luna, Marcela Ferrari Escolá.

Unidad Académica de Matemáticas, Universidad Autónoma de Guerrero,

México,

adgalun@gmail.com; marcela\_fe@yahoo.com.mx 

\end{datos}

En este reporte deseamos mostrar los avances de nuestra investigación,
bajo la perspectiva socioepistemológica (Buendía 2013 y Cantoral 2014),
desde una primera mirada al análisis de diferentes presentaciones
en escenarios no escolares del grupo de teatro guiñol, “Matetíteres”,
con la obra titulada: “La aldea de los rombos” (marcela 2010) diseñada
como el disparador de la discusión sobre clasificación inclusiva,
aquella que conlleva abstraer con mayor sutileza los elementos que
distinguen la “definición” de una figura de sus propiedades. 


\section{UNA EPISTEMOLOGÍA DE LOS USOS DE LA OPTIMIZACIÓN}

\begin{datos}

Tamara Del Valle Contreras, Astrid Morales Soto, Francisco Cordero
Osorio.

PUCV ,CINVESTAV,

Chile, México,

tamaradc.mat@gmail.com; ammorales@ucv.cl;

fcordero@cinvestav.mx 

\end{datos}

El discurso matemático escolar (DME) de la optimización es un proceso
mecánico, donde existe una mayor centración en la aplicación de métodos
que en los usos de ésta, quedando desprovista de significaciones,
procedimientos y argumentaciones. Esta investigación es un trabajo
en curso, que tiene por objetivo elaborar un marco de referencia de
los usos de la optimización U(op), para valorar la justificación funcional
que demandan otros dominios de conocimiento, como el caso de la ingeniería,
el cual nos permita resignificar los U(op) en el DME, con el fin de
estrechar la distancia existente entre la matemática y el cotidiano.
..


\section{UNA CARACTERIZACIÓN DEL CONCEPTO DE FUNCIÓN DESDE LA TEORÍA DE ANNA
SFARD – EXPERIENCIA CON ESTUDIANTES DE GRADOS NOVENO Y ONCE DE COLEGIOS
DE LA VIRGINIA RISARALDA –}

\begin{datos}

Alicia Murillo Hurtado, Vivian Libeth Uzuriaga López.

Colegio Liceo Gabriela Mistral, Universidad Tecnológica de Pereira,

Colombia,

alismurh@yahoo.com; vuzuriaga@utp.edu.co 

\end{datos}

Mostar resultados obtenidos en una investigación cuyo propósito fue
identificar y caracterizar niveles de comprensión del concepto de
función en estudiantes de grado noveno y once de los colegios públicos
del Municipio de La Virginia \_ Risaralda, desde la Teoría de Anna
Sfard sobre las definiciones de las concepciones operacional y estructural,
definidas como niveles de superación de logros. La primera, concepción
operacional, corresponde a actividades procedimentales realizadas
por un estudiante. La segunda, concepción estructural, se refiere
a la capacidad que tiene el alumno para ver el concepto como un objeto
que puede manipular para darle forma, teniendo en cuenta leyes, teoremas
o postulados.


\section{COMPARATIVO SOBRE LA EFICIENCIA DE CONOCIMIENTOS QUE LOGRAN LOS ESTUDIANTES
A PARTIR DE DOS ENFOQUES DE ENSEÑANZA: TRADICIONAL Y OTRO UTILIZANDO
EL PIZARRÓN DIGITAL INTERACTIVO PARA TÓPICOS DE CÁLCULO DIFERENCIAL}

\begin{datos}

Ruth E. Rivera Castellón, Maximiliano De Las Fuentes Lara, Milagros
Guiza Ezkauriatza, Ana Dolores Martínez Molina.

Universidad Autónoma de Baja California,

México,

rrivera@uabc.edu.mx; maximilianofuentes@uabc.edu.mx;

mguizae@gmail.com; ana.dolores.martinez.molina@uabc.edu.mx, 

\end{datos}

Se presenta un reporte de avance de un estudio explorativo y comparativo,
aplicado a dos formas de estructurar la enseñanza del concepto de
límite en un programa de cálculo diferencial en la Universidad Autónoma
de Baja California: a través de un esquema tradicional y mediante
la implementación de un juego didáctico que incorpora el pizarrón
digital interactivo. La investigación incluyó 751 estudiantes y la
experiencia didáctica se realizó con 121 de ellos, los resultados
de la administración de un instrumento de medición válido y confiable
que incluye indicadores de logro asociados al concepto de límite,
permite observar de manera significativa mayores niveles de eficiencia
en los indicadores, en particular cuando se trata de calcular el límite
de una función racional a partir de su representación algebraica,
además de disposición y protagonismo de aquellos estudiantes que utilizaron
el pizarrón digital.


\section{COMPETENCIAS MATEMÁTICAS Y ACTIVIDAD MATEMÁTICA DE APRENDIZAJE}

\begin{datos}

Bernardo García Quiroga, Arnulfo Coronado.

Universidad de la Amazonia.

Florencia - Caquetá - Colombia,

bgarciaquiroga@hotmail.com; arcoronado\_123@yahoo.es 

\end{datos}

El problema de investigación se focaliza en el desarrollo de competencias
matemáticas del estudiante, asumido como un proceso de enculturación
matemática formal, focalizado en compartir y desarrollar el significado
matemático a partir de la comunicación y la negociación cultural entre
los sujetos. Por ello, su actividad matemática de aprendizaje debe
permitirle resolver tareas y desarrollar procesos matemáticos de complejidad
creciente para poder relacionarse con las conceptualizaciones y simbolizaciones
de la cultura matemática. Desde una metodología cualitativa, apoyados
en grupos de discusión y talleres pedagógicos, se resignifcaron los
datos y emergieron nuevas categorías de análisis que orientaron un
nuevo conocimiento didáctico.


\section{ELEMENTOS CURRICULARES PARA LA FORMACIÓN INICIAL DEL PROFESIONAL
DE LA MATEMÁTICA EDUCATIVA.}

\begin{datos}

Judith Hernández$^{1}$, Crisólogo Dolores$^{2}$.

$^{1}$Universidad Autónoma de Zacatecas,

$^{2}$Universidad Autónoma de Guerrero,

México,

judith700@hotmail.com; cdolores2@gmail.com

\end{datos}

El objetivo general de este reporte es brindar elementos curriculares
que permitan caracterizar a los profesionales de la Matemática Educativa
(PME); lo anterior fue posible determinando como objeto de estudio
su campo académico. El marco teórico metodológico se realizó a través
del reconocimiento del campo y se acotó con las siguientes preguntas:
¿Qué problemas atienden los PME?, ¿dónde laboran y cómo se presentan?
y ¿de qué manera se propone su formación inicial? Los resultados dilucidan
al menos 3 tipos de PME; lo que se espera pueda reestructurar la forma
en que se constituye el campo y su incidencia en la formación inicial
de sus profesionales. Además se propone que la manera en que surgió
la ME en México, condiciona las formas en la que ésta y otras disciplinas
se presentan actualmente en la formación inicial de sus profesionales.


\section{REPRESENTACIÓN GRÁFICA Y NOCIÓN DE PERIODICIDAD EN LA EDUCACIÓN SECUNDARIA}

Gonzalo Zubieta Badillo, Xóchitl Yaraseth Reyes Cruz CINVESTAV-IPN,
MÉXICO gzubieta@cinvestav.mx yaraseth@gmail.com Geometría, educación
secundaria, investigación educativa

En esta investigación educativa el propósito principal fue conocer
nociones de estudiantes de educación secundaria respecto a la periodicidad
y sus representaciones gráficas. Se han realizado tres estudios exploratorios
y aquí se presenta uno de ellos. Se trata del reactivo liberado “El
faro” de la prueba PISA (Program for International Student Assessment)
donde los estudiantes de los tres grados de educación secundaria mostraron
qué reconocen como periodo en una representación gráfica y cómo representan
otros periodos.


\section{TRABAJO DE INVESTIGACIÓN: CUANDO LA MATEMÁTICA SE DEJA FOTOGRAFIAR}

\begin{datos}

Miguel Ángel Martínez, Silvia Verónica Fachal.

Universidad Nacional de Lomas de Zamora - Facultad de Ciencias Económicas,

Argentina,

lavalle1003@gmail.com; silvia.fachal@yahoo.com.ar

\end{datos}

El conocimiento matemático nos brinda la posibilidad de interpretar
muchos objetos del mundo que nos rodea y las relaciones que los sostienen.
Los objetos están, sólo falta descubrirlos. Tal vez, nuestros alumnos
no estén suficientemente incentivados a hacerlo o carezcan del lenguaje
adecuado para hacerse entender. Es por ello que nuestro proyecto de
investigación pretende analizar el lenguaje matemático que utilizan
los alumnos de la escuela secundaria, frente a la interpretación de
ciertas producciones fotográficas. En ese contexto nos parece importante
realizar una exploración del plano discursivo y la posibilidad de
descubrir en él los conocimientos matemáticos de nuestros estudiantes.


\section{EXPERIENCIA DE COMUNIDADES DE APRENDIZAJE EN LA TRANSFORMACIÓN DE
LA PRÁCTICA DOCENTE EN LA ENSEÑANZA DE LA LÓGICA MATEMÁTICA }

\begin{datos}

Rosario Celina Velázquez Ortega.

Zona Escolar de Primaria 431, Educación Pública,

México D.F.

celinveor@hotmail.com; celinveor@gmail.com

\end{datos}

En el trabajo de investigación acción “Experiencia de Comunidades
de Aprendizaje en la Transformación de la Práctica Docente para la
enseñanza de la lógica matemática”, desarrollada durante el ciclo
escolar 2013-2014, se proporciona información sobre la conformación
redes de Apoyo Técnico Pedagógico para la enseñanza y aprendizaje
de las matemáticas. Los 5\textdegree{} grados (punto de partida de
las reflexiones) de la Zona Escolar 431 en la Ciudad de México de
la SEP. El presente trabajo da cuenta del proceso sistémico en el
cambio en las competencias de lógica matemática en los involucrados,
es decir: alumnos, padres de familia, docentes, apoyos técnico pedagógicos,
directivos y supervisión escolar.


\section{DESARROLLO DE LA COMPETENCIA PARA EL CÁLCULO DE PERÍMETROS Y ÁREAS
EN PROFESORES Y ALUMNOS DE 5\textdegree{} Y 6\textdegree{} GRADOS
DE EDUCACIÓN PRIMARIA}

\begin{datos}

Romeo Froylán Caballero Ramos, Angelina Hernández Márquez, Sara Lizbeth
González Santiago. 

Benemérita Escuela Nacional de Maestros,

México D.F. 

rfroylanc@gmail.com

\end{datos}

En este reporte de investigación sobre “El desarrollo de la Competencia
para el Cálculo de Perímetros y Áreas en Alumnos de 5\textdegree{}
y 6\textdegree{} grados de Educación Primaria” desarrollada durante
el año escolar 2013-20134, se analiza la forma como los docentes de
educación primaria en México, enseñan y desarrollan en sus alumnos
la competencia para el cálculo de perímetros y áreas. Para realizar
este trabajo, se experimentó en 20 grupos una propuesta didáctica
y en esta ponencia se resumen los resultados obtenidos, que se pueden
considerar satisfactorios, pues un 78.7\% de los alumnos construyó
los aprendizajes esperados.


\section{LA TECNOLOGÍA EN EL AULA MATEMÁTICA}

\begin{datos}

Yacir Testa CEIBAL,

Consejo de Formación en Educación Uruguay - CICATA,

Uruguay,

prof.yacirtesta@gmail.com

\end{datos}

Se presentan resultados de la Investigación sobre las modificaciones
en la Integración de Tic en el Aula de Matemática luego que los Docentes
realizan el primer Cuso de Integración de TIC al Aula de Matemáticas
desde Plan Ceibal. El objetivo es presentar al docente distintas herramientas
TIC que permitan potenciar el aprendizaje de la matemática y el desarrollo
del pensamiento matemático del estudiante para integrarlas a sus cursos.
Plan Ceibal ha entregado a cada estudiante y Docente una computadora
portátil personal. En 2011 ofrece a los Maestros la oportunidad de
realizar cursos de integración de TIC en Matemática. 


\section{ESTRATÉGIAS PEDAGÓGICAS DE RESOLUÇÃO DE PROBLEMAS EM UM CURSO DE
FORMAÇÃO DE PROFESSORES}

\begin{datos}

Nielce Meneguelo Lobo da Costa, Aparecida Rodrigues Silva Duarte,
Edite Resende Vieira. 

Universidade Bandeirante de São Paulo – UNIBAN,

Brasil,

nielce.lobo@gmail.com; aparecida.duarte6@gmail.com;

edite.resende@gmail.com 

\end{datos}

Neste estudo se discutem estratégias pedagógicas para ensino de matemática
por resolução de problemas, as quais foram analisadas em uma formação
continuada para professores. Trata-se de pesquisa qualitativa, com
metodologia co-generativa (Greenwood; Levin, 2000) e fundamentada
em Shulman (1986), Ball, Thames e Phelps (2008), Polya (1995), Onuchic
(1999) e Bryant et al (2012). Um dos temas foi resolução de problemas
com números inteiros pela estratégia de jogo didático. Os desafios
propostos geraram ambiente propício à investigação e à descoberta,
de modo que vários professores ficaram estimulados a propor mudanças
promovendo aulas de matemática com um caráter mais exploratório e
investigativo. 

Palavras–Chave: Educação Continuada; Ensino de Matemática; Resolução
de Problemas; Jogos Didáticos; Números Inteiros.


\section{PREPARACIÓN PARA EL CÁLCULO DE ÁREAS Y PERÍMETROS. EXPERIMENTACIÓN
DE UNA PROPUESTA DE ENSEÑANZA-APRENDIZAJE CON ALUMNOS DE PREESCOLAR}

\begin{datos}

Espositora: Elena Raquel Castro Tapia,

Colectivo de docentes del Museo Didáctico de la Matemática y de Educación
Preescolar, Institución: Benemérita Escuela Nacional de Maestros,

México,

miss\_xolotl@yahoo.com

\end{datos}

La investigación “Preparación para el cálculo de áreas y perímetros.
Experimentación de una propuesta de enseñanza-aprendizaje con alumnos
de preescolar” desarrollada durante el año escolar 2013-2014, analiza
la forma como los alumnos de educación preescolar, construyen las
nociones espaciales básicas para el cálculo de áreas y perímetros
utilizando medidas no convencionales. Esto es importante, porque dichas
nociones servirán como antecedente para que en el siguiente nivel
educativo se inicien en la deducción de las fórmulas para el cálculo
de áreas y perímetros. También se analiza las diversas estrategias
que los alumnos aplican para la resolución de las situaciones problemáticas
propuestas.


\section{ACTITUD HACIA LA MATEMATICA: UN ESTUDIO COMPARATIVO ENTRE ESTUDIANTES
PANAMEÑOS Y MEXICANOS}

\begin{datos}

Luisa Mabel Morales Maure, José Gabriel Sánchez Ruiz, Orlando García
Marimón$^{1}$.

IDEN-UP, FEZ-Zaragoza, $^{1}$UDELAS,

México, $^{1}$Panamá,

lui.mora@hotmail.com ; josegsr@servidor.unam.mx;

nangarcia@hotmail.com 

\end{datos}

Esta investigación tiene el sentido de integrar la perspectiva afectiva
y cognitiva con los procesos de enseñanza aprendizaje de la Matemática.
El estudio se llevó a cabo con estudiantes de primer ingreso de las
universidades de Panamá y México, debido a las características de
la muestra y al problema de la investigación, se trata de un estudio
descriptivo en el que se caracterizan las actitudes. De lo anterior
se obtuvieron correlaciones significativas, ya que los alumnos que
tienen actitudes positivas hacia el aprendizaje matemático tienen
mejores calificaciones y su rendimiento académico es sobresaliente. 


\section{ESTUDIO DE CLASES PARA LA ARTICULACIÓN DE CONOCIMIENTOS EN FORMACIÓN
INICIAL}

\begin{datos}

Raimundo Olfos, Soledad Estrella, Sergio Morales.

Pontificia Universidad Católica de Valparaíso,

Chile,

Raimundo.olfos@ucv.cl; soledad.estrella@ucv.cl;

sergio.morales.candia@gmail.com 

\end{datos}

Este estudio de casos postula la integración del conocimiento teórico
con el práctico como clave en la formación del profesorado. Participan
alumnos de noveno grado, una futura profesora, el profesor mentor
del liceo y el académico responsable de un curso taller en la formación
inicial de la futura profesora. Los datos provienen de dos entrevistas,
la preparación de una clase y u implementación. Los principales hallazgos
se refieren al robustecimiento pedagógico y didáctico de la futura
profesora en virtud de las oportunidades que le proveyó el Estudio
de Clases para la integración de sus conocimientos. 


\section{VOLÚMENES E INTEGRALES MÚLTIPLES: UN ESTUDIO DESDE LA TEORÍA DE REPRESENTACIÓN
SEMIÓTICA }

\begin{datos}

Daniel Giovanni Proleón Patricio, Francisco Ugarte Guerra.

Pontificia Universidad Católica del Perú, Instituto de Investigación
sobre Enseñanza de las Matemáticas- IREM.

Perú,

dproleon@pucp.pe; fugarte@pucp.edu.pe

\end{datos}

En nuestro estudio analizamos cómo el uso de un software, que permite
la visualización de sólidos y sus proyecciones, facilita la conversión
entre el registro gráfico y el registro analítico. Dicha conversión
es necesaria para articular la noción de volumen e integral múltiple.
Los resultados de este análisis justifican la construcción de una
secuencia didáctica mediada por el software, que se incorpora al proceso
de conversión. La investigación se realizó con alumnos del segundo
año de estudios universitarios de ingeniería.


\section{EL PROCESO DE FORMACIÓN DE MAESTROS/AS DE MATEMÁTICA PARA EL NIVEL
PRIMARIO }

\begin{datos}

Isidro Báez Suero, José Manuel Ruiz Socarrás. 

Universidad Autónoma de Santo Domingo (UASD),

República Dominicana, Cuba,

ibaez13.18@hotmail.com; jose.ruiz@reduc.edu.cu

\end{datos}

El objetivo de la investigación fue caracterizar al proceso de formación
de maestros/as de matemática para el nivel primario, para lo cual
se hizo una revisión bibliográfica y una encuesta, concluyendo que
la formación y desarrollo del maestro en formación en la competencia
Dominio de la aritmética en la educación primaria en América Latina
y el Caribe, permite mejorar la enseñanza aprendizaje de la matemática.
En tal sentido para un adecuado funcionamiento del proceso de enseñanza
aprendizaje de la matemática en la carrera de formación de maestros,
como sistema, se requiere atender tanto la calidad del docente, del
plan de estudio, como del perfil de los estudiantes futuros maestros
y las relaciones entre estos componentes.


\section{HISTORIA DE LAS MATEMÁTICAS EN LA PRÁCTICA DE ENSEÑANZA DE LAS MATEMÁTICAS
EN LÌNEA Marger}

\begin{datos}

da Conceição Ventura Viana.

Universidade Federal de Ouro Preto (UFOP),

Brasil,

margerv@terra.com.br 

\end{datos}

Se presentan los resultados (informes y análisis de algunas de las
actividades) de una tesis de maestría en Educación Matemática, acerca
de una propuesta de enseñanza desarrollada para la asignatura Historia
de las Matemáticas: un enfoque metodológico para la enseñanza de las
matemáticas, que forma parte de la práctica de enseñanza de las Matemáticas
en la formación de profesores de matemáticas de un Centro de Educación
Abierta y a Distancia de una universidad estatal de Brasil. Se identificaron
algunas contribuciones de la propuesta para la formación de profesores
de Matemáticas en la modalidad a distancia y en ambientes virtuales
de aprendizaje.


\section{EL DESARROLLO DE LA HABILIDAD DE MODELAR EN MATEMÁTICA A TRAVÉS DEL
PROCESO DE ENSEÑANZA APRENDIZAJE DE LA PROGRAMACIÓN LINEAL: UN PROYECTO
DE INVESTIGACIÓN}

\begin{datos}

Juan Antonio Manzueta Concepción, Ramón Blanco Sánchez, Olga Lidia
Pérez González, Arturo Adames Tejeda.

Universidad Autónoma de Santo Domingo, Universidad de Camagüey, 

República Dominicana, Cuba,

jmanzueta2004@gmail.com; ramón.blanco@gmail.com; 

olguitapg@gmail.com

\end{datos}

En el presente trabajo se realiza la comunicación breve de un proyecto
de investigación que se encuentra en desarrollo, como parte de una
investigación de doctorado. En la propuesta se caracteriza la problemática
existente en relación a las dificultades que presentan los estudiantes
de ingeniería con el uso del lenguaje matemático para el desarrollo
de la habilidad de modelar matemáticamente


\section{NARRATIVAS DE LOS DERECHOS HUMANOS EN EDUCACIÓN MATEMÁTICA: EL CASO
DE LOS ESTUDIANTES DE GRADO SEXTO}

\begin{datos}

Juan Manuel Salas Martínez, Fernando Guerrero Recalde.

Universidad Distrital Francisco José de Caldas,

Colombia,

juanmanuelsalasmartinez@hotmail.com; nfguerreror@gmail.com

\end{datos}

Se pretende desarrollar una secuencia de actividades donde a partir
de la modelación matemática y las narrativas, los estudiantes de grado
sexto del colegio La Belleza los Libertadores IED, reflexionen sobre
la realidad social y analicen los efectos de algunos aspectos sobre
la conservación de la naturaleza, empleando las narrativas y la modelación
matemática de situaciones relacionadas con la conservación de la naturaleza,
para ejercer el derecho a la vida y el deber de protegerla, con la
introducción de una secuencia didáctica, la investigación es de tipo
cualitativo y se desarrollará a partir de la investigación acción,
Elliot (2005).


\section{UN ESTUDIO SOCIOEPISTEMOLÓGICO DEL USO DE LAS GRÁFICAS EN EL ENTORNO
DE PAM}

\begin{datos}

Yacir Testa CEIBAL.

Consejo de Formación en Educación Uruguay - CICATA,

Uruguay,

prof.yacirtesta@gmail.com

\end{datos}

Esta investigación, de corte socioepistemológico, busca categorizar
y analizar el “uso” que los docentes realizan de ciertas actividades
gráficas, centrándonos en cómo y por qué las seleccionan, si el contexto
de la PAM modifica el tipo de trabajo venían realizando en otros contextos.
En este marco analizaremos también la relación entre éste uso y el
desarrollo del PLV del estudiante. Esto permitirá construir modelos
de uso de conocimiento (Cordero y Flores, 2007), consideramos (Buendía,
2011, p. 42) “un uso que se desarrolla situacionalmente de tal manera
que es factible explorar la naturaleza del conocimiento matemático
involucrado y favorecer su resignificación.”


\section{COMPRENSIÓN DE NOCIONES DE ESPACIO Y CANTIDAD EN AULA DE EDUCACIÓN
ESPECIAL: ESTUDIO DE CASOS }

\begin{datos}

Sandra Patricia García Sánchez, Ignacio Garnica, Dovala.

Cinvestav-IPN - Departamento de Matemática Educativa,

México,

spgarcia@cinvestav.mx; igarnica@cinvestav.mx 

\end{datos}

Para identificar condiciones que limitan o favorecen la adquisición
de nociones matemáticas por siete niños de 6 a 8 años con distintas
discapacidades, se les aplicaron actividades de cantidad (agregaciones-desagregaciones,
conteo) y de espacio (recorridos en la escuela y en su exterior),
en sesiones semanales de una hora, con intervención de sus madres.
Los niños manifestaron nociones, diferenciadas, de cantidades hasta
de cuatro elementos de una colección, con dificultades para desagregar
y seriar; localizaron puntos referenciales en el espacio exterior
y trayectorias en el interior. La intervención materna permitió reconocer
modos de comunicación según las condiciones específicas de cada discapacidad.


\section{CONCEPCIONES SOBRE EL ARGUMENTO DE LA FUNCION SEN X EN PROFESORES
DEL NIVEL MEDIO SUPERIOR MEXICANO}

\begin{datos}

Yanira Pachuca Herrera, Gonzalo Zubieta Badillo,

Cinvestav,

México,

ypachucaherrera@yahoo.com ; gzubieta@cinvestav.mx 

\end{datos}

Esta investigación indaga sobre qué medida angular para la función
sen x utilizan los profesores de matemáticas que han impartido la
materia de trigonometría, al contestar un cuestionario donde se incluye
calcular sen x para valores reales del argumento. El uso de números
como 180 lo asocian a grados y los números donde aparece el valor
de lo interpretan como radianes, sin embargo cuando no aparece crea
ambigüedad. 


\section{ESTUDIO CORRELACIONAL ENTRE EL DESEMPEÑO EN MATEMÁTICAS Y LAS CONDICIONES
BIOLÓGICAS (ÍNDICE DE MASA CORPORAL. IMC). EN NIÑOS DE 6 A 12 AÑOS
DEL CENTRO PASTORAL Y DE SERVICIOS SAN MARCELINO CHAMPAGNAT.}

\begin{datos}

Yeimy Julieth Moreno Jiménez, Carlos Humberto Barreto Tovar, Julián
Gómez Niño, Marcela Ramírez, Roberto Pinzón, Yenny Paola Argüello
Gutiérrez, Liliana Rodríguez, Manuel Alejandro Henao, Gina Patricia
Cano.

Corporación Universitaria Iberoamericana , Comunidad de Hermanos Maristas
de la Enseñanza, 

Colombia - Bogotá D.C.

yeimy.moreno@iberoamericana.edu.co; carloshumbertobarreto@hotmail.com;

coordinadormatematica@maristasnorandina.org; cpschampagnat@maristasnorandina.org;

roalpico@yahoo.com; yenny.arguello@iberoamericana.com;

yuber.rodriguez@iberoamericana.com; solidaridad@maristasnorandina.org;

gina.cano@iberoamericana.edu.co 

\end{datos}

Los resultados presentados, teniendo como población de estudio los
niños de 6 a 12 años que asisten al Centro Pastoral y de Servicios
San Marcelino Champagnat de la ciudad de Bogotá D.C., muestran la
relación existente entre el Índice de Masa Corporal a través de la
Tabla de Crecimiento Infancia y Adolescencia de la Organización Mundial
de la Salud y su posible incidencia en el desempeño en Matemáticas
(referido a los componentes y las competencias evaluadas por el ICFES),
mediante las Pruebas Censales que aplica la comunidad de Hermanos
Maristas de la Enseñanza en sus obras educativas. 


\section{LOS SIGNIFICADOS DE FUNCIÓN Y FUNCIÓN DERIVADA DESARROLLADOS POR
LOS ESTUDIANTES AL ESTUDIAR LA VARIACIÓN EN EL CONTEXTO DE LOS PROBLEMAS
DE INGENIERÍA.}

\begin{datos}

Ramiro Ávila Godoy, Jesús Ávila Godoy$^{1}$, José María Bravo Tapia.

Universidad de Sonora, $^{1}$UABC, Universidad de Sonora,

México,

ravilag@gauss.mat.uson.mx; jag\_virgo@hotmail.com;

jmbravo@gauss.mat.uson.mx

\end{datos}

Este es un reporte parcial de un proyecto diseñado para investigar
el papel del contexto de la enseñanza en la formación y desarrollo
de los significados de los objetos matemáticos que construyen los
estudiantes. Específicamente se investigó sobre los significados de
los objetos matemáticos función y función derivada desarrollados por
un grupo de estudiantes de ingeniería al estudiar la variación en
el contexto de los problemas de optimización. La investigación se
desarrolla asumiendo las premisas de dos marcos teóricos: el Enfoque
Ontosemiótico de la Instrucción y la Cognición Matemática (EOS) de
Juan D. Godino y la Teoría de la Enseñanza Problémica de M. Majmutov
(TEP). 


\section{RESOLUCIÓN DE PROBLEMAS O EJERCITACIÓN DE ALGORITMOS}

\begin{datos}

GRACIELA HERNÁNDEZ TEXOCOTITLA .

INSTITUTO SUPERIOR DE CIENCIAS DE LA EDUCACIÓN DEL ESTADO DE MÉXICO,
UNAM,

MÉXICO,

texocotitla@yahoo.com

\end{datos}

La investigación es cualitativo interpretativo, usando la etnografía.
Se hicieron observaciones en el aula y entrevistas, con grupos de
diversos grados de primaria. El tema a investigar fue el significado
y práctica que realizan profesores y alumnos respecto a la resolución
de problemas. Encontramos que los algoritmos se suelen enseñar separadamente
de los problemas, e incluso antes que los problemas. Los alumnos le
dedican largas horas a dominar la técnica de un algoritmo fuera de
contexto en que se producen, desarrollan destrezas en una técnica
algorítmica vacía de significado: aprenden a multiplicar con un procedimiento,
pero no saben cuándo multiplicar.


\section{PRÁCTICAS DISCURSIVAS Y RECURSOS PEDAGÓGICOS EN CLASES DE GEOMETRÍA
EN LA EDUCACIÓN BÁSICA: EL CASO DEL ORIGAMI}

\begin{datos}

Ana Katherine Valencia.

Montenegro Universidad del Valle,

Colombia,

ankateva@univalle.edu.co 

\end{datos}

Reflexión sobre la enseñanza en situación. Resultado de la identificación
y análisis de prácticas discursivas de maestros cuando enseñaron geometría
vinculando la origámica, en el nivel educativo de educación básica.
La tesis de maestría que orientará este reporte destaca el enfoque
comunicacional de Anna Sfard, propuesto en esencia como principio
básico para el estudio de la cognición humana, donde el pensamiento
se puede conceptualizar como un caso de comunicación, es decir, como
la comunicación con uno mismo, lo que le da un estatus diferente al
lenguaje en relación con los recursos pedagógicos en el marco de la
geometría origámica. 


\section{DIFICULTADES DE LOS ALUMNOS EN EL TRABAJO CON LOS CONCEPTOS DEL CÁLCULO
DIFERENCIAL}

\begin{datos}

Neel Báez Ureña, Ramón Blanco Sánchez, Olga Lidia Pérez González.

Universidad Autónoma de Santo Domingo, Universidad de Camagüey,

República Dominicana, Cuba,

neelbaez@gmail.com$;$ ramón.blanco@reduc.edu.cu;

olguitapg@gmail.com 

\end{datos}

El objetivo es reportar los resultados parciales de un proyecto de
investigación que parte de la problemática de los estudiantes con
los conceptos del Cálculo Diferencial. Se ha podido argumentar que
dado el carácter no ostensivo de los objetos matemáticos y la insuficiente
cantidad de tareas, realizadas en clases, sobre la transferencia de
registros semióticos, se limita el trabajo conceptual, de los estudiantes,
con este contenido. El objetivo del proyecto es desarrollar un sub
sistema didáctico orientado a consolidar el nexo símbolo objeto, a
través de la transferencia de registros semióticos, mediante el uso
las TIC.


\section{ORGANIZACIÓN DEL PROCESO DOCENTE DE LA GEOMETRÍA Y TRIGONOMETRÍA
PLANA, FUNDAMENTADA EN LA ESTRUCTURA SISTÉMICA DE LA MATEMÁTICA: UN
REPORTE DE INVESTIGACIÓN}

\begin{datos}

Elizabeth Rincón Santana, José Manuel Ruíz Socarras, Ramón Blanco
Sánchez.

Universidad Autónoma de Santo Domingo, Universidad de Camagüey,

República Dominicana, Cuba,

te10elirisa@gmail.com; jose.ruiz@reduc.edu.cu; 

ramón.blanco@reduc.edu.cu

\end{datos}

Se realiza un reporte de una investigación que está en fase de desarrollo
y que forma parte de un proyecto de doctorado de la autora principal
del trabajo. La investigación parte de las dificultades que tienen
los estudiantes, de la carrera de Educación mención Matemática, de
la Universidad Autónoma de Santo Domingo, en el proceso de enseñanza
de la Geometría y la Trigonometría plana y sobre el conocimiento didáctico
que tienen sobre ellas. El objetivo del trabajo es exponer la lógica
que se está llevado en la investigación y las conclusiones parciales
que se han obtenido. 


\section{RESOLUCIÓN DE PROBLEMAS MATEMÁTICOS CON ENFOQUE MEDIOAMBIENTAL DESDE
LA UNIDAD 2: “TRABAJO CON VARIABLE, ECUACIONES, INECUACIONES Y SISTEMA
DE ECUACIONES”, DE DÉCIMO GRADO}

\begin{datos}

Mónica Santa López$^{1}$, Juan A. Manzueta, Migdalia$^{2}$  Fernández
Perón$^{3}$.

$^{3}$Universidad de ciencias pedagógicas José Martí. 

$^{1,2}$Rep. Dominicana, $^{3}$Camagüey - Cuba.

msanta@ucp.cm.rimed.cu;   

\end{datos}

La presente investigación constituye una propuesta de  problemas matemáticos
que contribuyen a la resolución de problemas y específicamente aquellos
que guardan relación con la educación ambiental. Este proceso investigativo
se inicio a partir de detectarse algunas insuficiencia que presentan
los estudiantes en la resolución de problemas matemáticos. Por lo
que se propone como objetivo Elaborar problemas matemáticos con enfoque
medioambiental en la unidad 2 “Trabajo con variable, ecuaciones, inecuaciones
y sistema de ecuaciones”, en décimo grado de Camagüey, Cuba. Para
el desarrollo de esta investigación se utilizaron métodos que permitieron
lograr una información detallada sobre el estado actual y final de
la preparación de los escolares para resolver problemas, además se
aborda los fundamentos teóricos metodológicos sobre la resolución
de problemas matemáticos. 


\section{DISCURSOS EN LA MODELACIÓN MATEMÁTICA Y SU INCIDENCIA EN LAS DINÁMICAS
DE INCLUSIÓN Y EXCLUSIÓN DESDE UNA PERSPECTIVA SOCIO CRÍTICA}

\begin{datos}

Oscar Alejandro Barrios Candil, Lesly Tatiana Galvis Bejarano.

Universidad Distrital Francisco José de Caldas,

Colombia,

oscalej2@gmail.com; matetag@gmail.com 

\end{datos}

Esta investigación pretende identificar la incidencia de los discursos
surgidos del proceso de modelación matemática en un escenario de investigación
respecto a las dinámicas de exclusión e inclusión desde la perspectiva
socio crítica (Skovsmose, 1999; Skovsmose \& Valero, 2012; y el trabajo
realizado por el grupo de investigación Diversidad y Didáctica de
las Matemáticas). Se busca partir del montaje de un escenario de aprendizaje
que involucre un ejercicio de modelación matemática para rastrear
las características polifónicas (Bajtin, 1982) de los discursos que
surgen en las interacciones entre estudiantes y docente, mediante
el uso de análisis antenarrativo (Mølberg, 2006). 


\section{ESTUDIO CUASI-EXPERIMENTAL SOBRE PROCESOS COGNITIVOS EN ESTUDIANTES
SECUNDARIOS DE LA REGIÓN DEL MAULE CHILE.}

\begin{datos}

María Aravena Díaz.

Facultad de Ciencias Básicas - Universidad Católica del Maule

Talca - Chile,

maravenadiaz@gmail.com 

\end{datos}

La investigación, aborda los procesos cognitivos que intervienen en
la resolución de problemas geométricos, enmarcado en el modelo de
Duval y las categorías descritas por Torregrosa y Quesada (2007).
Además, se consideró la interrelación entre los procesos cognitivos
y el modelo de Van-Hiele. La metodología fue cuantitativa y la muestra
de 645 alumnos mediante conglomerados de afijación proporcional. A
nivel de resultados el Grupo Experimental presenta diferencias significativas
sobre el Control en los procesos cognitivos. Sobre la relación entre
ambos modelos, la variable que contribuye en la predicción, para los
niveles de Van-Hiele, es la visualización y el proceso configural. 


\section{ENSEÑANZA DE LA MATEMÁTICA BASADA EN LA METODOLOGIA DE RESOLUCIÓN
DE PROBLEMAS, MEDIANTE LA EDUCACIÓN A DISTANCIA EN EL CURSO DE METODOLOGIA
DE ENSEÑANZA DE LA MATEMÁTICA, UNIVERSIDAD ESTATAL A DISTANCIA, COSTA
RICA 2013-2014.}

\begin{datos}

María Alejandra Chacón Fonseca. 

Universidad Estatal a Distancia,

Costa Rica,

mchacon@uned.ac.cr 

\end{datos}

El programa de Enseñanza y Aprendizaje de la Matemática de la Universidad
Estatal a Distancia, ante la aprobación de nuevos programas de enseñanza
de la matemática (2012), se planteó como propósito fundamental elaborar
e implementar una propuesta de planeamiento pedagógico que permitiera
el mejoramiento de la práctica pedagógica de docentes y estudiantes
de la Cátedra de Metodología, mediante la incorporación de la metodología
del aprendizaje basado en la resolución de problemas, facilitando
pasar de una inclusión teórica mínima a una inclusión teórica más
amplia constituyendo un gran aporte para la Educación a distancia
y a la comunidad educativa en general.


\section{TRATAMIENTO DIDÁCTICO DE LAS TAREAS DE TRANSFERENCIA DE REGISTROS
SEMIÓTICOS DE LOS OBJETOS MATEMÁTICOS EN EL PROCESO DE ENSEÑANZA APRENDIZAJE
DE LA MATEMÁTICA EN LA FORMACIÓN INICIAL DE MAESTROS }

\begin{datos}

Aury Rafael Pérez Cuevas, Isabel Yordi González, Jorge García Batan.

Universidad Autónoma de Santo Domingo, Universidad de Camagüey,

República Dominicana, Cuba,

auryp01@gmail.com; isabel.yordi@reduc.edu.cu; 

jorge.garcia@reduc.edu.cu 

\end{datos}

El trabajo parte del insuficiente tratamiento didáctico de las tareas
para la transferencia de registros semióticos de los objetos matemáticos,
en la formación inicial de maestros de Matemática, para el nivel de
secundaria. La idea que se defiende es que si en el proceso de enseñanza
aprendizaje, de la Matemática, se aplica una estrategia didáctica
sustentada en un modelo para fomentar la realización tareas, que tenga
en cuenta la transferencia de registros semióticos, y la relación
dialéctica entre “enseñar a enseñar” y “aprender a enseñar”, se contribuye
a elevar la calidad de la formación de estos maestros. 


\section{EL USO DE RESULTADOS DE INVESTIGACIÓN PARA EL DESARROLLO DE LA TEORÍA
DE LA MATEMÁTICA EDUCATIVA. EL CASO DEL APRENDIZAJE DEL ÁLGEBRA}

\begin{datos}

Margarita Itzel Curiel Neri, Claudia Margarita Acuña Soto.

Centro de Investigación y de Estudios Avanzados de Instituto Politécnico
Nacional (CINVESTAV IPN),

México,

micuriel@cinvestav.mx; claudiamargarita\_as@hotmail.com 

\end{datos}

En Matemática Educativa existen líneas de investigación con bases
epistemológicas y ontológicas distintas, ¿cómo se desarrollaron? y
¿cómo repercuten los llamados resultados de investigación en la actividad
docente en la actualidad? Los trabajos de Piaget y de Vygotski, son
fundamento para la producción científica de la matemática educativa,
y en particular para el estudio del aprendizaje del álgebra. Para
investigar la forma cómo los resultados de investigación están presentes
en nuevas teorías nos propusimos 
\begin{description}
\item [{1.}] Analizar dos de ellas: La Teoría APOS y la Teoría Cultural
de la Objetivación 
\item [{2.}] Establecer nuestros objetos de estudio en particular del uso
que se hace de los resultados de éstas teorías
\item [{3.}] Investigar hasta dónde los supuestos teóricos centrales pueden
ser usados adecuadamente o en forma ingenua en las actividades docentes.
\end{description}

\section{EL USO DE RECURSOS DIDÁCTICOS EN LA ENSEÑANZA DE LA MATEMÁTICA }

\begin{datos}

Aura Lucía Camargo Silva, Martha del Pilar Pacheco.

Universidad Pedagógica y Tecnológica de Colombia,

Colombia ,

camargoaura@hotmail.com; mapipacheco@gmail.com 

\end{datos}

El presente informe de investigación pretende resaltar la importancia
en el uso de recursos didácticos para la enseñanza de las matemáticas,
de forma que se convierta en un quehacer donde se potencialice el
conocimiento matemático, en el marco teórico se realizó un recorrido
por la concepción de recurso didáctico en el proceso de enseñanza
aprendizaje, su papel como herramienta dinamizadora y su importancia
en el aula de clase, se quiso identificar en 18 docentes de distintas
instituciones educativas públicas y privadas de la ciudad de Tunja
y municipios aledaños, cuál es el papel de los recursos didácticos
en relación con el desarrollo de los procesos de aprendizaje de la
matemática, abordando ventajas y debilidades en su utilización, el
devenir en la reflexión de cada docente y las estrategias utilizadas,
bajo en enfoque cualitativo y el diseño de investigación acción, se
encontró que el uso de los recursos, es una herramienta que usada
adecuadamente pude llegar a la comprensión de la matemática en un
alto nivel.


\section{\uppercase{ dominios ''en acción'' del conocimiento matemático para
enseñar geometría analítica a nivel universitario. el caso del Profesorado
en Matemática de la Universidad nacional de rosario}}

\begin{datos}

Virginia Ciccioli$^{1,2}$, Natalia Sgreccia$^{1,3}$.

$^{1}$FCEIA-UNR, $^{2}$ENS 33, $^{3}$CONICET,

Argentina,

ciccioli@fceia.unr.edu.ar$;$ sgreccia@fceia.unr.edu.ar 

\end{datos}

El conocimiento especializado del contenido, el conocimiento del contenido
y de la enseñanza y el conocimiento del contenido y de los estudiantes
son los dominios requeridos para enseñar Matemática más proclives
de ser observados en prácticas de aula, es decir, “en acción”. En
esta investigación interesa determinar la formación que el Profesorado
en Matemática (PM) de la Universidad Nacional de Rosario ofrece en
geometría analítica elemental. Para ello, entre otras acciones, se
analiza la activación de dichos dominios en la interacción alumno-profesor
producida así como las configuraciones de mensajes emergentes durante
las primeras clases de geometría analítica en el PM.


\section{EVALUACIÓN DEL CONOCIMIENTO DIDÁCTICO-MATEMÁTICO SOBRE PROBABILIDAD
EN PROFESORES DE PRIMARIA EN ACTIVO}

\begin{datos}

Claudia Vásquez, Àngel Alsina.

Pontificia Universidad Católica de Chile, Universidad de Girona,

Chile, España

cavasque@uc.cl; angel.alsina@udg.edu

\end{datos}

El conocimiento que un profesor necesita para enseñar ha sido ampliamente
investigado durante los últimos años, sin embargo existen pocos estudios
sobre profesores en activo. Se presenta un estudio que evalúa el conocimiento
didáctico-matemático sobre probabilidad en profesores de educación
primaria, centrado en aspectos de los componentes del modelo del conocimiento
didáctico-matemático. El análisis de los datos va a permitir, en primer
lugar, describir fortalezas y debilidades en relación a componentes
del conocimiento didáctico-matemático, en profesores de educación
primaria para enseñar probabilidad, y en segundo lugar, obtener información
relevante para orientar la formación inicial y continua del profesorado. 


\section{O ENSINO DE EQUAÇÕES DO PRIMEIRO GRAU À LUZ DA TEORIA DA APRENDIZAGEM
SIGNIFICATIVA: UMA PROPOSTA SOBRE A NOÇÃO DE EQUIVALÊNCIA COMO CONCEITO
SUBSUNÇOR}

\begin{datos}

Viviane Beatriz Hummes, Márcia Rodrigues Notare.

PPGEMAT/UFRGS ,

Brasil,

vivihummes@gmail.com; marcia.notare@gmail.com 

\end{datos}

Este trabalho tem a intenção de apresentar alguns resultados de um
estudo sobre o conceito de equivalência como conceito subsunçor fundamental
para o desenvolvimento da Aprendizagem Significativa de equações do
primeiro grau. À luz da Teoria da Aprendizagem Significativa, procuramos
investigar, em uma turma do oitavo ano do Ensino Fundamental, de uma
escola de Porto Alegre/RS, Brasil, se atividades propostas por um
Objeto Digital de Aprendizagem que relaciona o equilíbrio existente
em uma balança de dois pratos com uma igualdade entre os termos de
uma equação podem funcionar como organizadores prévios para facilitar
a Aprendizagem Significativa dos estudantes. 


\section{JOGOS DE LINGUAGEM DE ESTUDANTES DO ENSINO MÉDIO NA RESOLUÇÃO DE
POTÊNCIAS COM BASE RACIONAL E EXPOENTE NEGATIVO}

\begin{datos}

\textonesuperior{}Walter Aparecido Borges, \texttwosuperior{}Maria
Helena Palma de Oliveira .

$^{1,2}$Universidade Anhanguera de São Paulo,

Brasil,

\textonesuperior{}w53borges@gmail.com; \texttwosuperior{}mhelenapalma@gmail.com 

\end{datos}

O trabalho apresenta os processos de linguagem de alunos de 1º ano
do Ensino Médio de escola pública de São Paulo na resolução de atividades
de divisão de frações, como condição para a aprendizagem de potências
com base racional e expoente negativo, necessária para o entendimento
de função exponencial. O diálogo entre os alunos no processo de resolução
evidenciaram o uso de jogos de linguagem como recurso na superação
de dúvidas, ou como prática de resolução baseada em tecnicismo o que
expôs a falta de conhecimento teórico necessário para a efetiva aprendizagem. 


\section{EVOLUCIÓN DEL CONCEPTO DE INFERENCIA ESTADÍSTICA. INFLUENCIA EN SU
ENSEÑANZA}

\begin{datos}

Christiane Ponteville$^{1,2}$, Cecilia Crespo Crespo$^{1}$.

 $^{1}$Instituto Superior del Profesorado “Dr. Joaquín V. González”.
$^{2}$Facultad de Farmacia y Bioquímica. Universidad de Buenos Aires,

Argentina,

chponteville@gmail.com; crccrespo@gmail.com

\end{datos}

Podemos establecer dos tipos de tendencias en la enseñanza de conceptos
de la inferencia estadística: interpretarlos como modelos matemáticos
excluidos del contexto o como la aplicación de un algoritmo. Cualquiera
de las dos involucra perder algún aspecto en la generación del concepto
pues no tiene en cuenta su verdadera génesis en la historia de la
ciencia. Proponemos una revisión de los diferentes aspectos que participaron
en la evolución de los fundamentos de las probabilidades y la estadística
para entender los marcos históricos y epistemológicos en los cuales
se desarrollaron las pruebas de hipótesis como parte de la ciencia
estadística.


\section{EXPERIENCIAS EMOCIONALES DE ESTUDIANTES DE EDUCACIÓN MEDIA SUPERIOR
EN LA CLASE DE MATEMÁTICAS}

\begin{datos}

Gustavo Martínez Sierra, María S. García González

Cicata - IPN, Cinvestav - IPN.

México,

gmartinezsierra@gmail.com; mgargonza@gmail.com 

\end{datos}

Este trabajo presenta una investigación cualitativa cuyo objetivo
es identificar las experiencias emocionales de estudiantes de Educación
Media Superior en la clase de matemáticas. Para obtener información
se llevaron a cabo entrevistas en grupos focales con 22 estudiantes.
El análisis de los datos se realizó utilizando la teoría de la estructura
cognitiva de las emociones que especifica las condiciones desencadenantes
de cada emoción y las variables que afectan a la intensidad de cada
emoción.


\section{UN ESTUDIO SOBRE LAS PRÁCTICAS DOCENTES EN LA MATEMÁTICA DEL BACHILLERATO }

\begin{datos}

Carol Yaneth Corral López, Silvia Elena Ibarra Olmos.

Instituto Tecnológico de Monterrey, Universidad de Sonora,

México. 

mat.caroly.cl@gmail.com; sibarra@gauss.mat.uson.mx

\end{datos}

Se presentan los resultados de una investigación realizada con base
en el Enfoque Ontosemiótico del Conocimiento y la Instrucción Matemática,
la cual constó de dos etapas. En la primera se realizó un análisis
del libro de texto que es utilizado por los profesores de matemáticas
de un subsistema de bachillerato mexicano, con la intención de identificar
el significado institucional de referencia. En la segunda etapa, se
hizo un seguimiento del desarrollo de las prácticas docentes de algunos
de esos profesores. 


\section{\uppercase{ Diagnóstico de la Concepción de Variable y su modificación
dentro del estudio del cálculo }}

\begin{datos}

Adriana Cantú, Lorenza Illanes, y Ángeles Domínguez.

Tecnológico de Monterrey Campus Monterrey,

N.L - México 

adriana.cantu@itesm.mx; lillanes@itesm.mx;

angeles.dominguez@itesm.mx

\end{datos}

La concepción de variable es fundamental en el estudio del Cálculo,
concepción que se va modificando a través del estudio como lo demuestra
esta investigación a nivel superior. Se describen principalmente tres
concepciones: la variable como incógnita, la variable como número
generalizado y la variable como relación funcional. Estas concepciones
fueron estudiadas longitudinalmente a lo largo de tres semestres.
La investigación comprende los resultados de un diagnóstico del cual
se presenta un análisis cualitativo y estadístico de los resultados
después de haber aplicado un pretest y un postest en varios grupos
de Cálculo, y se observaron modificaciones de dichas concepciones.


\section{\uppercase{ Espacios de Desarrollo del Potencial Humano en EL APRENDIZAJE
DE LA INTEGRAL IMPROPIA}}

\begin{datos}

Lorenza Illanes, Armando Albert.

Tecnológico de Monterrey Campus Monterrey, 

N.L. México

lillanes@itesm.mx, albert@itesm.mx,

\end{datos}

Esta investigación presenta la unión del Desarrollo Humano y la Matemática
Educativa como una forma para aprender la Integral Impropia. El Desarrollo
Humano permite el desarrollo del potencial de la persona a partir
cuatro principios y de doce características del profesor facilitador.
A la par se desarrolló una didáctica qué surgió de un estudio histórico-epistemológico
de la Integral Impropia. Se investiga históricamente el concepto de
infinito, el concepto del límite, y el concepto de Integral Impropia.
Se construye actividades que siguen esta epistemología y se facilita
el aprendizaje con Desarrollo Humano. Se presentan resultados cualitativos
y cuantitativos de esta unión. 


\section{\uppercase{ MODELACIÓN Matemática con tecnología para EL APRENDIZAJE
de funciones POLINOMIALES}}

\begin{datos}

Rosario Arenas, Lorenza Illanes Tecnológico de Monterrey Campus Monterrey,

N.L. México,

rosarioarenas800@gmail.com; lillanes@itesm.mx 

\end{datos}

Esta investigación presenta como la Modelación Matemática al utilizar
diversas tecnologías, permite reconocer los conceptos formales relacionados
con las funciones polinomiales dentro del contexto de magnitudes que
cambian; apreciar la modelación como una herramienta teórica en del
estudio de las funciones polinomiales; y el uso de tecnologías dentro
de una problemática que modela diferentes movimientos para lograr
una experiencia integrada de aprendizaje. Se utilizan tecnologías
de diversas índoles y complejidades desde las que graban hasta las
que modelan. Se presentan los resultados de haber experimentado en
tres grupos a nivel profesional y un básico en un curso de Cálculo
I.


\section{MODELACIONES MATEMATICAS PRECURSORAS DEL MUNDO DE LA VIDA DE ADMINISTRADORES
Y ECONOMISTAS}

\begin{datos}

Mónica Soto Márquez, Leonora Díaz Moreno.

Pedagogía en Matemática y Estadística, Universidad Central,

Chile,

monicasotomarquez@gmail.com; leonoradm@gmail.com 

\end{datos}

Este artículo reporta la validación de una pregunta de investigación,
que tiene como propósito explorar modelaciones matemáticas del mundo
de la vida de administradores y economistas. Para ello se aplicó un
instrumento a un conjunto de estudiantes del área, que permitió obtener
evidencias de elementos precursores de modelaciones cotidianas de
esos profesionales. 


\section{\uppercase{ MODELACIÓN Matemática EN EL APRENDIZAJE HÍBRIDO de funciones
lineales }}

\begin{datos}

Rocio Cerecero, Adriana Cantú, Lorenza Illanes. 

Tecnológico de Monterrey Campus Monterrey, N.L.

México,

cerecero@itesm.mx; adriana.cantu@itesm.mx;

lillanes@itesm.mx

\end{datos}

Esta investigación presenta como la Modelación Matemática utilizando
Aprendizaje Híbrido, permite reconocer los conceptos formales relacionados
con la ecuación lineal en dos variables, dentro del contexto de magnitudes
que cambian uniformemente; apreciar la modelación como una herramienta
teórica en del estudio de la función lineal, dentro de una problemática
de llenado de tanques. Se investigaron las características híbridas
de aprendizaje: el estudiante construyó en el salón de clases, aprendió
en línea y adquirió una experiencia de aprendizaje integrada de la
modelación lineal. Se presentan los resultados de haber experimentado
en tres grupos a nivel profesional en un curso de Cálculo.


\section{APROPRIAÇÃO DE TECNOLOGIA DIGITAL NO ENSINO DE GEOMETRIA: EXPERIÊNCIAS
NO GRUPO DE ESTUDOS}

\begin{datos}

Edite Resende Vieira, Nielce Meneguelo Lobo da Costa.

Universidade Anhanguera de São Paulo – UNIAN,

Brasil ,

edite.resende@gmail.com; nielce.lobo@gmail.com

\end{datos}

Neste artigo, discutimos e refletimos sobre as ações e as operações
empreendidas pelas professoras Amora, Jade e La Reine ao realizarem
atividades para familiarização das ferramentas dos software SketchUp,
Régua e Compasso e Construfig3D. Tais discussões e reflexões originam-se
de episódios de uma pesquisa de doutoramento em Educação Matemática,
de natureza qualitativa e cunho co-generativo, realizada em um grupo
de estudos, a qual analisou o processo de apropriação de tecnologia
digital para o ensino de Geometria e o conhecimento profissional docente,
a partir dos estudos de Leontiev (2004), de Shulman (1986) e de Mishra
e Koehler (2006). 

Palavras-chave: Apropriação, Tecnologia Digital, Ensino de Geometria,
Anos Iniciais, Conhecimento Profissional Docente.


\section{DIÁLOGO E SIGNIFICAÇÃO NA RESOLUÇÃO DE POTÊNCIAS COM BASE RACIONAL
E EXPOENTE NEGATIVO DE ESTUDANTES DO 1º ANO DO ENSINO MÉDIO}

\begin{datos}

\textonesuperior{}Maria Helena Palma de Oliveira, \texttwosuperior{}Walter
Aparecido Borges. 

$^{1,2}$Universidade Anhanguera de São Paulo,

Brasil,

\textonesuperior{}mhelenapalma@gmail.com, \texttwosuperior{}w.borges.ltda@terra.com.br

\end{datos}

O estudo analisou o diálogo presente na aprendizagem de potências
com base racional e expoentes negativos de estudantes do 1º ano do
Ensino Médio de escola de São Paulo, como possibilidade de discurso
matemático. A interação verbal é fenômeno social que se realiza na
enunciação, seu tema e significação. Um diálogo comporta réplicas,
a comparação da própria palavra a uma contra-palavra. Evidenciou-se
a importância do diálogo de vozes que se expressaram sob condições
específicas e únicas de produção da linguagem. As relações entre os
participantes (inclusive as de poder) propiciaram a colaboração e
contribuíram para a construção do conceito matemático. 


\section{FACTORES QUE INFLUYEN EN LA DESVINCULACION DE LA ASIGNATURA DE MATEMÁTICA
POR PARTE DE ALUMNAS DE NIVEL SECUNDARIO }

\begin{datos}

Paula Vera Galarce, Claudia Urzúa Contreras, Marcelo Casis Raposo

Universidad Finis Terrae, Colegio Everest,

Chile,

pauvel18@gmail.com; c\_urzua24@hotmail.com ;

marcelcasis@gmail.com 

\end{datos}

Nuestro estudio pretende detectar los factores que inciden en que
una estudiante de enseñanza media (secundaria) termine desvinculándose
afectiva y cognitivamente de la asignatura de matemáticas cuando el
fracaso es reiterado. A través de un estudio de casos, indagamos en
aquellos aspectos que van más allá del propio fracaso y buscamos elementos
cognitivos y afectivos que lo expliquen y con ello, entreguen orientaciones
didácticas a los docentes con el fin de, pese a las malas evaluaciones,
evitar una desvinculación con la disciplina. 



%\setcounter{section}{312}


\section{ÁLGEBRA DE FUNCIONES CON APLICACIONES AL ÁREA ECONÓMICO ADMINISTRATIVAS}

\begin{datos}

Julio Moisés Sánchez Barrera.

Facultad de Estudios Superiores Cuautitlán UNAM,

México,

juliomoisessb@yahoo.com.mx 

\end{datos}

El diseño de éste trabajo está basado en la Ingeniería Didáctica,
la cual a su vez está sustentada principalmente en dos teorías: La
Teoría de Situaciones Didácticas (Guy Brousseau) y la Teoría de la
Transposición Didáctica (Chevallard), que son referentes de la Ingeniería
Didáctica. El objetivo es que el estudiante entienda ¿qué pasa? con
cada una de las operaciones del álgebra de funciones y pueda aplicarlo
en la vida cuando sea necesario, ya que durante el desarrollo de las
actividades plantadas comprenderá las diferentes articulaciones de
representaciones semióticas: el de la forma sagital, tabular, gráfica
y sus aplicaciones. 


\section{COHERENCIA INSTRUCCIONAL DEL FORMADOR DE PROFESORES DE MATEMÁTICA:
CONSTRUCCIÓN DE INDICADORES}

\begin{datos}

Francisco Rojasa, Eugenio Chandía.

Pontificia Universidad Católica de Chile,

Chile,

frojass@uc.cl; echandia@uc.cl

\end{datos}

Considerando que el formador es aquella persona que, en contextos
de formación inicial o permanente, tiene como tarea ayudar a los profesores
a desarrollar y mejorar la enseñanza de las matemáticas, nuestra investigación
se centra en la relación entre las prácticas instruccionales del formador
de profesores y la actividad del docente en el aula escolar. En esta
investigación construimos un conjunto de indicadores de coherencia
instruccional, por medio de grabaciones de clase, entrevistas y cuestionarios
de percepción de coherencia. Se espera consolidar comprensiones profundas
de cómo impacta la práctica instruccional en la construcción de modelos
de enseñanza en los estudiantes.


\section{EXPLORACIÓN DINÁMICA DE PARÁMETROS EN FUNCIONES}

\begin{datos}

Eduardo Basurto Hidalgo.

Benemérita Escuela Nacional de Maestros,

México,

basurtomat@hotmail.com

\end{datos}

Los estudiantes de enseñanza media se enfrentan al uso e interpretación
de los parámetros en funciones polinomiales, lugares geométricos y
expresiones algebraicas en general. Este hecho conduce a la necesidad
no sólo de diferenciar los parámetros de otro tipo de literales como
variables o incógnitas, sino también dar un sentido de uso a los mismos
con la finalidad de agrupar los objetos matemáticos en entidades más
generales como son las familias de funciones. Nuestra investigación
tiene como objetivo mostrar la influencia que puede tener el uso de
un recurso tecnológico dinámico en la comprensión de esta polisemia
de las literales.


\section{LA MODELACIÓN DEL MOVIMIENTO A PARTIR DE SU DESCOMPOSICIÓN}

\begin{datos}

$^{1}$Jaime Arrieta, $^{2}$Rafael Pantoja, $^{3}$Eduardo Carrasco,
$^{3}$Leonora Díaz 

Universidad Autónoma de Guerrero$^{1}$, Universidad de Guadalajara$^{2}$,
Universidad de los Lagos$^{3}$, 

$^{1,2}$México, $^{3}$Chile,

rpantoja@prodigy.net.mx; jaime.arrieta@gmail.com; 

ecarrasc@gmail.com; leonora.diaz@ulagos.cl 

\end{datos}

Se reportan los resultados de un taller de modelación matemática con
estudiantes de la Universidad de Los Lagos, campus Santiago, Chile,
bajo la premisa de la descomposición del movimiento de tres situaciones
problema: tiro parabólico, de un punto sobre una rueda girando sobre
su eje y rodando sin resbalar. Los estudiantes acondicionaron el escenario
y grabaron en video digital los movimientos para analizarlos con el
software Tracker. A partir del video se obtuvieron datos y gráficas
de los movimientos, que se discutieron, primero en equipo y luego
en sesión grupal, con los actores involucrados en la puesta en escena.


\section{DESEMPEÑO DIDÁCTICO DE LOS DOCENTES DE MATEMATICA, DEL PRIMER CICLO,
NIVEL PRIMARIO, Y SU RELACIÓN CON EL FRACASO ESCOLAR: UN REPORTE DE
INVESTIGACIÓN}

\begin{datos}

Geovanny Arturo Lachapell Maldonado, Cila Mola Reyes, Julio Alberto
Mora, Gerardo Quintero Pupo.

Ministerio de Educación de la República Dominicana, Universidad de
Camagüey,

República Dominicana, Cuba,

arturom96@gmail.com; julio.mora@reduc.edu.cu; 

cila.mola@reduc.edu.cu ; gquintero@ucp.cm.rimed.cu.

\end{datos}

El trabajo es un reporte de investigación de un proyecto de doctorado
que parte de las exigencias sociales para el sistema educativo, las
instituciones escolares y la profesionalización de los docentes. Su
objetivo es divulgar los resultados obtenidos en un estudio relacionado
con el desempeño didáctico de los docentes de Matemática del Nivel
Primario. El estudio es de tipo descriptivo-comparativo, se describe
la situación del dominio de los contenidos matemáticos, por parte
de los docentes, y se relaciona con el nivel de fracaso escolar de
los estudiantes de los centros educativos donde enseñan.


\section{DESPLAZAMIENTO DE PRÁCTICAS SOCIOESCOLARES CON BASE EN UNA EXPERIENCIA
DE MODELACIÓN}

\begin{datos}

Camila Contreras Bravo. 

Universidad Católica Silva Henríquez, Pedagogía en Matemáticas e Informática
Educativa,

Chile,

ccontrerasbravo@gmail.com 

\end{datos}

Reportamos el análisis a posteriori de la puesta en escena de un diseño
de aprendizaje basado en prácticas de modelación lineal. Los actores
de la puesta en escena son tres estudiantes de segundo año de enseñanza
media. Analizamos las producciones de los actores al transitar desde
una tabla de datos, del peso que se aplica a un resorte y la posición
de un indicador, al modelo algebraico. Este tránsito se realiza al
enfrentar, los estudiantes, situaciones de predicción que los obliga
a construir una trayectoria de procedimientos de predicción que va
de la tabla de datos a la ecuación .


\section{USO DE RECURSOS DIGITALES ABIERTOS Y SU IMPACTO EN LA COMPRENSIÓN
DEL CONCEPTO DE FRACCIÓN EN CONTEXTOS CONTINUOS EN ESTUDIANTES DE
GRADO SEXTO}

\begin{datos}

Uriel Ancízar López Mancipe, Elvira G. Rincón Flores, Leopoldo Zúñiga
Silva.

Tecnológico de Monterrey - Campus Monterey,

México,

urielop@gmail.com; elvira.rincon@itesm.mx;

lzs@itesm.mx

\end{datos}

A pesar del recelo que existe en Colombia ante la necesidad de renovar
los ambientes de aprendizaje, se propuso investigar el impacto que
tienen los recursos educativos abiertos en el tema de fracción en
contextos continuos. La población seleccionada estuvo situada en un
colegio público de Boyacá, Colombia, formado por 32 estudiantes del
grado sexto provenientes en su mayoría del sector rural. Como resultados
se hallaron: impactos en aspectos motivacionales y en la comprensión
de las relaciones parte y todo. También se encontró que la tecnología
puede ser una herramienta positiva, siempre y cuando, una planeación
analizada la respalde.


\section{PROPUESTA METODOLÓGICA DE FORMACIÓN CONTINUA Y REFORZAMIENTO ACADÉMICO
PARA PROFESORES DE SECUNDARIA. EL TRATAMIENTO DE LA GEOMETRÍA CON
ENFOQUE DINÁMICO}

\begin{datos}

Oliver Texta Mongoy.

Universidad Autónoma de Guerrero, Secretaría de Educación Pública,
Centro de Maestros No. 1218,

México,

matematico22@hotmail.com 

\end{datos}

El escrito de este autor, representa parte de los esfuerzos de un
trabajo de investigación doctoral concluido, divulgado y defendido
encaminado al proceso de formación continua de los profesores de matemáticas
en la escuela secundaria mexicana, y que el mismo autor del presente
tiene la inquietud de compartir en esta Relme XVIII. En él se abordan
entre otras cosas, el origen y el desarrollo que la geometría como
ciencia ha tenido para lograr su consolidación y pertinencia en el
ámbito escolar, así como el análisis que con respecto a su enseñanza
ha presentado en las últimas cuatro décadas del siglo pasado y la
primera del presente, por lo que hace imperiosa la necesidad de la
conformación de una propuesta metodológica para la formación continua
y reforzamiento académico para profesores de secundaria en nuestro
país. 


\section{PROBLEMAS AUTÉNTICOS Y MATEMÁTICA EDUCATIVA. UNA EXPERIENCIA EN EL
NIVEL MEDIO SUPERIOR DE LA UAGRO}

\begin{datos}

René Santos lozano, Santiago Ramiro Velázquez Bustamante.

Universidad Autónoma de Guerrero, Secretaría de Educación Guerrero,

México,

santos\_oasis@hotmail.com; sramiro@prodigy.net.mx 

\end{datos}

En este trabajo se describe y se presentan algunos avances de una
investigación centrada en el desarrollo de actitudes hacia las matemáticas
cuando los alumnos resuelven problemas auténticos en el nivel medio
superior, el objetivo es constatar que los alumnos pueden desarrollar
actitudes positivas al resolver estos problemas. Problemas auténticos
son los que abordan situaciones reales y de relevancia social, que
permiten reconocer diversos usos y significados del contenido matemático.
Para lograr el objetivo realizamos un estudio documental en este ámbito,
un reconocimiento de problemas auténticos y una actividad con profesores
y alumnos.


\section{NÚMERO IRRACIONAL. CONCEPCIONES Y ESTRATEGIAS DE ENSEÑANZA}

\begin{datos}

Prieto Gabriela, Parra S. Hugo.

Universidad del Zulia,

Venezuela,

gabrielacprietof@gmail.com; hps1710@yahoo.es 

\end{datos}

El trabajo que se presenta a continuación es una propuesta de investigación
para alcanzar el grado de magister en educación matemática. Tiene
como propósito identificar y exhibir las diferentes concepciones de
los docentes sobre el número irracional, así como también, las estrategias
que se emplean para su enseñanza. Para tal fin, se plantea un enfoque
cualitativo como metodología de investigación, en particular, la etnografía
educativa. Se espera determinar si las concepciones docentes sobre
el número irracional ejercen ciertas influencias sobre las estrategias
utilizadas, y a su vez, establecer a qué modelo de enseñanza de las
matemáticas responden.


\section{LA UNIVERSIDAD, COMO AGENTE GENERADOR DE CAMBIO EN EL DESARROLLO
DE NÚCLEOS PROBLÉMICOS EN EL ÁREA DE MATEMÁTICAS DE LA ESCUELA EN
EL MUNICIPIO DE VALLEDUPAR}

\begin{datos}

Amalfi Galindo Ospino, Liliana Patricia Barón Amaris.

Universidad Popular del Cesar,

Colombia,

amalfigalindo@unicesar.edu.co; lilianabaron@unicesar.edu.co

\end{datos}

La universidad, en el marco de su proyecto pedagógico institucional,
cumple una función social, que consiste en crear conocimientos y propagarlos,
para desarrollar acciones formativas de calidad. Los docentes, los
alumnos y el saber, son agentes principales para desarrollarla; este
trabajo pretende generar planteamientos, que posibiliten un cambio
en los docentes de matemáticas de la básica, en relación con el enfoque,
la práctica, la evaluación y la planificación de los procesos y el
desarrollo de núcleos problémicos apoyados en las herramientas proporcionados
por la web, sin desconocimiento de los factores que repercuten en
la enseñanza de las mismas.


\section{LA HISTORIA DE LAS MATEMÁTICAS EN EL CURRÍCULO PARA LA FORMACIÓN
DE PROFESORES DE MATEMÁTICA}

\begin{datos}

Marger da Conceição Ventura, Viana Milton Rosa.

Universidade Federal de Ouro Preto (UFOP),

Brasil,

margerv@terra.com.br; milrosa@hotmail.com 

\end{datos}

Es importante conocer qué obstáculos se produjeron en el desarrollo
de las matemáticas, pues problemas de aprendizaje pueden surgir de
los obstáculos epistemológicos; así es importante conocer varias interpretaciones
de los hechos históricos y para entenderlos se los puede confrontar
con ejemplos de situaciones que varían de una cultura a otra. En este
artículo se presentan las justificaciones y las alternativas a la
utilización de la historia de las matemáticas en la enseñanza según
investigaciones realizadas por el autor y otros investigadores; se
la puede utilizar de modo implícito o explícito. Se añaden ejemplos
prácticos de utilización en clase. 


\section{¿SE INTERESA EL ESTUDIANTE EN LEER INSTRUCCIONES EN LA CLASE DE MATEMÁTICA?}

\begin{datos}

Dalys Alvarado.

Universidad de Panamá (Centro Regional de San Miguelito),

Panamá,

crusamatys@gmail.com

\end{datos}

En todos los niveles escolares, la “capacidad lectora” influye significativamente
cuando los estudiantes siguen instrucciones y aplican conocimientos”
en la solución de problemas. No obstante “leer comprensivamente” es
una habilidad no propiamente matemática, sino más bien ligada al español.
Además malentender instrucciones o no seguirlas, puede motivar falla
en actividades de aprendizaje tanto cotidianas, como en exámenes formales
– finales. Esta investigación surge debido a la precaria habilidad
lectora observada al seguir instrucciones en matemática en grupos
de carreras distintas de la Universidad de Panamá, específicamente
en el Centro Regional de San Miguelito. 


\section{PROBLEMAS DE LA CONTEXTUALIZACIÓN DE LA MATEMÁTICA EN DOCENTES EN
FORMACIÓN}

\begin{datos}

Hugo Parra S. 

Universidad del Zulia,

Venezuela,

hps1710@yahoo.es

\end{datos}

Se presentan resultados parciales de una investigación que indagó
las dificultades halladas en docentes en formación para relacionar
la matemática con la realidad. La metodología fue la cualitativa-etnográfica.
Los resultados mostraron una fuerte tendencia a plantear situaciones
cotidianas (66 ,4\%) y una escasa vinculación de la matemática con
otras disciplinas científicas (3,2\%). Según la población estudiada
esto sucedió porque es la cotidianidad el referente fundamental para
poder vincular la matemática con la realidad, idea que tienen graves
implicaciones tanto en el ámbito de la formación de profesores como
en el interés por ofrecer una matemática en y para la vida.


\section{PUNTOS CRÍTICOS EN LA CONCEPTUALIZACIÓN DEL ÁREA QUE CONSTITUYEN
OBSTÁCULOS PARA LA COMPRENSIÓN DE LA INTEGRAL \uppercase{ Una aproximación
desde la teoría de los Modos de pensamiento}}

\begin{datos}

Martha Cecilia Mosquera Urrutia, Francisco Javier Reyes Bahamón, Raimundo
Olfos Ayarza.

Pontificia Universidad Católica de Valparaíso$^{1}$, Universidad
Surcolombiana$^{2}$

$^{2}$Colombia, $^{1}$Chile

martha.mosquera@usco.edu.co; raimundo.olfos@gmail.com 

\end{datos}

Nuestra investigación se enmarca en la línea del pensamiento matemático
avanzado y la problemática se evidencia en la dificultad que presentan
los estudiantes del programa de licenciatura en matemáticas para encontrar
contextos de significación para conceptos vinculados a la medida de
superficies, reduciéndose su conocimiento y su práctica a la deducción
de fórmulas y al manejo de algoritmos. Para su desarrollo se ocupa
el marco teórico de los modos de pensamiento y se diseñan, implementan
y evalúan un conjunto de lecciones de clase; para ilustrar presentamos
una clase sobre los procesos de razonamiento infinito vinculados con
el concepto de área.


\section{IDENTIFICACIÓN DE SITUACIONES DE MODELACIÓN QUE RECONSTRUYEN SIGNIFICADOS}

\begin{datos}

Miguel Solís Esquinca, Hipólito Hernández, Germán Muñoz y Cristóbal
Cruz.

Universidad Autónoma de Chiapas,

México,

solise@unach.mx; polito\_hernandez@hotmail.com;

yaltzil@unach.mx; cristobalcruzruiz@hotmail.com.

\end{datos}

Se presenta una propuesta donde se pretende identificar los usos del
conocimiento matemático y sus resignificaciones. Problematizaremos
la relación entre diferentes dominios de conocimiento que entran en
juego: el discurso matemático escolar, otras disciplinas científicas
y el cotidiano. La problemática se abordará a través de la identificación
de situaciones específicas, denominadas situaciones de modelación.
Las situaciones analizadas están referidas a las prácticas de la ingeniería,
aunque también se abordan algunas de otras disciplinas como la biología,
la química y la economía. Como resultado mostraremos episodios de
aprendizaje dentro de un nuevo marco de referencia, el de una matemática
funcional.


\section{COMPETENCIAS MATEMÁTICAS. DISEÑO Y SELECCIÓN DE TAREAS}

\begin{datos}

Sandra Gutiérrez Otálora, Alicia Guzmán Castro, Edgard Obonaga Garnica.

Escuela Colombiana de Ingeniería,

Colombia, sandra.gutierrez@escuelaing.edu.co; ana.guzman@escuelaing.edu.co;

edgar.obonaga@escuelaing.edu.co 

\end{datos}

Con el fin de contribuir a la organización del currículo de Matemáticas
en las carreras de pregrado de ingeniería, se presenta una propuesta
basada en competencias. Se asocian los procesos matemáticos que los
estudiantes aplican al tratar de resolver problemas con las competencias
matemáticas y se muestra la relación entre competencias, objetivos
específicos y contenidos, en las asignaturas de matemáticas de primer
año. También, se profundiza en el estudio de las competencias matemáticas,
a través del diseño y selección de tareas, que orienten el desarrollo
de las competencias matemáticas y que contribuyan al desarrollo del
perfil del estudiante de ingeniería.


\section{RECONSTRUCCIÓN DEL SIGNIFICADO DE REFERENCIA COMO RECURSO EN LA FORMACIÓN
DE DOCENTES}

\begin{datos}

Bencomo, Delisa, Franzone, Johanna.

Universidad Nacional Experimental de Guayana,

Venezuela,

dbencomo@uneg.edu.ve; jfranzone0113@gmail.com 

\end{datos}

En este trabajo de investigación nos proponemos mostrar cómo las nociones
del enfoque ontosemiótico de la cognición e instrucción matemática
propuesto por Godino y colaboradores, pueden ser útiles para la reconstrucción
del significado de referencia de un contenido, actividad necesaria
para el diseño de un proceso de instrucción adecuado. Para ello, realizaremos
un estudio de caso cualitativo que nos permitirá: a) Determinar los
campos de problemas que se resuelve con un contenido en particular,
b) Construir las configuraciones de objetos intervinientes y emergentes
de los sistemas de prácticas, y c) construir el significado de referencia
(Red epistémica) del contenido matemático.


\section{LA ROBÓTICA COMO MEDIO DIDÁCTICO EN LA ENSEÑANZA DE LAS MATEMÁTICAS
EN LA EDUCACIÓN PRIMARIA :}

\begin{datos}

María del Carmen Bonilla$^{1,2}$, Nancy Juvisa Huamán Avendaño$^{1,3}$,
Leandro Carlos de Souza Gomes$^{1,4}$.

$^{1}$APINEMA (Asociación Peruana de Investigación en Educación Matemática),
$^{2}$Universidad Peruana Cayetano Heredia, $^{3}$Pontificia Universidad
Católica del Perú, $^{4}$Universidade Estadual Da Paraíba,

Perú, Brasil, 

mc\_bonilla@hotmail.com; njhuaman@pucp.pe;

leandrouepb@hotmail.com. 

\end{datos}

En base al marco teórico de la Teoría Antropológica de lo didáctico,
y bajo la metodología de estudio de casos, se aplicó un proceso didáctico
de modelización matemática en el dominio de cambios y relaciones,
en estudiantes de 6to grado de primaria de tres I.E. Públicas, utilizando
el software de robótica educativa WeDo. En el desarrollo del trabajo
se analizó la praxis didáctica que engloba la propuesta. Se pudo demostrar
que el proceso didáctico estimuló una actitud científica en los estudiantes,
dado que ellos elaboraron sus propias construcciones estableciendo
nuevas relaciones entre diferentes variables.


\section{LA REFLEXIÓN DEL PROFESOR DE MATEMÁTICAS EN UNA EXPERIENCIA DE CONSTRUCCIÓN
SOCIAL }

\begin{datos}

Mayra A. S. Baéz Melendres, Rosa María Farfán Márquez.

Cinvestav, 

México,

mbaez@cinvestav.mx; rfarfan@cinvestav.mx 

\end{datos}

Formación de profesores, Medio Básico, Experimental Se pretende dar
cuenta del proceso reflexivo que lleva a cabo el profesor de matemáticas
cuando éste ha vivido una experiencia profesionalizante diseñada bajo
los fundamentos de la Socioepistemología. Partimos de considerar que
la reflexión constituye una herramienta fundamental que permite transformar
los conocimientos y prácticas que se poseen, por tanto, su estimación
en experiencias que consideran al saber matemático como una construcción
social deriva en acciones que contribuyen a su profesionalización.
En esta experiencia, la confrontación juega un papel importante al
normar su desarrollo y promover la reflexión.


\section{LA EVALUACIÓN POR COMPETENCIAS EN EL PROCESO DE ENSEÑANZA APRENDIZAJE
DEL ALGEBRA EN EL NIVEL MEDIO: UN REPORTE DE INVESTIGACIÓN }

\begin{datos}

Yanile Altagracia Valenzuela Calderón, Olga Lidia Pérez González,
Manuel Guardado Hernández, Nancy Montes de Oca.

Universidad Autónoma de Santo Domingo, Universidad de Camagüey,

República Dominicana, Cuba,

yanilevalenzuela@gmail.com; olguitapg@gmail.com;

manuel.guardado@reduc.edu.cu; nancy.montes@reduc.edu.cu 

\end{datos}

El objetivo del trabajo es hacer un reporte de los resultados parciales
de un proyecto de investigación que parte de la problemática existente
en relación a las insuficiencias en el tratamiento de la evaluación
en el proceso de enseñanza aprendizaje (PEA) del Álgebra en el nivel
medio. Se defiende la idea de que si se incide en la contradicción
dialéctica entre las exigencias sociales de la evaluación por competencias
y las concepciones y prácticas evaluativas de los profesores; adecuada
al contexto del Álgebra, se contribuye a favorecer el desarrollo de
la evaluación por competencias en el PEA del Álgebra. 


\section{ACEPTANDO LA EXISTENCIA DE UNA FIGURA INEXISTENTE: LOS ARGUMENTOS
DE PROFESORES Y ESTUDIANTES DE TOPOGRAFÍA}

\begin{datos}

Lidia Aurora Hernández Rebollar, José Antonio Juárez López, Josip
Slisko Ignjatov.

Facultad de Ciencias Físico Matemáticas - Benemérita Universidad Autónoma
de Puebla,

México, 

lhernan@fcfm.buap.mx; jajul@fcfm.buap.mx;

jslisko@fcfm.buap.mx 

\end{datos}

En esta investigación se reportan los diferentes argumentos que presentaron
un grupo de profesores y estudiantes de Topografía cuando se les cuestionó
acerca de la existencia de un terreno el cual apareció en el contexto
de una actividad del libro de texto de matemáticas de 5º grado de
primaria en México. El terreno es un polígono irregular pero que,
con las medidas de los lados que ahí aparecen, es imposible que exista
en la realidad. Se analizan los argumentos tanto de los que detectaron
el error como de los que no lo detectaron y se ofrece una clasificación
de éstos. 


\section{PROCESOS EN LA COMPRENSIÓN DEL CONCEPTO DE BASE DE UN ESPACIO VECTORIAL}

\begin{datos}

Vera Soria Ma. Guadalupe, Miranda Montoya Eduardo.

Universidad de Guadalajara, ITESO,

México,

guadalupe.vera@red.cucei.udg.mx; emiranda@iteso.mx

\end{datos}

Se presenta el avance de un proyecto doctoral en el que se investiga
cómo son los procesos en la comprensión del concepto de base para
un espacio vectorial en estudiantes universitarios. Se trata de un
estudio cualitativo e interpretativo que busca analizar y describir
la cadena de inferencias que los participantes realizan, a partir
de los lenguajes geométrico, simbólico y estructural. El marco teórico
para fundamentar la investigación es el modelo de la comprensión en
matemáticas de Anna Sierpinska (Sierpinska, 1994) y el modelo de la
distinción epistemológica teórico-práctico de Sierpinska y colaboradores
(Sierpinska, Nnadozie y Oktaç, 2002). 


\section{USO DE LA SUMATORIA PARA ACERCARSE AL CONCEPTO DE INTEGRAL COMO SUMA
DE RIEMANN}

\begin{datos}

José Daniel Galaz Arraño Universidad Central de Chile ,

Chile ,

jose.galaz@ucentral.cl

\end{datos}

Luego de un análisis de los programas curriculares chilenos y españoles,
se realiza una propuesta didáctica que proporciona al estudiante de
secundaria de temáticas propias de la educación universitaria, particularmente
la utilización de la sumatoria para acercarse al concepto de integral.
Se diseña una serie de actividades donde el estudiante es participe
de su propio aprendizaje, buscando diversas estrategias de resolución
a las problemáticas planteadas. Al analizar las producciones estudiantiles
luego de la aplicación de la propuesta, se puede observar qué tan
naturales son estos conceptos para los estudiantes, como nace la idea
de partición, suma inferior y superior.


\section{PARADOJA DE ZENON CON SOPORTE EN VIDEOS ¿RECURSO DIDÁCTICO PARA INTUIR
CONVERGENCIA? }

\begin{datos}

Víctor Arias, Claudio Báez, Pía Galaz, Juan Pablo López, Jorge Vargas,
Leonora Díaz.

Universidad Central de Chile,

Chile,

v.a.naipayan@gmail.com; profebaez21@gmail.com;

piagalaz12@gmail.com; juanpablopezc@gmail.com;

jorg.x3@gmail.com; leonoradm@gmail.com

\end{datos}

La comprensión y entendimiento de nociones principales del cálculo
requieren un alto nivel de abstracción. Es apremiante identificar
elementos precursores en la secundaria, que familiaricen tempranamente
a los estudiantes con la idea de convergencia. Esta investigación
exploró el estudio de la convergencia mediante interpretaciones, con
apoyo en un video de la paradoja de Zenón. Tomando como base una investigación
de diseño y experimentos de enseñanza se aplicaron tres versiones
de un diseño didáctico con estudiantes de secundaria. Los estudiantes
expresan nociones intuitivas de convergencia, y sería posible conjeturar
como nociones intuitivas, pueden ser precursoras de la noción matemática
de convergencia.


\section{LA TABLETA ELECTRÓNICA, LA GÉNESIS INSTRUMENTAL Y LOS PROFESORES
DE BACHILLERATO}

\begin{datos}

Jorge Alonso Santos Mellado; Claudia Margarita Acuña Soto.

Centro de Investigación y de Estudios Avanzados. 

México. 

lonchoboy@yahoo.com.mx ; claudiamargarita\_as@hotmail.com

\end{datos}

El presente trabajo reporta los resultados obtenidos luego de intervenir
en un curso para profesores de bachillerato cuyo objetivo es integrar
el uso de tabletas electrónicas en el aula a través de familiarizarse
con su funcionamiento y aprender a diseñar actividades utilizándola.
A partir de proponer la Génesis Instrumental como referente teórico,
observamos que tal curso se centra en la instrumentalización y desconoce
la instrumentación. Es necesario que los profesores construyan Esquemas
de Utilización que les permitan concebir la tableta como objeto educativo
y cambiar su percepción con respecto a su potencial y utilidad en
el salón de clase. 


\section{ROL DE TUTORES PARES EN CURSOS DE PRIMER AÑO DE INGENIERÍA }

\begin{datos}

Fredna Tatiana Riquelme Quezada.

Universidad Austral de Chile, 

Chile,

fredna.riquelme@uach.cl 

\end{datos}

El apoyo a los estudiantes de primer ingreso mediante tutores pares,
en los cursos de Álgebra y Geometría de la Facultad de Ciencias de
la Ingeniería de la Universidad Austral de Chile, es una iniciativa
reciente que apunta a orientar a los estudiantes en su proceso de
inserción a la Universidad. Este reporte recoge información de estudiantes,
profesores y tutores pares, para caracterizar el rol de un tutor par
en el proceso de inserción a las carreras de ingeniería, lo cual contribuye,
entre otros, a direccionar, de manera adecuada y coherente, el sistema
de tutorías por pares. 

Las categorías y subcategorías encontradas en la información, proporcionada
por los distintos estamentos, dicen relación con expectativas con
respecto a la inserción a la ingeniería y el rol del tutor par en
dicho proceso, cualidades personales del tutor, méritos académicos
y competencias asociadas a la labor de tutor. Todas ellas entregan
un panorama que permite caracterizar el perfil y el rol del tutor
en el contexto de las asignaturas de matemática del primer año de
ingeniería en la Universidad Austral de Chile.


\section{EL MUSEO DE CIENCIAS, ALTERNATIVA EN EDUCACIÓN MATEMÁTICA}

\begin{datos}

Henry Gallardo Pérez, Mawency Vergel Ortega, Olga Lucy Rincón.

Universidad Francisco de Paula Santander,

Colombia,

henrygallardo@ufps.edu.co; mawency@ufps.edu.co;

olgarincon@ufps.edu.co 

\end{datos}

Educación Superior, Modelación Matemática 

La experiencia permite implementar metodologías de enseñanza que le
permitan al estudiante adentrarse en el mundo matemático, conceptualizar
y aplicar el conocimiento adquirido con el propósito de profundizar
en los aprendizajes para desarrollar competencias matemáticas; también
le permite elaborar prototipos y montajes interactivos para exposición
en el Museo Interactivo a partir de dos componentes esenciales la
Tecnología y la Pedagogía; esta última tiene tres variables fundamentales:
interacción, mediación pedagógica y retroalimentación. Entre los principales
resultados se tienen: Desarrollo de competencias; Aprendizaje significativo
en matemáticas; Mejor rendimiento académico; Construcción de montajes
interactivos, Formación de orientadores y diseñadores para el Museo
Interactivo 


\section{LA CENTRACIÓN EN LOS OBJETOS MATEMÁTICOS COMO DIFICULTAD EN EL DESARROLLO
DEL PENSAMIENTO Y LENGUAJE VARIACIONAL }

\begin{datos}

Mario Caballero Pérez, Ricardo Cantoral Uriza.

Cinvestav, 

México,

macaballero@cinvestav.mx; rcantor@cinvestav.mx 

\end{datos}

Exponemos los resultados de una investigación cuyo objetivo fue el
de analizar las causas que originan dificultades en profesores de
bachillerato al llevar a cabo actividades del Pensamiento y Lenguaje
Variacional. Para tal fin se diseñaron un conjunto de actividades
que permitieron analizar las estrategias de respuesta a las que recurrían
los profesores. Los resultados muestran que la centración en los objetos
matemáticos no sólo no deja ver el carácter variacional del Cálculo,
sino que también es el origen de las dificultades para el desarrollo
del Pensamiento y Lenguaje Variacional.


\section{\uppercase{ Explorando el Conocimiento especializado de una maestra
investigativA y su relación con la práctica.}}

\begin{datos}

$^{1}$Álvaro Aguilar; $^{1}$Miguel Montes; $^{1}$José Carrillo,
$^{2,3}$C. Miguel Ribeiro.

$^{1}$Universidad de Huelva, $^{2}$Centro de Investigación sobre
el Espacio y las Organizaciones - Universidad de Algarve, $^{3}$UNESP
- Rio Claro 

República Dominicana, Portugal, Brasil,

alaguilargon@gmail.com; miguelmontesnavarro@gmail.com; 

cmribeiro@ualg.pt 

\end{datos}

En este artículo se reportan avances de una investigación buscando
una más amplia comprensión del conocimiento matemático especializado
y las relaciones de dicho conocimiento con la práctica de una maestra
(estudio de caso). Con ese objetivo analizamos la práctica según el
Mathematics Teachers Specialized Knowledge–MTSK (Carrillo et al.,
2013) en una clase de geometría que ha sido video grabada. Los primeros
resultados indican que el conocimiento matemático especializado junto
con la concepción que tiene la maestra sobre la perspectiva de Resolución
de Problemas, juega un papel fundamental para el desarrollo de una
actividad matemática, y cómo este conocimiento es un eje central de
la actividad.


\section{UN ESTUDIO SOBRE LA COMPRENSIÓN DE GRÁFICAS ESTADÍSTICAS DE PROFESORES
DE SECUNDARIA}

\begin{datos}

Fredy Carranza Herrera, Elika Sugey Maldonado Mejía.

Universidad Autónoma de Guerrero,

México,

fch\_mat2@hotmail.com; elika.mm@hotmail.com 

\end{datos}

La lectura e interpretación adecuada de las gráficas estadísticas
que contempla actualmente el currículo escolar mexicano y que se presentan
en los medios de información, nos brindan la posibilidad de tomar
decisiones acerca de sucesos importantes que pasan en nuestro entorno.
En este sentido, se pretende presenta un estudio acerca de la comprensión
gráfica de profesores de matemáticas de educación secundaria en el
estado de Guerrero, México, con la intención de identificar los niveles
de comprensión gráfica, así también las dificultades que presentan
los profesores al momento de trabajar con gráficas estadísticas.


\section{CÓNICAS DESDE LA DIALÉCTICA PUNTUAL - GLOBAL: UNA SECUENCIA DIDÁCTICA
INTEGRANDO UN AMBIENTE DE GEOMETRÍA DINÁMICA}

\begin{datos}

Edinsson Fernández Mosquera, María Fernanda Mejía.

Palomino Universidad de Nariño, Escuela Normal Superior Farallones
de Cali,

Colombia,

edi454@yahoo.com; edinfer@udenar.edu.co;

mafanda1216@yahoo.com.ar; mafanda1216gmail.com

\end{datos}

Esta investigación se efectuó en el marco de un curso de Geometría
Analítica con 25 profesores en formación de Licenciatura en Matemáticas,
en la Universidad de Nariño, Colombia. En la misma, se diseñó una
secuencia didáctica para el aprendizaje de las cónicas (parábola,
elipse e hipérbola) vistas como lugares geométricos y mediados con
el Ambiente de Geometría Dinámica Cabri II Plus. Para el marco teórico,
se tuvo en cuenta tres dimensiones: la Histórica – Epistemológica,
la Cognitiva y la Didáctica y como marco metodológico la micro-ingeniería
didáctica. Uno de los resultados fue el diseño de ocho situaciones
desde la dialéctica puntual – global.


\section{TRAYECTORIAS DE ALGORITMOS DE PREDICCIÓN EN MODELACIÓN LINEAL }

\begin{datos}

Carol Sepúlveda Herrera, Leonora Díaz Moreno, Jaime Arrieta Vera.

Universidad de los Lagos, Universidad Autónoma de Guerrero, 

Chile, México,

krol.cesh@gmail.com; leonora.diaz@gmail.com;

jaime.arrieta@gmail.com

\end{datos}

Este trabajo presenta el análisis de las producciones de estudiantes
de secundaria al participar en la puesta en escena de un diseño de
aprendizaje basado en la modelación lineal del llenado de un reciente
cilíndrico. Ante situaciones de modelación que el diseño propone los
estudiantes recurren a diferentes formas de predicción configurando
trayectorias que si bien son diferentes existen invariantes que dan
cuenta de una normativa. La perspectiva teórica que adoptamos es la
Teoría Socioepistemológica de la Matemática Educativa (Cantoral, 2014)
y la metodología utilizada corresponde a la “investigación de diseño
y experimentos de enseñanza” (Molina, Castro y Molina, 2011). 


\section{LA DEMOSTRACIÓN MATEMÁTICA EN EL CONTEXTO DE LA PEDAGOGÍA}

\begin{datos}

José Fermín Berríos Piña.

Universidad Pedagógica Experimental Libertador,

Venezuela,

joseferminp@gmail.com

\end{datos}

Pensamiento algebraico, superior (19-22 años), Etnográfico/interpretativo

Diversas investigaciones han hecho notar las dificultades que tienen
los estudiantes en torno a la demostración matemática. Esta investigación
tuvo como propósito interpretar y comprender, desde la perspectiva
de los actores involucrados, las dificultades con que se encuentran
los estudiantes de la carrera: Docencia en Matemática en la Universidad
Pedagógica Experimental Libertador, en el proceso de la demostración
matemática. Este estudio está enmarcado en el paradigma pospositivista,
bajo el enfoque cualitativo. El método utilizado, hermenéutico porque
se interpretó y comprendió el sentir de los estudiantes. Las técnicas
utilizadas fueron: observación participante, entrevista no estructurada
y el grupo focal. 


\section{CONSTRUYENDO SIGNIFICADOS PARA LA FACTORIZACIÓN DE EXPRESIONES CÚBICAS
EN EL CONTEXTO DEL CÁLCULO DE VOLUMENES DE RECIPIENTES}

\begin{datos}

Jorge Ávila Soria, Maricela Armenta.

Universidad de Sonora, 

México,

javilas9@gmail.com; maricela@gauss.mat.uson.mx 

\end{datos}

La factorización de ecuaciones es difícil de dominar para un alto
porcentaje de estudiantes, esto a pesar o quizás por el enfoque memorista
con que se enseñan los productos y factores notables. La enseñanza
del tema se convierte en un dolor de cabeza en el nivel superior donde
las materias de matemáticas de los primeros semestres de las carreras
de ingeniería lo pueden convertir en un filtro de así quererse. Creemos
que la construcción de significados para la factorización en un contexto
dado, permite que los estudiantes identifiquen la estructura del factor
y reconozcan elementos relacionados con el problema estudiado. 


\section{RESOLUÇÂO DE PROBLEMAS ADITIVOS ATRAVÈS DO JOGO }

\begin{datos}

Claudia Gomes Araujo, Gabriela dos Santos Barbosa UERJ/FEBF – Brasil
/ UERJ/FEBF – Brasil gomesaraujo67@gmail.com / gabrielasb80@hotmail.com

\end{datos}

Pensamento Numérico / Nível Básico / Estudo de casos

O objetivo deste trabalho é investigar como se caracterizam os processos
desenvolvidos pelos alunos do terceiro do ciclo na resolução de problemas
do campo aditivo. Fundamentadas na Teoria dos Campos Conceituais desenvolvemos
e analisamos um teste diagnóstico e uma intervenção de ensino. Durante
a intervenção, os alunos se depararam com várias situações-problema
e, à medida que a grandeza dos números aumentava, perceberam não ser
mais possível resolver os problemas contando nos dedos. Começaram
a refletir sobre as propriedades dos números e desenvolveram habilidades
de cálculo mental. Assim, simultaneamente avançavam na compreensão
das estruturas aditivas e no pensamento numérico.


\section{CONTRIBUIÇÕES DA INICIAÇÃO À DOCÊNCIA PARA A COMPREENSÃO DE FUTUROS
PROFESSORES SOBRE A CONSTRUÇÃO DO NÚMERO}

\begin{datos}

Gabriela dos Santos Barbosa.

UERJ - FEBF,

Gabrielasb80@hotmail.com

\end{datos}

Este artigo relata uma experiência vivenciada por uma aluna, bolsista
de Iniciação à Docência, de um curso de Licenciatura em Matemática
do Rio de Janeiro. Buscamos identificar as contribuições deste estágio
para a sua formação profissional. Em parceria com a coordenação do
projeto, a bolsista atuou como mediadora das atividades computacionais
que enfatizavam o conceito de número propostas às crianças que cursam
do 1º ao 5º ano do Ensino Fundamental na rede pública. Ao final, foi
possível à estudante: compreender a construção do conceito de número
como um processo, identificar as etapas e reconhecer o papel do erro
neste processo.


\section{“CUANDO UNA CRECE, LA OTRA DECRECE”… LA PROPORCIONALIDAD VA UN POCO
MÁS ALLÁ }

\begin{datos}

Daniela Reyes-Gasperini{*}, Ricardo Cantoral{*}, Gisela Montiel{*}{*}.

{*}Cinvestav-IPN, {*}{*}Instituto Politécnico Nacional - CICATA, 

México,

dreyes@cinvestav.mx, rcantor@cinvestav.mx;

gmontiel@ipn.mx 

\end{datos}

El análisis de marcos de referencia como los índices antropométricos,
leyes físicas o químicas, economía y Derecho permitirá entender la
idea germinal subyacente a la epistemología del pensamiento proporcional,
brindando una nueva base de significaciones y usos del saber, para
luego proponer una modificación en la epistemología que subyace al
dME. Concluimos hipotetizando que ante la incapacidad de medir lo
inconmensurable debemos comparar de ahí que la búsqueda de la “relación
adecuada” entre cosas diversas habrá de mantenerse en el ámbito normativo
humano y consecuentemente el principio de proporcionalidad, comienza
a tomar una relevancia hasta el momento no cuestionada.


\section{LA GRAFICACIÓN Y LA COMPARACIÓN COMO MEDIO PARA DISCERNIR LO CUADRÁTICO
DE LO NO CUADRÁTICO}

\begin{datos}

Rosa Gorocica Titla; Irene Pérez Oxté.

CINVESTAV-IPN,

México,

marosagtitla@gmail.com; iperezo@cinvestav.mx 

\end{datos}

Se presentan avances para la elaboración de una propuesta de diseño
para la construcción de la noción de función cuadrática. De acuerdo
a lecturas de investigación, el diseño considerará ejes centrales
como la argumentación y la optimización, así como tareas de cuantificación,
predicción y toma de decisiones. La finalidad es indagar las estrategias
que generan estudiantes de secundaria ante el diseño para discernir
entre un comportamiento cuadrático y uno no cuadrático. La articulación
de tales elementos en un diseño se consideran posibilitarán la construcción
de la noción de función cuadrática.


\section{EXPLORANDO LA COMPRENSIÓN DE FUNCIONES ELEMENTALES}

\begin{datos}

Asela Carlón M, Sergio Cruz C. 

UNAM. FES, ACATLÁN,

México,

asela.carlon@ gmail.com; correoaselasergio@gmail.com 

\end{datos}

A 70 alumnos de 4º semestre de bachillerato se les imparte una instrucción
con el propósito de promover un aprendizaje con comprensión de funciones
polinomiales elementales y desarrollar la automatización del cambio
de sus registros de representación gráfico y algebraico. Después de
lograr lo último, enfrentan una prueba de rendimiento que al analizar
sus resultados, los alumnos muestran haber alcanzado tal aprendizaje
con comprensión. El instrumento se diseña y los resultados se analizan
de acuerdo a los lineamientos teóricos de Hiebert y Carpenter (1992)
y Moschkovich, Schoenfeld y Arcavi (1992).

\setcounter{section}{352}


\section{HACIA UNA PROPUESTA DE FORMACIÓN DEL PROFESOR DE MATEMÁTICA DE EDUCACIÓN
MEDIA }

\begin{datos}

Sandra Castillo, Zoraida Pérez.

Universidad Nacional Experimental de Guayana,

Venezuela,

sandralilianacastillo@gmail.com; zoraidaperezs@gmail.com

\end{datos}

Con esta investigación se pretende hacer una aproximación a las realidades
percibidas por los propios profesores, en relación a su proceso de
formación permanente. Desde una perspectiva interpretativa y fenomenológica,
el proceso analítico se desarrollará de acuerdo a los principios y
procedimientos de la Teoría Fundamentada. Se espera que de este estudio
emerjan dimensiones esenciales de los datos, que ayuden a enriquecer
la fundamentación de propuestas curriculares integradoras, de tal
manera que puedan éstas contar con alta pertinencia y calidad para
la formación del profesor de matemática, donde se busque atender las
necesidades reales manifestadas por ellos, apuntando al mejoramiento
de la enseñanza de la matemática en la región. 

