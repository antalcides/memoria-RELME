
\pagestyle{headings}
\nocite{*}
\fontsize{7}{8}\selectfont
%\setlength{\baselineskip}{5pt}
\pagecolor{white} 

\onecolumn
\chapter{Pósters(Carteles) } 
\renewcommand\thesection{PO\ \nplpadding{3}\numprint{\arabic{section}}} 
\setcounter{section}{0}
\chaptertoc
\twocolumn
\balance





\section{DISEÑO Y GESTIÓN EN LA ENSEÑANZA-APRENDIZAJE DE LA FORMACIÓN DOCENTE
EN LA LICENCIATURA EN EDUCACIÓN BÁSICA CON ÉNFASIS EN MATEMÁTICAS
(LEBEM) EN EL PERIODO 2005 -2007}

\begin{datos}

Jenny Paola Rodríguez Gámez, Luis Gabriel González Cantor.

Universidad Distrital Francisco José de Caldas,

Colombia,

paolitastar200@hotmail.com; brothersgc@hotmail.com

\end{datos}

El documento presenta una descripción de las fases desarrolladas en
el proyecto de investigación “Desarrollo de las prácticas en la LEBEM
en el periodo 2005-2012” del grupo de investigación CRISALIDA, adscrito
a la Universidad Distrital; Así mismo presenta unas filtros de análisis
realizados en el desarrollo de la investigación de tipo documental,
sobre el análisis de la formación docente de los estudiantes para
profesores de matemática de la universidad en donde se desarrollan
las prácticas educativas, sistematizando y clasificando la información
de las unidades didáctica por medio de criterios fijados en el proyecto.


\section{PRÉ-VESTIBULAR NA UNIOESTE CAMPUS DE FOZ DO IGUAÇU: UM RESGATE HISTÓRICO }

\begin{datos}

Renata Camacho Bezerra, José Ricardo Souza.

Universidade Estadual do Oeste do Paraná-UNIOESTE - Campus de Foz
do Iguaçu/PR, Br. 

Brasil,

renata.bezerra@unioeste.br; jose.souza2@unioeste.br 

\end{datos}

Este artigo relata o trabalho desenvolvido por discentes e docentes
dos cursos de Enfermagem, Engenharia Elétrica, Engenharia Mecânica,
Letras, Matemática e Pedagogia, vinculados ao projeto de extensão
“Curso Pré-Vestibular” que acontece desde 2005 numa parceria entre
a Universidade Estadual do Oeste do Paraná – UNIOESTE Campus de Foz
do Iguaçu, a Fundação Parque Tecnológico de Itaipu – FPTI e o Instituto
de Tecnologia Aplicada e Inovação – ITAI e tem como objetivo principal
preparar alunos carentes, oriundos da rede pública de ensino, que
tenham boas notas e que pretendem prestar o vestibular na universidade
pública.


\section{MODELADO NUMERICO CON CURVAS BÉZIER }

\begin{datos}

Rogelio Ramos Carranza, Armando Aguilar Márquez, Frida María León
Rodríguez, Omar García León, Juan Rafael Garibay Bermúdez.

Universidad Nacional autónoma de México,

México,

egorrc@gmail.com; armandoa@unam.mx;

fridam@unam.mx; egor1131@unam.mx;

juragabe@unam.mx. 

\end{datos}

En esta investigación se plantea reunir los documentos que conforman
el modelado de curvas mediante el algoritmo de Bézier. Una vez conjuntados
los documentos necesarios se propondrán para su estudio y aplicación
utilizando una aproximación mediante la teoría Vygotskyana. 

Se denomina curvas de Bézier a un método desarrollado en 1960, para
el trazado de dibujos técnicos, en el diseño aeronáutico y en el de
automóviles. Su denominación es en honor a Pierre Bézier, quien ideó
un método de descripción matemática de las curvas. Proporcionando
un modelo numérico para describir un modelo geométrico. Las curvas
fueron desarrolladas por Paul de Casteljau.


\section{UN JUEGO DE DOMINÓ PARA LA ENSEÑANZA DE LOS CONCEPTOS DE COMBINATORIA
Y PROBABILIDAD}

\begin{datos}

José Marcos Lopes, Jaime Edmundo Apaza Rodriguez.

Universidade Estadual Paulista “Júlio de Mesquita Filho” – UNESP,

Brasil,

jmlopes@mat.feis.unesp.br; jaime@mat.feis.unesp.br 

\end{datos}

En este trabajo presentamos un juego de dominó (original) asociado
al triangulo de Pascal, el cual puede ser utilizado para la enseñanza
de los conceptos fundamentales referentes al cálculo de probabilidades
y combinatoria. Este juego utiliza las 28 piezas de un dominó comun.
La disposición de las piezas debe ser realizada en un tablero de forma
triangular y sujeta a determinadas reglas. Ese formato debera corresponder
a las siete primeras filas del triangulo de Pascal. Con base en este
juego, formulamos algunos problemas que pueden ayudar a subsidiar
el trabajo del professor que se dedica a enseñar estos conceptos.


\section{CONSTRUCCION DE OBJETOS TRIGONOMETRICOS POR MEDIO DE UNA VISUALIZACION
GEOMETRICACONTENIDA EN UN REGISTRO DE REPRESENTACION SEMIOTICA DE
DUVAL}

\begin{datos}

Oscar Jesús San Martín Sicre sicreo@outlook.es Universidad 261 Pedagógica
Nacional-Instituto de Formación Docente, Zonora, México 

\end{datos}

Se presenta un poster o cartel que contiene una visualización geométrica
contenida en un registro de representación semiótica de Duval que
ha posibilitado un tratamiento novedoso de algunos contenidos trigonométricos.
Esta visualización permite construir y representar por medios geométricos
diversos objetos trigonométricos tales como: las definiciones de las
funciones trigonométricas, las identidades trigonométricas y las ecuaciones
trigonométricas. Se describe la figura que sirve de base para la elaboración
del cartel. Esta visualización provee además una base geométrica para
trabajar estos mismos resultados utilizando softwares de geometría
dinámica tales como Cabrí Geometre, Geometers Sketch Pad o Geogebra. 


\section{EXPERIENCIA METODOLÓGICA SOBRE RESOLUCIÓN DE PROBLEMAS INGENIERILES
QUE SE RESUELVEN A TRAVÉS DE LA INTERPOLACIÓN Y LA REGRESIÓN. }

\begin{datos}

Edfram Rodríguez Pereira, Pedro Castañeda Porras, Sergio Abraham,
Pedro Fernández de Córdoba, Patricia Mayo, Arelys Quintero Silverio. 

Universidade do Estado do Amazonas, Universidad de pinar del Río,
Universidad Politécnica de Valencia,

Brasil, Cuba, España,

pcasta@mat.upr.edu.cu; sabraham@mat.upv.es;

pfernandez@mat.upv.es; pmayo@mat.upv.es; edfram\_2008@hotmail.com
;arelys@mat.upr.edu.cu. 

\end{datos}

En este trabajo trataremos algunas experiencias sobre cómo enseñar
en las carreras ingenieriles los contenidos relacionados con la Interpelación
Polinomica y la Regresión. Dicha experiencia está enmarcada en un
proyecto de investigación que llevamos a cabo entre las Universidades
Politécnica de Valencia, España y la de Pinar del Río, Cuba. Queremos
resaltar una metodología para resolver algunos problemas de carácter
ingenieril que se resuelven a través de la interpolación y la regresión,
sobre todo hacer hincapié en el uso de un asistente matemático y así
los estudiantes pueden conocer más profundamente los algoritmos.


\section{PROPUESTA METODOLÓGICA PARA EL DESARROLLO DEL ALGEBRA LINEAL EN CARRERAS
DE PERFIL INFORMATICO CENTRADA EN EL METODO DE PROYECTO}

\begin{datos}

Oscar A. González Chong, Rogerio dos Reis Gonçalves , Miguel Tadayuki
Koga.

Universidade Estadual de Mato Grosso, 

Brasil,

oscarchong@unemat-net.br; rogerio@ unemat-net.br;

miguelkoga@unemat-net.br 

\end{datos}

El trabajo es una propuesta metodológica para modificar la disciplina
de Algebra Lineal en carreras de perfil informático, que muestre la
vinculación de esta disciplina con la actividad profesional. Nuestra
propuesta está centrada en el método de proyecto, que sugiere variaciones
en las formas de enseñanza, en cuanto al aumento del número de actividades
prácticas y laboratorios, donde los estudiantes utilizarán los conceptos
de esta disciplina en disímiles aplicaciones, resolverán ejercicios
básicos de la informática, robótica y telecomunicaciones, y desarrollarán
habilidades en el uso de un asistente matemático que será el medio
donde desarrollarán su proyecto final, el cual presentarán en seminarios. 


\section{ECUA-PARQUÉS UNA ALTERNATIVA INNOVADORA PARA LA ENSEÑANZA Y EL APRENDIZAJE
DE LAS ECUACIONES LINEALES}

\begin{datos}

Sandra Viviana Mora Gómez, William Andrés Cárdenas, Lyda Constanza
Mora Mendieta.

Universidad Pedagógica Nacional,

dma\_sgomez913@pedagogica.edu.co; dma\_wcardenas989@pedagogica.edu.co;

lmendieta@pedagogica.edu.co

\end{datos}

El “Ecua-parqués” es un material didáctico manipulativo, que se creó
basándose en la idea de usar un tablero de parqués tradicional para
resolver algunas ecuaciones lineales, que modificado, permite solucionar
cualquier ecuación lineal de la forma $x+a=b$ planteada con ayuda
de un par de dados, en donde $a,\text{\,\ensuremath{b}}$ y $x$ son
números enteros, $-6\leq a\leq6$ y $-12\leq b\leq12$ , mostrando
de forma directa la propiedad uniforme de la igualdad con la adición
entre números enteros y la existencia del opuesto aditivo, promoviendo
en los estudiantes la abstracción, mediante la experimentación a través
del juego y la competencia entre pares.


\section{LAS CÓNICAS DE LA GEOMETRÍA DEL TAXISTA }

\begin{datos}

Daniela Bonilla Barraza, Marcela Parraguez González, Leonardo Solanilla
Chavarro.

Universidad de La Serena (Chile). Pontificia Universidad Católica
de Valparaíso (Chile). Universidad del Tolima (Colombia). danielabonillab@gmail.com;
marcela.parraguez@ucv.cl;

leonsolc@ut.edu.co.

\end{datos}

La presente investigación, muestra elementos de una propuesta didáctica
que busca la comprensión de las cónicas en estudiantes entre 16 y
18 años, bajo un enfoque cognitivo, donde se utiliza los modos de
pensamiento de Anna Sierpinska como marco teórico. Postulamos como
hipótesis de investigación, que para lograr una comprensión de las
cónicas es necesario que el aprendiz de estos tópicos transite entre
los distintos modos de comprenderla: SG (figuras que las representan),
AA (pares ordenados que satisfacen una ecuación) y AE (lugar geométrico),
para ello, utilizamos como sistema de referencia el plano discreto
de la geometría del taxista. 


\section{“PROYECTO AULA: ACTIVACION FISICA CON MATEMATICAS” }

\begin{datos}

Guillermo Carrasco García.

Centro de Estudios Científicos y Tecnológicos 9 del I.P.N.,

México,

utejdb@ipn.mx

\end{datos}

Proyecto Aula es una propuesta metodológica utilizada en los CECyTs
del IPN, que incorpora las Asignaturas del semestre para la solución
de un problema, a partir de un proyecto, aplicando a través del proceso
enseñanza-aprendizaje, estrategias didácticas que permitan al estudiante
adquirir conocimientos, habilidades y actitudes. Este trabajo muestra
los resultados de la actividad realizada por los estudiantes de Geometría
y Trigonometría, que consistió en planear y ejecutar una “Rutina de
activación física con base en la realización de figuras geométricas”,
como parte del Proyecto “Importancia de la actividad física en el
ámbito escolar para una vida sana”.


\section{PROYECTO FUNDER ETNOMATEMATICA: CONSTRUCCIÓN DE UNIDADES DIDÁCTICAS
CONTEXTUALIZADAS}

\begin{datos}

Ana Patricia Vásquez Hernández, Eithel Trigueros Rodríguez.

Universidad Nacional de Costa Rica,

Costa Rica,

patrimate76@gmail.com; eitheltr@gmail.com 

\end{datos}

El presente poster muestra la descripción de un proyecto de etnomatemática
con fondos para el desarrollo de las regiones, con vigencia 2014-2015,
coordinado por el Campus Sarapiquí de la Universidad Nacional de Costa
Rica. Su objetivo es desarrollar capacidades académicas para la confección
colectiva de obras didácticas de matemática contextualizadas y validadas
por territorios indígenas, para el fortalecimiento de la identidad
cultural y el respeto por el derecho indígena a un sistema de educativo
intercultural. 


\section{NIVELES DE RAZONAMIENTO ALGEBRAICO EN LA ACTIVIDAD MATEMÁTICA DE
MAESTROS EN FORMACIÓN: ANÁLISIS DE TAREAS ESTRUCTURALES}

\begin{datos}

$^{1}$Lilia Aké, $^{2}$Walter F. Castro, $^{1}$Juan D. Godino. 

$^{1}$Universidad de Granada, $^{2}$Universidad de Antioquia

España$^{1}$, Colombia$^{2}$,

lake86@gmail.com; wfcastro82@gmail.com;

jgodino@ugr.es

\end{datos}

Diversas investigaciones y propuestas curriculares recomiendan la
introducción de contenidos algebraicos desde los primeros niveles
educativos con el fin de enriquecer la matemática escolar y facilitar
la transición hacia la matemática de la secundaria. La introducción
presupone una visión ampliada de la naturaleza del álgebra escolar
y una pluralidad de significados. En este trabajo presentamos un modelo
de caracterización de la actividad algebraica en la que se diferencian
tres niveles de razonamiento algebraico, abordando particularmente
tareas de tipo estructural, que puede utilizarse para proporcionar
a los maestros en formación la posibilidad de promover el pensamiento
algebraico en la educación primaria.


\section{UNA REPRESENTACION ELECTRONICA COMO INNOVACION DIDACTICA PARA LA
ENSEÑANZA DE VECTORES EN EL PLANO }

\begin{datos}

José Armando Murillo Contreras.

Universidad Pedagógica Experimental Libertador,

Caracas-Venezuela,

Josem2523@hotmail.com

\end{datos}

En esta oportunidad hablaremos acerca de una idea de innovación didáctica
para la enseñanza de la matemática. En este caso se expondrá una innovación
didáctica para la enseñanza de los vectores en el plano. Para ello
utilizaremos la electrónica para lograr una representación electrónica
en la cual, con esa representación trataremos de que se visualice
de manera entretenida y original los vectores para que los alumnos
logren un aprendizaje significativo y así lograr llamar la atención
de dichos alumnos 


\section{ARTICULACIÓN DE LA TRANSFORMADA DE LAPLACE CON LA RESOLUCIÓN DE PROBLEMAS
DE CIRCUITOS ELÉCTRICOS QUE CONDUECEN A ECUACIONES DIFERENCIALES.}

\begin{datos}

Francilene Dos Santos Cruz, Pedro Castañeda Porras, Pedro Fernández
de Córdoba, Juan. L. Martínes-González Santander.

Universidad de Pinar del Río, Universidad Politécnica de Valencia,
Universidad Católica de Valencia,

Universidade do estado Amazonas, Brasil. pcasta@mat.upr.edu.cu, franci\_78sl@hotmail.com,
pfernandez@mat.upv.es, martinez.gonzalez@ucv.es

\end{datos}

  En la experiencia que contaremos va dirigida a los estudiantes de
la carrera de ingeniería en Telecomunicaciones y Electrónica de la
Universidad de Pinar del Río en conjunto con un proyecto de colaboración
docente con la Universidad Politécnica de Valencia. Se pretende articular
las asignaturas de la disciplina matemática con la de circuitos eléctricos,
aplicar, la Transformada de Laplace a la resolución de problemas de
circuitos que conducen a ecuaciones diferenciales, que históricamente
determinaron el inicio del uso sistemático de la transformada de Laplace.
Los circuitos que vamos a considerar son aquellos constituidos por
los resistores, condensadores y los inductores.


\section{\uppercase{ Fracciones y la relación parte-todo. Una experiencia
didáctica}}

\begin{datos}

Sonia Bibiana Benítez, Lidia María Benítez.

Facultad de Cs. Naturales e I.M.Lillo. - Universidad Nacional de Tucumán,

Argentina,

Soniabenitez2001@hotmail.com; lidiabenitez@hotmail.com

\end{datos}

El objetivo de este trabajo es presentar la secuencia didáctica llevada
a cabo en escuelas, en el marco del Plan Matemática Para Todos, con
la intención de generar un efecto multiplicador que abarque a todas
las escuelas públicas y privadas del país. Dicho Plan es a nivel Nacional
de capacitación a docentes de escuelas públicas primarias, que tiende
a desarrollar en los alumnos competencias necesarias para un trabajo
autónomo en el área. Desde el Ministerio de Educación de la Nación
y desde nuestra propia práctica se asumió la responsabilidad de que
la escuela sea el lugar en el que todos aprendan y se garantice el
acceso a la herencia cultural de cada uno de nuestros niños. 


\section{O PAPEL LÚDICO DOS JOGOS NO ENSINO DA MATEMÁTICA}

\begin{datos}

Jessica de Abreu Barbosa.

Universidade de Brasília – UnB,

Brasil ,

jessica.xml@hotmail.com

\end{datos}

Os jogos matemáticos são de enorme importância para a educação matemática.
Devido às dificuldades encontradas pelos alunos no aprendizado da
matemática, o desenvolvimento de atividades alternativas como um apoio
às aulas formais é fundamental. Por ter um caráter lúdico, o jogo
demonstra um papel relevante no processo de ensino-aprendizagem. Os
jogos possuem efeitos cognitivos e socializantes de grande impacto
aos estudantes; observam-se melhorias e aquisição de habilidades,
tais como: raciocínio lógico, resolução de problemas, concentração,
trabalho em equipe, criatividade, reflexão e investigação. Existem
diversos tipos de jogos matemáticos: jogos que envolvem geometria,
jogos de tabuleiro, jogos de estratégia, jogos espaciais, tangrans,
quebra-cabeças, entre outros. 


\section{LABORATORIO DE CÁLCULO CON LA APLICACIÓN DESMOS}

\begin{datos}

Rubén Darío Santiago Acosta, Ma de Lourdes Quezada Batalla.

Tecnológíco de Monterrey - Campus Estado de México,

México,

ruben.dario@itesm.mx; lquezada@itesm.mx, 

\end{datos}

En este trabajo presentamos un laboratorio de Cálculo elaborado con
el software libre DESMOS, cuyas principales característica son: versatilidad,
facilidad y portabilidad ya que se puede usar en diferentes dispositivos
electrónicos (laptps, tabletas, smartphones). El laboratorio se usa
actualmente en diferentes cursos de matemáticas en el Tecnológico
de Monterrey, Campus Estado de México. El laboratorio contiene prácticas
sobre derivación e integración de funciones de una y varias variables.
Cada tema termina con una práctica interactiva de evaluación donde
se utilizan las herramientas Weebly y JotForm. Al final presentamos
resultados de una encuesta de ambiente del aula sobre la utilidad
del laboratorio. 


\section{DIFICULTADES ASOCIADAS AL CONCEPTO CONJUNTO GENERADOR EN NIVEL SUPERIOR}

\begin{datos}

Esteban Mendoza Sandoval, Flor Monserrat Rodríguez Vásquez.

Universidad Autónoma de Guerrero,

México,

emendoza@uagro.mx; flor.rodriguez@uagro.mx 

\end{datos}

Dada la naturaleza abstracta del álgebra lineal, nos interesa proponer
una vía alternativa en la enseñanza del concepto conjunto generador,
acuñando a la teoría APOE como sustento teórico. Por tanto en este
trabajo, como primera parte de una investigación en desarrollo, mostramos
algunas dificultades asociadas a dicho concepto en el nivel medio
superior, pues es fundamental para la propuesta alternativa que reconozcamos
tales dificultades como parte de la descomposición genética que se
debe realizar en la contribución del desarrollo del pensamiento matemático
avanzado. 


\section{POSTER: PROYECTO de MUSEO MATEMÁTICO: EL OMNIPOLIEDRO. }

\begin{datos}

Zenón Eulogio Morales Martínez.

Pontificia Universidad Católica del Perú – PUCP, Instituto de Investigación
en Enseñanza de las Matemáticas – IREM-PUCP, Institución Educativa
Particular Agroestudio,

Perú ,

morales.ze@pucp.edu.pe 

\end{datos}

En este poster se presentan las imágenes del proyecto del Museo Matemático
Agroestudio MMAE (Perú). En el año 2013, el proyecto del MMAE, presentó
el Omnipoliedro en el Congreso Iberoamericano de Educación Matemática
en Montevideo, Uruguay. Para este año se han propuesto continuar con
la elaboración de otras muestras para este museo matemático. Se logró
aumentar el interés por los objetos matemáticos, debido a la manipulación
de las muestras elaboradas. Promovamos el mensaje en catalán que se
muestra en el Museo de Catalunya (España): “Les matemátiques entre
per les mans” que nos dice: “Las matemáticas entra por las manos”. 


\section{LA GEOMETRIA ESPACIAL Y LOS RECURSOS DIDÁCTICOS }

\begin{datos}

Agda Jéssica de Freitas Galletti, Ana Maria Redolfi Gandulfo.

Universidad de Brasília,

Brasil,

aj.mat@hotmail.com; gandulfo@uol.com.br 

\end{datos}

Este poster tiene la finalidad de mostrar propuestas metodológicas
innovadoras de enseñanza y de aprendizaje de la geometría por medio
de modelos didácticos y también enfatizar la importancia de los recursos
didácticos como instrumentos para la visualización y la comprensión
de la geometría espacial. Abordamos los elementos, las representaciones
y las relaciones de algunos poliedros, así como sus propiedades, clasificaciones
e identificamos las diferentes figuras espaciales en la naturaleza,
el arte y la arquitectura. También tratamos los temas simetría y dualidad,
destacando los rellenados del espacio, sus clasificaciones y formaciones. 


\section{EL USO DE LAS REDES SOCIALES EN LA EDUCACION SUPERIOR }

\begin{datos}

José Dionicio Zacarias Flores, Yazmin Jiménez Jiménez, Gladys Denisse
Salgado Suárez BUAP,

Puebla - México,

jzacarias@fcfm.buap.mx; yazjim2\_26@hotmail.com;

gladys008@hotmail.com

\end{datos}

Las redes sociales muestran un fuerte crecimiento e impacto en toda
la sociedad, convirtiéndose en la nueva forma de interactuar de las
personas (Romero, 2011), de donde surgen las preguntas: ¿Pueden ser
útiles en la enseñanza y aprendizaje de la matemática a nivel superior?,
si es así ¿de qué forma? Se presenta una revisión del estado del arte
en el uso de las redes sociales con la finalidad de conocer si pueden
o no dárseles un uso en el sector educativo a nivel superior, buscando
la posibilidad de integrarlas, como estrategias de enseñanza, y no
como canal de comunicación.


\section{NARRATIVAS DE LOS DERECHOS HUMANOS EN EDUCACIÓN MATEMÁTICA: EL CASO
DE LOS ESTUDIANTES DE GRADO SEXTO }

\begin{datos}

Juan Manuel Salas Martínez, Fernando Guerrero Recalde.

Universidad Distrital Francisco José de Caldas,

Colombia,

juanmanuelsalasmartinez@hotmail.com; nfguerreror@gmail.com 

\end{datos}

Se pretende desarrollar una secuencia de actividades donde a partir
de la modelación matemática y las narrativas, los estudiantes de grado
sexto del colegio La Belleza los Libertadores IED, reflexionen sobre
la realidad social y analicen los efectos de algunos aspectos sobre
la conservación de la naturaleza, empleando las narrativas y la modelación
matemática de situaciones relacionadas con la conservación de la naturaleza,
para ejercer el derecho a la vida y el deber de protegerla, con la
introducción de una secuencia didáctica, la investigación es de tipo
cualitativo y se desarrollará a partir de la investigación acción,
Elliot (2005).


\section{SECUENCIA DIDÁCTICA DE MODELACIÓN-GRAFICACIÓN EN LA RESIGNIFICACIÓN
Y CONSTRUCCIÓN DE LA FUNCIÓN CUADRÁTICA}

\begin{datos}

Fredy de la Cruz Urbina, Hipólito Hernández Pérez.

Universidad Autónoma de Chiapas,

México,

frecu@hotmail.com; polito\_hernandez@hotmail.com 

\end{datos}

Construir el diseño y modelo algebraico que representa a un fenómeno
o situación, se ha vuelto complejo para el alumno. Por ello abordamos
el estudio de la función cuadrática en el nivel medio superior, desde
una mirada Socioepistemológica, para averiguar que prácticas sociales
permiten la resignificación y construcción de la concepción de “función
cuadrática” a partir de la modelación-graficación y con el uso de
la Ingeniería Didáctica diseñar secuencias didácticas basadas en el
movimiento con sensores, llenado y vaciado de recipientes; creemos
que estas actividades experimentales ayudarán a construir y resignificar
lo cuadrático en la estructura mental del alumno. 


\section{LA VISUALIZACIÓN EN EL PROCESO DE ENSEÑANZA-APRENDIZAJE DE LA GEOMETRÍA.
UN EJEMPLO}

\begin{datos}

Élgar Gualdrón, Ángel Gutiérrez.

Grupo de Investigación EDUMATEST - Universidad de Pamplona, Universidad
de Valencia,

Colombia, España,

elgargualdron@yahoo.es; angel.gutierrez@uv.es

\end{datos}

Se muestra los resultados de una investigación cualitativa que desarrollamos
al interior de la línea de investigación en “Pensamiento Geométrico”
en el grupo de investigación EDUMATEST de la Universidad de Pamplona
(Colombia), sobre la mejora del aprendizaje geométrico, concretamente
en semejanza de figuras planas. El objetivo central del trabajo es
analizar y caracterizar el proceso de aprendizaje de la semejanza
de figuras planas, en la enseñanza secundaria, usando el modelo de
razonamiento de Van Hiele y la visualización. En este cartel daremos
cuenta del uso que hacen los estudiantes de imágenes mentales y habilidades
de visualización en la resolución de tareas matemáticas relacionadas
con la semejanza. 


\section{CINCO PROBLEMAS CLAVES PARA EL DESARROLLO DEL PROCESO DE ENSEÑANZA
APRENDIZAJE DE LOS ESPACIOS VECTORIALES}

\begin{datos}

Ángela Mercedes Martín Sánchez, Olga Lidia Pérez González, Laura Casas
Fuentes, Lisandra Docampo López, Lenniet Coello, Isabel Yordi González;
Cila Mola Reyes.

Universidad Autónoma de Santo Domingo, Universidad de Camagüey,

República Dominicana, Cuba

m.angela@gmail.com; olguitapg@gmail.com;

laura.casas@reduc.edu.cu ; lisandra.docampo@reduc.edu.cu;

lenniet.coello@reduc.edu.cu; isabel.yordi@reduc.edu.cu; 

cila.mola@reduc.edu.cu

\end{datos}

Campo de investigación: pensamiento algebraico, Nivel Superior, Investigación
empírico/teórico

Se proponen 5 problemas claves que actúan como hilo conductor, utilizando
la combinación lineal de vectores como la célula que genera cada uno
de dichos problemas (Hernández, 1989). Estos 5 problemas muestran
el aspecto común entre los procedimientos que se estudian en los espacios
vectoriales, evidenciando la relación entre ellos y permitiendo orientar
al estudiante en la ejecución de las tareas. 


\section{LA TRANSFERENCIA DE REGISTROS SEMIÓTICOS COMO ESTRATEGIA PARA LA
NIVELACIÓN DE LOS ESTUDIANTES EN LOS CURSOS PROPEDÉUTICOS}

\begin{datos}

Wendy Heredia Soriano, Juan Manzueta, Francisco Vegazo; Ramón Blanco
Sánchez, Olga Lidia Pérez González, Alexia Nardín, Carlos Basulto
Morales, Adolfo Álvarez, Ibis Ramos.

Universidad Autónoma de Santo Domingo, Universidad de Camagüey,

República Dominicana, Cuba,

ramón.blanco@reduc.edu.cu,olguitapg@gmail.com; alexia.nardin@reduc.edu.cu;

carlos.basulto@reduc.edu.cu; adolfo.alvarez@reduc.edu.cu; 

ibis.ramos@reduc.edu.cu

\end{datos}

El objetivo es reportar los resultados parciales de un proyecto de
investigación que parte de la problemática relacionada con el tratamiento
didáctico que se le da en los cursos propedéuticos a la nivelación
de los estudiantes desde el punto de vista de la formación conceptual.
El objetivo es poner ejemplos de cómo abordar esta nivelación desde
la perspectiva de la transferencia de los registros semióticos de
los conceptos matemáticos. Se expone la experiencia llevada a la práctica
y los resultados obtenidos que sirven a su vez de orientaciones metodológicas
para que otros maestros la pongan en práctica.


\section{MODELACIÓN DE PARABOLOIDES COMO REPRESENTACIONES ARQUITECTONICAS}

\begin{datos}

Lucero Antolínez Quijano, Yetza Ximena Díaz Pinzón.

Universidad Pedagógica Nacional, Universidad de Boyacá ,

Colombia,

luceroaq@gmail.com;yetzad@gmail.com

\end{datos}

Las herramientas informáticas posibilitan modelar curvas y superficies
desde sus ecuaciones paramétricas. Al aplicar elementos matemáticos,
se pueden optimizar algunas condiciones estructurales para generar
espacios arquitectónicos versátiles y novedosos, haciendo un análisis
referencial de la teoría y las construcciones existentes. Esto promueve
el diseño y construcción de estructuras con su respectiva caracterización
matemática y sugerirlas como soluciones arquitectónicas innovadoras.
Particularmente, los paraboloides hiperbólicos son una alternativa
para construir cubiertas que proveen gran luminosidad y hacen visiblemente
atractivos aquellos espacios que demandan áreas importantes de uso
común, sumando la caracterización anticlastica como elemento esencial
para lograr la rigidez de las cubiertas. 


\section{JUEGO INTERACTIVO DE APOYO PARA NIÑOS CATALOGADOS CON TDAH. }

\begin{datos}

Alicia Rocendo Ramírez. 

ton.ali24.07@gmail.com 

Asesores: MCC. Hilda Icela Garzón Barrientos, DRA. Guadalupe Cabañas-Sánchez

\end{datos}

El presente estudio se enfoca en el desarrollo de aplicaciones interactivas
utilizando Scratch en el proceso de enseñanza de niños con hiperactividad,
en el interés de centrar su atención, que es la fuente principal de
sus problemas en el salón de clases. Algunas investigaciones han documentado
que los estudiantes que se distraen fácilmente, pueden concentrarse
mejor si las tareas se realizan en un computador, en razón de que
la tecnología ofrece a docentes opciones para adaptar la instrucción
a necesidades específicas de los alumnos (e.gr. Correa et al, 2012).
Para los niños el principal estimulo es la motivación y la atracción
de la aplicación, ya que esta incluye sonido, instrucciones en audio,
imágenes atractivas para ellos, y que requiere de un corto tiempo
para ejecutar la aplicación.

Referencias bibliográficas Armstrong, T. (2008). Síndrome de Déficit
de Atención con o sin Hiperactividad ADD/ADHD. Estrategias en el aula.
España: PAIDÓS. Correa, S., Reséndiz, E., Llanos, R., Salazar, M.
\& Sánchez, J. (2012). Diseño, desarrollo y evaluación de objetos
de aprendizaje en matemáticas básicas. En Flores, R. (Ed.). (2012).
Acta Latinoamericana de Matemática Educativa, Vol. 25. México, DF:
Colegio Mexicano de Matemática Educativa A. C. y Comité Latinoamericano
de Matemática Educativa A. C. Lázaro, N. (2012). Estrategia metodológica
para potenciar el uso del software elementos matemáticos en la secundaria
básica. En Flores, R. (Ed.). (2012). Acta Latinoamericana de Matemática
Educativa, Vol. 25. México, DF: Colegio Mexicano de Matemática Educativa
A. C. y Comité Latinoamericano de Matemática Educativa A. C. PO029 


\section{LA TRANSFERENCIA DE REGISTROS SEMIÓTICOS EN LA FORMACIÓN DEL CONCEPTO
DERIVADA. }

\begin{datos}

Neel Báez Ureña, Olga Lidia Pérez González, Alexia Martín Anarela,
Adolfo Álvarez, Arnaldo Espíndola, Cila Mola, Ibis Ramos, Ramón Blanco
Sánchez.

Universidad Autónoma de Santo Domingo, Universidad de Camagüey,

República Dominicana, Cuba,

neelbaez@gmail.com; ramón.blanco@reduc.edu.cu;

olguitapg@gmail.com; alexia.nardin@reduc.edu.cu;

adolfo.alvarez@reduc.edu.cu; arnaldo.espindola@reduc.edu.cu; 

cila.mola@reduc.edu.cu; ibis.ramos@reduc.edu.cu 

\end{datos}

El objetivo es reportar los resultados parciales de un proyecto de
investigación que parte de la problemática de los estudiantes con
los conceptos del Cálculo Diferencial. Se ha podido argumentar que
dado el carácter no ostensivo de los objetos matemáticos y la insuficiente
cantidad de tareas, realizadas en clases, sobre la transferencia de
registros semióticos, se limita el trabajo conceptual, de los estudiantes,
con este contenido. El objetivo del proyecto es exponer la experiencia
del trabajo con la transferencia de registros semióticos para la formación
del concepto derivada. Se muestran los resultados obtenidos con su
aplicación.


\section{LA TRANSFERENCIA DE REGISTROS SEMIÓTICOS COMO ESTRATEGIA PARA LA
NIVELACIÓN DE LOS ESTUDIANTES EN LOS CURSOS PROPEDÉUTICOS}

\begin{datos}

Wendy Heredia Soriano, Juan Manzueta, Francisco Vegazo, Ramón Blanco
Sánchez, Olga Lidia Pérez González, Alexia Nardín, Carlos Basulto
Morales, Adolfo Álvarez, Ibis Ramos.

Universidad Autónoma de Santo Domingo,

Universidad de Camagüey,

República Dominicana, Cuba,

neelbaez@gmail.com; ramón.blanco@reduc.edu.cu;

olguitapg@gmail.com; alexia.nardin@reduc.edu.cu;

carlos.basulto@reduc.edu.cu; adolfo.alvarez@reduc.edu.cu; 

ibis.ramos@reduc.edu.cu

\end{datos}

El objetivo es reportar los resultados parciales de un proyecto de
investigación que parte de la problemática relacionada con el tratamiento
didáctico que se le da en los cursos propedéuticos a la nivelación
de los estudiantes desde el punto de vista de la formación conceptual.
El objetivo es poner ejemplos de cómo abordar esta nivelación desde
la perspectiva de la transferencia de los registros semióticos de
los conceptos matemáticos. Se expone la experiencia llevada a la práctica
y los resultados obtenidos que sirven a su vez de orientaciones metodológicas
para que otros maestros la pongan en práctica.


\section{PROCESO ESTRATEGICO PARA RESOLVER 20 IDENTIDADES TRIGONOMETRICAS
SENCILLAS }

\begin{datos}

Elizabeth Rincón Santana, José Manuel Ruíz Socarras, Ramón Blanco
Sánchez.

Universidad Autónoma de Santo Domingo, Universidad de Camagüey,

República Dominicana, Cuba,

te10elirisa@gmail.com; jose.ruiz@reduc.edu.cu;

ramón.blanco@reduc.edu.cu

\end{datos}

Se presenta un póster o cartel para la resolución de identidades trigonométricas
sencillas, con el objetivo de favorecer el pensamiento deductivo de
los estudiantes y proporcionarles estrategias básicas que ayuden a
simplificar expresiones trigonométricas. El diseño es una guía que
forma parte del conjunto de actividades que tributan al proyecto de
doctorado de la autora principal del trabajo, el cual está orientado
a investigar sobre las estrategias docentes para la enseñanza aprendizaje
de la Geometría y Trigonometría plana, sustentado en la estructura
sistémica de la Matemática y su lógica de desarrollo intrínseca.


\section{CONCEPCIONES DE LOS PROFESORES AL USAR MATERIAL CONCRETO Y HERRAMIENTAS
TECNOLÓGICAS PARA EXPLICAR EL TEOREMA DE PITÁGORAS. }

\begin{datos}

Jesús Grajeda Rosas, Eliza Minnelli Olguín Trejo, Claudia Rodríguez
Muñoz.

Departamento de Matemática Educativa, CINVESTAV-IPN,

México iq\_jesusgr@hotmail.com; minnelli\_angel@yahoo.com.mx;

claurom65@yahoo.com

\end{datos}

Presentamos un estudio sobre la
valoración
que
los docentes, inscritos
en un laboratorio, hicieron sobre las ventajas
y desventajas que el
uso
de
las herramientas tecnológicas tienen en la educación, frente al
material concreto. El Laboratorio fue impartido en la XVI Escuela
de Invierno de Matemática Educativa en México a un grupo de profesores
de nivel medio en torno al uso de material concreto, en este caso
papel, y herramientas tecnológicas, tal es el caso del software GeoGebra,
poniendo como objeto matemático al Teorema de Pitágoras. 


\section{VISUALIZACIÓN DE LA MODELACIÓN MATEMÁTICA EN ALUMNOS DEL BACHILLERATO}

\begin{datos}

David Alfonso Romero, Liliana Suárez Téllez, José Luis Torres Guerrero.

Instituto Politécnico Nacional, 

México,

darbunk24@gmail.com; lilianasuarezt@gmail.com;

jeluistg@yahoo.com.mx

\end{datos}

Los alumnos de NMS se enfrentan a problemáticas para la comprensión
de las Matemáticas fuera del ámbito escolar, preguntándose el uso
de las mismas en lugares distintos al aula de clases, por ello inculcarlos
en la resolución de experimentos de la materia, donde se da un enfoque
práctico mediante el trabajo en equipo y uso de la tecnología es como
se busca que los alumnos del bachillerato interpreten los conocimientos
adquiridos en las cursos de Matemáticas en la vida diaria, dando solución
y uso de estrategias para concretar un conocimiento de la modelación
de las Matemáticas.


\section{ENSEÑANZA DE SUMA Y RESTA DE NÚMEROS NATURALES A NIÑOS CON SÍNDROME
DE DOWN }

\begin{datos}

Liliana Hernández Martínez.

Universidad Pedagógica y Tecnológica de Colombia, 

Colombia,

lili\_girla@hotmail.com; lili.girla.hernandez@gmail.com 

\end{datos}

Este proyecto de investigación cualitativa tiene como objetivo diseñar,
aplicar y evaluar una estrategia para estudiantes con Síndrome de
Down (SD) de la Escuela Normal Superior Santiago de Tunja (Boyacá),
que permita un aprendizaje significativo de los conceptos de suma
y resta, facilitando la construcción del conocimiento matemático.
El marco referencial comprende la inclusión educativa, las reformas
curriculares que trae consigo y las características del SD, entre
otros. Éste se desarrollará según el método investigación-acción y
en sus etapas se utilizarán la observación y la entrevista como recolectores
de información, determinando así las características que puede tener
dicha estrategia, adecuada a las capacidades, necesidades e intereses
de los estudiantes.


\section{SIGNIFICADO DE REFERENCIA DE LA EXPRESIÓN NUMÉRICA DE LA FORMA “$\frac{a}{b}$” }

\begin{datos}

Bencomo Delisa, Franzone Johanna,

Universidad Nacional Experimental de Guayana,

Venezuela,

dbencomo@uneg.edu.ve; jfranzone0113@gmail.com 

\end{datos}

En este cartel nos proponemos mostrar la reconstrucción del significado
de la expresión numérica de la forma “a/b”. Esta reconstrucción se
realizó a partir del reconocimiento de los campos de problemas del
contexto de uso informal, de la identificación de los elementos de
significados que emergen de esos sistemas de prácticas de contexto
informal, de la construcción de las configuraciones empíricas a partir
de las relaciones entre los elementos de significado, del reconocimiento
de los contextos de uso formales asociado, equivalentes y/o relacionados
con las configuraciones empíricas, y de las configuraciones teóricas
que emergen de esos contextos formales. 


\section{EJEMPLOS PARA LA EVALUACIÓN POR COMPETENCIAS EN EL PROCESO DE ENSEÑANZA
APRENDIZAJE DEL ALGEBRA EN EL NIVEL MEDIO}

\begin{datos}

Yanile Altagracia Valenzuela Calderón, Olga Lidia Pérez González,
Manuel Guardado Hernández, Nancy Montes de Oca.

Universidad Autónoma de Santo Domingo, Universidad de Camagüey,

República Dominicana, Cuba,

yanilevalenzuela@gmail.com; olguitapg@gmail.com; 

manuel.guardado@reduc.edu.cu; nancy.montes@reduc.edu.cu

\end{datos}

El objetivo del trabajo es hacer un reporte de los resultados parciales
de un proyecto de investigación que parte de la problemática existente
en relación a las insuficiencias en el tratamiento de la evaluación
en el proceso de enseñanza aprendizaje (PEA) del Álgebra en el nivel
medio. Se exponen ejemplos que favorecen el desarrollo de la evaluación
por competencias en el PEA del Álgebra. 


\section{MODELACIÓN MATEMÁTICA COMO HERRAMIENTA EDUCATIVA DESDE LA MIRADA
DE ESTUDIANTES}

\begin{datos}

Jhosep Jonathan Huerta Martínez, Verónica Camacho Salinas, Liliana
Suárez Téllez, Claudia Flores Estrada, Adriana Gómez Reyes.

Instituto Politécnico Nacional,

México,

jjhm3796@gmail.com; veronica00307@gmail.com;

lsuarez@ipn.mx; cfloreses@ipn.mx;

orodelsilencio@yahoo.com.mx 

\end{datos}

La enseñanza de las matemáticas pretende que los alumnos sean capaces
de emplear y hacer uso de los conocimientos para resolver problemas
en su vida cotidiana y otras áreas. Debido a esto la matemática educativa
a lo largo de su desarrollo ha buscado generar estrategias y herramientas
que den pauta a cambios en el proceso de enseñanza-aprendizaje. El
eje del presente trabajo es la búsqueda y análisis, por parte de dos
estudiantes de bachillerato, de tres afirmaciones sobre la modelación
matemática como una herramienta que permite al alumno y al docente
cambios positivos en el proceso de enseñanza-aprendizaje.


\section{CURSO E-LEARNING COMO APOYO EN LA ENSEÑANZA DE UNA DISTRIBUCIÓN BINOMIAL
EN LA ASIGNATURA DE ESTADÍSTICA}

\begin{datos}

Miguel de Nazareth Pineda Becerril, Armando Aguilar Márquez, Juan
Carlos Axotla García, Frida María León Rodríguez, Omar García León.

mnazarethp@fesc.cuautitlan2.unam.mx; armandoa@servidor.unam.mx;

jc\_axotla@fesc.unam.mx; fridam@servidor.unam.mx

\end{datos}

Para facilitar el estudio del tema de Distribución Binomial en las
asignaturas de estadística que se imparten en la Facultad de Estudios
Superiores Cuautitlán, se desarrolló este tema con diferentes actividades
bajo la plataforma de Dokeos, Dentro de este trabajo se propone que,
aunado a la enseñanza de los profesores en el aula, los alumnos aborden
este tema mediante un curso E-learning, el cual contiene applets,
teoría, chat, videos, etc. En este curso E-learning se cuanta con
diferentes herramientas para que el alumno obtenga una mejor comprensión
del tema de distribución binomial 


\section{LOS CURSOS PROPEDÚTICOS, LA NIVELACIÓN CONCEPTUAL Y LA APROPIACIÓN
Y APLICACIÓN DE CONCEPTOS: ¿LO LOGRAMOS? }

\begin{datos}

Danitza Soledad Mirabal, Wendy Heredia Soriano, Francisco Vegazo;
Ramón Blanco Sánchez, Olga Lidia Pérez González, Alexia Nardín, Carlos
Basulto Morales, Adolfo Álvarez, Ibis Ramos.

Universidad Autónoma de Santo Domingo, Universidad de Camagüey,

República Dominicana, Cuba,

dsm\_536@hotmail.com; ehssoriano@gmail.com; 

ramón.blanco@reduc.edu.cu; olguitapg@gmail.com;

alexia.nardin@reduc.edu.cu; carlos.basulto@reduc.edu.cu; 

adolfo.alvarez@reduc.edu.cu; ibis.ramos@reduc.edu.cu

\end{datos}

El objetivo es reportar los resultados parciales de un proyecto de
investigación que parte de la problemática relacionada con el tratamiento
didáctico que se le da en los cursos propedéuticos a la nivelación
de los estudiantes desde el punto de vista de la formación conceptual.
El objetivo es exponer los resultados de un estudio empírico sobre
las concepciones y creencias que tienen los profesores de cómo abordar
esta nivelación desde la perspectiva de la transferencia de los registros
semióticos de los conceptos matemáticos. 


\section{EL ANÁLISIS DE LAS COMPETENCIAS COMO UNA OPORTUNIDAD DE DESARROLLO
PROFESIONAL EN ESTUDIANTES DE PREPARATORIA.}

\begin{datos}

Carlos Oropeza Ugalde, José Isaac Sánchez Guerra, Carlos Oropeza Legorreta. 

CBT Gabriel V. Alcocer - Facultad de Estudios Superiores Cuautitlán,
UNAM,

México, 

lambo.r.gini@hotmail.com; joejade@hotmail.com;

coropeza96@hotmail.com.

\end{datos}

Las inteligencias permiten el desarrollo de capacidades, habilidades
y aprendizajes en el mundo del trabajo. En ellos, los modelos organizacionales
han incorporado nuevas formas de concebir las competencias laborales,
influidas por los aportes de la psicología, la antropología, la sociología,
la economía y la medicina, disciplinas que han contribuido a problematizar
el tema. ¿Cuál es la relación entre competencias, capacidades e inteligencias?,
¿qué tipo de inteligencias se necesitan para los mercados de trabajo
en la economía? ¿hasta dónde las competencias básicas son el referente
fundamental para el mundo del trabajo? y ¿qué tipo de inteligencia
o inteligencias permiten su desarrollo? 


\section{MAPAS CONCEPTUALES EN EL PROCESO DE ENSEÑANZA Y APRENDIZAJE DE LA
MATEMÁTICA}

\begin{datos}

Daysi García Cuéllar, Daniel Proleón Patricio, Edisson Pérez Sotelo.

Pontificia Universidad Católica del Perú - Instituto de Matemática
Pura y Aplicada,

garcia.daysi@pucp.pe; dproleon@pucp.pe; 

eperez@imca.edu.pe

\end{datos}

La experiencia educativa muestra la importancia de los mapas conceptuales
en el proceso de enseñanza y aprendizaje de la matemática los cuales
son considerados como una importante herramienta para el logro de
aprendizajes significativos, constructivos y por descubrimiento, desde
las posturas de Ausubel, Piaget y Bruner, respectivamente. Esta experiencia
fue realizada con las alumnas del segundo año de secundaria. Ellas
utilizaron como herramienta el programa Cmaptools que les permitió
reconocer algunas característica de los mapas conceptuales como: La
jerarquía, palabras enlaces, entre otros.


\section{UNA ALTERNATIVA PARA LA RESOLUCIÓN DE PROBLEMAS DE MATEMÁTICA: TRABAJO
EN GRUPO}

\begin{datos}María Elena Villanueva Pinedo.

Universidad Nacional Agraria La Molina,

Perú ,

villanuepi@lamolina.edu.pe

\end{datos}

Este trabajo surge de la necesidad de iniciar un proceso de innovación
de la enseñanza e incorporar metodologías activas como medios para
el aprendizaje eficaz. Se realizó una actividad en grupo en el curso
de Cálculo Diferencial con estudiantes de carreras relacionadas con
el agro que tienen diferentes habilidades y conocimientos. La finalidad
fue la de que verifiquen y afiancen sus avances y aprendan unos de
otros. Posteriormente, se realizará la evaluación del rendimiento
(específicamente, del puntaje obtenido en la pregunta) y de los resultados
obtenidos del trabajo en grupo, a través de técnicas estadísticas,
con el objetivo de determinar si estas variables son dependientes.
En caso haya habido una mejora en el rendimiento después de realizar
la actividad, se recomendará plantear este tipo de trabajo en otras
partes del curso que requieren de la resolución de problemas. 
\begin{description}
\item [{1.}] Bologna, E. (2011). Estadística para Psicología y Educación.
Córdova: Editorial Brujas.
\item [{2.}] Rue, J. (2009). El aprendizaje Autónomo en Educación Superior.
Madrid: NARCEA, S. A. de Ediciones.
\item [{3.}] Torres, P. (2003). Estrategias de Resolución de Problemas.
UPC, Lima. 
\item [{4.}] Villanueva, M. (2009). Relación entre las notas de matemática
obtenidas en el nivel secundario y en curso de matemática de los estudiantes
que recién ingresan a la universidad. Actas Enseñanza de las matemáticas
IV Coloquio Internacional. Lima: Departamento de Ciencias – PUCP.\end{description}


