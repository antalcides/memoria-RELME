
\pagestyle{headings}
\nocite{*}
\fontsize{7}{8}\selectfont
%\setlength{\baselineskip}{5pt}
\pagecolor{white} 

\onecolumn
\chapter{Comunicaciones Breves } 
\renewcommand\thesection{CB\ \nplpadding{3}\numprint{\arabic{section}}} 
\setcounter{section}{0}
\chaptertoc
\twocolumn
\balance



\section{UN PROBLEMA PARA ESTUDIANTES DE INGENIERÍA INDUSTRIAL }

\begin{datos}

Yoana Acevedo Rico. Universidad Pontificia Bolivariana - Seccional
Bucaramanga,

Colombia,

yoana.acevedo@upb.edu.co 

\end{datos}

En esta comunicación se aborda el diseño, implementación y evaluación
de una estrategia didáctica con estudiantes de primer semestre de
Ingeniería Industrial de la Universidad Pontificia Bolivariana Seccional
Bucaramanga. La estrategia consiste en la resolución de un problema
para aplicar los ejes temáticos de las asignaturas: Geometría y trigonometría
y Cálculo diferencial. Se formuló el siguiente problema: “Diseño de
una lámpara comercial en papel inspirada en un fractal y análisis
de optimización de la función producción”. Es una investigación de
métodos mixtos. Se encontró en el grupo donde se implementó la estrategia
un mayor rendimiento académico (a través de dos pruebas escritas y
comparando con un grupo control) y cambio de hábitos de estudio (en
las observaciones realizadas a través de la implementación de la propuesta).


\section{O PAPEL DE GRUPOS COLABORATIVOS EM COMUNIDADES DE PROFESSORES QUE
ENSINAM MATEMÁTICA}

\begin{datos}

Renata Camacho Bezerra, Klinger Teodoro Ciríaco.

Universidade Estadual do Oeste do Paraná, Campus de Foz do Iguaçu/PR,

Universidade Federal do Mato Grosso do Sul, Campus de Naviraí/MS,

Brasil,

renata.bezerra@unioeste.br; klingerufms@hotmail.com 

\end{datos}

Este artigo discute as ideias que envolvem grupos colaborativos e
baseia-se em estudos teóricos e empíricos/experimental. Nos últimos
tempos o processo de ensino e aprendizagem de conceitos e a formação
do professor que ensina Matemática tem ganhado destaque no cenário
mundial e de forma especial no Brasil, isso se dá em grande medida
pelos resultados alcançados pelos nossos alunos nos exames nacionais,
e pelo fato de que é atribuída a Matemática grande responsabilidade
pela repetência e evasão escolar. Diante desse panorama é necessário
pensarmos em alternativas, e uma delas tem sido o investimento na
formação de professores inicial e continuada.


\section{DISEÑO Y GESTIÓN EN LA ENSEÑANZA-APRENDIZAJE DE LA FORMACIÓN DOCENTE
EN LA LICENCIATURA EN EDUCACIÓN BÁSICA CON ÉNFASIS EN MATEMÁTICAS
(LEBEM) EN EL PERIODO 2005 -2007. }

\begin{datos}

Jenny Paola Rodríguez Gámez, Luis Gabriel González Cantor.

Universidad Distrital Francisco José de Caldas,

Colombia,

paolitastar200@hotmail.com; brothersgc@hotmail.com 

\end{datos}

Este documento muestra una descripción de las fases desarrolladas
en el proyecto de investigación “Desarrollo de las prácticas en LEBEM
en el periodo 2005-2012” del grupo de investigación CRISALIDA, de
la Universidad Distrital; se presentará la planeación del desarrollo
de la investigación y las actividades que se desenvolverán dentro
la misma, en torno al análisis de contenido que se trabaja en las
unidades didácticas de la práctica intermedia I en el periodo 2005-2007,
centrando la mirada en evidenciar progresos y desarrollos en el diseño
y planeación de actividades y su relevancia en el proceso de enseñanza-aprendizaje
de la matemática educativa en el aula de clase. 


\section{LAS PRÁCTICAS DE ENSEÑANZA EMPLEADAS POR DOCENTES DE MATEMÁTICAS
DE GRADO SÉPTIMO, EL CONCEPTO DE FRACCIÓN, SUS OPERACIONES Y LA SOLUCIÓN
DE SITUACIONES COTIDIANAS}

\begin{datos}

Alexander Murillo Moreno, Leonardo Ceballos Urrego.

Universidad de Antioquia,

Colombia,

alexanderm54@gmail.com ; lceu0457@gmail.com

\end{datos}

Este trabajo de investigación relaciona teorías y aportes significativos,
frente a posibles conexiones entre las prácticas de enseñanza empleadas
por docentes de matemáticas, y como ellas, a partir del tratamiento
de las fracciones, se reflejan en los estudiantes en: la comprensión
del concepto, sus operaciones, y destrezas que desarrollan al resolver
situaciones cotidianas derivadas. La investigación se enmarca en el
paradigma de investigación cualitativa, bajo el enfoque de la teoría
fundada y con diseño metodológico basado en el estudio de casos. Se
centra en estudios y experiencias nacionales, confrontados con la
literatura existente en Latinoamericana, Norteamérica y algunos países
europeos.


\section{AUTOEVALUACIONES DINÁMICAS EN PROCESOS DE APRENDIZAJE}

\begin{datos}

Teresita Terán, Omar Córdoba, Juan José Cámpora.

Facultad de Ciencias Veterinarias – Universidad Nacional de Rosario,

Argentina,

teresitateran@hotmail.com; odcordoba@hotmail.com

\end{datos}

La evaluación es uno de los momentos fundamentales dentro del proceso
de enseñanza-aprendizaje. La calificación está incluida dentro de
la evaluación, pero esta última comprende otras etapas, como ofrecer
al alumno una retroalimentación sobre su aprendizaje, permitir que
los docentes conozcan la efectividad de su actuación, certificar los
resultados, evaluar la calidad de la metodología, revisar el material
didáctico. Por ello se considera a la evaluación como un proceso dinámico.
Se propone un sistema de autoevaluaciones online a alumnos de Bioestadística
para que evalúen su metacognición con el fin de que los docentes observen
el proceso de enseñanza-aprendizaje.

\setcounter{section}{6}


\section{LA ENSEÑANZA DE LA MATEMÁTICAS UTILIZANDO EL APRENDIZAJE COLABORATIVO }

\begin{datos}

Rosario del Pilar Gibert Delgado, José Guadalupe Torres Morales.

Departamento de Ingeniería en Comunicaciones y Electrónica, IPN-Esime
Culhuacan,

México,

giberty42@hotmail.com; jgtorresm@hotmail.com

\end{datos}

La enseñanza de las matemáticas hoy en día es un problema real que
requiere de soluciones adecuadas a nivel del conocimiento humanístico
de nuestra sociedad que también debe estar a la par de la tecnología
con la que cuenta, buscando no solamente facilitar el aprendizaje
colaborativo, sino también generar nuevas expectativas y estrategias
dentro de la enseñanza. La premisa del aprendizaje colaborativo es
“llegar al consenso entre los miembros del grupo”. Pedagógicamente
hablando, el aprendizaje colaborativo comprende el espectro entero
de las actividades de los grupos estudiantiles, que trabajan juntos
y fuera de la clase.


\section{ESTRATÉGIAS PEDAGÓGICAS DE RESOLUÇÃO DE PROBLEMAS EM UM CURSO DE
FORMAÇÃO DE PROFESSORES}

\begin{datos}

Nielce Meneguelo Lobo da Costa, Aparecida Rodrigues Silva Duarte,
Edite Resende Vieira.

Universidade Bandeirante de São Paulo – UNIBAN,

Brasil,

nielce.lobo@gmail.com; aparecida.duarte6@gmail.com;

edite.resende@gmail.com

\end{datos}

Neste estudo se discutem estratégias pedagógicas para ensino de matemática
por resolução de problemas, as quais foram analisadas em uma formação
continuada para professores. Trata-se de pesquisa qualitativa, com
metodologia co-generativa (Greenwood; Levin, 2000) e fundamentada
em Shulman (1986), Ball, Thames e Phelps (2008), Polya (1995), Onuchic
(1999) e Bryant et al (2012). Um dos temas foi resolução de problemas
com números inteiros pela estratégia de jogo didático. Os desafios
propostos geraram ambiente propício à investigação e à descoberta,
de modo que vários professores ficaram estimulados a propor mudanças
promovendo aulas de matemática com um caráter mais exploratório e
investigativo. 


\section{GRÁFICA DE LAS FUNCIONES SENO Y COSENO UTILIZANDO LOS DEDOS DE LAS
MANOS Y LAS HERRAMIENTAS DE POWER POINT}

\begin{datos}

Eduin Segundo Peláez Cotera.

San Vicente de Paúl,

Colombia ,

edwinp-1408@hotmail.com 

\end{datos}

Este proyecto se basa en un código escrito en los dedos de las manos
para calcular el valor de las funciones trigonométricas para ángulos
múltiplos de 10 grados con herramientas de Power Point. El código
es 017345064768693981, escribiéndose así: 0, pulgar izquierdo; 17,
índice, 34; medio, 50, anular y 64, meñique. Mano derecha: 76, meñique;
86, anular; 93, medio; 98 índice y 1, pulgar. Ejemplo, hallar sen30\textdegree{}:
se abre mano izquierda, se cuenta de 10\textdegree{} en 10\textdegree{},
partiendo de 0\textdegree{}, donde queda 30\textdegree{} es el anular
izquierdo, que equivale a 0,50. Luego, con Power Point, se gráfica
Seno y Coseno.


\section{CONFIGURACIONES EPISTÉMICAS PREVIAS PARA DAR SIGNIFICADO GLOBAL AL
OBJETO MATEMÁTICO “MÉTODO DE INTEGRACIÓN POR PARTES”}

\begin{datos}

Enrique Mateus Nieves.

Universidad Distrital Francisco José de Caldas,

Bogotá- Colombia,

jeman124@gmail.com

\end{datos}

Esta investigación doctoral indaga si el estudiante puede alcanzar
un significado global del objeto matemático “integración por partes”
a partir de la potenciación de 4 configuraciones epistémicas {[}CE{]}
definidas para el objeto matemático llamado la integral. En lo avanzado
de la investigación se evidencia, como un primer resultado, que si
estas CE no se potencian el proceso de enseñanza de este método carecerá
de significado para el estudiante y seguirá siendo solo una regla,
un algoritmo más que nada le aporta a su formación integral; donde
él debe reconocer el doble valor que tienen las matemáticas: como
ciencia y como herramienta. 


\section{O ENSINO DE MATEMÁTICA POR MEIO DAS RENDAS DE BILROS: UM OLHAR PARA
O CONTEÚDO DE SIMETRIA}

\begin{datos}

Maria José Costa dos Santos, Iran Abreu Mendes,

UFC-Brasil , UFRN,

Brasil,

mazeautomatic@gmail.com; iamendes@gmail.com 

\end{datos}

Formação de professores; básico; empírico experimental Este trabalho
é o recorte de uma pesquisa realizada em nível de doutorado acerca
das possibilidades de construções matemáticas a partir da Visualização
Matemática por meio das rendas de bilro. A simetria é um conteúdo
de certo modo, esquecido nas escolas do Ensino fundamental, no entanto,
este trabalho tem como objetivo apresentar a importância desse conteúdo
na escola, de forma, mais significativa, dinâmica e divertida, apresentando
os entrelaces e entremeios possíveis para o ensino e aprendizagem
visando a interdisciplinaridade das temáticas. Para tanto, promopomos
uma discussão no âmbito da cientificidade apresentando a prática das
rendeiras de bilro como fonte de criatividade, de imaginação para
o desenvolvimento do conceito de simetria e padrões simétricos. Foi
necessário promover um diálogo, transversalizante entre os saberes
(matemática x rendas de bilro), tendo por desafio compreender os enlaces
das temáticas. Considerando que o conteúdo de simetria é relevante
para dar fundamentação conceitual para outros conteúdos matemáticos,
como as isometrias, entendemos que esse trabalho contribui para ampliar
e consolidar a educação matemática por meio da valorização de uma
prática artesanal, o que implica em outras discussões, como a conscientização
da Matemática na e para a cidadania.


\section{SOBRE LOS SIGNOS DE LAS RAZONES TRIGONOMÉTRICAS CON GEOGEBRA. UNA
MANERA DE TRASCENDER LAS REGLAS NEMOTÉCNICAS}

\begin{datos}

Stephanie Chiquinquirá Díaz Urdaneta, Juan Luis Prieto González.

Grupo Tecnologías en la Educación Matemática, Centro de Estudios Matemáticos
y Físicos, Universidad del Zulia, 

Venezuela,

stephanie.diaz@aprenderenred.com.ve; juan.prieto@aprenderenred.com.ve

\end{datos}

En este trabajo se describe una secuencia de análisis del comportamiento
geométrico de las razones seno, coseno y tangente de un ángulo, por
medio del software GeoGebra. Con esto se busca dotar de sentido a
los signos de estas razones en los distintos cuadrantes del plano
cartesiano, considerando la noción de razón trigonométrica de un ángulo
desde una perspectiva geométrica y vinculada a una circunferencia
unitaria. En este contexto, centramos la atención en los cambios de
sentido de los vectores representativos de las razones que se dibujan
sobre la circunferencia, como consecuencia de la variación de un ángulo
central asociado. 


\section{EL PAPEL DIDÁCTICO DE LAS TESELACIONES PARA EL ESTUDIO DE LOS POLÍGONOS
EN EL BACHILLERATO }

\begin{datos}

Josefa Osuna Márquez, Martha Cristina Villalva, Gutiérrez.

Universidad de Sonora,

México,

osunamj@gmail.com; mcris@gauss.uson.mx

\end{datos}

Se presenta la descripción del diseño de una secuencia de actividades
didácticas para el estudio de los polígonos en el bachillerato a través
de una organización matemática (OM) fundamentada en la Teoría Antropológica
de lo Didáctico. Se parte de una cuestión que, en el ambiente de la
geometría dinámica, permite promover el desarrollo de los procesos
cognitivos del pensamiento geométrico que son clasificados, según
Raymond Duval, en procesos de visualización, de construcción y de
razonamiento. Este diseño busca incidir en el desarrollo de competencias
que plantea el enfoque impulsado por la Reforma Integral de Educación
Media Superior en México.


\section{CONOCIMIENTO MATEMÁTICO PARA LA ENSEÑANZA DE LA CORRELACIÓN Y REGRESIÓN.
ANÁLISIS DE SUS COMPONENTES}

\begin{datos}

María M. Gea, Emilse Gómez-Torres, Carmen Batanero, Gustavo R. Cañadas.

Universidad de Granada, Universidad Nacional de Colombia,

España, Colombia,

mmgea@ugr.es; egomezt@unal.edu.co;

batanero@ugr.es; grcanadas@ugr.es

\end{datos}

Se analizan los componentes del conocimiento matemático para la enseñanza
de la correlación y regresión en el currículo de España. La propuesta
sobre conocimiento común y ampliado parte del estudio del tema en
16 libros de texto, y de un estudio histórico. Los conocimientos del
contenido y el estudiante y del contenido y la enseñanza se deducen
de una síntesis de la investigación sobre razonamiento covariacional
y comprensión de correlación y regresión, de las propuestas para la
enseñanza de estos temas, y del análisis de los textos. Para el conocimiento
del currículo se han analizado las directrices españolas y norteamericanas.


\section{UN RECURSO CON GEOGEBRA PARA ANALIZAR LA REFRACCIÓN Y REFLEXIÓN TOTAL
INTERNA}

\begin{datos}

Angela Cervantes, Leonela Rubio, Germain Montiel.

Grupo Tecnologías en la Educación Matemática, Centro de Estudios Matemáticos
y Físicos, Universidad del Zulia,

Venezuela,

angela.cervantes@aprenderenred.com.ve; leonela.rubio@aprenderenred.com.ve;

germainmontiel1@hotmail.com 

\end{datos}

En ocasiones, la falta de insumos afecta el desarrollo de las clases
de laboratorio de Física en secundaria. El uso de GeoGebra como simulador
de fenómenos físicos ofrece una alternativa para suplir esta falta.
Lo anterior amerita integrar las tecnologías en las prácticas pedagógicas,
lo que supone que los profesores comprendan los contenidos a enseñar
y conozcan las posibilidades didácticas del programa. Por ello este
trabajo presenta una secuencia para analizar la Refracción y Reflexión
total interna, utilizando GeoGebra, proporcionándole al profesorado
la oportunidad de conocer algunas bondades del software y guiándolo
en la integración de tecnologías en sus clases. 


\section{FORMAÇÃO DE PROFESSORES DOS ANOS INICIAIS: CONHECIMENTO PROFISSIONAL
DOCENTE AO EXPLORAR A INTRODUÇÃO DO CONCEITO DE FRAÇÃO}

\begin{datos}

Angélica da Fontoura Garcia Silva, Maria Gracilene de Carvalho Pinheiro,
Tânia Maria Mendonça Campos.

Universidade Bandeirante Anhanguera- UNIBAN,

Brasil ,

angelicafontoura@gmail.com; gracilenepinheiro@gmail.com;

taniammcampos@hotmail.com 

\end{datos}

Esta comunicação tem o propósito de analisar os conhecimento conhecimentos
necessários ao ensino de frações de um grupo de professoras que leciona
para os anos iniciais da Educação Básica que lecionam na rede estadual
de São Paulo. A investigação desenvolveu-se em um processo formativo,
por meio do qual foi possível analisar e refletir sobre a utilização
de situações quociente para a introdução do conceito de fração a partir
de uma sequência de tarefas em que é explorado esse significado. As
respostas do grupo indicaram que a análise do significado quociente
favoreceu a prática pedagógica do professor.


\section{MATEMÁTICAS Y LECTURA: UN BINOMIO INSEPARABLE EN EL APRENDIZAJE POR
COMPETENCIAS EN EL NIVEL MEDIO SUPERIOR}

\begin{datos}

Lucila Mendoza Toro, José Trinidad Ulloa Ibarra, Elsa García de Dios.

Universidad Autónoma de Nayarit. Cet Mar No. 26. 

México,

luzzy\_m13@hotmail.com; jtulloa@hotmail.com;

elsagd03.gmail.com 

\end{datos}

En la actualidad todo nuestro mundo está codificado mediante las palabras,
no se entiende una existencia independiente en este mundo si no conocemos
la forma de expresión reina: el lenguaje. Desde una perspectiva matemática,
no podemos decir que un alumno es competente matemáticamente si simplemente
sabe operar, es decir, si no tiene la capacidad de resolver problemas.
Pero, ¡cómo vamos a poder resolver un problema si no sabemos leerlo,
si no sabemos qué datos nos dan! Nuestro objetivo general propone
determinar el nivel de correlación entre la lectocomprensión y los
resultados en Geometría Analítica e identificar los componentes lectomatemáticos
problemáticos.


\section{LAS MATEMÁTICAS Y EL AJEDREZ, UNA EXPERIENCIA FORMATIVA DESDE EL
AULA}

\begin{datos}

Judith Bertel Behaine.

Institución educativa la Unión-Universidad de Sucre,

Colombia,

judithbertel@gmail.com

\end{datos}

Esta experiencia se desarrolla con el propósito, de motivar a los
estudiantes de la institución Educativa la Unión, por el estudio de
la matemática a través, de la formación y la práctica del ajedrez.
Con el juego ciencia como medio y siguiendo las etapas de iniciación,
fundamentación y perfeccionamiento como parte del desarrollo de este
proyecto transversal, se ha logrado desarrollar pensamiento matemático
a través de potenciar las habilidades mentales como la atención, memoria,
creatividad, toma de decisiones, el cálculo, planeación, comparación,
además fomentar los valores humanos, el aprovechamiento del tiempo
libre y el uso de la tecnología.


\section{MATEMÁTICA COMO AÇÃO POLÍTICA: UMA PROPOSTA PARA O ENSINO DE MATEMÁTICA
EM EDUCAÇÃO DE JOVENS E ADULTOS }

\begin{datos}

Arlete de Jesus Brito.

UNESP RIO CLARO, 

Brasil,

arlete@rc.unesp.br

\end{datos}

Nessa comunicação breve exporemos uma experiência desenvolvida em
aulas de primeiro ano do ensino médio na Educação de Jovens e Adultos
(EJA), de uma escola de periferia da cidade de Rio Claro, Brasil.
Essa experiência educativa foi guiada pelos pressupostos de Paulo
Freire. A partir da escolha, por parte dos alunos, de um tema a ser
estudado e das palavras geradoras envolvidas, quais sejam, salários
e gastos, foram elaboradas atividades em que a matemática foi usada
para analisar situações vivenciadas por esses alunos, o que colaborou
para a aprendizagem desse campo do saber.


\section{CONSIDERACIONES METODOLÓGICAS SOBRE LA DIAGONALIZACIÓN DE FORMAS
CUADRÁTICAS}

\begin{datos}

Luis Antonio Perfetti Villamil. 

Universidad Central “Martha Abreu de Las Villas”,

Cuba,

perfetti@uclv.edu.cu

\end{datos}

En el trabajo se muestra una metodología para la demostración de la
condición de diagonalizable para cualquier matriz cuadrada, simétrica
y de componentes reales, lográndose la diagonalización mediante una
transformación ortogonal de coordenadas, es decir, mediante una matriz
inversible cuya inversa es igual a su transpuesta. Se sustento en
tres resultados clásicos: El primero es que los autovalores de toda
matriz simétrica real son todos reales; el segundo, que toda matriz
cuyos autovalores pertenecen al campo de sus componentes es semejante
a una matriz triangular; y el tercero, que toda base, en un espacio
euclidiano de dimensión finita, se puede sustituir por una base ortonormal,
siendo la matriz de transición una matriz triangular superior. Consideramos
que esta forma de abordar el problema es de gran utilidad didáctica,
pues las demostraciones son sencillas, fáciles de comprender por los
estudiantes, quedando además implícito el método o algoritmo para
realizar la reducción. El método puede reforzarse con ilustrativas
interpretaciones geométricas, cuando se trata de formas cuadráticas
en espacios de dimensiones 2 o 3.


\section{DISEÑO DE UNA SITUACIÓN DE APRENDIZAJE PARA EL TEMA DE ELIPSE EN
ALUMNOS DE BACHILLERATO BASADA EN EL MODELO DE VAN HIELE}

\begin{datos}

Jonathan E. Martínez Medina, Sergio Dávila.

Espinosa Universidad Autónoma de San Luis Potosí, Facultad de Ciencias,

México,

jonathan\_martinez@alumnos.uaslp.edu.mx; sergio.davila@uaslp.mx 

\end{datos}

El presente trabajo comparte el diseño de una situación de aprendizaje
para el tema de elipse que se incluye en los cursos de Geometría Analítica,
impartidos en bachillerato (México) o equivalente. Dicha situación
está sustentada en el marco de referencia de la Evaluación PISA 2012
que proporciona un referente básico sobre los procesos, capacidades,
conocimientos y contextos implicados en el desarrollo de competencia
matemática; paralelamente se utiliza el modelo de razonamiento geométrico
de Van Hiele para el diseño de la secuencia de actividades de enseñanza-aprendizaje
para guiar a los estudiantes desde niveles de razonamiento intuitivos
hasta razonamientos abstractos. Se presenta al alumno el reto de dibujar
un plano de una réplica de la Plaza de San Pedro, la cual incluye
una elipse, en un pliego de papel albanene adaptado a las dimensiones
de un terreno específico. Se detallan, aprendizajes esperados, productos,
criterios de evaluación y secuencia de actividades. 


\section{UNA FORMULACIÓN CONCEPTUAL: CURVAS B-SPLINE}

\begin{datos}

Rogelio Ramos Carranza, Armando Aguilar Márquez, Frida María León
Rodríguez, Omar García León, Juan Rafael Garibay Bermúdez.

Universidad Nacional autónoma de México,

México,

egorrc@gmail.com; armandoa@unam.mx;

fridam@unam.mx; egor1131@unam.mx;

juragabe@unam.mx.

\end{datos}

Usando los principios de la transposición didáctica, nos proponemos
mediante este trabajo preparar al estudiante para la aprehensión del
objeto numérico, curvas b-spline; a partir de la apropiación de los
conceptos requeridos en la discusión y definición de las curvas b-spline.
Se trata de una investigación documental en la que se tiene el propósito
de mostrar y poner al alcance de los estudiantes, los fundamentos
del algoritmo b-spline; el documento se centrará en la discusión de
los conceptos matemáticos que conforman el modelo de curva considerado.
Una curva spline es una secuencia de segmentos que se conectan formando
una curva continua. 


\section{POR LA ABOLICIÓN DEFINTIVA DELALGORITMO DE LA RAÍZ CUADRADA DE LA
ESCUELA! }

\begin{datos}

Antonio Ramón Martín Adrián.

Colegio público,

España ,

tonycapicua@yahoo.es

\end{datos}

Cuando se hace referencia a la raíz cuadrada, rápidamente nos viene
a la mente el algoritmo, sin que la mayoría de las personas que se
han encontrado con este concepto (raíz cuadrada) sigan sin tener claro
que significa, prolongándose este desconocimiento a lo largo de la
vida escolar y personal. En esta comunicación se presentan situaciones
reales de enseñanza-aprendizaje, donde veremos a alumnas y alumnos
de 5º y 6º de Primaria apropiándose del concepto de raíz cuadrada.
Partiendo de situaciones problemáticas sobre el cálculo de áreas de
cuadrados y la relación con el lado.

\setcounter{section}{24}


\section{UNA FORMA DIVERTIDA DE APRENDER MATEMÁTICA}

\begin{datos}

Mayra Murillo.

Departamento de Matemática,- Universidad de Panamá,

mayramaxell@hotmail.com

\end{datos}

En ocasiones el docente se enfrenta a retos en la educación, uno de
ellos es que sus estudiantes se interesen por estudiar Matemática;
existen distintas estrategias metodológicas en la que el docente puede
seleccionar y realizar sus clases más amenas; y despertar el interés
por el estudio de la Matemática. Las actividades lúdicas ayudan a
que los saberes se conviertan en una opción para que los estudiantes
no vean con apatía la matemática. El objetivo de esta propuesta es
presentar distintas actividades que ayudan a que los saberes se refuercen
y permitan las interrelaciones entre los estudiantes, esta se enfoca
en temas de aritmética, geometría y algebra.


\section{CELDAS SOLARES, ACTIVIDAD DIDÁCTICA PARA PROMOVER LOS SIGNIFICADOS
DEL COEFICIENTE DE CORRELACIÓN, UTILIZANDO METODOLOGÍA ACODESA }

\begin{datos}

Irma Nancy Larios Rodríguez, Benjamín Morán Medina, Enrique Hugues
Galindo, Gerardo Gutiérrez Flores.

Universidad de Sonora,

México,

nancy@gauss.mat.uson.mx; bmoran@cetmar03.com.mx;

ehugues@gauss.mat.uson-mx; gerardo@ gauss.mat.uson.mx 

\end{datos}

En este trabajo se presenta una actividad didáctica titulada “Celdas
solares”, la actividad didáctica es la última de una secuencia de
actividades, cuya intención es promover un acercamiento intuitivo
al concepto de correlación lineal en estudiantes del curso de Probabilidad
y Estadística del Técnico en Electrónica (TE), (CoSDAc, 2010) del
Centro de Estudios Tecnológicos del Mar 03 Guaymas (CETMar 03) el
cual es considerado como un bachillerato tecnológico dentro del Sistema
Nacional de Bachillerato (SNB) de la Reforma Integral del Sistema
Medio Superior de México (RIEMS). 


\section{CONFLICTOS SEMIÓTICOS EN PRÁCTICAS ALGEBRAICAS DE DIVISIBILIDAD DE
ALUMNOS DE PROFESORADO EN MATEMÁTICA}

\begin{datos}

Marcel David Pochulu, Ricardo Fabian Espinoza.

Universidad Nacional de Villa María, Universidad Nacional del Nordeste,

Argentina,

marcelpochulu@hotmail.com ; rrfespinoza@gmail.com

\end{datos}

El trabajo tuvo por objetivo la determinación de conflictos semióticos
en prácticas algebraicas de estudiantes de Álgebra I, asignatura correspondiente
a la carrera Profesorado en Matemática de la Universidad Nacional
del Nordeste. Se tomó como contexto de estudio las demostraciones
de una propiedad de Divisibilidad elaboradas por los alumnos en la
segunda instancia de evaluación formal del curso 2013. A partir del
marco referencial del Enfoque Ontosemiótico, se avanzó en una relación
dialéctica entre prácticas institucionales y personales, lo cual permitió
determinar conflictos semióticos relacionados fundamentalmente con
la confusión entre objetos matemáticos tales como Relación divide,
División y Fracción.


\section{TRÁNSITO DE REGISTROS EN EL APRENDIZAJE DE LA MATEMÁTICA UNIVERSITARIA
BÁSICA}

\begin{datos}

Juan Carlos Sandoval Peña.

Instituto de Investigación sobre la enseñanza de las matemáticas (IREM),

Perú,

sandovaljc007@gmail.com ; Sandoval.j@pucp.edu.pe 

\end{datos}

En el trabajo se analizarán los aciertos y dificultades de 20 estudiantes,
futuros docentes de educación bilingüe, al realizar tratamientos y
conversiones de acuerdo a la teoría de Registros de Representación
semiótica de Duval. La investigación se centra en el estudio de los
números racionales, los estudiantes se enfrentan a situaciones de
contexto real, se analiza la tendencia sobre los registros y si logran
realizar tratamientos y conversiones. En la investigación utilizamos
la metodología de design expirements, cuyo foco de investigación es
entender e interpretar datos y discursos de estudiantes. La evaluación
final confirma el buen uso de los registros. 


\section{ENSEÑANZA DE LAS PROGRESIONES USANDO FICHAS DE CARTÓN.}

\begin{datos}

Pineda Lobo Dairo José, Urbiñez Herazo Daniel, Zabaleta Flórez Rafael
A. 

Universidad de Sucre, 

Colombia,

dajolp9110@gmail.com; leinadurbis@hotmail.com;

razf18@hotmail.com. 

\end{datos}

Este proyecto se fundamenta principalmente con la teoría del aprendizaje
creativo; con las fichas de cartón se espera desarrollar en los estudiantes
de nivel medio el concepto de progresión, además, el proceso de modelación
mediante talleres, en los cuales los estudiantes mediante tareas diversificadas
basadas en distintos arreglos creados por el docente, de acuerdo a
preguntas orientadoras hechas por él, estarán próximos a realizar
ciertas abstracciones que les permitirán crear sus propios modelos;
se espera que el docente sin necesidad de decirle al estudiante el
concepto lo induzca a éste y a los resultados que se esperan de cada
actividad.


\section{CINEMA E MATEMÁTICA: PROPOSTAS ENVOLVENDO A RESOLUÇÃO DE PROBLEMAS }

\begin{datos}

Leandro Millis da Silva, Isabel Cristina Machado de Lara.

Pontifícia Universidade Católica do Rio Grande do Sul – PUCRS,

Brasil,

prof\_millis@yahoo.com.br ; isabel.lara@pucrs.br

\end{datos}

Este estudo apresenta alguns resultados sobre uma pesquisa de Mestrado
desenvolvido com a temática cinema e matemática. Objetiva apresentar
possibilidades de utilização de filmes de ficção associados à resolução
de problemas, em sala de aula, verificando os diferentes papéis que
o filme pode assumir. Participam da pesquisa 4 professores de matemática
e 19 licenciandos em matemática. Para tanto, perfaz quatro etapas:
fundamentação teórica, intervenções dos pesquisadores, elaboração
das propostas, análise das aplicações. Identifica potencialidades
do uso do cinema como introdutor, motivador e contextualizador de
conteúdos, bem como apresenta dificuldades encontradas pelos professores
e ocorrências sobre a aceitação dos estudantes.


\section{MODELOS NUMÉRICOS DE INTERPOLAÇÃO E AJUSTE DE CURVAS COMO MÉTODO
DE CÁLCULO, APROXIMAÇÃO E CARACTERIZAÇÃO DE TENDÊNCIA DE DADOS EXPERIMENTAIS}

\begin{datos}

José Luiz Giarola Andrade, Dimas Felipe de Miranda, João Bosco Laudares.

Pontifícia Universidade Católica de Minas Gerais – PUC Minas,

Brasil,

giarolaandrade@hotmail.com; dimasfm48@yahoo.com.br; 

jblaudares@terra.com.br 

\end{datos}

Este trabalho relata a exploração, o comportamento e a caracterização
de modelos matemáticos de aproximação de funções, aplicados em amostras
de dados discretos, inserido em ambiente computacional, como estratégia
de ensino e aprendizagem em aulas de Cálculo Numérico nos cursos de
engenharia. O embasamento metodológico é sustentado pela Modelagem
Matemática descrita por Bassanezi, visando à utilização de Modelos
Matemáticos, como conjunto de símbolos e relações matemáticas que
representam de alguma forma o objeto estudado. Foram utilizados dois
softwares livres (Visual Cálculo Numérico e Scilab) permitindo uma
manipulação mais dinâmica dos modelos estudados.

\setcounter{section}{32}


\section{CONOCIMIENTO DEL CONTENIDO MATEMÁTICO DE UN PROFESOR PARA LA ENSEÑANZA
DE LAS FUNCIONES EN CUARTO AÑO DE SECUNDARIA EN COSTA RICA }

\begin{datos}

Ariana Rodríguez, Miguel Picado, Jonathan Espinoza.

Universidad Nacional de Costa Rica,

Costa Rica,

arodriguez89@gmail.com; miguepicado@hotmail.com;

jonaespinoza@una.cr

\end{datos}

El conocimiento del profesor sobre las matemáticas y su didáctica
constituye un tema de interés en el estudio del desempeño docente
en esta área. Esta investigación enfoca el conocimiento del contenido
matemático del profesor para la enseñanza de los conceptos básicos
sobre función. Se enmarca en las investigaciones sobre el conocimiento
matemático del profesor para la enseñanza. Corresponde a una investigación
cualitativa descriptiva basada en los estudios de caso en su fase
de ejecución. Se espera establecer indicadores para caracterizar el
conocimiento del contenido matemático del profesor sobre los conceptos
básicos de función en Costa Rica.


\section{EL PRINCIPIO DE MÍNIMA ACCION COMO ESCENARIO PARA LA OPTIMIZACION
Y MODELACIÓN MATEMÁTICA. }

\begin{datos}

David Maldonado Rico.

Universidad La Gran Colombia,

Colombia,

Davidc75814@hotmail.com 

\end{datos}

Formación de profesores, Nivel superior, Tipo de investigación: Teórico

Este principio enunciado de manera intuitiva y filosófica por el sabio
francés del siglo XVII Pierre Louis Moreau de Maupertuis fue expresado
por él de la siguiente manera como principio general: “En todo cambio
que se produzca en la naturaleza la cantidad de acción para tal cambio
ha de ser la mínima posible”. Si bien, el carácter abstracto de la
matemática es lo que ha permitido diferenciarla de otras ciencias,
también es cierto que una saturación de la misma en la enseñanza de
la matemática, puede ocasionar animadversión hacia aquellos conceptos
poco evidenciables y carentes de contexto en el proceso de aprendizaje
llevado en el área de matemáticas.


\section{EL LABORATORIO DE LA LEY DE HOOKE COMO ESCENARIO PARA LA CONSTRUCCIÓN
DE LA FUNCION LINEAL}

\begin{datos}

Carlos León, Camilo Sáchica, Marlon Gama, David Maldonado, Michael
Ocampo, Jenny Supelano, Cristian Gutiérrez.

Universidad La Gran Colombia,

Colombia,

carlos.leon@ugc.edu.co; jefer.sachica@ugc.edu.co;

marlon.gama@hotmail.com; davidc75814@hotmail.com

\end{datos}

Se plantea la física como escenario para la construcción de conocimiento
matemático, teniendo como metodología las prácticas generadas alrededor
de la experimentación: Medir, estimar, predecir, calcular, graficar,
comunes en Física, mas no en matemáticas. Valiéndonos de esto y del
desinterés estudiantil, el semillero diseña escenarios para la construcción
de conocimiento, donde se dé un nuevo significado del concepto, de
función lineal. Establecemos cuatro fases de diseño y análisis experimental.
Desarrollo teórico de leyes de Newton seguido de montaje experimental.
Recopilación de datos como parejas ordenadas. Representación gráfica
y análisis de datos. Finalmente una fase de discusión por parte de
escolares.

\setcounter{section}{36}


\section{DISEÑO DE UNA UNIDAD DIDÁCTICA PARA LA ENSEÑANZA DE LA GEOMETRÍA
EN DÉCIMO AÑO MEDIANTE LA RESOLUCIÓN DE PROBLEMAS }

\begin{datos}

Eithel Eduardo Trigueros Rodríguez.

Instituto Tecnológico de Costa Rica,

eitheltr@gmail.com

\end{datos}

En esta comunicación breve se presenta el diseño de una unidad didáctica
para el aprendizaje de la Geometría en décimo año de la Educación
Costarricense, aplicando la estrategia metodológica de la resolución
de problemas. Esta unidad corresponde al resultado de un proyecto
final de graduación de la licenciatura en Enseñanza de la Matemática
Asistida por computadora del Instituto Tecnológico de Costa Rica.


\section{CAÍDA LIBRE DE UN OBJETO: APLICACIÓN PRÁCTICA DE UN MODELO}

\begin{datos}

Claudia Martínez Pacheco.

Universidad Nacional, 

Costa Rica,

claudia.martinez.pacheco@una.cr 

\end{datos}

La física usa las matemáticas para explicar relaciones entre mediciones
en fenómenos naturales. Éstas reciben el nombre de “ley física” y
se expresan mediante fórmulas, resumiendo resultados de muchas experiencias
científicas. El estudiantado comprenderá mejor estas relaciones mediante
actividades en el aula que involucren conceptos físicos y matemáticos.
Para lograr la comprensión adecuada de algunas leyes físicas se llevó
a cabo una actividad, que resultó ser una buena experiencia de aula.
Los resultados obtenidos fueron muy buenos, se abarcaron conceptos
físicos y matemáticos, relacionándolos entre sí, y haciendo notar
el vínculo de las matemáticas con otras ciencias.


\section{LA ESTADÍSTICA: UNA MIRADA A TRAVÉS DE LOS REGISTROS SEMIÓTICOS. }

\begin{datos}

Zenón Eulogio Morales Martínez. 

Instituto de Investigación en Enseñanza de las Matemáticas – IREM-PUCP,

Perú,

morales.ze@pucp.edu.pe 

\end{datos}

Analizaremos las dificultades cuando el alumno estudia la estadística
descriptiva, estas dificultades se analizan a través de la Teoría
de Registros de Representaciones Semióticas, Duval (1995), que propone
un enfoque cognitivo en el aprendizaje de las matemáticas. Se analiza
el registro verbal para identificar la naturaleza de las variables
estadísticas; el registro tabular para representar tablas de frecuencias;
y el registro gráfico donde el alumno se enfrenta a conversiones semióticas
frente a una diversidad de gráficos estadísticos como el gráfico de
barras, gráficos circulares, gráficos de bastones, histogramas, polígonos
de frecuencia y ojivas que permiten representar las distintas variables
estadísticas.


\section{LA EPISTEMOLOGÍA DE LAS MATEMÁTICAS EN LOS PROGRAMAS DE LICENCIATURA
EN MATEMÁTICAS EN COLOMBIA}

\begin{datos}

Liliana Barón Amaris.

Universidad Popular del Cesar, 

Colombia,

lilianabaronamaris@gmail.com

\end{datos}

La epistemología reviste gran importancia especialmente en el campo
de la educación. Los elementos que aporta esta disciplina para el
estudio y análisis de la realidad desde diferentes posiciones epistemológicas
no se pueden desconocer. Esta investigación revisa la situación actual
de la epistemología de las matemáticas en los programas de Licenciatura
en Matemáticas en el contexto colombiano. En el país son ofertados
32 programas afines con la Licenciatura en Matemáticas, sin embargo
sólo cuatro programas contemplan dentro de su plan de estudio la asignatura
Epistemología de las Matemáticas; se revisa también los contenidos
de las asignaturas relacionadas con la epistemología.


\section{LAS GRÁFICAS CARTESIANAS Y LA SUPERACIÓN DE OBSTÁCULOS COGNITIVOS}

\begin{datos}

Miryan Trujillo Cedeño. 

Universidad de La Salle. Bogotá,

Colombia,

mtrujillo@unisalle.edu.co

\end{datos}

El trabajo hace parte de un proyecto de investigación iniciado dentro
de un programa de Doctorado en Educación Matemática. El estudio está
enmarcado en la Socioepistemológia, perspectiva que considera el carácter
social de la matemática. El problema se ha planteado en torno a la
presencia de obstáculos cognitivos en el estudio de la variación y
el cambio., pretendiéndose hacer una indagación de cómo se argumenta
con las gráficas, cuando se estudian obstáculos referidos a la variación,
a través del diseño de un ambiente de trabajo gráfico variacional.
La metodología para el diseño didáctico es la ingeniería didáctica. 

\setcounter{section}{42}


\section{INVESTIGACIONES SOBRE EL PENSAMIENTO ALGEBRAICO, UNA REVISIÓN }

\begin{datos}

Sergio Damián Chalé Can, Claudia Margarita Acuña Soto.

Centro de Investigación y de Estudios Avanzados del Instituto Politécnico
Nacional Cinvestav-IPN,

México,

schalecan@gmail.com 

\end{datos}

En este escrito, pretendemos compartir nuestras reflexiones acerca
de los resultados de investigación relacionados con el Pensamiento
Algebraico. Considerando los trabajos realizados en el seno de la
reunión del Conference of International Group of the Psychology of
Mathematics Education en sus 30 años, así como trabajos de investigación
actuales publicados en diversas revistas internacionales. La finalidad
es entender la naturaleza y los fundamentos teóricos de los resultados
de investigación que se vinculan con los lineamientos de la organización
de los contenidos, los cursos y los libros de texto actuales de la
enseñanza del álgebra en México.


\section{EDUCACIÓN MATEMÁTICA DESDE LA PRÁCTICA PEDAGÓGICA INVESTIGATIVA}

\begin{datos}

Alberto Iriarte Pupo, Ubaldo Buelvas Solórzano.

Universidad de Sucre,

Colombia ,

albertoiriarte4@yahoo.es; ubaldo959@hotmail.com 

\end{datos}

La Práctica Pedagógica Investigativa, en el Programa de Licenciatura
en Matemáticas de la Universidad de Sucre Colombia, está orientada
hacia la denominada investigación formativa, la cual se centra en
problemas que se presentan en la enseñanza y en el aprendizaje de
las matemáticas, particularmente los que se dan en las aulas de clase
de las escuelas y colegios del departamento de Sucre y la región,
teniendo en cuenta el entorno sociocultural en que las instituciones
se hallan ubicadas a fin de establecer la relación existente entre
el programa de matemáticas y las comunidades. 


\section{TRATAMIENTO HISTÓRICO COMO MÉTODO DE ENSEÑANZA DEL CÁLCULO INTEGRAL
EN ESTUDIANTES DE INGENIERÍA EN CIENCIAS INFORMÁTICAS.}

\begin{datos}

Noel Márquez Batista, Idania Esther Urrutia Romani. 

Universidad de las Ciencias Informáticas (UCI),

Cuba,

nmarquez@uci.cu; idania@matcom.uh.cu

\end{datos}

El trabajo tiene como objetivo exponer el resultado de una investigación
que permitió utilizar el tratamiento histórico como método de enseñanza
a raíz de dificultades encontradas en el aprendizaje del Cálculo Integral
en el primer año de la carrera de ingeniería en ciencias informáticas.
Se realiza un sistema de tareas mediante problemas que condujeron
de alguna forma a la creación del cálculo integral, para que sea un
recurso en el aprendizaje, atractivo y motivador que puesto en manos
del estudiante y del profesor puede contribuir a desarrollar las habilidades
requeridas para el cálculo de integrales. 


\section{INTRODUCCIÓN DE CONCEPTOS DEMOGRÁFICOS EN LAS CLASES DE MATEMÁTICAS
DEL CURSO DE FORMACIÓN INTERCULTURAL DE PROFESORES INDÍGENAS FIEI}

\begin{datos}

Vanessa Sena Tomaz, Rafael Andrés Urrego Posada.

Universidad Federal de Minas Gerais,

Brasil,

vanessastomaz@gmail.com; raurregop@gmail.com

\end{datos}

Este trabajo relata la experiencia de introducir conceptos demográficos
en clases de matemáticas del curso de Formación Intercultural de Educadores
Indígenas, de la Universidad Federal de Minas Gerais. Esta experiencia
revela que la metodología transdisciplinar utilizada en las clases
es una forma de concientización, pues se evidenció como los estudiantes
tomaron consciencia de problemáticas de sus comunidades y de cómo
poder transformarlas. Además, la introducción de conceptos demográficos
en las clases, sirvió para subvertir la práctica tradicional de la
Demografía, históricamente asociada al control estatal de las poblaciones,
colocando sus técnicas y conceptos al servicio de poblaciones socialmente
excluidas.


\section{SISTEMA DE TAREAS PARA DESARROLLAR EL TALENTO INFORMÁTICO DESDE EL
PROCESO DE ENSEÑANZA APRENDIZAJE DEL CÁLCULO DIFERENCIA E INTEGRAL}

\begin{datos}

Luis Eduardo Benítez Oliva, Manuel Villanueva Betancourt.

Universidad de las Ciencias Informáticas (UCI),

Cuba,

lebenitez@uci.cu; manuelvb@uci.cu 

\end{datos}

En este trabajo se exponen las principales características de un sistema
de tareas sobre el cálculo diferencial e integral de funciones reales
de una variable real, concebido por el uso determinante de las Tecnologías
de la Informática y las Comunicaciones (TICs) en la formación, para
que sea un recurso de aprendizaje, atractivo y motivador, como apoyo
a la preparación de estudiantes potencialmente talentosos en el área
de la informática, y para su desempeño dinámico y protagónico dentro
de la clase y en proyectos de producción de software.


\section{SUDOKU PARA INCENTIVAR LA ATENCION EN PREESCOLARES }

\begin{datos}

Alberto Iriarte, Rosario Ariza.

Institución Educativa San Antonio Club de Leones, 

Colombia,

Rossi0507@gmail.com; albertoiriarte4@yahoo.es 

\end{datos}

El proceso puesto en marcha en el aula de preescolar, admitió llevar
a cabo una investigación descriptiva de corte cualitativo, donde se
pudo establecer la relación positiva existente entre un juego como
el Sudoku y el desarrollo de habilidades en la orientación espacial,
direccionalidad, la atención, concentración, la percepción, discriminación
y memoria visual, conteo, orden, seriación y resolución de problemas.
La intervención de aula se realizó en tres momentos o fases: diagnóstico,
adquisición de conceptos, y puesta en marcha del juego. 


\section{“ERRORES FRECUENTES EN EL TEMA MATRICES” }

\begin{datos}

Martínez Irma Zulema, Crespo Sergio Hernán.

Facultad de Ciencias Exactas-Facultad de Ciencias Económicas - Universidad
Nacional de Salta,

Argentina,

irmart@unsa.edu.ar screspo@ucasal.net 

\end{datos}

Observando distintas modalidades de evaluaciones, en temas de asignaturas
de 1º Año de diversas carreras de la UNSa, y como parte del plan de
mejoras en el proceso de enseñanza aprendizaje, se viene realizando
el estudio y análisis de errores y dificultades, en ésta ocasión el
tema: Matrices para aprovechar con ello un mejor tratamiento integral
del tema, planteando estrategias, como también la incorporación del
soft Maple. Estas actividades se incorporan a las tradicionales, permitiendo
que los alumnos amplíen, sus producciones por la rapidez de resolución
induciendo así, mayor dedicación, motivada por el uso de recursos
de la tecnología educativa. 


\section{ETNOCIÊNCIA: UM OLHAR SOBRE OS SABERES TRADICIONAIS DA PESCA ARTESANAL }

\begin{datos}

Mayara de Araujo Saldanha, Isabel Cristina Machado de Lara.

Pontifícia Universidade Católica do Rio Grande do Sul,

Brasil,

mayara.saldanha@acad.pucrs.br; isabel.lara@pucrs.br

\end{datos}

Nossa investigação que visa explicar como o sentido de um grupo é
construído, mantido e transformado, constitui um estudo etnomedológico,
cujo propósito é compreender os saberes tradicionais da pesca artesanal.
Apresenta, em um primeiro momento, uma breve discussão a respeito
da maneira como os diferentes modos de pensar e conhecer foram sendo
colocados à margem da sociedade, apontando para a necessidade de valorizar
os saberes dos diferentes grupos culturais. Em seguida, aborda alguns
aspectos quanto às concepções de Etnomatemática e Etnociência, evidenciando
as razões pelas quais os estudos “etnos” têm sido reconhecidos como
uma possibilidade de resgatar esses saberes que de algum modo se perderam
ao longo da história. Voltando à atenção, em particular, para os saberes
dos pescadores artesanais, como produtores de um saber que lhes é
próprio apresenta uma breve descrição da cultura da pesca na Ilha
da Pintada 


\section{MODELACIÓN MATEMATICA: UN CASO DE LA FUNCION POR PARTES}

\begin{datos}

Gloria Ines Neira Sanabria, Manuel Andres Castiblanco Acosta.

Universidad Distrital Francisco Jose de Caldas,

Bogotá - Colombia,

nicolauval@yahoo.es; mcastiblanco7@yahoo.es

\end{datos}

El presente trabajo trata de mostrar qué elementos emergen en la construcción
de modelos matemáticos, la investigación se realizó con un grupo de
seis estudiantes de grado noveno de un colegio público de Bogotá,
como tema de investigación se abordó la situación “El Valor del servicio
de acueducto a nivel residencial en función del consumo de agua potable”.
Para desarrollar el concepto de función por partes, función constante
y parte entera. El marco teórico se fundamentó en los trabajo de Maria
Salet Biembengut y Jhony Alexander Villa Ocho, dos exponentes de la
perspectiva educativa de la Modelación matemática.


\section{PRUEBA DE INGRESO A LA EDUACIÓN SUPERIOR EN MATEMÁTICAS}

\begin{datos}

Marco Antonio Ramirez Porras, Frey Rodríguez Pérez.

Corporación Universitaria Minuto de Dios – UNIMINUTO – Sede Bogotá,

Colombia , mramirez@uniminuto.edu ; frodriguez@uniminutio.edu

\end{datos}

El Departamento de Ciencias Básicas, de UNIMINUTO Sede Bogotá Colombia,
consciente de las debilidades en el campo de la matemática de los
estudiantes cuando ingresan a la educación superior desarrolló una
investigación centrada en identificar los elementos que se deberían
evaluar en una prueba de manera que permitiera identificar las dificultades
y carencias de los estudiantes y no discriminarlos por su desempeño.
Luego de aplicar tres pruebas se estableció que aquella que estuvo
centrada en pre habilidades matemáticas permitió diagnósticar dificultades
y errores de una manera clara y precisa en comparación con las centradas
en competencias generales y conceptos previos. 


\section{OS PROCESSOS DE RESOLUÇÃO DE SISTEMA DE EQUAÇÕES DO 1º GRAU COM DUAS
INCÓGNITAS POR ALUNOS DO 9º. ANO DO ENSINO FUNDAMENTAL}

\begin{datos}

Maurílio Antonio Valentim, Maria Helena Palma de Oliveira.

Universidade Anhanguera de São Paulo,

Brasil,

valenttinos@yahoo.com.br; mhelenapalma@gmail.com 

\end{datos}

Este artigo é parte de um estudo maior sobre linguagem e produção
de significados por parte de alunos do 9º ano do Ensino Fundamental
de uma escola pública de Minas gerais na resolução de atividades com
sistema de equações do 1º grau com duas incógnitas. Buscou-se classificar
as resoluções das atividades dos alunos como decorrentes do uso de
técnicas ou de tentativa e erro. Participaram da pesquisa 18 alunos
do 9º ano do Ensino Fundamental. Ficou evidenciado que o ambiente
onde se coletou os dados e alguns aspectos sócio-culturais influenciaram
a escolha do tipo de resolução adotado pelos alunos.


\section{TRATAMIENTO DIDÁCTICO DE TEMAS ESPECÍFICOS EN TEXTOS DE GEOMETRÍA
ESCOLAR}

\begin{datos}

Élgar Gualdrón.

Grupo de investigación EDUMATEST- Universidad de Pamplona,

Colombia,

elgargualdron@yahoo.es

\end{datos}

Una de las líneas de investigación en Didáctica de las Matemáticas,
que ha tomado fuerza en tiempos recientes, se ha enfocado en analizar
los libros de texto escolares, principalmente por la importancia que
tienen en el proceso de enseñanza/aprendizaje de esta disciplina.
Hemos realizado un estudio que consistió en realizar una réplica de
la investigación reportada por Jaime, Chapa y Gutiérrez (1992), los
cuales caracterizaron los errores que se presentan en libros de texto
de matemáticas de la Educación Básica, en relación con el enunciado
de definiciones de triángulos y cuadriláteros. Los resultados sugieren
que, después de más de dos décadas, los hallazgos no tuvieron el impacto
que se hubiera esperado dentro de la comunidad internacional, a pesar
que la publicación de los resultados es de fácil acceso.

\setcounter{section}{55}


\section{CONCEPTUALIZACION DE LA DERIVADA CON GEOGEBRA }

\begin{datos}

María Celina Morales Rivera, Jorge Armado Peralta Samano.

Cecyte-UAEM,

México,

macelimo@hotmail.com; samano@uaem.mx 

\end{datos}

Se propone una estrategia didáctica en la que se visualizan los efectos
de los parámetros dados en el deslizador del GeoGebra. Está dirigida
a estudiantes de cuarto semestre de bachillerato que cursan la asignatura
de Cálculo, misma que se encuentra dentro del marco curricular común
(MCC) de la Reforma Integral de la Educación Media Superior (RIEMS).
La metodología consiste en el diseño de una secuencia de actividades
para obtener la conceptualización de la derivada desde el punto de
vista gráfico y algebraico. Nuestro marco teórico se basa en la visualización. 


\section{PROPUESTA DIDÁCTICA PARA EL APRENDIZAJE AUTOGESTIVO DE LA DERIVADA
DE UNA FUNCIÓN PRESENTADA EN AMBIENTE MULTIMEDIA.}

\begin{datos}

Karla Liliana Puga Nathal, Eliseo Santoyo Teyes, Felipe Santoyo Telles,
María Eugenia Puga Nathal. 

Instituto Tecnológico de Cd. Guzmán Jalisco, Centro Universitario
del Sur de la Universidad de Guadalajara Centro de Bachillerato Tecnológico
Industrial y de Servicios 226 (CBTis 226) Cd. Guzmán Jalisco,

México,

esantoyo25@hotmail.com; santf22@hotmail.com;

karlalpn4@hotmail.com; kenapn@hotmail.com

\end{datos}

Este trabajo promueve la interacción con el concepto de “derivada
de una función” a partir del desarrollo de cuadernos de trabajo presentados
en ambiente multimedia, en los cuales se abordan tanto la interpretación
física de la derivada, como la interpretación geométrica de la misma
y los algoritmos de derivación, a partir de la resolución de situaciones
problema que permitan al estudiante apropiarse de modo autogestivo
de los conceptos, y adquirir habilidades en la resolución de problemas,
se aborda desde diversos marcos de representación, considerando que
el significado se desprende de las acciones que el estudiante ejecuta
sobre los objetos matemáticos. 


\section{PRÁTICAS CURRICULARES EM MATEMÁTICA: DAS FORMAS DE SE APRENDER AS
FORMAS DE SE CONDUZIR}

\begin{datos}

Adriani Mello Felix, Márcia Souza da Fonseca.

Universidade Federal de Pelotas,

Brasil,

adrianifelix@gmail.com; mszfonseca@gmail.com; 

\end{datos}

Este trabalho analisa as condições discursivas que possibilitaram
as reformas curriculares para o Ensino Médio no Brasil, com um olhar
mais atento ao Ensino Médio Politécnico no Rio Grande do Sul. Buscamos
compreender, nos discursos curriculares oficiais, como a matemática
tornou-se um dispositivo na condução dos sujeitos via currículo e
políticas públicas endereçadas à educação. Através do conceito de
práticas discursivas e governamentalidade em Foucault, estudamos as
articulações entre contextualização, saber estatístico e matemática.
O espaço de análise se constituiu dos documentos curriculares para
o ensino médio e de sua materialização em Projetos Vivenciais, do
seminário integrado no RS. 


\section{POSIBILIDADES DE DESARROLLO DE LA HABILIDAD DE UNA PERSONA PARA RESOLVER
PROBLEMAS MATEMÁTICOS}

\begin{datos}

Norma Constanza Sarmiento Benavidez, William Becerra Salamanca.

Universidad Militar Nueva Granada,

Colombia,

norma.sarmiento@unimilitar.edu.co ; william.becerra@unimilitar.edu.co

\end{datos}

El proyecto de investigación cuyos aspectos esenciales se reportan
en la exposición, se realizó con el propósito inicial de indagar acerca
de los procesos cognitivos involucrados en las dificultades que tienen
los estudiantes de Cálculo Diferencial para resolver problemas de
optimización. Se desarrollan, por otra parte los fundamentos de una
propuesta didáctica enfocada al desarrollo de la habilidad para resolver
problemas matemáticos. El marco teórico de la mencionada propuesta
se encuentra esencialmente en los tratamientos heurísticos desarrollados
por George Polya, Allan Schoenfeld y Luz Manuel Santos Trigo, y en
algunas consideraciones de carácter semiótico referidas al empleo
de diferentes formas de representación desarrolladas por Raymond Duval.


\section{EPISTEMOLOGÍA DE LOS PROFESORES SOBRE LA NATURALEZA DEL CONOCIMIENTO
MATEMÁTICO: UN ESTUDIO SOCIOEPISTEMOLÓGICO DE LA ENSEÑANZA DEL TEOREMA
DE PITÁGORAS}

\begin{datos}

Karla Sepúlveda Obreque.

Instituto Politécnico Nacional-CICATA,

México,

ksepulveda@uct.cl; ksepulvedao1400@alumno.ipn.mx

\end{datos}

A continuación se presenta una propuesta de investigación doctoral
que busca dar respuesta a la pregunta ¿Cuál es la epistemología de
los profesores sobre la naturaleza del conocimiento matemático que
enseñan? La intención es compartir con otros colegas en la búsqueda
de comentarios que orienten de mejor manera la pregunta y el desarrollo
de este estudio. Se parte de la base que América Latina está dominada
por el conocimiento moderno racional de carácter eurocéntrico, esta
característica de colonización cognitiva permea el curriculum invisibilizando
otras formas de conocimiento locales distintas al conocimiento científico.
Desde aquí el interés de conocer cuáles son las epistemologías consientes
o subyacentes del profesorado respecto al conocimiento matemático
que enseñan. 


\section{\uppercase{ ACTIVIDADES QUE planifica el profesor PARA su clase
de Matemática}}

\begin{datos}

Jeannette Galleguillos.

Universidad Estadual Paulista UNESP,

Brasil,

jeannette.galleguillos@gmail.com

\end{datos}

En este trabajo se explora la forma en que el profesor planifica su
clase de matemática focalizándose en las actividades que prepara.
Para ello se analizaron entrevistas a profesores de escuelas básicas
rurales. Se obtiene, que para planificar actividades los profesores
consideran el entorno y los intereses de los estudiantes. Los profesores
utilizan problemas matemáticos, juegos didácticos y actividades en
que está presente el material concreto. Se omiten elementos dinámicos
que aparecen en las actividades de otras asignaturas como canciones
y actuación de papeles, lo que intuye una visión de la matemática
como una disciplina difícil y exenta de dinamismo.


\section{IDONEIDAD DIDÁCTICA. IMPLICACIONES PARA LA FORMACIÓN DE PROFESORES.}

\begin{datos}

Damian A. Clemente Olague, Sabrina P. Canedo Ibarra.

Centro de Estudios Universitarios Vizcaya de las Américas, Manzanillo,

México,

damian.alex03@gmail.com; sabrinacanedo@hotmail.com

\end{datos}

El trabajo como docente tiene un carácter eminentemente práctico en
que es indispensable un cuerpo teórico que explique el por qué se
actúa de una forma determinada en el aula y cuáles son las fundamentaciones
de esas actuaciones. El enfoque ontosemiótico de la cognición e instrucción
matemática aporta herramientas teóricas que articulan diversas facetas
implicadas en el proceso de enseñanza matemática para el análisis
conjunto del pensamiento matemático. Este trabajo se propone como
un estudio de caso que permita la caracterización de la instrucción
matemática en la educación media superior para la mejora en los proceso
de formación del profesorado.


\section{AULA INVERTIDA: UNA EXPERIENCIA DIDÁCTICA EN UN CURSO DE CÁLCULO
PARA NEGOCIOS}

\begin{datos}

Elvira G. Rincón Flores, Dora Elia Cienfuegos Zurita, Delia Galván
Sánchez, María de la Luz Fabela Rodríguez.

Instituto Tecnológico y de Estudios Superiores de Monterrey - Campus
Monterrey,

México,

elvira.rincon@itesm.mx; dcienfue@itesm.mx;

delia.galvan@itesm.mx; mfabela@itesm.mx

\end{datos}

El aula invertida pretende que el alumno aprenda por cuenta propia
ayudado de la tecnología, luego regresa al aula a resolver situaciones
didácticas colaborativamente y con la guía del docente. La estrategia
se aplicó en un curso de Cálculo para negocios. Durante la experiencia
se observó una mayor interacción entre los alumnos así como entre
los alumnos y el profesor, el ambiente en el aula se volvió, más dinámico,
más rico en experiencias y más interesante, además, el estudiante
pudo comprobar su capacidad de autoaprendizaje a pesar de tratarse
de un curso de Cálculo.


\section{SOFTWARE LIBRE EN UN CURSO DE MÉTODOS NUMÉRICOS }

\begin{datos}

Lenin Augusto Echavarría Cepeda.

Unidad Profesional Interdisciplinaria de Ingeniería Campus Guanajuato
- IPN, 

México,

laugusto@ipn.mx 

\end{datos}

Uno de los problemas que pueden enfrentar los estudiantes en cualquier
curso es el acceso a recursos de aprendizaje tales como lecturas,
software y materiales de evaluación. En el diseño de un curso de métodos
numéricos, se integraron tres herramientas de software libre para
proveer de esos recursos a los estudiantes. Con este diseño se logra
tener actualidad en las herramientas disponibles para los futuros
ingenieros en la resolución de problemas matemáticos. Se describirá
el papel que juega cada una de estas herramientas, lo cual puede ser
útil para futuros proyectos de investigación.


\section{NIVEL DE DESARROLLO DEL PENSAMIENTO MATEMÁTICO DE LOS ESTUDIANTES
DEL DEPARTAMENTO DE SUCRE}

\begin{datos}


 Alfonso E. Chaucanés Jácome, Jairo Escorcía Mercado, Eugenio Therán
Palacio, Atilano Medrano Suarez. Edgar Miguel Madrid Cuello.

Universidad de Sucre,

Colombia,


 chaucane@yahoo.com; escorciamercadojairo@yahoo.es;

eugeniotheran@gmail.com 

\end{datos}

Este artículo da a conocer los resultados de una investigación realizada
con el propósito de determinar el nivel de desarrollo del pensamiento
matemático en los estudiantes de Educación Secundaria del Departamento
de Sucre, Colombia. Se realiza un análisis comparativo de los niveles
de desempeño de los estudiantes. La información, insumo del trabajo,
fueron los resultados obtenidos por los estudiantes en las XIII Olimpiadas
de Matemáticas 2012 Unisucre. El trabajo es de tipo descriptivo, con
un enfoque cuanti-cualitativo. La investigación se dio en dos momentos:
el de la recolección de la información y el relacionado con el procesamiento
de la misma.

\setcounter{section}{66}


\section{ESTRATEGIA PARA LA ENSEÑANZA DE LAS MATEMÁTICAS EN AULAS INCLUYENTES
CON ESTUDIANTES QUE PRESENTAN LIMITACIÓN VISUAL.}

\begin{datos}

Camilo Salgado, Claudia Castro.

Universidad Distrital - Francisco José de Caldas,

Bogotá,

Colombia,

camiloud@gmail.com; mathclaudiacastro@yahoo.com

\end{datos}

La estrategia de enseñanza se desarrolló durante el segundo semestre
del año 2011, en el Centro Educativo Distrital LA O.E.A, en la ciudad
de Bogotá, colegio que cuenta con aulas incluyentes para estudiantes
que presentan limitación visual y baja visión. Se abordó con el fin
de generar herramientas en las aulas incluyentes con recursos didácticos
adaptados para todos los estudiantes. Se trabajó con una secuencia
de actividades, encaminadas en los pensamientos métrico, geométrico
y variacional.

\setcounter{section}{69}


\section{ECUACIONES DIFERENCIALES ORDINARIAS ASISTIDAS POR COMPUTADOR }

\begin{datos}

Yolima Álvarez Polo, Asdrúbal Moreno Mosquera.

Universidad Distrital Francisco José de Caldas,

Colombia,

yalvarezp@udistrital.edu.co; asmorenomosquera@gmail.com

\end{datos}

En el desarrollo de los cursos de ecuaciones diferenciales ordinarias
en Facultades de Ingeniería en Colombia se hace necesario emplear
herramientas computacionales que permitan una mayor comprensión de
los conceptos allí tratados. Las técnicas cualitativa y numérica son
un complemento importante a la parte analítica que se suele enseñar
en dichos cursos, y se apoyan fundamentalmente en el uso de software
especializado, como Derive, MatLab y Mathematica. En este trabajo
presentamos resultados de nuestra experiencia en el aula con estudiantes
de Ingeniería de la Universidad Distrital Francisco José de Caldas. 


\section{ASPECTOS MEDIACIONALES PARA LA ENSEÑANZA DEL VOLUMEN EN EDUCACIÓN
MEDIA GENERAL.}

\begin{datos}

Yraima Ramos, Angélica María Martínez, Mario Arrieche.

Universidad Pedagógica Experimental Libertador de Maracay.

Venezuela, 

yraimaramos@gmail.com; angelicmar5@gmail.com;

marioarrieche@hotmail.com

\end{datos}

Se presenta a través de este trabajo una experiencia didáctica cuyo
objetivo principal consistió en establecer el grado de adecuación
de los recursos materiales y temporales que se utilizaron para llevar
a cabo una estrategia de enseñanza y aprendizaje de volumen de cuerpos
geométricos, fundamentada en el uso de materiales concretos. Teóricamente,
se basa en el modelo del enfoque ontosemiótico de la cognición e instrucción
matemática (EOS) y metodológicamente se trata de una investigación
de estudio de casos, cuyos sujetos informantes fueron 38 estudiantes
del primer año de Educación Media General, de la Unidad Educativa
“Hipólito Cisneros”, de Venezuela. 


\section{FUNDAMENTOS DE LA GEOMETRIA DEL ESPACIO-TIEMPO DE MINKOWSKI }

\begin{datos}

Asdrúbal Moreno Mosquera, Yolima Álvarez Polo.

Universidad Distrital Francisco José de Caldas,

Colombia,

asmorenomosquera@gmail.com; yalvarezp@udistrital.edu.co

\end{datos}

Siendo el Álgebra Lineal tan asequible como los números complejos
a los estudiantes universitarios, se plantea estudiar las propiedades
métricas seudoeuclídeas en el plano complejo. Se propone desarrollar
la cinemática del Universo de Minkowski, analizando la causalidad,
las líneas de universo, la contracción de longitudes y la dilatación
del tiempo en el marco de la geometría hiperbólica basada en la circunferencia
unitaria Minkowskiana. Adicionalmente se pretende realizar un contraste
entre la concepción Newtoniana del Universo en el marco de la geometría
euclidea y la interpretación Minkowskiana en términos de la geometría
de un espacio hiperbólico.


\section{APROXIMACIÓN AL NÚMERO RACIONAL A PARTIR DE LAS TABLAS DE VALOR NUTRICIONAL
DE LOS ALIMENTOS QUE CONSUMEN LOS ESTUDIANTES }

\begin{datos}

Wilfaver Hernández Montañez.

Universidad Pedagógica Nacional - Colegio Distrital El Rodeo,

Bogotá,

Colombia,

hhh.mmm.hernandez@gmail.com 

\end{datos}

Tomando como referencia la formación matemática, que debe girar en
torno a la utilización de las matemáticas para solucionar situaciones
en contexto, la capacidad lectora que incluye la habilidad para leer
materiales escolares y no escolares y la formación científica que
incluye la capacidad de resolver problemas en situaciones del mundo
real que pueden afectarnos como individuos (OCDE), se diseñó, implementó
y evaluó un proyecto de aula enfocado hacia el estudio de los números
racionales a partir de las tablas de valor nutricional de los alimentos
consumidos por los estudiantes de grado séptimo del colegio Distrital
El Rodeo (Bogotá, Colombia). 


\section{ALFABETIZACIÓN ESTADÍSTICA: UN PENDIENTE DEL CURRÍCULO DEL BACHILLERATO
EN MÉXICO}

\begin{datos}

José Luis Torres Guerrero, Blanca R. Ruiz Hernández, Liliana Suárez
Téllez.

CECyT 7 Cuauhtémoc, Instituto Politécnico Nacional,

México,

jeluistg@yahoo.com.mx; bruiz@itesm.mx;

lilianasuarezt@gmail.com

\end{datos}

Recientemente en México se han realizado una Reforma Integral de la
Educación Media Superior y establecida la obligatoriedad de la Educación
Media Superior. Esto ha llevado a un rediseño curricular en todos
los subsistemas de este nivel educativo. Se establecieron ocho competencias
en matemáticas, pero ninguna establece la necesidad de lograr la alfabetización
estadística en los estudiantes. Por otro lado, la pertinencia de esto
se manifiesta por el uso de la estadística para la toma de decisiones
que afectan al ciudadano común y por resultados de investigación de
matemática educativa. Existen también dificultades para el aprendizaje
a tomar en cuenta. 


\section{COMUNIDAD DE CONOCIMIENTO MATEMÁTICO}

\begin{datos}

Claudia Méndez; Claudio Opazo, Teresa Parra, Rosario Pérez, Francisco
Cordero. 

Centro de Investigación y de Estudios Avanzados del Instituto Politécnico
Nacional,

México,

clmendezb@cinvestav.mx; opazoferrari\_claudio@hotmail.com; 

parra.tere@gmail.com;

\end{datos}

Ésta es una reflexión colegiada con base en investigaciones en desarrollo
de corte socioepistemológico que intentan caracterizar al colectivo
que genera nociones matemáticas a partir de su práctica en diferentes
contextos, donde adquieren una diversidad de significados mediante
sus distintos usos: Si hay conocimiento hay una comunidad que lo construye,
a la que llamamos comunidad de conocimiento matemático, la cual se
explica mediante la triada: Localidad, intimidad y reciprocidad, y
los ejes identidad e institucionalización.


\section{FORMACIÓN DE PROFESORES DE BÁSICA PRIMARIA EN EL MARCO DEL ENFOQUE
ONTOSEMIÓTICO (EOS) MEDIANTE LA METODOLOGÍA ESTUDIO DE CLASE (MEC).}

\begin{datos}

William Tapasco Gañán, Eliécer Aldana Bermúdez.

Universidad del Quindío,

Armenia – Quindío - Colombia ,

williamtapasco@hotmail.com; eliecerab@uniquindio.edu.co

\end{datos}

Este reporte hace parte de un estudio en proceso que tiene como objetivo
la formación de profesores en ejercicio en el área de matemáticas
grados 4\textdegree{} y 5\textdegree{} de básica primaria de una institución
educativa oficial de Colombia. En el acompañamiento que se realiza
se pudo evidenciar las dificultades y limitaciones que presentan los
profesores en la enseñanza del concepto de fracción. Esta situación
es hallada también en estudios anteriores, realizados en diferentes
contextos. La investigación se aborda desde el enfoque ontosemiótico,
en las facetas epistémica y cognitiva, aplicando para el estudio la
metodología estudio de clase. 


\section{SENTIDO CRÍTICO EN LA EDUCACIÓN MATEMÁTICA. RECUENTO DESDE LA MODELIZACIÓN. }

\begin{datos}

Arnaldo Mendible, José Ortiz.

Universidad Nacional Experimental Politécnica de la Fuerza Armada
- Universidad de Carabobo,

Venezuela,

arnmen2005@yahoo.com; ortizbuitrago@gmail.com 

\end{datos}

El sentido crítico es manifestación del hombre. En las interacciones
sociales se construye conocimientos. La ingeniería cambia la naturaleza
y transforma el ambiente, se registra este esfuerzo, propiciando interpretaciones
críticas. El ingeniero debe formarse en el desarrollo de habilidades
laborales. Los problemas permiten acercar la realidad que cambian
al ambiente. Se analizaron las respuestas a un cuestionario acerca
de las experiencias de estudiantes de Matemáticas III en una Universidad
venezolana. Se concluyó que la crítica subyace a los criterios de
evaluación. Además se constataron relaciones entre las competencias
de los estudiantes y la realidad cotidiana que se critica. 


\section{UNA APROXIMACIÓN AL ESTUDIO DE LA NOOSFERA: “LA CONSTRUCCIÓN DE LOS
PARALELOGRAMOS EN EL NIVEL PRIMARIO Y SECUNDARIO” }

\begin{datos}

Lidia Ibarra, Blanca Formeliano, Ivone Patagua, Silvia Baspiñeiro,
Florencia Alurralde, Mirta Velásques, Graciela Méndez.

Universidad Nacional de Salta- Facultad de Ciencias Exactas – CIUNSa,

Argentina,

ibarra@unsa.edu.ar; blafor@hotmail.com;

florencialurralde@hotmail.com; ivonepatagua@gmail.com,

\end{datos}

La enseñanza de la geometría ha sido postergada durante años en nuestro
sistema educativo, en este sentido, el grupo de investigación ha realizado
diversos trabajos en esta temática (Construcción de Triángulos con
regla y compás), teniendo como marco teórico la Teoría Antropológico
de lo Didáctico (TAD). En este aspecto, nos hemos propuesto continuar
con el estudio de construcciones de los paralelogramos con regla y
compás. Esta comunicación, tiene como finalidad presentar resultados
del estudio de la noosfera, a partir del análisis de los documentos
curriculares correspondientes y los libros de textos escolares, elementos
propios del Modelo Epistemológico de Referencia (MER).


\section{\uppercase{ La situación didáctica como dispositivo formativo en
la formación inicial del profesorado de educación primaria}}

\begin{datos}

Miguel Angel Quinteros, Mariana Alaniz.

Universidad Nacional de la Patagonia Austral – Unidad Académica de
Rio Turbio,

Argentina,

quinteros594@hotmail.com ; malynaia@yahoo.es 

\end{datos}

El desafío de la Didáctica de la Matemática en la formación del profesorado
consiste en lograr que los estudiantes resignifiquen su formación
inicial al llevar adelante la práctica; propiciar experiencias integrando
conocimientos teóricos y prácticos. En tal sentido, se expone una
experiencia de abordaje áulico, en la que se ponen en juego conocimientos
previos y otros nuevos a través de la resolución de situaciones problemáticas.
Experiencia que servirá como insumo para aprender, desde lo vivencial,
los aspectos sobresalientes de las situaciones didácticas de Brousseau. 


\section{\uppercase{ Fracciones y la relación parte-todo. Una experiencia
didáctica}}

\begin{datos}

Sonia Bibiana Benítez, Lidia María Benítez.

Facultad de Cs. Naturales e I.M.Lillo - Universidad Nacional de Tucumán,

Argentina,

Soniabenitez2001@hotmail.com; lidiabenitez@hotmail.com

\end{datos}

El objetivo de este trabajo es presentar la secuencia didáctica llevada
a cabo en escuelas, en el marco del Plan Matemática Para Todos, con
la intención de generar un efecto multiplicador que abarque a todas
las escuelas públicas y privadas del país. Dicho Plan es a nivel Nacional
de capacitación a docentes de escuelas públicas primarias, que tiende
a desarrollar en los alumnos competencias necesarias para un trabajo
autónomo en el área. Desde el Ministerio de Educación de la Nación
y desde nuestra propia práctica se asumió la responsabilidad de que
la escuela sea el lugar en el que todos aprendan y se garantice el
acceso a la herencia cultural de cada uno de nuestros niños. 

\setcounter{section}{81}


\section{MUSEO DE HISTORIA Y FILOSOFÍA DE LAS MATEMÁTICAS “JUAN FÉLIX MARTÍNEZ”:
SU EVOLUCIÓN Y ALCANCES PARA LA FORMACIÓN DOCENTE EN COSTA RICA}

\begin{datos}

Margot Martínez R., Jesennia Chavarría V., Ma. Elena Gavarrete.

V. U. Nacional, 

Costa Rica,

margomr@gmail.com; jesenniach@gmail.com;

marielgavarrete@gmail.com 

\end{datos}

La presente COMUNICACIÓN BREVE corresponde a una exposición que integra
reflexiones de tipo teórico-filosófico y resultados de trabajo cooperativo-colaborativo
de corte empírico-experimental. Se encuadra en los niveles: superior
y educación continua y está relacionada con las categorías de estudios
socioculturales, etnomatemáticas y formación de profesores. Este documento
expone la evolución del proyecto del Museo de Historia y Filosofía
de las Matemáticas “Juan Félix Martínez” desde la visión de tres de
sus participantes.


\section{UMA INVESTIGAÇÃO SOBRE A (RE) CONSTRUÇÃO DO CONHECIMENTO DE PROFESSORES
PARTICIPANTES DE UM GRUPO QUE ESTUDA O CAMPO CONCEITUAL ADITIVO }

\begin{datos}

Angélica da Fontoura Garcia Silva, Mirtes Pereira de Souza.

Universidade Bandeirante de São Paulo – UNIBAN,

Brasil,

angelicafontoura@gmail.com; mieducacaocife@yahoo.com.br 

\end{datos}

O propósito deste estudo é analisar a (re) construção dos conhecimentos
necessários para o ensino do Campo Conceitual Aditivo de 15 professores
participantes de um grupo de estudos formado em uma escola pública
do estado de São Paulo- Brasil. A pesquisa, de natureza qualitativa
foi desenvolvida no âmbito do Projeto Observatório da Educação, promovido
pela UNIBAN. As análises das informações coletadas mostrou evidências
da (re)construção do conhecimentos dos professores, sobretudo, em
relação a percepção sobre a necessidade de se ofertar aos alunos uma
maior diversidade de situações a fim de possibilitar a ampliação dos
esquemas utilizados. 


\section{APRENDIENDO HISTORIA DE LA MATEMÁTICA A TRAVÉS DE SU PRÁCTICA, EXPERIENCIA
DE AULA.}

\begin{datos}

Yoilyn Rojas Salazar, Cristian Quesada Fernández.

Universidad Estatal a Distancia,

Costa Rica,

yorojas@uned.ac.cr; cquesadaf@uned.ac.cr 

\end{datos}

La primera experiencia del curso Matemática a través de la Historia,
del Programa de Enseñanza de la Matemática de la Universidad Estatal
a Distancia, abordando contenidos desde una perspectiva práctica y
de contextualización de los mismos, mostró que la historia de la matemática
puede ser una gran herramienta tanto para la generación de interés
y motivación, como para la enseñanza misma de contenidos en secundaria.
Enfrenándose a dificultades algorítmicas, simbólicas y de lenguaje,
analizando las dificultades para generar, descubrir y heredar los
conocimientos matemáticos, los estudiantes del curso le dieron un
nuevo sentido y valor a su profesión y formación.


\section{UNA EXPERIENCIA DE TRABAJO DE TÍTULO}

\begin{datos}

María Soledad Montoya González.

Universidad Alberto Hurtado,

Chile,

mmontoya@uahurtado.cl

\end{datos}

Esta ponencia está inserta en la Formación Inicial de profesores y
tiene como objetivo presentar la experiencia de desarrollar Estudios
de Clases articulando dos cursos de la malla curricular de la carrera
Pedagogía en Matemáticas de una universidad chilena. En uno de los
cursos el estudiante debe realizar una investigación vinculada a los
procesos de enseñanza aprendizaje de un contenido matemático y el
otro curso es un seminario de la práctica docente final que realiza
en la escuela. De este modo, se pensó en fortalecer la formación de
profesores como investigadores de su propia práctica.


\section{TUTORIAL DE CÁLCULO DE VARIAS VARIABLES EN WEB}

\begin{datos}

Ma de Lourdes Quezada Batalla, Rubén Darío Santiago Acosta.

Tecnológico de Monterrey - Campus Estado de México,

México,

lquezada@itesm.mx; ruben.dario@itesm.mx

\end{datos}

En este trabajo se presenta un sistema entrenador cuasi-inteligente
de conceptos y algoritmos de Cálculo Multivariable utilizando el WWW.
El sistema se elaboró en el estándar html5, lo que garantiza su visualización
en cualquier tipo de dispositivo inteligente. El sistema consta de
ejemplos resueltos paso a paso, prácticas de ejercicios con sugerencias
y retroalimentación, prácticas de autoevaluación y un sistema de evaluación
automatizada, todo generado aleatoriamente. Para su construcción se
utilizaron los paquetes Mathematica, MatJax y \LaTeX{}. Se utilizó
Google-Drive como base del sistema de evaluación en línea, lo cual
permite que el profesor realice estudios estadísticos sobre los resultados. 


\section{LA RECTA Y LAS ESTRATEGIAS DE SOLUCIÓN NUMÉRICAS.}

\begin{datos}

Rubén Darío Santiago Acosta, Ma. de Lourdes Quezada Batalla.

Tecnológico de Monterrey - Campus Estado de México,

México,

lquezada@itesm.mx; ruben.dario@itesm.mx

\end{datos}

En este trabajo presentaremos la red de problemas que se utiliza en
algunos grupos del curso de Cálculo Diferencial del Tecnológico de
Monterrey, Campus Estado de México, para desarrollar habilidades y
estrategias de solución numéricas en problemas complejos. En el trabajo
se discuten tres problemas relacionados con el movimiento de objetos,
cohetes y planetas. En ellos los estudiantes requieren vincular los
conceptos del cálculo con sus similares aritméticos mediante rectas.
Se presentan además las soluciones típicas formuladas por equipos
de estudiantes. Al final se presentan algunos resultados y conclusiones
sobre lo obtenido en el aula.

\setcounter{section}{88}


\section{ELEMENTOS DE DISEÑO PARA UNA CLASE DE MATEMÁTICAS A TRAVÉS DE LA
MODELACIÓN}

\begin{datos}

Ruth Rodríguez, Samantha Quiroz.

Tecnológico de Monterrey, CIMATE Tecnológico de Monterrey,

México,

ruthrdz@itesm.mx; samanthaq.rivera@gmail.com

\end{datos}

El presente estudio pretende mostrar una propuesta para el diseño
de una clase de Ecuaciones Diferenciales en el contexto de Circuitos
Eléctricos basada en modelación matemática. Se reconoce a la modelación
matemática como el medio para el uso o construcción de modelos que
permitan la resolución de problemas en contextos cotidianos. A través
del uso de un esquema de dicho proceso se describen además de la manera
en que fue diseñada e implementada la clase, los resultados encontrados
así como las dificultades detectadas en la realización de las diversas
actividades propuestas. 


\section{SEMINARIOS VIRTUALES DE CIENCIA Y FORMACIÓN DOCENTE. UNA MIRADA ALTERNATIVA
SOBRE LA CIENCIA Y LA FORMACIÓN DOCENTE}

\begin{datos}

Carlos Grande, Liber Aparisi.

Instituto Nacional de Formación Docente,

Argentina,

carlosgrande@infd.edu.ar; laparisi@infd.edu.ar 

\end{datos}

Presentaremos las experiencias de formación docente en Matemática,
llevadas a cabo durante 2012-2013, en el contexto de los seminarios
virtuales “Ciencia y Formación Docente” desarrollados desde el área
de Políticas Estudiantiles del Instituto Nacional de Formación Docente
(Ministerio de Educación, Argentina). Mostraremos los lineamientos
de los seminarios y en particular abordaremos los trabajos realizados
en el campo de la educación matemática. Informaremos sobre las propuestas
pedagógicas y los resultados obtenidos, utilización de plataforma
para el cursado virtual, herramientas TIC y finalmente categorizaremos
las devoluciones que han realizado los cursantes, todos ellos, futuros
profesores de nivel medio de nuestro país.


\section{HABILIDADES SOCIALES EN ALUMNOS DE MEDICINA VETERINARIA: EL APRENDIZAJE
COOPERATIVO COMO DISPOSITIVO PEDAGÓGICO PARA SU DESARROLLO}

\begin{datos}

Lilian Cadoche; Darío Manzoli, Karina Miño, Mª Candelaria Prendes,
Omar Zoratti.

Universidad Nacional del Litoral - Facultad de Ciencias Veterinarias,

Argentina,

lcadoche@fcv.unl.edu.ar; dmanzoli@fcv.unl.edu.ar;

karina094@hotmail.com

\end{datos}

En un estudio en Argentina, en ingresantes a la Universidad, se encontró
que las mayores dificultades se relacionan con la falta de competencias
sociales. En la Facultad de Veterinaria de la UNL, desde el 2009 se
realizan tareas en las que se desarrollan y ponderan estas competencias,
en propuestas de Aprendizaje cooperativo. Entre las ya realizadas
resaltamos el diseño de material didáctico para el aprendizaje cooperativo
y grillas de valoración de aspectos como comunicación, trabajo en
equipo, confianza. Las experiencias permiten inferir que la cooperación
es eficaz para el logro de mejores adaptaciones y confianza y los
resultados muestran interesantes asociaciones entre rendimiento y
comunicación y otras competencias.


\section{CONCIENCIA METACOGNITIVA EN FUTUROS DOCENTES DE MATEMÁTICAS: EL CASO
DE LA UNIVERSIDAD ANTONIO NARIÑO}

\begin{datos}

Grace Judith Vesga.

Bravo Universidad Antonio Nariño, 

Colombia,

gvesga@uan.edu.co 

\end{datos}

El objetivo de este estudio fue determinar el nivel de conciencia
metacognitiva en futuros docentes de matemáticas. Se utilizó el “Metacognitive
Awareness Inventory”, instrumento creado por Schraw \& Denninson (1994).
El cuestionario tiene 52 ítems distribuidos en dos categorías el conocimiento
de la cognición y la regulación de la cognición. Se aplicó a 27 estudiantes
de la licenciatura en matemáticas de la Universidad Antonio Nariño.
Los resultados evidencian que tienen una baja regulación de la cognición,
especialmente lo que se refiere a la planeación de tiempos, metas
de aprendizaje y recursos; y la evaluación de la efectividad de las
estrategias usadas. 


\section{DESARROLLO DEL PENSAMIENTO LÓGICO GEOMÉTRICO ESPACIAL MEDIANTE LÚDICAS
CON ESTUDIANTES BAJO RIESGO}

\begin{datos}

Juan David Firigua Bejarano, Angélica Janneth Montero Cortes.

Universidad de Cundinamarca,

Colombia ,

Juan.firigua@gmail.com; angelica.montero4@gmail.com 

\end{datos}

El desarrollo del pensamiento matemático es la base de todos los niveles
de educación, por ello, con la “pedagogía innovadora” y a través de
las teorías didácticas se pretende estimular la enseñanza de la geometría.
Por tanto, se llevó a cabo una experiencia con el noveno grado del
Instituto de Promoción Social de la Beneficencia de Cundinamarca en
Fusagasugá, que se buscó poner a prueba el pensamiento lógico y geométrico
espacial para la resolución de problemas en la construcción de modelos
solidos – geométricos en una comunidad estudiantil considerada vulnerable
o bajo riesgo, utilizando el origami. Este proceso usó la creatividad
para la producción divergente y el componente geométrico o matemático
para garantizar la producción convergente.

\setcounter{section}{94}


\section{UNA EXPERIENCIA CON DOCENTES SOBRE EL TEMA FUNCION CUADRÁTICA }

\begin{datos}

Rodríguez María Rosa, Benítez Sonia Bibiana.

Fac. de Cs Económicas y Fac. de Cs Naturales - UNT,

Argentina,

mrrodriguez@face.unt.edu.ar ; soniabenitez2001@hotmail.com

\end{datos}

Una tarea compleja de los docentes del nivel medio es intentar cambios
en los modelos tradicionales de enseñanza. Con este fin se implementó
un Aula-taller dirigido a docentes interesados en nuevas estrategias
de enseñanza. La experiencia se inició con un problema cuyo objetivo
fue construir el concepto de Función Cuadrática. Se utilizaron estrategias
participativas, derivadas de la concepción del trabajo grupal como
una forma de resolver problemas. La participación de los docentes
fue muy buena y demostró gran capacidad para experimentar nuevos tratamientos
al tema considerado. Pensamos que estas experiencias nos permitirán
avanzar para diseñar mejores estrategias de enseñanza. 


\section{MANEJO DE LAS ESTRUCTURAS ADITIVAS Y LAS PRINCIPALES DIFICULTADES
AL TRABAJAR CON ELLAS.}

\begin{datos}

María Camila Tuberquia, Deimer Rafael Aguilar, José de los santos
Gil Lidueña, Carlos Guillermo Hernández.

Universidad de Sucre,

Colombia,

carlos.hernandezcontreras@yahoo.com.co; camilatuberquia@hotmail.com;

 andeimer500@hotmail.com 

\end{datos}

Durante el desarrollo del trabajo de investigación realizado con estudiantes
del grado cuarto de una institución educativa pública, se identificaron
a través de un diagnóstico, los diferentes errores cometidos por los
dicentes al realizar operaciones con estructuras aditivas, originando
una investigación a las posibles alternativas para solucionar dichas
problemática, por ello utilizamos fundamentos teóricos que permitan
reconocer dificultades de diferentes naturaleza, que se generan en
el proceso de aprendizaje matemático, ya que según Duval (2004) “la
actividad matemática requiere que los alumnos pongan en práctica modo
de funcionamiento cognitivos donde apliquen conocimientos para adaptarse
constructivamente a las distintas situaciones matemáticas”.


\section{MOTIVACIÓN Y TECNOLOGÍAS EN EL APRENDIZAJE DE LAS MATEMÁTICAS}

\begin{datos}

Hilda Olivia Albarrán Padilla, Evangelina López Ramírez.

Universidad Autónoma de Baja California,

México,

hilda.albarran@uabc.edu.mx; evangelina.lopezramirez@yahoo.com.mx

\end{datos}

El siguiente trabajo de investigación fue desarrollado en el nivel
básico y se ubica en el primer grado de secundaria, en la asignatura
de Matemáticas I. Las áreas que se consideraron son Geometría básica
y Proporcionalidad. Se realizó con el diseño de secuencias didácticas
que incluían estrategias distintas a las que regularmente se utilizan,
como técnicas de integración para considerar el aspecto afectivo,
y tecnologías para llevar a cabo la enseñanza, considerando los aspectos
afectivos para un mejor aprendizaje de los mismos. Tanto en el aspecto
cognitivo como en el emocional se obtuvieron resultados satisfactorios.


\section{ERRORES Y DIFICULTADES QUE PRESENTAN ALGUNOS ESTUDIANTES DE 8\textdegree{}
AL FACTORIZAR}

\begin{datos}

Arnidis Baltazar, Juan Rivera, Rosa Martínez, Hernando Cárdenas.

Universidad de Sucre,

Colombia,

Arnidis\_b.m@hotmail.com; juanramon48.com@gmail.com;

rosa\_que\_linda\_eres@hotmail.com 

\end{datos}

En Las matemáticas, se han notado dificultades con la factorización;
por esto, se tiene como meta indagar sobre las dificultades que presentan
los discentes. Nuestra metodología busca detectar dificultades en
la aplicación de ésta; y para ello, se aplicó una prueba a 36 estudiantes
de 8\textdegree{}B, donde se encontraron dificultades respecto al
uso de los signos, despeje de ecuaciones y resolución de operaciones
aritméticas; la mayoría presentan falencias no por falta de conocimiento,
sino por interés; esto llevó a concluir que se deben implementar recursos
didácticos y clases teórico-prácticas más dinámicas para proporcionar
al estudiante interés por los contenidos. 


\section{Propuesta Curricular para la Formación Docente Inicial de Profesores
para la Enseñanza Secundaria en Matemática. Provincia de Entre Ríos,
ARGENTINA.}

\begin{datos}

Vertone, Claudia; Gay, Mabel Alicia. Instituto de Formación Docente
“Profesor Rogelio Leites”, La Paz, Entre Ríos. Instituto de Profesorado
Concordia D-54 ; Concordia, Entre Ríos, Argentina. clodine468@hotmail.com;
mabelgay@gmail.com 

\end{datos}

El fracaso en los aprendizajes de los jóvenes y adultos, la rigidización
de las formas de enseñar y la pérdida de sentido para docentes y estudiantes,
nos abren un nuevo camino al repensar la formación inicial de los
profesores y preguntarnos: ¿qué docentes queremos formar y cómo lo
haremos?, ¿cuál sería entonces el proceso curricular que contemple
por un lado la necesidad de trasformaciones en el campo cognitivo,
y que, a la vez, supere las lógicas de adaptación instrumental a un
orden sin buscar alternativas superadoras? Esto sin lugar a dudas
nos interpela en cuanto a la construcción de subjetividades y en la
generación, distribución y apropiación de conocimientos, desde las
propias instituciones formadoras, desde la producción académica.

\setcounter{section}{100}


\section{EL DOCENTE DE MATEMÁTICA Y SU PROCESO DE INSTRUCCIÓN EN LA ERA DIGITAL }

\begin{datos}

Ricardo E. Valles P. 

Universidad Simón Bolívar-Sede del Litoral,

Venezuela,

revalles@usb.ve 

\end{datos}

La educación se encuentra influenciada por el creciente desarrollo
tecnológico; las Universidades se encuentran en un constante equipamiento
técnico con la intensión de mantenerse actualizadas y brindar condiciones
vanguardistas para el desarrollo educativo. El nuevo rol que el docente
de matemática consistirá básicamente en adaptar su praxis educativa
al lenguaje que manejan los estudiantes de hoy, En esta perspectiva,
se considera pertinente plantear dos preguntas centrales para esta
ponencia: ¿Qué papel juega el docente en matemática en la era digital?,
¿Cómo la enseñanza de la matemática debe ser rediseñada en función
de los nativos digitales? 


\section{EL PROYECTO DE TRABAJO: UNA PROPUESTA METODOLÓGICA PARA EL DESARROLLO
DE LA DISCIPLINA DE CÁLCULO DE UNA VARIABLE REAL EN LAS UNIVERSIDADES
DE LA REGIÓN AMAZÓNICA. }

\begin{datos}

Oscar A. González Chong, Miguel Tadayuki Koga, Rogerio dos Reis Gonçalves.

Universidade Estadual de Mato Grosso,

Brasil,

oscarchong@unemat-net.br; miguelkoga@unemat-net.br;

rogerio@unemat-net.br

\end{datos}

Nuestro trabajo propone a través del método de proyecto contribuir
a la formación de valores como: Amor a la naturaleza y cientificidad
ecológica. Los temas de proyectos abordan aristas diferentes de la
actividad humana de la región, como: patrones de la selva amazónica,
construcciones típicas de la región (ocas), control de las reservas
de agua durante la seca, la cacería de animales, control ambiental
y poblacional. Para solucionar el problema principal del proyecto
los estudiantes deben aplicar el cálculo diferencial e integral. Cada
proyecto contiene título, objetivo, información previa, preguntas
a responder y orientaciones metodológicas.


\section{ALTERNATIVA PARA LA ENSEÑANZA DE LOS CONCEPTOS DEL CÁLCULO }

\begin{datos}

Antonio Rey Roque.

Universidad de las Ciencias Informáticas, 

Cuba,

antrey@uci.cu 

\end{datos}

La alternativa propone introducir los conceptos básicos del Cálculo
y desarrollar algunas aplicaciones del cálculo diferencial e integral.
Realizar este proceso mediante la formulación de problemas que se
acerquen a la significación del concepto dentro de la profesión, que
sean en lo posible generalizables, que permitan escalar desde las
funciones de una variable independiente a funciones de varias variables
y a funciones vectoriales. La solución de los problemas enunciados
debe permitir introducir las aplicaciones que tradicionalmente se
estudian en los dos primeros semestres de la disciplina, lo cuales
se formularán mediante la resolución de tareas docentes. 


\section{MODELOS Y TÉCNICAS DE ENSEÑANZA ELABORADAS POR EL DOCENTE EN EL DESARROLLO
CON FRACCIONES ALGEBRAICAS EN ESTUDIANTES DE OCTAVO GRADO}

\begin{datos}

Bravo José Carlos, Fernández Miguel, Moreno Edwin, Ortega Karen, Tous
Wendy.

Universidad de Sucre,

Colombia,

Josecar1718@hotmail.com; migue\_12@hotmail.es;

eddamope04@hotmail.com; kagiorpe125@hotmail.com

\end{datos}

El proceso de enseñanza – aprendizaje tiene como fin formar un estudiante
integral, se debe tener en cuenta el método que el docente utiliza
en el momento de enseñar y lo que éste constituye en los estudiantes
ya sean debilidades o fortalezas en su desarrollo intelectual. Con
base en este proyecto se realizaron diversas observaciones en un plantel
educativo específicamente en el grado octavo con el propósito de analizar
las posibles fallas que existen en la relación docente –alumno y estrategias
– desarrollo cognitivo de los estudiantes. Como conclusión de esta
investigación, se logró identificar la gran importancia de los significados
y definiciones correspondientes al tema en cuestión, así como las
estrategias metodológicas en el desarrollo cognoscitivo de los estudiantes,
formando estructuras constituidas como debilidades o fortalezas, dependiendo
del estilo y modelo de aprendizaje empleados en este proceso.


\section{DIFICULTADES ASOCIADAS AL CONCEPTO CONJUNTO GENERADOR EN NIVEL SUPERIOR}

\begin{datos}

Esteban Mendoza Sandoval, Flor Monserrat Rodríguez Vásquez.

Universidad Autónoma de Guerrero,

México,

emendoza@uagro.mx; flor.rodriguez@uagro.mx 

\end{datos}

Dada la naturaleza abstracta del álgebra lineal, nos interesa proponer
una vía alternativa en la enseñanza del concepto conjunto generador,
acuñando a la teoría APOE como sustento teórico. Por tanto en este
trabajo, como primera parte de una investigación en desarrollo, mostramos
algunas dificultades asociadas a dicho concepto en el nivel medio
superior, pues es fundamental para la propuesta alternativa que reconozcamos
tales dificultades como parte de la descomposición genética que se
debe realizar en la contribución del desarrollo del pensamiento matemático
avanzado. 


\section{A EXPLORAÇÃO DAS PROPRIEDADES DO TRIÂNGULO DE PASCAL NO ENSINO DE
ANÁLISE COMBINATÓRIA COM ATIVIDADES INVESTIGATIVAS}

\begin{datos}

Christiano Otávio de Rezende Sena, João Bosco Laudares.

Pontifícia Universidade Católica de Minas Gerais – PUC Minas,

Brasil,

christianosena@gmail.com; jblaudares@terra.com.br

\end{datos}

Esse artigo apresenta resultados de uma Pesquisa com atividades investigativas
visando o ensino de Análise Combinatória. O propósito é construir,
implementar e analisar uma proposta de ensino de Análise Combinatória
tendo como referência o Triângulo de Pascal, explorando suas propriedades
e padrões, fundamentando na metodologia de Resolução de Problemas,
Investigação Matemática e Análise de Erros. O Triângulo de Pascal
permite muitas explorações enriquecedoras no desenvolvimento da capacidade
do aluno de produzir argumentos que justifiquem um raciocínio combinatório,
bem como a capacidade de generalizar padrões e elaborar argumentações
matemáticas relacionadas a problemas de contagem. 


\section{DISCALCULIA: PERCEPÇÕES DE PROFESSORES DO PRIMEIRO ANO DO ENSINO
FUNDAMENTAL}

\begin{datos}

Isabel Cristina Machado de Lara, Letícia da Silva Pimentel.

Pontifícia Universidade Católica do Rio Grande do sul – PUCRS,

Brasil,

isabel.lara@pucrs.br; leticia.pimentel@pucrs.br

\end{datos}

Este artigo apresenta parte dos resultados de uma pesquisa de Mestrado,
em andamento, sobre discalculia. Objetiva identificar a percepção
que professores do 1o. ano do Ensino Fundamental possuem acerca desse
transtorno e as estratégias utilizadas em classe para desenvolver
as habilidades matemáticas que o indicam. Metodologicamente, realiza
uma Análise Textual Discursiva dos dados levantados por meio de um
questionário aplicado em seis escolas públicas. O referencial está
fundamentado em estudos voltados ao transtorno da discalculia e evidencia
que a maioria dos professores não recebem subsídios teóricos que possibilitem
ao menos a suspeita de indícios dos estudantes com tais dificuldades.


\section{RAZONAMIENTO LÓGICO-MATEMÁTICO EN SECUNDARIA }

\begin{datos}

Islande Cristina Delgado Monge, Yoilyn Rojas Salazar.

Universidad Estatal a Distancia,

Costa Rica ,

isdelg13@hotmail.com; yoi2901@hotmail.com

\end{datos}

La experiencia aquí presentada surge de la necesidad existente en
secundaria de poner en práctica actividades que potencien el uso del
razonamiento lógico-matemático en la resolución de problemas académicos
y en lo cotidiano. Se lleva a cabo mediante la coordinación entre
instancias de la universidad y las instituciones donde se aplicó.
Mediante el ejercicio dinámico entre el estudiante y el docente de
resolver problemas, analizar diferentes procesos de solución, así
como la introducción de contenidos que no se abarcan en secundaria,
se logra potenciar en el estudiante, la capacidad de análisis y relación
de conceptos propiciando una mejora en los resultados académicos de
los estudiantes.

\setcounter{section}{109}


\section{ANÁLISIS DE LA VARIACIÓN EN LOS LIBROS DE TEXTO: EL CASO DE LAS RELACIONES
TRIGONOMÉTRICAS }

\begin{datos}

Ferney Tavera Acevedo, Jhony Alexander Villa Ochoa.

Universidad de Medellín , Universidad de Antioquia,

Colombia ftavera827@yahoo.es; javo@une.net.co

\end{datos}

En esta comunicación presentamos los resultados de un estudio cualitativo
realizado a algunos libros de texto de matemáticas. En particular,
nos centramos en las diferentes tareas que se proponen para abordar
la temática de las relaciones trigonométricas. Los resultados muestran
que este tipo de recurso exhibe mayor cantidad de situaciones, que
denominamos “estáticas”, en relación con aquellas situaciones en las
cuales la noción de variable y de variación tenga presencia, por lo
tanto, consideramos que estos libros “descuidan” algunos aspectos
variacionales propios del contexto en el que se plantean dichas tareas.
De este análisis surgió la necesidad de proporcionar experiencias
para que los estudiantes observen fenómenos de cambio y de variación,
valorando el papel mediador que hace la tecnología en el diseño de
tales situaciones.


\section{PROCESO DE GENERALIZACION: UNA MIRADA DE ESTUDIANTES DE BÁSICA PRIMARIA}

\begin{datos}

Ányela Xiomara Corredor Santos, Mónica Adriana Pineda Ballesteros,
Solange Roa Fuentes.

Universidad Industrial de Santander, EDUMAT, UIS.,

Colombia,

Xiomy\_1121@hotmail.com; monypin23@gmail.com;

roafuentes@gmail.com

\end{datos}

Con esta investigación pretendemos analizar cómo estudiantes entre
9 y 12 años abordan el proceso de generalización a partir del estudio
de situaciones sobre patrones en diferentes representaciones. Para
este trabajo hemos implementado las fases del proceso de generalización
propuestas por el grupo Azarquiel (1993): Ver, Describir y Escribir.
Los resultados muestran que después de un proceso de instrucción,
los estudiantes identifican algunos patrones en secuencias numéricas
y geométricas, ayudándose de estrategias que les permiten identificar
el patrón; sin embargo, tienen grandes dificultades para pasar de
una fase a otra; les resulta complejo llegar a una generalización
ya sea de manera verbal o mediante una expresión general.


\section{DESARROLLANDO RAZONAMIENTO PROPORCIONAL EN ALUMNOS DEL NIVEL MEDIO
BÁSICO (13-15 AÑOS) }

\begin{datos}

Yenhy Lizeth Silva Aguilar,

Estudiante de la licenciatura en educación secundaria con especialidad
en matemáticas en la Escuela Normal Superior Pública del estado de
Hidalgo,

México,

yenhysilva@gmail.com

\end{datos}

Hablar de razonamiento proporcional en nivel medio básico implica
reconocer un gran reto en el campo de labor para los docentes, por
ello este análisis da cuenta de algunas de las dificultades que enfrenté
al abordar contenidos relacionados a las estructuras multiplicativas,
aquí se analizan las situaciones didácticas, las respuestas de los
alumnos y principalmente se hace un contraste con el marco teórico
haciendo un intento por validar éste con el propósito de tomar conciencia
de las actividades que se proponen en las aulas y los procesos cognitivos
que favorecen.


\section{O USO DE SEQUÊNCIAS DIDÁTICAS UTILIZANDO O SOFTWARE GEOGEBRA PARA
O ENSINO DE GEOMETRIA PLANA: PROPRIEDADES DO TRAPÉZIO.}

\begin{datos}

Alexandre Botelho Brito , Kewla Dias Pires Brito , Lizandra Almeida
Araujo, Weslley Florentino de Oliveira.

Instituto Federal de Educação, Ciência e Tecnologia do Norte de Minas
Gerais (IFNMG),

Brasil, 

alexandre.brito@ifnmg.edu.br; kewla.brito@ifnmg.edu.br; 

lizandra-almeida@hotmail.com;

\end{datos}

Educadores Matemáticos vêm buscando metodologias de ensino diversificadas
para o ensino de um conteúdo abjurado pelos docentes, a Geometria.
Diante da sua importância e das grandes dificuldades enfrentadas no
seu ensino-aprendizado, propôs-se a elaboração de Sequências Didáticas
(SD) que recorrem ao software GeoGebra no ensino de algumas propriedades
do trapézio, fundamentadas nos pressupostos da Engenharia Didática.
Constatou-se, a partir da presente pesquisa, que estas Sequências
Didáticas serão melhor exploradas se aplicadas na modalidade presencial.
Observou-se também, que a combinação de SD e TDIC auxilia no processo
de ensino-aprendizagem, pois norteia o aluno frente ás diversas possibilidades
que o GeoGebra oferece.


\section{EL PAPEL DE LA ETNOMATEMATICA EN EL MEJORAMIENTO DEL APRENDIZAJE
ESCOLARIZADO.}

\begin{datos}

Fortunato Morales Pastelín.

Universidad Pedagogica Nacional,

México,

moralespast@hotmail.com; fortunato.morales.pastelin@gmail.com

\end{datos}

El Sistema Educativo Nacional en México ha implantado con los ajustes
necesarios modelos educativos homogéneos, contradiciendo ésta política
operativa con preceptos legales como la Constitución Política Nacional,
la Ley Estatal de Educación y otros documentos, en dónde se resalta
el carácter pluricultural y multilingüe, característica de la sociedad
mexicana. Es determinantemente necesario incluir dentro de los Planes
de estudio de educación básica, una serie de contenidos matemáticos
que se adquieren en la socialización familiar y comunitaria de los
individuos de los diversos grupos étnicos, de tal manera que su perfil
de egreso muestre conocimientos de las dos cosmovisiones.


\section{Grafeq e Matemática no Ensino Médio: algumas propostas interdisciplinares}

\begin{datos}

Isabel Cristina Machado de Lara, Cláudia Helena Fettermann Batistela.

Pontifícia Universidade Católica do Rio Grande do Sul – PUCRS,

Brasil ,

isabel.lara@pucrs.br; claudiah@pucrs.br 

\end{datos}

Este estudo objetiva verificar como o software GrafEQ contribui para
o ensino de Matemática, no Ensino Médio, numa perspectiva interdisciplinar.
Participam como colaboradores quatro professores de escolas públicas
e 20 licenciandos em Matemática, que compõem um programa de iniciação
à docência. Metodologicamente, percorre quatro etapas: intervenções
pedagógicas; elaboração das propostas; aplicação das propostas; validação.
Após as intervenções, realizadas pelas autoras, os participantes elaboraram
propostas a serem aplicadas nas escolas envolvidas. Ao validar as
propostas evidenciam-se possibilidades do tratamento de temas diversos
que envolvem professores de diferentes áreas contribuindo para melhorar
o desempenho em Matemática e aumentar o interesse do estudante.


\section{EXPERIENCIAS DERIVADAS DE INNOVACIONES EN LA PRÁCTICA DE LA EVALUACIÓN
DEL APRENDIZAJE: EL ROL ACTIVO DEL ALUMNO EN LA EVALUACIÓN}

\begin{datos}

Alejandro Martínez Castellini.

Universidad de las Ciencias Informáticas (UCI),

Cuba,

alexmc@uci.cu

\end{datos}

Con este trabajo el autor se propone comunicar sus experiencias derivadas
de innovaciones en la práctica de la evaluación teniendo el alumno
un papel activo. Se innovó en el uso de técnicas alternativas de evaluación,
los exámenes, las funciones de la evaluación, así como los respectivos
roles del profesor y el alumno. La observación de la actitud de los
alumnos durante las evaluaciones, sus respectivos criterios y las
observaciones de clases permiten arribar a la conclusión que la insuficiente
implicación activa del alumno en la evaluación es un aspecto negativamente
influyente en los resultados de aprendizaje.


\section{CONTRIBUCIÓN DESDE EL ESTUDIO DEL TEMA ESPACIOS VECTORIALES DEL ÁLGEBRA
LINEAL AL DESARROLLO DEL PENSAMIENTO ALGORÍTMICO DE LOS ESTUDIANTES
DE PRIMER AÑO DE LA CARRERA INGENIERÍA EN CIENCIAS INFORMÁTICAS. }

\begin{datos}

Yisel Reyes Cardoso, Annia García Ruiz.

Universidad de las Ciencias Informáticas (UCI),

Cuba,

ycardoso@uci.cu; agarciar@uci.cu 

\end{datos}

Durante el desarrollo del trabajo se fundamenta cómo desde el tema
Espacios Vectoriales de la asignatura Álgebra Lineal, impartida en
el primer año de la carrera de Ingeniería en Ciencias Informáticas,
se contribuye al desarrollo del pensamiento algorítmico de los estudiantes.
Fue escogido el tema de Espacios Vectoriales por la diversidad de
procedimientos que se desarrollan. De los distintos mecanismos para
representar algoritmos, fue adoptado el pseudocódigo, un lenguaje
algorítmico menos rígido que un lenguaje de programación, considerándose
valiosa su contribución para otras asignaturas que se imparten en
la carrera como Programación. 


\section{ALGUNOS FACTORES QUE PUEDEN AFECTAR AL P-VALOR }

\begin{datos}

Irma Nancy Larios Rodríguez, Gudelia Figueroa Preciado, Maria Elena
Parra Ramos. 

Universidad de Sonora,

México,

nancy@gauss.mat.uson.mx;

gfiguero@gauss.mat.uson.mx;

meparra@gauss.mat.uson.mx

\end{datos}

El análisis estadístico es una herramienta fundamental de la mayoría
de los investigadores en las diferentes disciplinas científicas. Cuando
se realizan pruebas estadísticas con el uso de software estadístico,
el resumen de resultados obtenido, generalmente utiliza el p-valor.
Desafortunadamente, algunos usuarios analizan de forma mecánica los
resultados obtenidos, al guiarse solamente por el resultado del p-valor
obtenido sin considerar que tal valor puede ser influido por diversos
factores entre los que se encuentran los cambios en la hipótesis nula
y tamaños de muestra, lo que puede llevar a conclusiones equivocadas
y a la toma de decisiones no adecuada. 


\section{ANALISIS ESTADISTICO DE LOS DATOS DE ADMISION DEL INSTITUTO PEDAGOGICO
DE CARACAS PARA EL AÑO 2011}

\begin{datos}

Mayra Alejandrina Freites Villegas .

Instituto Pedagógico de Caracas,

Venezuela,

mayrafreites@gmail.com

\end{datos}

En el presente trabajo se plantea como predecir el rendimiento académico
de los estudiantes del Instituto Pedagógico de Caracas usando como
técnica estadística la regresión multinivel a partir de datos socioeconómicos.
Para estudiar el rendimiento estudiantil se realizo un análisis descriptivo
de los datos para determinar sus medidas de centralización, de variabilidad,
además identificar la normalidad o no de dichos datos y así justificar
el uso de las técnicas de regresión y modelos multiniveles. Finalmente
se desea establecer asociaciones entre las variables consideradas.


\section{COMPETENCIAS PARA EL ANÁLISIS COGNITIVO DE TAREAS SOBRE RAZONAMIENTOS
ALGEBRAICO ELEMENTAL POR PROFESORES EN FORMACION. }

\begin{datos}

Walter F. Castro, Juan D. Godino, Patricia M. Konic, Mauro Rivas. 

Universidad de Antioquia, Universidad de Granada, Universidad Nacional
de Río Cuarto,

Universidad de los Andes, Colombia, España, Brasil, Venezuela,

wfcastro82@gmail.com; jgodino@ugr.es;

pkonic@gmail.com; rmauro@ula.ve

\end{datos}

En este documento se informa sobre las competencias exhibidas por
profesores en formación para el análisis cognitivo de tareas sobre
razonamiento algebraico elemental (RAE). La inclusión del álgebra
en la escuela primaria requiere de profesores preparados para reconocer
el álgebra y promoverla. La metodología adoptada fue cualitativa,
y estudio de caso. Las conclusiones se agrupan en dos: a) Conclusiones
sobre los Análisis realizados por los maestros; y b) Conclusiones
sobre las creencias. Los hallazgos coinciden con los resultados de
Stump y Bishop (2002), quienes reportan sobre las concepciones en
términos de solución de ecuaciones, hallazgo de valores desconocidos.


\section{EMPLEO DE ASISTENTES MATEMÁTICOS EN EL PROCESO ENSEÑANZA APRENDIZAJE}

\begin{datos}

Iván J Valido González, Sandy Díaz Ramos.

Universidad de las Ciencias Informáticas, 

Cuba,

sdiaz@uci.cu; ivalido@uci.cu 

\end{datos}

En el presente artículo se exponen algunas experiencias sobre el empleo
de asistentes matemáticos (AM) en la resolución de problemas, establecimiento
de conjeturas y visualización durante el proceso de enseñanza y aprendizaje.
Tanto en la sala de clases como en los laboratorios se utilizan diversos
asistentes atendiendo a las características del tema a estudiar; entre
ellos: Derive 6.1, Matlab, Mupad, Geometra y Octave. 


\section{¿OLIMPIADA DE MATEMÁTICA ASISTIDA POR COMPUTADORA? }

\begin{datos}

Sandy Díaz Ramos, Iván J Valido González.

Universidad de las Ciencias Informáticas - La Habana, 

Cuba,

sdiaz@uci.cu ; ivalido@uci.cu 

\end{datos}

En este artículo se describe la experiencia de los autores en la organización
de la Olimpiada de Matemática con el empleo de la Informática (OMEI)
y la preparación de los estudiantes para la misma. Además se presenta
una caracterización de este evento.

\setcounter{section}{123}


\section{ALCANCES Y BENEFICIOS DE CONOCER EL ESTILO DE PENSAMIENTO MATEMÁTICO
DE JOVENES CHILENOS DE 15 AÑOS}

\begin{datos}

Pamela Reyes-Santander, Rita Borromeo-Ferri.

Universidad de Valparaíso, Universidad de Kassel,

Chile, Alemania,

pamela.reyes@uv.cl; borromeo@mathematik.uni-kassel.de 

\end{datos}

En este trabajo se muestran alcances y beneficios que tiene el conocer
el estilo de pensamiento matemático (EPM) de jóvenes chilenos. Beneficios
en el trazado de líneas de acción para la enseñanza de la matemática,
basadas en el conocimiento de los EPM. Conocer los EPM, repercutirá
en la forma de enseñar y como consecuencia en resultados de pruebas
internacionales. También, al mostrar que el EPM del profesor influencia
el EPM del estudiante, se tiene que flexibilizar la formación del
profesor, para que incluya en sus aulas diferentes EPM, que reconozca
el suyo propio y que fortalezca el de sus estudiantes. 


\section{INGRESO A UNA NUEVA ETAPA ESCOLAR (EDUCACIÓN MEDIA GENERAL) CON CONOCIMIENTOS
MATEMÁTICOS }

\begin{datos}

Ramírez Yuraima, Irving Valencia.

Universidad Pedagógica Experimental Libertador,

Venezuela yura2572@hotmail.com; irving.valencia@gmail.com

\end{datos}

El propósito de este estudio fue diagnosticar los conocimientos previos
de matemática que tenían los estudiantes al ingresar en la Educación
Media General. Para esta evaluación se aplicaron pruebas con ejercicios
y problemas que se usan en el Sistema Nacional de Evaluación y Aprendizaje
(SINEA), con la finalidad de que los docentes que trabajaron con estos
grupos pudieran reforzar y así realizar la planificación de los contenidos
que se implementarían en cada lapso. El estudio se enmarcó en una
investigación cuantitativa. Los resultados obtenidos evidenciaron
que la mayoría de los estudiantes presentaban mayor dificultad en
geometría y resolución de problemas.

\setcounter{section}{127}


\section{VISUALIZACIÓN DE LAS FUNCIONES RACIONALES}

\begin{datos}

Juan Alfonso Oaxaca Luna, María del Carmen Valderrama Bravo, José
Juan Contreras Espinosa, Carlos Oropeza Legorreta, José Isaac Sánchez
Guerra.

Facultad de Estudios Superiores Cuautitlán UNAM,

México,

joxaca@unam.mx; carmenvalde@yahoo.com.mx;

jjuancon2000@yahoo.com.mx; coropeza96@hotmail.com,

\end{datos}

La investigación muestra situaciones de enseñanza y aprendizaje de
la función racional, en la que los informantes pertenecen a la Facultad
de Estudios Superiores Cuautitlán UNAM. El objetivo fue diseñar y
aplicar estrategias didácticas empleando la visualización. Es de tipo
cualitativo aplicando la Investigación Acción Participante, a situaciones
desarrolladas dentro del salón de clases. La visualización fue aplicada
a la expresión algebraica y gráfica de la función. Los resultados
mostraron que el estudiante puede dibujar la gráfica cuando visualiza
la expresión algebraica, sin embargo es difícil, cuando el estudiante
primero visualiza la gráfica para obtener la expresión algebraica. 


\section{CB129 MODELO DIDÁCTICO PARA EL DESARROLLO DE LA HABILIDAD ALGORITMIZAR
A TRAVÉS DEL ÁLGEBRA LINEAL}

\begin{datos}

Anelys Vargas Ricardo, Olga Lidia Pérez González, Ángela Martín, Ramón
Blanco Sánchez.

Universidad de las Ciencias Informáticas, Universidad de Camagüey,
Universidad Autónoma de Santo Domingo, Universidad de Camagüey, 

Cuba, República Dominicana, 

anelys@uci.cu, olga.perez@reduc.edu.cu; m.angela24@gmail.com;

ramon.blanco@reduc.edu.cu

\end{datos}

Resolución de problemas; Nivel Superior; Teórico/filosófico

El presente trabajo tiene como objetivo poner a consideración, de
la comunidad de Matemática Educativa de la región, un modelo didáctico
para el desarrollo de la habilidad Algoritmizar desde el Álgebra Lineal
con la particularidad de la importancia que tiene la misma para los
graduados de las titulaciones de Informática y Ciencias de la Computación.
Este trabajo forma parte de la segunda etapa de un proyecto de investigación
que tendrá como resultado la defensa de una tesis doctoral y en él
se abordan los aspectos teóricos tomados en cuenta en el modelo.


\section{RELAÇÕES PESSOAIS DE ESTUDANTES DE SÃO PAULO DOS ENSINOS FUNDAMENTAL,
MÉDIO E SUPERIOR SOBRE AS REPRESENTAÇÕES DOS NÚMEROS RACIONAIS }

\begin{datos}

José Valério Gomes da Silva, Marlene Alves Dias UNIBAN,

Brasil,

valerio.gomes@yahoo.com.br; alvesdias@ig.com.br 

\end{datos}

Apresentamos nesse trabalho parte de nossa pesquisa sobre “A evolução
das relações pessoais de estudantes da educação básica e do ensino
superior sobre a noção de números racionais”. Nosso objetivo é compreender
as marcas das relações institucionais sobre as relações pessoais dos
estudantes. Para tal utilizamos a TAD, a noção de quadro e mudança
de quadros, de níveis de conhecimento esperados dos estudantes e os
diferentes significados identificados por Cavalcanti e Guimarães.
As análises tendem a mostrar que é reduzido o número de estudantes
que evoluem e são capazes de aplicar as técnicas associadas à representação
dos números racionais.

\setcounter{section}{131}


\section{CB132 LA INCIDENCIA DE UN SOFTWARE EN LA RESOLUCIÓN DE PROBLEMAS
DE CÁLCULO MULTIVARIABLE. UN ESTUDIO DE CASO}

\begin{datos}

Giovanni Ruiz Faúndez, Andrea Silvina Seoane, Liliana Milevicich,
Alejandro Lois.

Universidad Tecnológica Nacional - Facultad Regional General Pacheco,

Argentina,

gruizfaundez@gmail.com; seoane\_andrea@yahoo.com.ar;

liliana\_milevicich@yahoo.com.ar 

\end{datos}

En el marco de los trabajos de nuestro equipo de investigación vinculados
a la incorporación de nuevas tecnologías en la enseñanza de la Matemática
en carreras de Ingeniería, abordamos un estudio de caso longitudinal
con enfoque mixto, sobre la incidencia del software en las producciones
de los alumnos. La toma de datos en dos etapas, nos ha permitido analizar
la evolución de dos grupos y observar que la decisión sobre la utilización
o no-utilización del software estuvo vinculada a la necesidad del
mismo como herramienta, por lo que no se observa una dependencia ni
automatización en su uso.


\section{INVESTIGACIÓN COLABORATIVA COMO ESTRATEGIA DE RESOLUCIÓN DE PROBLEMÁTICAS
PEDAGOGICAS: EL CASO DEL MUNICIPIO DE GUACAMAYAS }

\begin{datos}

Camilo Fuentes Leal, Albeiro Tarazona Mojica, Yarleny Jiménez Niño,
Ovidio Archila, Luis Alfredo Niño.

Universidad Distrital Francisco José de Caldas, Institución Educativa
Técnica San Diego De Alcalá,

Colombia,

cristianfuentes558@hotmail.com; albetamo\_07@hotmail.com;

yarleny\_81@hotmail.com 

\end{datos}

Se presentará una experiencia de investigación colaborativa de un
grupo de docentes del municipio de Guacamayas como proceso de reflexión
sobre las problemáticas y las necesidades presentes en sus prácticas
pedagógicas, fruto esto se encontró el poco interés y gusto de las
matemáticas por parte de los estudiantes, pues ellos consideraban
que éstas son un producto ajeno a su contexto social, como respuesta
a esta situación se propuso el diseño de actividades para la enseñanza
de las matemáticas teniendo en cuenta el contexto cultural y las prácticas
sociales del municipio en especial la elaboración de artesanías en
fique y paja.


\section{FRAÇÃO: SITUAÇÕES PARTE-TODO E QUOCIENTE NAS QUESTÕES DE NOMEAR E
RACIOCÍNIO. }

\begin{datos}

Raquel Factori Canova, Tânia Maria Mendonça Campos, Angélica Fontoura
Garcia Silva.

UNIBAN,

Brasil ,

fraquelc@yahoo.com.br; taniammcampos@hotmail.com;

angelicafontoura@gmail.com

\end{datos}

Nesse trabalho o objetivo é investigar o desempenho dos alunos em
problemas de fração nas situações parte-todo e quociente no que se
refere à questão de nomear e questão de raciocínio (equivalência e
ordem). Foi aplicado um teste com oito problemas a 378 alunos do Ensino
Fundamental de escolas públicas do estado de São Paulo. O referencial
teórico foi a Teoria dos Campos Conceituais. Os resultados apontam
que o desempenho dos alunos é melhor nas questões de nomear frações
na situação parte-todo enquanto que nas questões de raciocínio o melhor
desempenho é na situação quociente.


\section{LA ADICIÓN EN LOS TEXTOS DEL PROGRAMA TODOS A APRENDER-MEN}

\begin{datos}

Juan Barboza Rodríguez, Enyel Arias Mercado, José Garrido Peralta.

Universidad de Sucre - grupo de investigación Proyecto Pedagógico,

Colombia,

juan.barboza@unisucre.edu.co; engels20@hotmail.es;

jo.cam12@hotmail.com

\end{datos}

El propósito de este estudio es analizar la visión que los textos
de la serie proyecto Sé del MEN tienen y promueven sobre la operación
adición, en relación con los referentes teóricos y pedagógicos que
la comunidad académica y el MEN asumen como orientadores para el desarrollo
de la calidad educativa. Se aplica la técnica de análisis de texto,
desde el enfoque mixto de investigación. Las categorías de análisis
son: contexto del enunciado, palabras involucradas, estructura semántica,
estructura sintáctica y estructura del texto. Los resultados señalan
que los problemas con mayor presencia son los de tipo numérico y de
combinación. 


\section{ETNOCIÊNCIAS E ETNOMATEMÁTICA: NARRATIVAS DE UM COLONO ALEMÃO DE
SANTA MARIA DO HERVAL}

\begin{datos}

Isabel Cristina Machado de Lara, Ketlin Kroetz.

Pontifícia Universidade Católica do Rio Grande do Sul, 

Brasil,

isabel.lara@pucrs.br ; ketlin.kroetz@acad.pucrs.br

\end{datos}

Esse artigo apresenta resultados de uma pesquisa de Mestrado em Educação
em Ciências e Matemática da Pontifícia Universidade Católica do Rio
Grande do Sul desenvolvida na cidade de Santa Maria do Herval. Objetiva
analisar a presença da Etnomatemática e Etnociência nas narrativas
de um colono alemão de baixa escolarização. Expõe a história da imigração
alemã na cidade para situar seu contexto, utilizando a Etnomatemática
e Etnociência para o estudo dos saberes e práticas produzidas pelo
colono. Conclui a presença de práticas matemáticas e diversos saberes
em seu cotidiano, reformuladas em sua cultura continuadamente e passadas
de geração em geração.


\section{LABORATORIO DE MATEMÁTICA: UNA PROPUESTA PARA EL FORTALECIMIENTO
DE LA ENSEÑANZA Y APRENDIZAJE}

\begin{datos}

Edilmo Carvajal, Thais Arreaza, Mayra Freites, José Fajardo, Yuraima
Ramírez Irving,

Valencia Universidad Pedagógica Experimental Libertador,

Venezuela,

qedcarvajal@gmail.com; mayrafreites@gmail.com;

irving.valencia@gmail.com; tarreaza@gmail.com; 

jgfr70@gmail.com, yura2572@gmail.com.

\end{datos}

El laboratorio de matemática más que un espacio donde se lleva a cabo
el proceso de enseñanza y aprendizaje, es un conglomerado de estrategias
pedagógicas, de materiales concretos y el uso de las TIC. El objetivo
principal de esta propuesta es la creación de un laboratorio de matemática.
Se utilizará un diseño metodológico enmarcado en el paradigma cualitativo,
con un diseño de Investigación-Acción Participativa. Esta metodología,
permitirá que los investigadores observen los procesos de aprendizaje
de los participantes, corrijan las fallas detectadas durante la aplicación
de las actividades y si es necesario reorganizarlas hasta que el estudiante
logre la competencia.


\section{EL ROL DE LA VARIABILIDAD ESTADÍSTICA EN LA FORMACIÓN INICIAL DE
PROFESIONISTAS EN CIENCIAS SOCIALES }

\begin{datos}

Enrique Hugues G., Gerardo Gutiérrez F., I. Nancy Larios R.

Universidad de Sonora,

México,

ehugues@gauss.mat.uson.mx; gerardo@gauss.mat.uson.mx;

nancy@ gauss.mat.uson.mx

\end{datos}

Una reforma once años atrás trajo nuevos planes y programas de estudio
para las diferentes carreras en nuestra institución. Tiempo de práctica
suficiente para una evaluación de sus alcances y de las acciones complementarias
que le han acompañado. En el caso particular de la educación estadística
en carreras de ciencias sociales, como avances de un proyecto, se
describen directrices del mismo así como los primeros resultados de
valorar su estado actual y de la revisión de alternativas para el
desarrollo del curso de estadística descriptiva, especialmente las
formas de considerar a la variabilidad estadística en este curso básico
pero fundamental. 


\section{AJEDREZ EN LA ENSEÑANZA DE LAS MATEMATICAS A ESTUDIANTES DE ORIGEN
MAPUCHE}

\begin{datos}

Iván Collins Silva.

Universidad Mayor, Temuco, 

Chile,

collins.iv@gmail.com 

\end{datos}

La presente experiencia pedagógica, se enmarca en contexto de escuela
rural de la cuidad de Temuco, donde el 98\% de sus estudiantes provienen
o viven en comunidades mapuches. Dicha experiencia busca dar a conocer
del cómo, en un intento desesperado por enseñar los contenidos de
la asignatura de matemática, con el único fin de obtener resultados
de aprendizaje positivos, nace la idea de enseñar los movimientos
básico de las piezas de ajedrez, transformándose en una alternativa
se superación y desarrollo cognitivos para dichos estudiantes, contagiando
a la escuela y al sector rural, al que pertenecían.


\section{MATEMATCH: UNA EXPERIENCIA LÚDICA FUERA DEL AULA}

\begin{datos}

Adrián Alberto Cahuana Garboza, Felipe Asmad Falcon, Elizabeth Caycho
Ñuflo, Melissa Denisse Castillo Medrano, Victor Fernando Garro Moreno,
Myrian Luz Ricaldi Echevarria, Manuel Villarán Paredes.

Colegio de la Inmaculada - Jesuitas,

Lima - Perú,

acahuanag@ci.edu.pe; fasmadf@ci.edu.pe; 

ecaychoñ@ci.edu.pe; mcastillom@ci.edu.pe

\end{datos}

La experiencia pretende trabajar la matemática a través del desarrollo
de actividades lúdicas colaborativas en espacios abiertos. Dicha experiencia
se lleva a cabo en el contexto de la semana estudiantil de una institución
educativa privada de la ciudad de Lima y está dirigida a alumnos desde
6to de primaria hasta 5to de secundaria. Algunos de los juegos matemáticos
gigantes son: el ajedrez, el ludo, el juego de la oca, el cranium
y el tangram. Esta actividad es valorada positivamente por los alumnos
y permite percibir la matemática no solo en escenarios académicos,
sino también como una actividad humana y divertida.


\section{MODELACIÓN DE FENÓMENOS MEDIANTE FUNCIONES LINEALES}

\begin{datos}

Saúl Ernesto Cosmes Aragón, Silvia Elena Ibarra Olmos.

Instituto Tecnológico de Sonora, Universidad de Sonora,

México,

saul.cosmes@itson.edu.mx; sibarra@gauss.mat.uson.mx 

\end{datos}

El trabajo que se presenta consiste en el diseño de una secuencia
didáctica para abordar el estudio de la modelación de fenómenos de
ingeniería mediante funciones lineales. Dicha propuesta tiene como
base fundamental la consideración de que el abordaje de los contenidos
matemáticos debe ser funcional, es decir, se debe promocionar una
enseñanza que produzca un conocimiento útil y duradero, para que cada
vez que este conocimiento se necesite como producto de las prácticas
de un individuo en sociedad, tal conocimiento emerja. El diseño de
las actividades fue realizado con la metodología ACODESA.


\section{UM ESTUDO SOBRE O PROCESSO DE INTEGRAÇÃO DAS TECNOLOGIAS AO CONTEÚDO
DA MATEMÁTICA NA FORMAÇÃO DOCENTE}

\begin{datos}

Ana Karina de Oliveira Rocha, Maria Elisabette Brisola Brito Prado.

Universidade Anhanguera de São Paulo,

Brasil,

anakarina@ufs.br; bette.prado@gmail.com

\end{datos}

Este artigo tem como objetivo descrever a analisar as experiências
realizadas em um curso de formação continuada de professores visando
favorecer a construção do conhecimento pedagógico e tecnológico de
conteúdos matemáticos. Essa pesquisa foi desenvolvida no âmbito do
Projeto Observatório da Educação com professores que atuam com alunos
do 6º ao 9º ano de escolas públicas e utilizou a linguagem de programação
Scratch nas atividades pedagógicas. O resultado dessa experiência
mostrou que os professores reconheceram as potencialidades da programação
para propiciar aos alunos a construção do conhecimento matemático
a partir da criação de algoritmos e de softwares educativos.


\section{LA MODELACIÓN MATEMÁTICA EN LA FORMACIÓN INICIAL DE INGENIEROS}

\begin{datos}

Paula Andrea Rendón Mesa, Jhony Alexander Villa Ochoa, Pedro Vicente
Esteban Duarte.

Universidad de Antioquia, Universidad EAFIT,

Colombia,

rendonmesa@gmail.com; javo@une.net.co;

pesteban@eafit.edu.co

\end{datos}

La literatura internacional resalta a la modelación matemática como
un medio para que los estudiantes comprendan, los conceptos matemáticos
aplicados a su área de conocimiento y, la matematización de diferentes
procesos que desarrollan en su formación. Se analizan diferentes episodios
provenientes de una investigación cualitativa, que indaga por aspectos
que en un ambiente de modelación se articulan con el conocimiento
matemático y con las necesidades de formación del ingeniero. Los resultados
preliminares sugieren que conocimientos relativos a la ingeniería,
otras ciencias y la matemática interactúen y se complementen en un
ambiente de modelación matemática.


\section{¿POR QUÉ ALGUNOS ESTUDIANTES PRESENTAN DIFICULTADES AL RESOLVER SITUACIONES
PROBLEMAS QUE INVOLUCRAN FRACCIONES?}

\begin{datos}

Dayana Paola Escobar Álvarez, Lilian Carina Fuentes Monterroza, Moisés
David Arcia Benavides.

Universidad de Sucre,

Colombia,

dapaesal93@hotmail.com; lilianfuentes1995@hotmail.com;

moisesarcia9a@hotmail.com

\end{datos}

Con este trabajo se quiere compartir los inicios de un proyecto de
investigación en el marco del programa Licenciatura en Matemáticas
de la Universidad de Sucre, buscando recibir de los asistentes, significativos
aportes para mejorarlo. La investigación en educación matemática se
ha convertido en una de las áreas más activas de los estudios de educación.
Algunos autores han dado a conocer algunos modelos que permiten entender
el significado y lo que implica aprender matemáticas y también pautas
que se deben seguir en las actividades para lograr en los estudiantes
un aprendizaje significativo.


\section{CÁLCULO INTEGRAL ATRAVÉS DE RESOLUCIÓN DE PROBLEMAS EN INGENIERIA }

\begin{datos}

Gloria Angélica Flores Rivera, Martha Chairez Jiménez.

Universidad Autónoma de Baja California,

México,

angelica.flores@uabc.edu.mx; mchairez@ uabc.edu.mx

\end{datos}

El objetivo de esta ponencia es presentar los resultados de un estudio
de casos realizado con estudiantes de segundo semestre de ingeniería
en una universidad pública En el estudio se analizaron, interpretaron
y valoraron los resultados obtenidos en el proceso de enseñanza aprendizaje
de la materia de cálculo integral. Para lo cual se utilizaron secuencias
didácticas diseñadas con la finalidad de integrar, los conocimientos
previos del aprendiz, la resolución de problemas en aplicaciones de
la ingeniería, los afectos matemáticos, las tecnologías de la información
y la comunicación durante la intervención educativa que tuvo una extensión
de 40 horas clases. 


\section{DESARROLLO DE AMBIENTES DE APRENDIZAJE EN GEOMETRÍA ANALÍTICA }

\begin{datos}

Edgar Edgar Allan Romero Angulo, Evangelina López Ramírez.

Universidad Autónoma de Baja California,

México,

allan.romero@uabc.edu.mx; evangelina.lopezramirez@yahoo.com.mx 

\end{datos}

El siguiente trabajo escrito es una intervención educativa basada
en una investigación cualitativa que fue desarrollado en educación
media superior, en la asignatura de “Matemáticas 3”, que involucra
los conocimientos de geometría analítica. Para ello se creo un ambiente
de aprendizaje desarrollado por secuencias didácticas que tomaban
en cuenta el aspecto emocional, el aspecto cognitivo y el uso de TIC´s.
Todo esto para poder desarrollar el aprendizaje y la comprensión en
la asignatura. Se tomó en cuenta como fundamento teóricos de Lucie
Sauve y del ITESM (sobre ambientes de aprendizaje), y de Gómez Chacón
(sobre aspecto afectivo y cognitivo). 

\setcounter{section}{147}


\section{ASPECTOS EMOCIONALES QUE IMPACTAN EL DESEMPEÑO DE LOS ESTUDIANTES
EN EL AULA DE MATEMÁTICA.}

\begin{datos}

Oswaldo Martínez Padrón, María Analía Contarino, Jorge Ávila Contreras.

$^{1}$Universidad Pedagógica Experimenta Libertador , $^{2}$Instituto
de Formación Docente Nº 1, Universidad Católica Silva Henríquez$^{3}$, 

Venezuela$^{1}$, Argentina$^{2}$, Chile$^{3}$,

ommadail@gmail.com; pspmanaliac@hotmail.com;

javila@ucsh.cl 

\end{datos}

Aquí se reporta un compendio de aspectos emocionales que tienen que
ver con el desempeño de los estudiantes al momento de aprender Matemática.
Dicha investigación está motivada por el hecho de que en el aula de
clases de Matemática aún persiste un desinterés por darle preponderancia
a aquellos aspectos que tienen que ver no sólo con las emociones sino
con otros factores del dominio afectivo y social. Se corresponde con
un estudio documental que revisa reportes que informan sobre el impacto
que tienen dichos aspectos en relación con la Matemática que aprenden
o dejan de aprender los estudiantes. 


\section{MODELANDO TUS FINANZAS. UNA PROPUESTA DE EDUCACIÓN ECONÓMICA Y FINANCIERA
DESDE UNA PERSPECTIVA SOCIO-POLÍTICA DE LA EDUCACIÓN MATEMÁTICA }

\begin{datos}

Fanny Aseneth Gutiérrez Rodríguez, Yael Carolina Rodríguez Moreno,
Francisco Javier Camelo Bustos.

Universidad Distrital Francisco José de Caldas, Universidad Federal
de Minas Gerais,

Colombia, Brasil,

fagutierrezr@correo.udistrital.edu.co; yacrodriguezm@correo.udistrital.edu.co, 

\end{datos}

Se da cuenta de adelantos de una tesis de maestría que surge desde
tres aspectos: una revisión teórica de la perspectiva socio-política
de la Educación Matemática, un proyecto de educación económica y financiera
(Colombia) y un análisis del contexto de estudiantes en un colegio
de Bogotá. A partir de lo anterior, proponemos a estudiantes analizar
el uso del crédito Codensa para generar un ambiente de aprendizaje
en el que la modelación matemática desde una perspectiva socio-política
surja como posibilidad de trabajo en el aula. Mediante una metodología
crítica y cualitativa, analizamos los discursos, evidenciando posibilidades
y limitaciones en su implementación.


\section{LA INTERPRETACIÓN DE MEDIDAS DE TENDENCIA CENTRAL A TRAVÉS DEL APRENDIZAJE
BASADO EN PROBLEMAS CON APOYO DE LAS TIC’S.}

\begin{datos}

Iosmany Ruiz Puertas, Maikel Quintana Núñez.

Universidad de las Ciencias Informáticas (UCI),

Cuba,

osmany@uci.cu; mquintanan@uci.cu 

\end{datos}

La propuesta que se presenta desde la metodología didáctica del Aprendizaje
Basado en Problemas (serán empleadas las siglas ABP para hacer referencia
a la metodología) pretende contribuir a la formación de un pensamiento
aleatorio en la resolución de problemas de Estadística Descriptiva
y a mejorar tanto el proceso de interpretación como el de comprensión
para los estudiantes de tercer año de la carrera de Ingeniería en
Ciencias Informáticas de la Universidad de Ciencias Informáticas.


\section{ESTRATEGIA DIDACTICA PARA EL APRENDIZAJE DE LAS MEDIDAS DE TENDENCIA
CENTRAL A TRAVES DE LAS TECNOLOGIAS DE LA INFORMACION Y LA COMUNICACIÓN
EN EL MARCO DEL MODELO EDUCATIVO POR COMPETENCIAS.}

\begin{datos}

Miguel Ángel Rangel Romero, Karla Liliana Puga Nathal, Eliseo Santoyo
Teyes, Felipe Santoyo Telles.

Centro Universitario del Sur de la Universidad de Guadalajara, Instituto
Tecnológico de Cd. Guzmán Jalisco, Centro de Bachillerato Tecnológico
Industrial y de Servicios 226- Guzmán Jalisco,

México,

marangel@cusur.udg.mx; karlalpn4@hotmail.com; 

esantoyo25@hotmail.com; santf22@hotmail.com

\end{datos}

Se propone un proceso sistemático para desarrollar una competencia
interpretativa en estadística, mediante una secuencia de acciones
que debe realizar un estudiante, utilizando las tecnologías de la
información y la comunicación, lo cual debe contribuir a la formación
de un pensamiento crítico, basado en la valoración de la evidencia
obtenida. Se aborda un objeto de conocimiento, medidas de tendencia
central, a partir de las necesidades reales de personas en un contexto
y con una problemática particular, dado que resulta necesario procesar,
analizar y difundir la información producida en diversos sectores,
misma que es utilizada en procesos de toma de decisiones. 


\section{UN ESTUDIO EXPLORATORIO DE UNA PROPUESTA DIDÁCTICA EN 1er AÑO DE
MEDIA GENERAL PARA LA APLICACIÓN DE DIFERENTES CONCEPCIONES DE LA
NOCIÓN DE FRACCIÓN}

\begin{datos}

Yuraima Lilibeth Ramírez Rondón.

Universidad Pedagógica Experimental Libertador, Instituto Pedagógico
de Caracas,

Venezuela,

yura2572@gmail.com

\end{datos}

Números racionales y proporcionalidad. Nivel Medio Básico. Estudio
de casos.

La necesidad de crear una herramienta didáctica que permita al docente
gestionar de manera idónea los conocimientos que los jóvenes poseen
sobre fracción, decimales y porcentajes fue la motivación para el
diseño de una propuesta de enseñanza. Ésta pretende captar la atención
del estudiante sobre el proceso con el cual aprendió estos temas y,
reconstruya sus conocimientos en caso de necesitarlo, uniendo la manipulación
y la realidad del contexto social hasta alcanzar un aprendizaje significativo.
Se aplicó a alumnos del 1er año de educación media general y se utilizó
una investigación observacional participativa de tipo cualitativo,
bajo el paradigma interpretativo. 

\renewcommand\thesection{\tiny CB\ \nplpadding{3}\numprint{\arabic{section}}\_{}A}

\setcounter{section}{151}


\section{ SIMILITUDES Y DIFERENCIAS ENTRE LOS ELEMENTOS CURRICULARES ESTABLECIDOS
EN LOS PROGRAMAS DE SECUNDARIA EN MATEMÁTICA DEL MINISTERIO DE EDUCACIÓN
PÚBLICA (MEP) —COSTA RICA— Y LOS DESARROLLADOS POR LOS DOCENTES}

\begin{datos}

Jessenia Chavarría Vásquez, Marcela García Borbón, Manuel Emilio Salazar
Morales.

Universidad Nacional de Costa Rica (UNA),

Costa Rica,

jesenniach@gmail.com; magarcib@gmail.com;

manuelsalm@hotmail.com

\end{datos}

En 2013, comenzó a regir una propuesta curricular en Matemática desde
el nivel educativo básico hasta el medio superior costarricense, basada
en la resolución de problemas y modelización; aunque esta no se aplicó
por completo en todos los niveles, ya que se implementó un plan de
transición y se dieron capacitaciones a docentes. Así, el estudio
buscará determinar las diferencias y semejanzas entre lo que establece
el programa vigente y lo que desarrolla curricularmente el docente
en el aula para el tema de triángulos en el nivel de sétimo año. Esta
comparación será analizada a la luz de los elementos básicos del currículum
y la formación docente.

\renewcommand\thesection{CB\ \nplpadding{3}\numprint{\arabic{section}}}

\setcounter{section}{153}


\section{DISEÑO CURRICULAR Y EL POTENCIAL GENERATIVO DE CONCEPTOS}

\begin{datos}

Javier Triana, Ninfa Navarro, José Luis Orozco Tróchez, Alfonso Palomá,
Maryory López, Vasken Stepanian Bassili, Edgar Alberto Guacaneme Suarez.

Colegio Champagnat de Bogotá,

Colombia,

javiertriana@colegiochampagnat.edu.co; joseluisorozco@colegiochampagnat.edu.co

\end{datos}

A través del análisis crítico de las actividades encontradas en las
cartillas del proyecto \textquotedbl{}JUEGA Y CONSTRUYE LA MATEMATICA\textquotedbl{}
y de las tareas diseñadas por los profesores, se evidencio el potencial
generativo de conceptos matemáticos relacionados con el pensamiento
variacional y el razonamiento covariacional que poseen. En esta comunicación
se presentan las reflexiones como resultado de la experiencia de los
profesores cuando gestionan una de estas tareas, en dos sentidos,
en cuanto a la formación profesional como docentes indagadores en
ejercicio y en cuanto al desarrollo del pensamiento variacional en
los estudiantes. 

\newpage\setcounter{section}{155}


\section{BANCO DAS FUNÇÕES: UMA ADAPTAÇÃO PEDAGÓGICA DO JOGO BANCO IMOBILIÁRIO
PARA A SALA DE AULA}

\begin{datos}

Evanilson Landim Alves, Victor Louis Rosa de Souza, Isa Layane Máximo
UPE Campus Petrolina,

Brasil,

evanilson.landim@upe.br; victor\_louis@hotmail.com;

i-s-a\_layanne@hotmail.com

\end{datos}

O artigo ora apresentado trata do processo de desenvolvimento do Jogo
Banco das Funções que tem como objetivo contribuir com o processo
de aprendizagem de conceitos relativos às funções polinomiais e trigonométricas.
O jogo foi desenvolvido no decorrer de um projeto de extensão da UPE
Campus Petrolina com estudantes secundaristas da Escola Estadual Antônio
Padilha. Os estudantes participaram não apenas da vivência do jogo,
mas de todas as etapas da sua construção. Na sala de aula, o jogo
tem contribuído e motivado os estudantes a compreenderem conceitos
importantes no estudo das funções. 


\section{O JOGO DESAFIO DAS OPERAÇÕES NA SALA DE AULA: UMA INVESTIGAÇÃO COM
ESTUDANTES DO 7º ANO DO ENSINO FUNDAMENTAL}

\begin{datos}

Andrea Paula Monteiro de Lima, Conceição de Lourdes Farias Brandão,
Acácia Silva Pereira.

UFPE ,

Brasil ,

a.p.ml@hotmail.com; clfb\_05@hotmail.com;

acácia.ps@gmail.com

\end{datos}

Este artigo apresenta as potencialidades didáticas do Jogo Desafio
das Operações aplicado numa turma de 7º ano do Ensino Fundamental.
O jogo é resultado de um projeto desenvolvido pela UFPE com 3 600
professores de matemática do estado de Pernambuco. A motivação para
o Desafio das Operações surge a partir das dificuldades dos estudantes
da Educação Básica no processo de aprendizagem dos números inteiros.
Os resultados indicam que o jogo contribui com a aprendizagem dos
conceitos realtivos às operações com números inteiros, principalmente
no caso da multiplicação e divisão destes números.


\section{LA DISTANCIA ENTRE 2 PUNTOS TERRESTRES MEDIADA POR M-LEARNING }

\begin{datos}

Alvaro Joserand Camargo, John Alvaro Munar.

Universidad La Gran Colombia, 

Colombia,

alvaroj.camargo@ugc.edu.co; john.munar@ugc.edu.co

\end{datos}

Una de las dificultades que presenta el docente en la enseñanza de
los conceptos de medida, es el de implementar situaciones didácticas
que permitan atender la necesidad de acercar a los estudiantes al
concepto de longitud en entornos reales. Geo2metrica es una estrategia
didáctica diseñada por los autores, que permite el cálculo de distancias
entre dos puntos terrestres a partir de un sistema de referencia con
soporte GPS. Esta oportunidad basada en el m-learning, propone un
cambio en las prácticas de aula mediadas con dispositivos móviles
que desarrolla el aprendizaje interactivo y colaborativo de sus participantes. 


\section{TRATAMIENTO ESCOLAR DE LA GEOMETRÍA A TRAVÉS DEL DISEÑO DE ACTIVIDADES
INTEGRANDO MATERIALES MANIPULATIVOS }

\begin{datos}

Gilbert Andrés Cruz, Carlos Esteban Montenegro Narváez.

Universidad del Valle,

Santiago de Cali - Colombia,

andrescruz1008@hotmail.com; cemn\_1987@hotmail.com

\end{datos}

A partir del trabajo realizado se pretende reconocer que en años recientes,
un cuerpo creciente de investigaciones en didáctica de las matemáticas
han identificado algunas dificultades en relación con la enseñanza
y aprendizaje de contenidos temáticos, procesos y contextos relacionados
con el pensamiento espacial y sistemas geométricos, siendo comúnmente
atribuidas a causas de orden epistemológico, cognitivo, curricular
y didáctico. En este marco se genera la necesidad de integrar recursos,
específicamente materiales manipulativos, al currículo y a las prácticas
escolares, que permitan fortalecer en los estudiantes los conocimientos
obtenidos para resolver algunos problemas de su entorno escolar y
cotidiano.


\section{ATIVIDADES DO MESTRADO PROFISSIONAL NA PRÁTICA}

\begin{datos}

Adil Ferreira Magalhães, Davidson Paulo Azevedo Oliveira, Tereza Raquel
Couto de Lima.

Universidade do Estado Minas Gerais – UEMG – Unidade Ibirité, Instituto
Federal de Minas Gerais – IFMG/Campus Ouro Preto, Instituto Federal
de Minas Gerais – IFMG/Campus Ouro Preto,

Brasil,

magalhaesadil@gmail.com; davidson.oliveira@ifmg.edu.br;

tereza.lima@ifmg.edu.br

\end{datos}

Apresentamos resultados de atividades desenvolvidas Mestrado Profissional
(MP) em Educação Matemática e aplicadas na prática. No MP é indispensável
a elaboração de um produto de natureza educacional destinado a professores.
Partiu-se de um trabalho que procurou analisar dificuldades que alunos
do curso de Licenciatura em Matemática apresentam na resolução de
inequações por meio de uma sequência de atividades cuja concepção
partiu da retomada do desenvolvimento do pensamento algébrico e funcional
dos estudantes. Para dar luz a análise foi utilizado a Teoria das
Representações Semióticas. A partir do produto Educacional atividades
adaptadas e utilizadas com alunos do ensino básico analisando as representações
gráfica e algébrica dos alunos. Ressalta-se a importância dos diversos
registros de representações semióticas no ensino de inequações, além
do papel do Mestrado Profissional e do Produto Educacional produzido
das pesquisas aplicadas no ensino.


\section{LABORATORIO DE DIDÁCTICA DE LAS MATEMÁTICAS: LA EXPERIENCIA DEL PROYECTO
CURRICULAR DE LICENCIATURA EN EDUCACIÓN BÁSICA CON ÉNFASIS EN MATEMÁTICAS}

\begin{datos}

Christian Camilo Fuentes Leal.

Universidad Distrital Francisco José de Caldas,

Colombia,

cristianfuentes558@hotmail.com 

\end{datos}

Se presentará una experiencia de creación del laboratorio de didáctica
de las matemáticas de la licenciatura en educación básica con énfasis
en matemáticas, como un espacio que brinda apoyo a la formación de
profesores de matemáticas, por medio del préstamo de material didáctico,
bibliográfico y electrónico, para ello se mostrarán su justificación
en el marco de la propuesta del proyecto curricular, además de sus
servicios, necesidades, y proyecciones. 


\section{LOS FUTUROS MAESTROS PLANTEAN PAEV ADITIVOS: EL PAPEL DE LOS INDICIOS
VERBALES}

\begin{datos}

Angela Castro, Núria Gorgorió, Montserrat Prat.

Universitat Autônoma de Barcelona,

España,

angicastro-27@hotmail.com; nuria.gorgorio@uab.cat;

Montserrat.prat@uab.cat 

\end{datos}

Partiendo de los problemas propuestos por 128 alumnos del Grado de
Educación Primaria, en este estudio discutimos sobre: (a) la percepción
que tienen los futuros maestros de las palabras clave en los problemas
aditivos de enunciado verbal de una etapa, y (b) el uso que hacen
de las mismas en el planteamiento de este tipo de problemas. Constatamos
que, mayoritariamente proponen enunciados construidos a partir de
palabras clave y reflexionamos acerca de la importancia que tiene
en la formación de maestros la reflexión sobre el uso de palabras
clave y el impacto que puede tener en los alumnos de primaria. 


\section{LOS EQUIPOS GEOMETRICOS EN LOS ÚTILES ESCOLARES NO SON ESTRICTAMENTE
NECESARIOS}

\begin{datos}

Oscar Jesús San Martín Sicre.

Universidad Pedagógica Nacional – Instituto de Formación Docente del
Estado de Sonora,

México,

sicreo@outlook.es 

\end{datos}

Se presenta una propuesta de investigación que puede ser desarrollada
en el aula con grupos naturales o intactos. Se intentaría responder
a la pregunta ¿Qué efectos didácticos se derivan de la aplicación
en el aula de una propuesta didáctica consistente en la construcción
por los estudiantes de objetos geométricos tales como paralelas, perpendiculares,
bisectrices, mediatrices, cuadrados, triángulos equiláteros, etc,
sin utilizar todas las herramientas contenidas en los juegos geométricos
utilizando para ello únicamente al transportador? Se esbozan algunos
de los fundamentos teóricos matemáticos (Mascheroni, Poncelet, Steiner,
Severi) y didácticos (Piaget, Vygotsky, Aebli. Brousseau) que fundamentarían
al estudio de caso correspondiente.


\section{CONSTRUCCIÓN DEL CONCEPTO DE DERIVADA, CON APOYO DE LA COMPUTADORA }

\begin{datos}

Marisol Radillo Enríquez, Lucía González Rendón, Maricruz Martínez
Ruiz.

Universidad de Guadalajara,

México,

marisol.radillo@red.cucei.udg.mx; lgrendon2@yahoo.com.mx;

markruzel@hotmail.com

\end{datos}

Se presenta una serie de actividades diseñadas desde una perspectiva
constructivista, apoyada en la visualización, para la construcción
del concepto de la derivada de una función. La propuesta se basa en
la manipulación de imágenes y la traducción de la representación geométrica
de una función a las notaciones numérica y algebraica, para lo cual
se utiliza el programa geogebra. Las actividades de visualización
se centran en el significado geométrico de la derivada, los casos
en que una función carece de derivada en un punto dado, la relación
entre la gráfica de la función y la gráfica de su derivada. 


\section{CONOCIMIENTO PEDAGÓGICO DEL CONTENIDO QUE UTILIZA UN PROFESOR DE
MATEMÁTICA PARA ENSEÑAR LOS CONCEPTOS BÁSICOS DE FUNCIÓN EN CUARTO
AÑO DE LA EDUCACIÓN SECUNDARIA COSTARRICENSE}

\begin{datos}

Carlos Eduardo Román López, Jonathan Espinoza González, Miguel Picado
Alfaro.

Universidad Nacional,

Costa Rica,

odraude11121982@yahoo.com; jonaespinoza@una.cr;

miguelpicado@hotmail.com

\end{datos}

Este estudio está relacionado con las investigaciones sobre el conocimiento
matemático para la enseñanza y pretende caracterizar el conocimiento
pedagógico del contenido matemático que posee un profesor de matemática
para enseñar los conceptos básicos de función en el cuarto año de
la Educación Secundaria en Costa Rica. La investigación se enmarca
en el paradigma cualitativo descriptivo basado en los estudios de
caso. Para describir los subdominios del conocimiento pedagógico del
contenido, en los profesores participantes, se utilizarán las categorías
del Análisis Didáctico.


\section{REFLEXIÓN DE FUTUROS PROFESORES DE MATEMÁTICAS DURANTE LAS PRÁCTICAS
DE ENSEÑANZA}

\begin{datos}

Flores Martínez Pablo, Moreno Verdejo Antonio, Castellanos Sánchez
María Teresa.

Universidad de Granada , Universidad de los Llanos ,

España, Colombia,

pflores@ugr.es; amvermejo@ugr.es;

castellanosmt@ugr.es 

\end{datos}

Para estudiar el proceso de iniciación al desarrollo profesional de
futuros profesores de matemáticas durante la realización de sus prácticas
de enseñanza, se re-formula e implementa el curso del último año de
formación (Practica-Profesional-Docente-PPD) de la licenciatura en
matemáticas, facilitando a futuros profesores atravesar ciclos de
reflexión sobre problemas de la enseñanza-(matemáticas) derivados
de sus prácticas, el curso cubre las etapas de una investigación de
diseño que permite planificar, implementar y revisar cada ciclo antes
de pasar al siguiente. El enfoque cualitativo de tipo interpretativo
orienta la comprensión de acciones y relaciones reveladas en los procesos
de reflexión de los participantes.


\section{UNA MIRADA A LA ETNOMATEMÁTICA EN CHILE}

\begin{datos}

Pilar Peña-Rincón.

Centro de Investigación en Ciencias y Tecnología Avanzada del Instituto
Politécnico Nacional de México,

México,

pilaralejandrapena@yahoo.es 

\end{datos}

El objetivo de esta comunicación es mostrar el estado del arte del
desarrollo de la Etnomatemática en Chile como campo de formación y
de investigación. El análisis se realizó utilizando cinco categorías
que aluden a las distintas áreas en las que es posible observar muestras
del desarrollo de la Etnomatemática en Chile: Política educativa,
investigación, formación docente, encuentros académicos, redes académicas.
Luego de analizar las posibilidades y dificultades para el desarrollo
de la Etnomatemática en Chile se concluye que el desarrollo de este
campo recién comienza, pero existen amplias posibilidades de desarrollarla
en el contexto de los proyectos de educación intercultural. 


\section{GRUPO DE ESTUDOS DE PROFESSORES: UM EPISÓDIO SOBRE RESOLUÇÃO DE PROBLEMAS
NO ENSINO DE MATEMÁTICA }

\begin{datos}

Rosana Jorge Monteiro Magni, Nielce Meneguelo Lobo da Costa.

Universidade Bandeirante de São Paulo – UNIBAN,

Brasil,

rosanamagni@ig.com.br; nielce.lobo@gmail.com

\end{datos}

O propósito deste trabalho é apresentar um relato sobre um grupo de
estudos, objeto de uma pesquisa de doutorado que se desenvolve em
um processo de formação continuada, do projeto “Educação Continuada
do Professor de Matemática do Ensino Médio: Núcleo de Investigações
sobre a Reconstrução da Prática Pedagógica” do Programa Observatório
da Educação, que aqui denominamos Projeto Observatório Práticas, proposto
pela Universidade Bandeirante de São Paulo – Uniban/ Brasil em parceria
com a Secretaria da Educação do Estado de São Paulo. Discutimos no
artigo as reflexões do grupo de estudos ao aplicar atividades sobre
Resolução de Problemas para alunos do Ensino Fundamental.


\section{LA MATEMATICA Y SU INFLUENCIA EN LA ELECTRONICA }

\begin{datos}

José Armando Murillo Contreras.

Universidad pedagógica experimental libertador UPEL caracas,

Venezuela,

josem2523@hotmail.com 

\end{datos}

La electrónica y la matemática son dos ciencias importantes en la
actualidad. Y a lo largo del tiempo su relación ha ido de menos a
más. En este caso hablaremos de las relaciones existentes en el ámbito
educativo de la electrónica y la matemática. Teniendo más énfasis
en la gran ayuda que puede proporcionar la matemática en la enseñanza
de la electrónica. Ya que la matemática ayuda al alumno a pensar de
manera mas critica y lógica.


\section{EXPERIMENTO DE ENSINO SOBRE DISTRIBUIÇÃO NORMAL DE PROBABILIDADES
COM PROFESSORES DE MATEMÁTICA DO BRASIL}

\begin{datos}

José Ivanildo F. de Carvalho, Robson Macedo Candeias, Ruy C. Pietropaolo.

Universidade Anhanguera de São Paulo,

Brasil,

ivanfcar@hotmail.com; profmacedo@uol.com.br; 

rpietropaolo@gmail.com

\end{datos}

Nossa pesquisa está inserida no Projeto Observatório da Educação no
Brasil, que visa contribuir com a formação continuada do professor
de matemática da educação básica. Objetivamos analisar o desenvolvimento
do raciocino dos professores na vivência de um experimento de ensino
com a distribuição normal de probabilidades. Tal experimento de ensino
propiciará uma discussão envolvendo os conhecimentos que dão base
para compreensão da distribuição normal de probabilidades como a noção
de desvio-padrão e a curva normal. Com vistas à atender nossos objetivos
optamos pela metodologia do Design Experiments.


\section{LA NUEVA PROPUESTA CURRICULAR PARA LA ENSEÑANZA DE LA GEOMETRÍA EN
LA EDUCACIÓN PRIMARIA DE COSTA RICA}

\begin{datos}

Edwin Chaves Esquivel,

Universidad de Costa Rica, Universidad Nacional,

Costa Rica,

echavese@gmail.com; edwin.chaves.esquivel@una.ac.cr 

\end{datos}

La ponencia que se está presentando realiza una descripción de la
propuesta curricular para la enseñanza de la Geometría en los nuevos
programas de Matemáticas de Costa Rica. Se considera la Geometría
como organizadora de los fenómenos del espacio y la forma, se ven
los objetos geométricos como patrones o modelos de muchos fenómenos
de la realidad. A diferencia de lo que tradicionalmente se ha enseñado
la Geometría basada en el análisis de objetos ideales y abstractos,
en estos programas se propone una enseñanza geométrica basada en problemas
del contexto y fundamentalmente en los entornos espaciales.


\section{DIFICULTADES ASOCIADAS AL CONCEPTO CONJUNTO GENERADOR EN NIVEL SUPERIOR}

\begin{datos}

Esteban Mendoza Sandoval, Flor Monserrat Rodríguez Vásquez.

Universidad Autónoma de Guerrero, 

México,

emendoza@uagro.mx; flor.rodriguez@uagro.mx 

\end{datos}

Dada la naturaleza abstracta del álgebra lineal, nos interesa proponer
una vía alternativa en la enseñanza del concepto conjunto generador,
acuñando a la teoría APOE como sustento teórico. Por tanto en este
trabajo, como primera parte de una investigación en desarrollo, mostramos
algunas dificultades asociadas a dicho concepto en el nivel medio
superior, pues es fundamental para la propuesta alternativa que reconozcamos
tales dificultades como parte de la descomposición genética que se
debe realizar en la contribución del desarrollo del pensamiento matemático
avanzado. 


\section{RELAÇÕES PESSOAIS DE ESTUDANTES DE SÃO PAULO DOS ENSINOS FUNDAMENTAL,
MÉDIO E SUPERIOR SOBRE AS REPRESENTAÇÕES DOS NÚMEROS RACIONAIS}

\begin{datos}

José Valério Gomes da Silva, Marlene Alves Dias.

UNIBAN,

Brasil,

valerio.gomes@yahoo.com.br; alvesdias@ig.com.br

\end{datos}

Apresentamos nesse trabalho parte de nossa pesquisa sobre “A evolução
das relações pessoais de estudantes da educação básica e do ensino
superior sobre a noção de números racionais”. Nosso objetivo é compreender
as marcas das relações institucionais sobre as relações pessoais dos
estudantes. Para tal utilizamos a TAD, a noção de quadro e mudança
de quadros, de níveis de conhecimento esperados dos estudantes e os
diferentes significados identificados por Cavalcanti e Guimarães.
As análises tendem a mostrar que é reduzido o número de estudantes
que evoluem e são capazes de aplicar as técnicas associadas às representações
de frações. 


\section{MODELANDO TUS FINANZAS. UNA PROPUESTA DE EDUCACIÓN ECONÓMICA Y FINANCIERA
DESDE UNA PERSPECTIVA SOCIO-POLÍTICA DE LA EDUCACIÓN MATEMÁTICA}

\begin{datos}

Fanny Aseneth Gutiérrez Rodríguez, Yael Carolina Rodríguez Moreno,
Francisco Javier Camelo Bustos.

Universidad Distrital Francisco José de Caldas, Universidad Federal
de Minas Gerais,

Colombia, Brasil,

fagutierrezr@correo.udistrital.edu.co; yacrodriguezm@correo.udistrital.edu.co

\end{datos}

Se da cuenta de adelantos de una tesis de maestría que surge desde
tres aspectos: una revisión teórica de la perspectiva socio-política
de la Educación Matemática, un proyecto de educación económica y financiera
(Colombia) y un análisis del contexto de estudiantes en un colegio
de Bogotá. A partir de lo anterior, proponemos a estudiantes analizar
el uso del crédito Codensa para generar un ambiente de aprendizaje
en el que la modelación matemática desde una perspectiva socio-política
surja como posibilidad de trabajo en el aula. Mediante una metodología
crítica y cualitativa, analizamos los discursos, evidenciando posibilidades
y limitaciones en su implementación.


\section{LA CONSTRUCCIÓN DE CURVAS FRACTALES COMO OBJETOS QUE TRASCIENDEN
DE PROCESOS ITERATIVOS INFINITOS }

\begin{datos}

Diana Villabona Millán, Solange Roa Fuentes.

Universidad Industrial de Santander, Grupo de Investigación EDUMAT-UIS,

Colombia,

diana.villabona@gmail.com; sroa@matematicas.uis.edu.co

\end{datos}

En este trabajo usaremos los elementos de la teoría APOE (Acrónimo
de Acción, Proceso, Objeto y Esquema) para analizar evidencias empíricas
de la forma en que estudiantes de posgrado en Matemáticas y Educación
Matemática construyen curvas fractales como objetos trascendentes
a partir de procesos iterativos infinitos. La transformación iterada
que nos permite construir las curvas fractales guarda procesos que
pueden ser de naturaleza diversa, indagaremos cómo influye esta naturaleza
en la construcción del infinito matemático en un contexto fractal
específico: la curva de Koch y el triángulo de Sierpinski. 


\section{UNA PROPUESTA DE ENSEÑANZA PARA LOS SISTEMAS DE TRES ECUACIONES LINEALES
CON TRES INCÓGNITAS}

\begin{datos}

Carolina Wa Kay Galarza, Christian Yáñez Villouta.

Universidad de Santiago de Chile,

Chile,

carolina.wakay@usach.cl; christian.yanez@usach.cl 

\end{datos}

Este trabajo corresponde a una tesis de licenciatura, reporta la implementación
de una propuesta de enseñanza en base a sistemas de tres ecuaciones
lineales de primer grado con tres incógnitas. Se considera la teoría
de los Modos de Pensamiento como marco teórico, la teoría del Aprendizaje
por descubrimiento como marco metodológico, y la utilización de Geogebra
5.0 Beta que permite el trabajo en un ambiente tridimensional. A partir
de investigaciones hechas en el ámbito, interesa llevar estas teorías
al aula como métodos de enseñanza y cuantificar el impacto en los
estudiantes, además de recoger su opinión en relación al tema.

\setcounter{section}{177}


\section{UN CAMINO HACIA LA PERSPECTIVA CRÍTICA DE LA EDUCACIÓN MATEMÁTICA:
CONFESIONES DE UN MAESTRO}

\begin{datos}

Magda Liliana González Alvarado, Gabriel Mancera Ortiz, Francisco
Javier Camelo Bustos.

Universidad Distrital Francisco José de Caldas, Universidad Federal
de Minas Gerais,

Colombia, Brasil,

mlgonzaleza@udistrital.edu.co; gmancerao@udistrital.edu.co;

fjcamelob@udistrital.edu.co 

\end{datos}

Presentamos intereses y presupuestos teóricos que consideramos relevantes
para asumir una consolidación de preocupaciones en y para la Educación
Matemática (EM) desde una postura socio-política. Para ello, narraremos
experiencias que como educadores e investigadores en EM nos permitieron
quebrar una visión tradicional y nos condujeron a aceptar una perspectiva
crítica. Particularmente haremos énfasis en aspectos sociales y políticos
de la perspectiva, mostrando obstáculos identificados en dicho posicionamiento
y dando relevancia a los aspectos institucionales y culturales. Al
final, plantearemos algunas preocupaciones actuales sobre las que
consideramos importante profundizar para avanzar en una consolidación
de un visión socio-política de la EM en Colombia.


\section{LOS SISTEMAS DE ECUACIONES LINEALES: EVIDENCIAS DEL TRÁNSITO ENTRE
LOS MODOS DE PENSAMIENTO EN ESTUDIANTES UNIVERSITARIOS }

\begin{datos}

Doris Evila González Rojas, Solange Roa.

Fuentes Universidad Industrial de Santander,

Colombia,

dorevigonroj@gmail.com; roafuentes@gmail.com

\end{datos}

El Álgebra Lineal es un área de las matemáticas que tiene aplicabilidad
diferentes áreas. Por tanto nuestro trabajo busca responder a la pregunta:
¿pueden los estudiantes de un curso de ecuaciones diferenciales transitar
entre los modos de pensamiento propuestos por Sierpinska en la resolución
de situaciones en la que involucran sistemas de ecuaciones? Nuestro
objetivo fue identificar los modos de pensamiento con que los estudiantes
de ingeniería abordan situaciones relacionadas con sistemas de dos
y tres ecuaciones lineales con dos incógnitas (2x2 y 3x2); así como
la manera en que pueden transitar de un modo de pensamiento a otro. 


\section{REFLEXIONES HISTÓRICAS Y FILOSÓFICAS SOBRE LA MODELACIÓN MATEMÁTICA
COMO PRÁCTICA PEDAGÓGICA}

\begin{datos}

Yadira Marcela Mesa, Carlos Mario Jaramillo López, Jhony Alexánder
Villa Ochoa.

Universidad de Antioquia,

Colombia,

yadiramarcela@gmail.com; javo@une.net.co; 

cama@matematicas.udea.edu.co

\end{datos}

Hablar de la modelación matemática como práctica pedagógica implica
centrarla en un nivel epistemológico que posibilite una reflexión
desde la historia y la filosofía de las matemáticas de tal manera
que permita analizar algunos imaginarios y creencias acerca de los
modelos matemáticos, los cuales también demandan una postura acerca
de las concepciones sobre las matemáticas y de su relación con otras
ciencias, lo cual posibilita fuentes de reflexión sobre algunas implicaciones
al plantear ejercicios modeladores en las aulas escolares, de tal
manera que haya una toma de conciencia del maestro acerca de los objetos
centrales de la formación en matemáticas.

\setcounter{section}{182}


\section{CARACTERIZACIÓN DE PROBLEMA MATEMÁTICO}

\begin{datos}

Oscar Alonso Ramírez Castro.

Corporación Universitaria Minuto de Dios (UNIMINUTO),

Colombia orcaramirez@gmail.com ; osramire@uniminuto.edu

\end{datos}

El presente escrito muestra el referente teórico de una investigación
que se encuentra en curso en cuanto al planteamiento y solución de
problemas. Su objetivo consiste en identificar algunas características
del problema matemático las cuales permiten analizar, desde una perspectiva
didáctica, el planteamiento y solución de problemas en el aula de
clase como resultado de la interacción dinámica entre estudiante,
docente y conocimiento matemático.

\setcounter{section}{185}


\section{TALLER DE MATEMÁTICAS EMOCIONALES PARA ALUMNOS DE PREPARATORIA}

\begin{datos}

Miriam Lemus.

Universidad Nacional Autónoma de México,

México,

Miriam.lemusg@gmail.com 

\end{datos}

El aprendizaje y aprovechamiento de las matemáticas, en Bachillerato
es muy importante. En quinto año de Preparatoria los alumnos eligen
área de conocimiento a fin de elegir su carrera posteriormente. Si
las matemáticas se le dificultaron o tienen un historial de reprobación,
la disyuntiva a favor de ésta área, es prácticamente nula.El taller
se realizó con alumnos de cuarto de preparatoria con una característica
en común: son buenos alumnos en todas las materias excepto en matemáticas.
Proyecto de intervención: “El taller de Matemáticas Emocionales” dio
sustento a la utilización de los conceptos asociados con el dominio
afectivo de las matemáticas.


\section{EL PLANTEAMIENTO DE PROBLEMAS EN CLASES DE MATEMÁTICA}

\begin{datos}

Johan Espinoza González. 

Universidad Nacional de Costa Rica.

Costa Rica,

jespinoza@una.cr 

\end{datos}

La resolución de problemas ha sido históricamente en la educación
matemática una componente importante del currículo escolar y un gran
objetivo de instrucción. A partir de ahí surge el planteamiento de
problemas como una línea de investigación y es tal que algunos distinguidos
investigadores en Educación Matemática la reconocen como actividad
importante de la experiencia matemática de cualquier estudiante. De
esta forma se presenta en qué consiste el proceso de invención de
problemas, para qué podría emplearse, cuáles son los principales aportes
que ha dado a los procesos de enseñanza y aprendizaje y cuáles son
las implicaciones que conlleva plantear una actividad de invención
de problemas. La ponencia tiene como fin ubicar al lector en las bondades
que presenta la invención de problemas y motivarlo para que emplee
este tipo de actividades en sus clases.


\section{HABILIDADES MATEMÁTICAS Y RESOLUCIÓN DE PROBLEMAS EN EL NIVEL BÁSICO}

\begin{datos}

María Rosado Ocaña, Genny Uicab Ballote, María Ordáz Arjona.

Universidad Autónoma de Yucatán,

México,

rocana@uady.mx; uballote@uady.mx;

oarjona@uady.mx 

\end{datos}

La enseñanza en la escuela primaria privilegia la resolución de problemas
como la fuente principal de generación de conocimiento matemático.
El propósito del presente trabajo es compartir el análisis de una
experiencia de trabajo en resolución de problemas con niños de quinto
y sexto grado de primaria, en busca de respuestas a la pregunta: ¿Existe
relación entre las habilidades matemáticas desarrolladas por los niños
y las estrategias empleadas en resolución de problemas en el nivel
básico?, haciendo énfasis en las relaciones entre habilidades matemáticas
(aritméticas y geométricas) y las soluciones desarrolladas por los
niños en los problemas.


\section{HACIA UN APRENDIZAJE AUTORREGULADO}

\begin{datos}

Noemí Hilda Carione, Miguel Ángel Martínez.

ISP “Joaquín V. González” CABA y ISFDYT Nº 24 Quilmes,

Argentina,

nhcarione@gmail.com; lavalle1003@gmail.com

\end{datos}

Se presentan dos experiencias que propician la autorregulación del
aprendizaje. La primera de ellas consiste en un proyecto de investigación-acción
llevado a cabo en una institución secundaria y está centrado en la
enseñanza por competencias. La segunda, se realizó durante la enseñanza
de la asignatura Álgebra I del Profesorado de Matemática y el objetivo
fue avanzar en la autonomía en el aprendizaje de los estudiantes.
Se apeló a que contaran con habilidades metacognitivas y a la instrumentalidad
de una tarea ligada a su futuro rol docente. Ambas experiencias se
fundamentan en el aprendizaje autorregulado como un proceso activo
y constructivo.


\section{RELAÇÕES METODOLÓGICAS NA TRANSIÇÃO 5º E 6º ANO DO ENSINO FUNDAMENTAL
BRASILEIRO PARA O CASO DAS FIGURAS PLANAS E FIGURAS ESPACIAIS}

\begin{datos}

Sirlene Neves de Andrade, Marlene Alves Dias.

DER – Diretoria Regional Sul 3 , UNIBAN,

Brasil ,

sirlene-neves@hotmail.com; alvesdias@ig.com.br

\end{datos}

Nesse trabalho apresentamos um estudo das metodologias relacionadas
à transição entre o 5º e 6º ano do Ensino Fundamental paulista para
as noções matemáticas de figuras planas e espaciais, ressaltando a
importância dos professores compreenderem as diferenças de comunicação
existentes entre essas duas etapas da escolaridade. O objetivo é compreender
melhor quais conhecimentos os estudantes podem mobilizar quando ingressam
no 6º ano do Ensino Fundamental e identificar aqueles que devem ser
revisitados para atingir a aprendizagem esperada. A metodologia utilizada
é a análise documental. Observamos a importância de o professor escolher
as atividades em função dos conhecimentos mobilizáveis dos estudantes. 


\section{PLAN CEIBAL Y MATEMÁTICA }

\begin{datos}

Yacir Testa ,

CEIBAL, Consejo de Formación en Educación, CICATA, 

Uruguay,

prof.yacirtesta@gmail.com 

\end{datos}

Se presentaran distintas acciones que el Plan Ceibal está llevando
adelante en relación a la Enseñanza y al Aprendizaje de la Matemática:
{*}Cursos de Integración de la tecnología al Aula de Matemática.{*}Herramientas
como la Plataforma Adaptativa de Matemática (PAM) que por primera
vez a nivel mundial se ha puesto a disposición de todos los Docentes
y Estudiantes. {*}Actividades que se están imprentando en el marco
del Proyecto de Robótica Educativa, talleres y capacitaciones a Estudiantes
y Profesores de Matemática. Buscan potenciar el aprendizaje de distintos
conceptos matemáticos, y el desarrollo del pensamiento matemático
del estudiante, usando como soporte la tecnología. 


\section{CREACIONES LITERARIAS CON FINES DIDÁCTICOS EN LA ENSEÑANZA DE LAS
MATEMÁTICAS}

\begin{datos}

Néstor Briceño Estepa.

Universidad Pedagógica Nacional,

Colombia ,

nestbric@gmail.com

\end{datos}

Este trabajo indaga acerca de elementos que son dispuestos cuando
se incluyen textos narrativos en el aula de matemáticas y cómo constituyen
imaginarios frente a las matemáticas. El objetivo es reconocer cómo
en las narrativas de lo literario; se declaran unos asuntos en relación
con las matemáticas e imaginarios en que están inmersas. La propuesta
consiste en hacer un análisis de contenido a textos que usan el recurso
literario en relación a las matemáticas y facilitan un escenario para
reconocer la formalización del lenguaje, normalización en el aula
de matemáticas entre otros en relación al poder formativo de las matemáticas.


\section{RAZONAMIENTO PROBABILÍSTICO EN ESTUDIANTES DE UNDÉCIMO GRADO}

\begin{datos}

José Alcides Romero Martínez, Mónica Andrea Vergara Chávez, Felipe
Jorge Fernández Hernández.

Universidad Pedagógica Nacional,

Colombia,

mdma\_jaromerom698@pedagogica.edu.co; mdma\_mavergarac163@pedagogica.edu.co

\end{datos}

El número de investigaciones en educación probabilística en Colombia
es escaso; siendo necesario el desarrollo de trabajos en este campo.
Por ello, se propone un sistema de indicadores, basado en la Taxonomía
SOLO, que describa los resultados de aprendizaje del razonamiento
probabilístico en los estudiantes bajo los enfoques intuitivo, frecuencial
y clásico a través de los siguientes elementos de análisis: frecuencias
relativas y convergencia estocástica, comparación de probabilidades,
tratamiento de intuiciones, tablas de frecuencia, espacio muestral,
eventos, aplicación de la ley de Laplace, cálculo de probabilidades
en experimentos compuestos y gráficos estadísticos.


\section{RAZONAMIENTO ACERCA DEL SIGNIFICADO DE LOS PARÁMETROS EN LOS MODELOS
DE PROBABILIDAD EN ESTUDIANTES UNIVERSITARIOS}

\begin{datos}

Álvaro Cortínez, Norma Alamilla, Armando Albert, José Guadalupe Rios.

Universidad de Tarapacá, Universidad Politécnica del Centro, Tecnológico
de Monterrey,

México, Chile, 

acortinezp@uta.cl; norma.alamilla@gmail.com;

albert@itesm.mx; jrios@itesm.mx

\end{datos}

En las últimas décadas la estadística bayesiana está siendo usada
cada vez más en áreas de la salud, investigación en ingeniería y negocios,
entre otros. Esto demanda una necesidad de formación inicial desde
al menos el nivel de pregrado. Diversas universidades del mundo están
incluyendo en sus programas de estudio elementos de estadística bayesiana.
La inclusión de ideas bayesianas en el aula universitaria nos representa
un reto didáctico importante por tener sus propias dificultades. En
esta investigación nos proponemos hacer una exploración sobre concepciones
y dificultades de estudiantes universitarios alrededor del concepto
clave de la estadística bayesiana: variabilidad del parámetro.


\section{EL DOCENTE DE MATEMÁTICA Y SU PROCESO DE INSTRUCCIÓN EN LA ERA DIGITAL}

\begin{datos}

Ricardo E. Valles P.

Universidad Simón Bolívar - Sede del Litoral,

Venezuela,

revalles@usb.ve

\end{datos}

La educación se encuentra influenciada por el creciente desarrollo
tecnológico; las Universidades se encuentran en un constante equipamiento
técnico con la intensión de mantenerse actualizadas y brindar condiciones
vanguardistas para el desarrollo educativo. El nuevo rol que el docente
de matemática consistirá básicamente en adaptar su praxis educativa
al lenguaje que manejan los estudiantes de hoy, En esta perspectiva,
se considera pertinente plantear dos preguntas centrales para esta
ponencia: ¿Qué papel juega el docente en matemática en la era digital?,
¿Cómo la enseñanza de la matemática debe ser rediseñada en función
de los nativos digitales? 

\setcounter{section}{199}


\section{\uppercase{ fORMACIÓN DE pROFESORES DE mATEMÁTICA: UN LUGAR PARA
PENSAR LAS PRÁCTICAS dOCENTES. Provincia de Entre Ríos, ARGENTINA.}}

\begin{datos}

Gay Mabel Alicia, Vertone Claudia.

Instituto de Profesorado Concordia D-54, Instituto de Formación Docente
“Profesor Rogelio Leites”,

Concordia y La Paz - Entre Ríos - Argentina,

mabelgay@gmail.com; clodine468@hotmail.com 

\end{datos}

En esta comunicación se pretende poner en tensión, dar a pensar los
espacios de Práctica Docente en la transformación curricular de la
provincia de Entre Ríos, como recorridos posibles, complejos y multidimensionales.
Los cambios en el Nivel Secundario plantean desafíos y urgencias que
nos interpelan. Reconocer estas tensiones, resistencias, acuerdos
y desacuerdos, nos invita a visibilizar y problematizar los espacios
de la Formación Práctica a fin de proponer experiencias formativas
desde el comienzo de la Formación Inicial que permitan construir modos
de actuación basados en la reflexión de la realidad con niveles crecientes
de autonomía.


\section{FUNDAMENTACIÓN TEÓRICA DE LA APROPIACIÓN CONCEPTUAL CON AYUDA DE
LAS TECNOLOGÍAS DE LA INFORMACIÓN Y LAS COMUNICACIONES, EJEMPLIFICADO
EN LA DERIVADA. }

\begin{datos}

Neel Báez Ureña, Ramón Blanco Sánchez, Olga Lidia Pérez González.

Universidad Autónoma de Santo Domingo, Universidad de Camagüey,

República Dominicana, Cuba,

neelbaez@gmail.com; ramón.blanco@reduc.edu.cu;

olguitapg@gmail.com

\end{datos}

En el presente, se hace cada vez más cotidiano el uso de las tecnologías
de la información y las comunicaciones (TIC) en las clases de Matemática,
pero lamentablemente esta cotidianidad es resultado de la intención
de estar a la moda, o de cumplir orientaciones institucionales, pero
sin una clara conciencia por parte de los maestros del por qué, cuándo
y cómo, deben incorporar estas tecnologías en su clase. Por tales
razones el presente trabajo tiene como objetivo ilustrar, algunos
ejemplos, del por qué, cuándo y cómo se deben integrar estas nuevas
tecnologías con la clase de Matemática tradicional, de modo que resulte
un producto pedagógico que posea de forma holística la experiencia
de lo tradicional y el aporte de lo nuevo.


\section{ESTUDIO SOBRE LA ORGANIZACIÓN DEL PROCESO DE ENSEÑANZA APRENDIZAJE
DE LA GEOMETRÍA Y TRIGONOMETRÍA PLANA }

\begin{datos}

Elizabeth Rincón Santana, José Manuel Ruíz Socarras, Ramón Blanco
Sánchez.

Universidad Autónoma de Santo Domingo, Universidad de Camagüey,

República Dominicana, Cuba,

te10elirisa@gmail.com; jose.ruiz@reduc.edu.cu; 

ramón.blanco@reduc.edu.cu

\end{datos}

Se realiza una investigación empírica de corte cualitativo, con el
objetivo de indagar sobre la opinión de los profesores, de la carrera
de Educación, mención Matemática, de la Universidad Autónoma de Santo
Domingo, en relación a las causas por las que los contenidos de las
asignaturas de Matemática en dicha carrera, no pueden ser enseñados
con el nivel de profundidad que exige el programa de la asignatura.
Para desarrollar el estudio se aplica una encuesta. El diseño y resultados
de la encuesta son analizados teniendo como fundamentos teóricos la
investigación que trata la organización de contenidos en la Matemática.


\section{FORMAS DE ACCIÓN, REFLEXIÓN Y EXPRESIÓN ASOCIADAS A TAREAS DE TIPO
ADITIVO}

\begin{datos}

Óscar Leonardo Pantano Mogollón, Rodolfo Vergel Causado.

Universidad Pedagógica Nacional, Universidad Pedagógica Nacional,

Bogotá - Colombia,

mdma\_olpantanom661@pedagogica.edu.co; rodolfovergel@gmail.com

\end{datos}

El interés de esta propuesta de investigación consiste en indagar
las formas de acción y reflexión asociadas al desarrollo conceptual
de los estudiantes de cuarto grado de primaria al resolver tareas
de tipo aditivo en los naturales, desarrollo que estará caracterizado
por los medios semióticos de objetivación y los procesos de objetivación
desarrollados por ellos. Esta propuesta y su interés se plantean a
partir de la revisión de antecedentes y de los constructos teóricos
asociados a la Teoría cultural de la objetivación, con el propósito
que saquen a la luz el problema y la pregunta de investigación que
se pretende abordar.

\setcounter{section}{204}


\section{PENSAMIENTO CRÍTICO EN ESTUDIANTES DE GRADO OCTAVO (EMC)}

\begin{datos}

Angello David Chaparro Fonseca, Henry Geovanny Cardozo Rozo.

Colegio República de China IED,

Colombia ,

Angeloud07@hotmail.com; profemathenry@hotmail.com 

\end{datos}

Esta investigación buscó determinar las contribuciones al desarrollo
del pensamiento crítico en estudiantes de grado octavo, promovidas
mediante la implementación de ambientes de aprendizaje en torno a
la estadística y enmarcados en la teoría de la Educación Matemática
Crítica, por medio de la investigación acción y sus cuatro fases:
la Planificación, la Acción, la Observación y la Reflexión. Pare ello
se tuvo en cuenta los seis diferentes tipos de Ambientes de aprendizaje
desde la EMC y como categorías de análisis se tomaron las seis diferentes
destrezas o habilidades del Pensamiento Crítico.


\section{UNA PROPUESTA DIDÁCTICA PARA LA COMPRENSIÓN DE LA  FUNCIÓN DERIVADA
EN SECUNDARIA DESDE LA TAD.}

\begin{datos}

Daniela Bonilla Barraza, Jocelyn Díaz Pallauta.

Universidad de la Serena,

Chile,

danielabonillab@gmail.com; jocelyndiazpallauta16@gmail.com

\end{datos}

La presente comunicación consiste en el diseño de una propuesta didáctica,
donde se aproxima a los estudiantes de secundaria al estudio de la
función derivada de una función polinómica. En el diseño se utilizan
elementos de la Teoría Antropológica de lo Didáctico (TAD), se distingue
una organización matemática que radica en determinar la función derivada
de una función polinomial, a través del tránsito de la gráfica de
las rectas tangentes a la curva de la función hacia la caracterización
y gráfica de la función derivada asociada, para su puesta en escena
se utiliza el software de geometría dinámica, geogebra. 


\section{APRENDER-ENSEÑAR MATEMÁTICA EN INGENIERÍA}

\begin{datos}

Zoraida Pérez Sánchez.

Universidad Nacional Experimental de Guayana,

Venezuela ,

zoraidaperezs@gmail.com

\end{datos}

El propósito de esta propuesta de investigación es comprender el fenómeno
de la educación matemática en las carreras de ingeniería, a partir
de la interpretación de significados atribuidos por sus actores a
experiencias e interacciones durante la actividad de aprender-enseñar
matemáticas. Corresponde a un estudio de caso cualitativo, instrumental
en el contexto de la Universidad Nacional Experimental de Guayana,
apoyado en premisas epistemológicas, metodológicas y analíticas de
la Teoría Fundamentada. Como fin último y práctico, se propondrán
lineamientos para la elaboración de un programa de formación para
la especialización de profesores de matemática en las carreras de
ingeniería. 


\section{PROPUESTA DIDÁCTICA PARA LA ENSEÑANZA Y EL APRENDIZAJE DE CÁLCULOS
DE PERÍMETRO Y ÁREA DE FIGURA GEOMÉTRICA, ORIENTADA A LA INTEGRACIÓN
DE ESTUDIANTES CON DISCAPACIDAD VISUAL}

\begin{datos}

Carlos Castillo.

Universidad Pedagógica Experimental Libertador, Instituto Pedagógico
de Barquisimeto “Luis Beltrán Prieto Figueroa”,

Venezuela,

carloscastillo.13@hotmail.com 

\end{datos}

La investigación propone diseñar una propuesta didáctica para la enseñanza
y el aprendizaje de cálculos de perímetro y área de figura geométrica,
orientada a la integración de estudiantes con discapacidad visual.
La misma se llevo a cabo en dos fases. Primera Fase: Estudio Diagnóstico,
a fin de diagnosticar las dificultades que presentan los docentes
de matemática para trabajar con estudiantes que presenten discapacidad
visual. Segunda Fase: atendiendo a los resultados que se obtuvieron,
se procedió con el diseño de la propuesta didáctica. Con esta se espera
generar estimulo en los docentes, respecto a implementar este tipo
de programaciones donde se busque desarrollar estrategias creativas
e innovadoras, que involucren a todos los estudiantes cualquiera sea
su condición, y capacidades.


\section{CÓMO INTERPRETAN LOS ESTUDIANTES DE LOS GRADOS TRANSICIÓN, PRIMERO
Y SEGUNDO EL NÚMERO CERO. UNA INVESTIGACIÓN EN EL COLEGIO JORGE ISAACS}

\begin{datos}

John Jairo Getial Arteaga, Kelly Johana De Arco Jiménez, Bryan Eduardo
Rodríguez Díaz, Karen Lisbeth Vallejo Chacón, Brayan David Gómez.

Franco Universidad Distrital Francisco José de Caldas – L.E.B.E.M,

Bogotá - Colombia,

jhonarteaga209@hotmail.com; qlly1993@gmail.com;

brianrodriguezd@gmail.com; 

\end{datos}

El presente trabajo de investigación describe y analiza las interpretaciones
que realizan los estudiantes en los grados transición, primero y segundo
en cuanto al número cero, por medio de una encuesta estandarizada
bajo el estudio de tres categorías valor posicional, cantidad y origen,
realizada a 45 estudiantes entre los 4 y 5 años de edad del Colegio
Jorge Isaacs, para identificar falencias que se puedan presentar con
este número “tan misterioso”.


\section{MODELO EDUCATIVO BASADO EN COMPETENCIAS EN BACHILLERATO, TEXTOS Y
PRÁCTICAS DOCENTES DE LOS PROFESORES}

\begin{datos}

Agustín Grijalva Monteverde, Silvia Elena Ibarra.

Olmos Universidad de Sonora, 

México,

guty@gauss.mat.uson.mx; sibarra@gauss.mat.uson.mx 

\end{datos}

Se presenta un texto de matemáticas para estudiantes del Colegio de
Bachilleres del Estado de Sonora, en México. Se plantean las posibilidades
de emplearlo para promover el desarrollo de competencias en estudiantes,
así como instrumento para la formación de profesores de bachillerato,
que contribuya a la modificación de sus prácticas docentes. El texto
fue elaborado por expertos en Matemática Educativa de la Universidad
de Sonora y se establecieron mecanismos de seguimiento a las prácticas
docentes de los profesores y con base en ello se plantean futuras
modificaciones del material, involucrando a los profesores en la discusión
sobre su actividad académica.


\section{CONCEPTO DE NÚMERO IRRACIONAL EN TEXTOS UNIVERSITARIOS}

\begin{datos}

Alfonso Segundo Gómez Mulett.

Universidad de Cartagena, 

Colombia,

agomezm1@unicartagena.edu.co

\end{datos}

En este trabajo se reporta el estudio sobre la introducción del concepto
de número irracional en los libros de texto universitarios. La información
fue obtenida de una muestra intencional de libros utilizados para
la enseñanza de dicho concepto, en el primer curso del currículo de
matemáticas, en los cuales se estudió la definición de número irracional,
a través del análisis de contenido, encontrándose cuatro categorías
en las que pueden enmarcarse las definiciones. La pesquisa dejó ver
que con la variedad de definiciones persisten problemas en la presentación
del concepto de número irracional, convirtiéndose en un obstáculo
epistemológico para su aprendizaje. 


\section{LAS VALORACIONES SOCIALES EN LA TRAYECTORIA DE APRENDIZAJE DE LAS
MATEMÁTICAS. EL CASO DE LUCHO. }

\begin{datos}

Julián Ricardo Gómez Niño, Luis Guillermo Marín Saboya, Gloria García
Oliveros.

Universidad Pedagógica Nacional,

Bogotá - Colombia,

jurioni@hotmail.com; maringuillermo@yahoo.com;

gloriagarciaoliveros@gmail.com

\end{datos}

En nuestra trayectoria como docentes encontramos estudiantes que deciden
no participar en el aprendizaje de las matemáticas. Estudios que conciben
el aprendizaje como forma de participación social muestran que estas
decisiones están influenciadas por razones como: las valoraciones
sociales negativas expresadas por sus compañeros y/o profesores a
partir de sus intervenciones, las identidades negativas producidas
por dichas valoraciones, y la poca motivación para involucrase en
el aprendizaje de la matemáticas. Esta ponencia presenta un análisis
de las razones sociales-culturales por las cuales un estudiante toma
la decisión de interrumpir su participación durante una actividad
grupal propuesta en la clase matemáticas.


\section{REGISTROS DE REPRESENTACIÓN SEMIÓTICA QUE MOVILIZAN LOS ESTUDIANTES
ADOLESCENTES AL RESOLVER SISTEMAS DE ECUACIONES LINEALES}

\begin{datos}

William Andrés Cárdenas, Wilfaver Hernández Montañez. 

Universidad Pedagógica Nacional,

Bogotá - Colombia,

dma\_wcardenas989@pedagogica.edu.co; whernandezm@pedagogica.edu.co

\end{datos}

Con el propósito de indagar sobre los diferentes registros de representación
movilizados por los adolescentes al momento de abordar la variable
como incógnita y lograr una mejor comprensión de su uso, se muestran
los resultados del diseño, implementación y evaluación de una unidad
didáctica en torno a la solución de situaciones problema que involucran
sistemas de ecuaciones lineales. Se plantean actividades que involucran
diferentes sistemas de representación a partir de las cuales los estudiantes
deben realizar tránsito entre estas, realizar operaciones entre registros
semióticos (al interior de un sistema o entre diferentes sistemas)
para dar solución a las situaciones planteadas.


\section{ANÁLISIS DE TAREAS MATEMÁTICAS CON EL USO DE CALCULADORAS ALGEBRAICAS
DESDE LA TEORÍA ANTROPOLÓGICA DE LO DIDÁCTICO.}

\begin{datos}

María Fernanda Mejía Palomino, Edinsson Fernández Mosquera.

Institución Educativa Normal Superior Farallones de Cali, Universidad
de Nariño,

Colombia,

mafanda1216@gmail.com; edi454@yahoo.com

\end{datos}

Investigación relacionada con el álgebra escolar y la integración
de TIC realizada con estudiantes de noveno grado de la educación básica
colombiana en el marco de una tesis para optar el título de magister
en educación. Comunicación breve. Esta ponencia presenta el análisis
de unas tareas algebraicas diseñadas para la enseñanza de la factorización
de polinomios teniendo presente la complementariedad de las técnicas
Lápiz/Papel (L/P) y del sistema de álgebra computacional (conocidos
como CAS, sigla en inglés). Estas tareas se diseñaron y se realizaron
teniendo presente la Teoría Antropológica de lo Didáctico y la génesis
instrumental bajo la metodología de una micro-ingeniería didáctica.
El análisis se divide en dos momentos, el primero es un análisis a
priori y el segundo es el análisis de los resultados obtenidos en
la experimentación con estudiantes de grado noveno. 

\setcounter{section}{216}


\section{SABERES FUNCIONALES EN LAS PRÁCTICAS DE UN INGENIERO ELECTRÓNICO}

\begin{datos}

Diana Sarait Gómez Leal, Edith Miriam Soto Pérez.

Universidad Autónoma de San Luis Potosí,

México,

dianagomez.matedu@gmail.com; miriam@fciencias.uaslp.mx

\end{datos}

El propósito de este trabajo es analizar las prácticas sociales en
el contexto profesional de un ingeniero electrónico, a fin de reconocer
los saberes matemáticos funcionales, para poder categorizarlos y describirlos,
desde la visión Socioepistemología. Partimos de que el Discurso Matemático
Escolar en los cursos de ingeniería, no logra tener relación clara
con lo que necesita saber un ingeniero en su vida profesional. El
contexto al que nos referimos corresponde al de un proyecto en el
área de robótica; en él encontramos que para la formación y la representación
de una imagen observada por un robot se usa una función.


\section{REGISTROS DE REPRESENTACIÓN SEMIÓTICA DE LA FRACCIÓN EN UN TEXTO
ESCOLAR.}

\begin{datos}

Jorge Eliécer Beltrán Triana. 

Secretaria de Educación Distrital, Universidad Distrital Francisco
José de Caldas,

Bogotá - Colombia,

jorgeeliecerb@gmail.com 

\end{datos}

El artículo presenta algunos resultados de un trabajo de maestría
en educación matemática el cual describe el tratamiento didáctico
que hace el libro de texto “Casa de las matemáticas 4” del concepto
de fracción. El trabajo pone en evidencia que en las operaciones de
tratamiento y conversión se encuentran varias de las dificultades
que presentan los estudiantes de cuarto grado de primaria en el aprendizaje
de la fracción. En particular, se destaca la dificultad asociada con
la complejidad de escoger la representación apropiada, en un registro
semiótico determinado, para una situación donde se involucre el concepto
de fracción.


\section{EVALUACIÓN DEL CONOCIMIENTO DIDÁCTICO-MATEMÁTICO SOBRE PROBABILIDAD
EN PROFESORES DE PRIMARIA EN ACTIVO}

\begin{datos}

Claudia Vásquez , Àngel Alsina.

Pontificia Universidad Católica de Chile, Universidad de Girona,

Chile, España,

cavasque@uc.cl; angel.alsina@udg.edu

\end{datos}

El conocimiento que un profesor necesita para enseñar ha sido ampliamente
investigado durante los últimos años, sin embargo existen pocos estudios
sobre profesores en activo. Se presenta un estudio que evalúa el conocimiento
didáctico-matemático sobre probabilidad en profesores de educación
primaria, centrado en aspectos de los componentes del modelo del conocimiento
didáctico-matemático. El análisis de los datos va a permitir, en primer
lugar, describir fortalezas y debilidades en relación a componentes
del conocimiento didáctico-matemático, en profesores de educación
primaria para enseñar probabilidad, y en segundo lugar, obtener información
relevante para orientar la formación inicial y continua del profesorado.


\section{ACTITUDES DE ESTUDIANTES DE SECUNDARIA HACIA EL TRABAJO CON SITUACIONES
DE APRENDIZAJE}

\begin{datos}

María del Socorro García González, Rosa María Farfán Márquez.

Cinvestav, IPN,

México,

mgargonza@gmail.com; rfarfan@cinvestav.mx 

\end{datos}

El presente trabajo da cuenta de un trabajo experimental del estudio
de las actitudes de estudiantes de un grupo particular de secundaria
cuando trabajan con una situación de aprendizaje sobre la proporcionalidad.
Para el estudio de la actitud se adoptó el modelo tripartita de la
actitud y se organizó un diseño metodológico que respondiera al modelo
adoptado. Identificamos dos tipos de actitudes, manifestadas por todos
los estudiantes. 1) Aceptación a la actividad y 2) Colaboración entre
compañeros.


\section{LINEALIDAD DEL POLINOMIO EN UNA SITUACIÓN DE LA FÍSICA. LA CAÍDA
DE OBJETOS}

\begin{datos}

Irene Pérez Oxté, Maria Gorocica Titla, Rodolfo Fallas Soto, Mauricio
Orozco del Castillo, Francisco Cordero Osorio.

CINVESTAV-IPN,

México,

iperezo@cinvestav.mx; mgorocicat@cinvestav.mx;

rfallass@cinvestav.mx; morozcoc@cinvestav.mx

\end{datos}

En este escrito se presentan resultados de la puesta en escena de
un diseño centrado en la resignificación de los parámetros de la ecuación
cuadrática través del contexto de un fenómeno físico. Dicho diseño
se basó en una epistemología de prácticas enmarcado en la Teoría Socioepistemológica.
En el estudio se buscó favorecer argumentaciones por parte de los
estudiantes para resignificar los parámetros de la ecuación cuadrática
de tal suerte que establecieran relaciones entre expresiones algebraicas
y gráficas. Se identificó que las experiencias de los estudiantes
así como la situación fueron elementos determinantes para el éxito
en la resolución del diseño.

\setcounter{section}{222}


\section{MICRO-INGENIERÍA DIDÁCTICA EN LA ENSEÑANZA DE LOS CONCEPTOS DE ÁREA
Y PERÍMETRO UTILZANDO EL TANGRAMA CHINO EN ALUMNOS DE QUINTO AÑO BÁSICO.}

\begin{datos}

Yohana Swears Pozo.

Universidad Iberoamericana de Ciencias y Tecnología,

Chile,

yswears@gmail.com; yswears@ibero.cl 

\end{datos}

La enseñanza de las matemáticas y en especial la de geometría se ha
volcado hacia una mecanización, entrega de fórmulas más que propiciar
una reflexión de contenidos y mucho menos realizar una construcción,
articulación, apropiación de saberes matemáticos. En este trabajo
se toma como Marco Teórico la Ingeniería Didáctica de Artigue, la
Teoría de Situaciones Didácticas de Guy Brousseau para enseñar los
conceptos de Área y Perímetro con la utilización del Tangrama Chino.
Con distintas Situaciones Didácticas de descubrimiento, exploración,
formulación y cálculos de conceptos básicos en geometría como Superficie,
Contorno, puedan construir e integrar estos contenidos con otros posteriores.


\section{DÍA DEL JUEGO MATEMÁTICO }

\begin{datos}

Mónica Carolina Cárcamo Recinos, Claudia Jeannette Cojulún Juárez.

Universidad Panamericana,

Guatemala,

carol\_carcamo@hotmail.com; clau27\_7@hotmail.com

\end{datos}

Las herramientas y actividades que apoyen y constituyan un componente
que ayude a desarrollar la creatividad de los docentes de las escuelas
primarias y secundarias por medio de la convivencia y el compartir
experiencias docentes, permite ayudar a que se creen círculos de mejora
continua en la didáctica de la matemática. El día del Juego matemático
es un proyecto realizado por un grupo de docentes que tiene el deseo
de mejorar la enseñanza de la matemática a través de talleres simultáneos
impartidos anualmente por otros docentes que han ideado actividades
lúdicas exitosas para enseñar temas específicos y desean compartir
sus ideas.

\setcounter{section}{228}


\section{CONSTRUCCIÓN DE LECCIONES DIDÁCTICAS DE PROBABILIDAD PARA UN ENTORNO
VIRTUAL DE APRENDIZAJE }

\begin{datos}

Gladys Denisse Salgado Suárez, José Dionicio Zacarías Flores, Yazmín
Jiménez Jiménez.

FCFM, BUAP,

México,

gladys008@hotmail.com; jzacarias@fcfm.buap.mx;

yazjim2\_26@hotmail.com

\end{datos}

Se presenta la construcción de lecciones didácticas para el aprendizaje
de la probabilidad para estudiantes de nivel medio superior, diseñadas
para utilizarse en un Entorno Virtual de Aprendizaje (EVA) con las
que se busca promover el aprendizaje de los primeros conceptos de
probabilidad, asimismo reducir parte de la problemática observada
en alumnos de este nivel. Para el desarrollo de dichas lecciones y
la interfaz para su presentación se recurre a los elementos teóricos
de la didáctica de Cuevas y Pluvinage y la integración de la tecnología
digital. Se presentan los resultados alcanzados en la aplicación del
primer prototipo.

\setcounter{section}{230}


\section{INFERENCIA INFORMAL: EL VALOR P, EN PRUEBAS DE ALEATORIZACIÓN }

\begin{datos}

María Inés Rodríguez, José Armando Albert.

Universidad Nacional de Río Cuarto, Tecnológico de Monterrey,

Argentina, México,

mrodriguezbriguet@gmail.com; albert@itesm.mx.

\end{datos}

Todo curso introductorio de estadística a nivel universitario tiene
como propósito llegar a desarrollar métodos inferenciales. Siendo
éste uno de los temas más enseñados resulta ser el peor comprendido
y utilizado. Esto ha preocupado a la comunidad internacional dedicada
al estudio e investigación en Educación Estadística, quienes sugieren
estimular el desarrollo del razonamiento inferencial de manera informal,
a partir de los 14 años. En este trabajo presentamos algunas reflexiones
surgidas a partir de la bibliografía revisada y describimos una actividad
áulica con simulación, que contribuye al desarrollo del razonamiento
inferencial informal del estudiante, facilitando la comprensión e
interpretación del valor p.


\section{ACTITUDES QUE MANIFIESTAN HACIA LAS MATEMÁTICAS LOS ESTUDIANTES DE
CHILE DE 4º AÑO DE EDUCACIÓN BÁSICA }

\begin{datos}

Marcelo Casis Raposo, Daniela Bravo Valdivia.

Universidad Finis Terrae, Universidad Metropolitana de Cs. de la Educación, 

Chile,

marcelocasis@gmail.com; daniela.bravo.v@hotmail.com

\end{datos}

En este documento pretendemos describir las actitudes que manifiestan
los estudiantes de 4º básico de Chile hacia las matemáticas. Proponemos
un estudio estadístico descriptivo, que permita contar con información
válida y relevante sobre los aspectos afectivos de estos estudiantes,
la que podría explicar y anticipar las causas del fracaso escolar
en matemáticas. Este informe resume una propuesta de trabajo colaborativo
de los autores a través de una visión didáctico matemática que oriente
la formación de profesores, promoviendo la creatividad docente, la
contextualización activa y la valoración y fortalecimiento del dominio
afectivo de los estudiantes hacia la educación matemática.


\section{UN ANÁLISIS SOBRE LA VISUALIZACIÓN MATEMÁTICA CON ALUMNOS DE PROFESORADO.}

\begin{datos}

Celis, María Belén; Muñoz, Lorena Raquel; Vozzi, Ana María; Zelaya
Galera, Cristina Lorena.

Facultad de Ciencias Exactas Ingeniería y Agrimensura, Universidad
Nacional de Rosario, 

Argentina,

mbcelis@fceia.unr.edu.ar; lorena.raquel.m@gmail.com;

amvozzi@fceia.unr.edu.ar

\end{datos}

El presente trabajo se encuadra en el Proyecto de Investigación “Interaccionismo
entre el lenguaje Matemático y el Aprendizaje” (ING 476) dirigido
por la Profesora Martha Elena Guzmán, que se desarrolla en la FCEIA,
UNR. Esta experiencia se realizó con estudiantes de 2º año de la carrera
Profesorado en Matemática, en la asignatura Álgebra Lineal y Geometría
Analítica. Analizamos el trabajo que realizaron en grupos, mediante
una presentación en Power Point desarrollando el tema Cónicas para
exponerlo frente a sus compañeros, con el objeto de fomentar a través
de este soporte la visualización como medio de aportación de los conocimientos
geométricos. 


\section{APROXIMACIÓN AL ESTUDIO SOCIOPISTEMOLÓGICO DEL PRINCIPIO DE LA ESTABILIDAD
DEL CAMBIO: EL CASO DEL CAOS DETERMINISTA}

\begin{datos}

Jesús Enrique Hernández Zavaleta, Ricardo Cantoral Uriza.

CINVESTAV,

México,

jherza@gmail.com; rcantor@cinvestav.mx 

\end{datos}

Esta primera aproximación es parte de una investigación en curso que
pretende dar cuenta del carácter determinista del cambio ligado a
la inestabilidad de un sistema en situaciones en donde se presenta
la construcción de los conocimientos matemáticos. Se pretende mostrar
una parte del análisis histórico – epistemológico que se encuentra
orientado hacia una epistemología de prácticas en donde se encuentran
asentadas las nociones del Principio de la Estabilidad del Cambio
utilizadas por L. Euler y H. Poincaré, entre otros, hasta llegar a
los principales exponentes de la noción de Caos Determinista.


\section{ETNOMATEMÁTICA E COGNIÇÃO CORPORIFICADA: DIÁLOGO TEÓRICO}

\begin{datos}

Olenêva Sanches Sousa.

Universidade Anhanguera de São Paulo,

Brasil,

oleneva.sanches@gmail.com

\end{datos}

Essas reflexões contemplam alguns aspectos da relação entre o Programa
Etnomatemática e a Teoria da Cognição Corporificada, inserindo-se
em estudos de Doutorado em Educação Matemática, que objetivam também
o reconhecimento desse Programa como uma teoria geral transdisciplinar
do conhecimento. Tomando como parâmetro o contexto educacional, e,
em especial da Matemática escolar, estabelece um diálogo teórico entre
os ciclos etnomatemáticos do conhecimento e vital e algumas defesas
da consideração da experiência vivida como essencial à compreensão
de um sistema cognitivo que priorize a reunificação do eu. Busca reafirmar
o conhecimento como sentido da própria vida.


\section{ESTUDIO QUE PROMUEVE LA ARTICULACIÓN DE ARGUMENTOS TANTO ANALÍTICOS
COMO GEOMÉTRICOS EN LA COMBINACIÓN LINEAL DE MATRICES DE 2X2.}

\begin{datos}

José Isaac Sánchez Guerra, Carlos Oropeza Legorreta.

Facultad de Estudios Superiores Cuautitlán, UNAM

México,

joejade@hotmail.com; coropeza96@hotmail.com. 

\end{datos}

Cuando hablamos de álgebra lineal estamos convencidos que su estudio
incluye una gran variedad de temas y conceptos que se encuentran íntimamente
relacionados con otras asignaturas. El objetivo de este trabajo es
precisamente la articulación de argumentos de corte algebraico, vinculados
con fundamentos de cálculo diferencial e integral y analizar regularidades
de corte geométrico en contraste con los cálculos analíticos correspondientes
al área generada por los vectores columna en la combinación lineal
de matrices de 2x2. En el desarrollo se incluyen cuatro formas diferentes
de solución (utilizando integrales, cálculo de polígonos regulares,
determinantes y cálculo de paralelogramos).


\section{MECANISMOS DE CONSTRUCCIÓN DE CONOCIMIENTO EN LA BASE DEL PENSAMIENTO
Y LENGUAJE VARIACIONAL }

\begin{datos}

Mario Caballero Pérez, Ricardo Cantoral Uriza.

Cinvestav, 

México,

macaballero@cinvestav.mx; rcantor@cinvestav.mx

\end{datos}

En este escrito exponemos las ideas iniciales de un proyecto de investigación
que tiene como objetivo identificar aquellos mecanismos que permiten
el desarrollo de un pensamiento y lenguaje variacional. En el Cálculo
predomina una enseñanza centrada en los objetos matemáticos, que enfatizan
en aspectos memorísticos soslayando el desarrollo de ideas variacionales.
Esta enseñanza representa un obstáculo para desarrollar el pensamiento
y lenguaje variacional, por lo que resulta importante localizar aquellos
mecanismos que permitan superar este obstáculo y generar su desarrollo,
lo que proveerá de referentes para la generación de herramientas de
intervención para superar dificultades en el aprendizaje del Cálculo.


\section{NOCIONALIZACIÓN DE LAS OPERACIONES MATEMÁTICAS BÁSICAS POR MEDIO
DEL MATERIAL CONCRETO Y LOS ARREGLOS RECTANGULARES.}

\begin{datos}

Carlos Alberto Díez Fonnegra, Óscar Leonardo Pantano Mogollón, John
Edison Castaño Giraldo, Daniel Andrés Fandiño Ríos, Juan Carlos Vega
Vega.

Liceo Hermano Miguel La Salle,

Bogotá - Colombia,

carlosd@fundacionedp.org; leonardopantanom@gmail; 

jecastanogi@gmail.com; daanfari23@gmail.com

\end{datos}

El interés de esta comunicación consiste en mostrar el uso del material
concreto y la construcción de arreglos rectangulares como medios transversales
para la nocionalización de las operaciones básicas en los primeros
grados de escolaridad, en este caso la multiplicación, la división,
la potenciación y la radicación. Inicialmente se realizará la manipulación
del material así como la representación de diferentes cantidades,
posteriormente se mostrará la construcción de las operaciones básicas
mencionadas y finalmente se establecerán algunas relaciones entre
la representación simbólica y la representación geométrica, ventaja
brindada por la implementación del material concreto en el aula de
clase.


\section{LEITURA E ESCRITA MATEMÁTICA NA PERSPECTIVA DO LETRAMENTO}

\begin{datos}

Luanna Priscila da Silva Gomes, Claudianny Amorim Noronha.

Universidade Federal do Rio Grande do Norte,

Brasil ,

luanna.gomes1@gmail.com; cnoronha.ufrn@gmail.com 

\end{datos}

A presente pesquisa objetiva analisar como as práticas de letramento
podem colaborar para a aprendizagem do conteúdo e desenvolvimento
da escrita e leitura na matemática. O letramento matemático é pautado
na teoria da existência de múltiplos letramentos, o que implica adotar
a leitura e escrita matemática como habilidades essenciais para o
exercício de atividades básicas na sociedade. A pesquisa-ação se concretiza
em uma escola pública do município de Natal/RN, numa turma de terceiro
ano do ensino fundamental. Nossa proposta é a realização de um projeto
de letramento matemático. 

\setcounter{section}{240}


\section{ESTRATEGIA DIDÁCTICA PARA IDENTIFICAR UNA ECUACIÓN DE UNA RECTA PERPENDICULAR
A OTRA RECTA DADA, EN SU REPRESENTACIÓN ANALÍTICA O GRÁFICA }

\begin{datos}

Manuel Alfredo Urrea Bernal, Luis Enríquez Chapa.

Universidad de Sonora,

México,

maurr@gauss.mat.uson.mx; luis.cetmar@gmail.com 

\end{datos}

Pensamiento geométrico, Medio superior, Etnográfico

Se presenta una estrategia didáctica para promover, en estudiantes
de nivel medio superior, la identificación de la ecuación de una recta
perpendicular a una recta dada, en forma analítica o gráfica. La estrategia
consiste en presentar una secuencia didáctica organizadas en tres
fases, la primera se espera que se pongan en juego o recuperen los
objetos matemáticos intervinientes, en la siguiente fase es donde
se incorporan las actividades que promoverán el significado institucional
del objeto de estudio en cuestión, y la de cierre que es donde se
institucionalizan los objetos matemáticos que se espera emerjan en
las actividades de desarrollo. 


\section{DOCENMAT: UN ESPACIO DE APRENDIZAJE Y COLABORACIÓN ENTRE PROFESORES
DE MATEMÁTICAS}

\begin{datos}

Rebeca Flores, Luis Arturo Serna, Javier Lezama.

CICATA – IPN,

México,

rebefg@gmail.com; luisarturo\_sernamartinez@yahoo.com.mx;

jlezamaipn@gmail.com,

\end{datos}

La Red de Docencia de Matemáticas (RDM) es un espacio virtual que
congrega a profesores de matemáticas de distintos niveles, así como
investigadores, estudiantes de profesorado y formadores de profesores
en donde todos ellos comparten un interés común que es la enseñanza-aprendizaje
de las Matemáticas. Uno de los planteamientos medulares de este proyecto
es tratar de dejar ver cómo la Red de Docencia de Matemáticas (RDM)
permite construir un espacio de formación para los profesores de matemáticas
en servicio, a corto, mediano y largo plazo que aliente y ayude a
la mejora del desempeño profesional de los miembros. 


\section{DIFICULTADES DE LOS DOCENTES EN FORMACIÓN CON EL ESTUDIO DE LAS CUADRATURAS
DE LAS FIGURAS PLANAS Y SU TRABAJO AULICO CON BASE EN TIPOLOGÍA DIDÁCTICA
DE BROUSSEAU. }

\begin{datos}

Francisco Guillermo Herrera Armendia, Marcos Fajardo Rendón.

Escuela Normal Superior de México,

México,

Harmendia@gmail.com; 3651433@gmail.com

\end{datos}

En la presente contribución describo las principales dificultades
que se observan en los estudiantes durante el proceso de aprendizaje
en la asignatura Figuras y Cuerpos Geométricos, al abordar el estudio
de las cuadraturas de figuras planas propuesta por Hipócrates que
incluyen al rectángulo, al triángulo de cualquier tipo, a los polígonos
cualesquiera, un caso particular de lúnula y la cuadratura del círculo.
El trabajo en el aula se basó en la tipología didáctica de Brousseau,
y el trabajo geométrico se abordó a través de problemas, teoremas
y porismas.


\section{DIFICULTADES QUE ENFRENTAN LOS ESTUDIANTES AL MOMENTO DE RESOLVER
SUSTRACCIONES EN FORMA MENTAL}

\begin{datos}

María Paz Thieller, Karen Suárez Matus, Marcelo Casis Raposo.

Universidad Finis Terrae, 

Chile ,

pazthieller@gmail.com; ksuarez@gmail.com;

marcelocasis@gmail.com

\end{datos}

En el presente documento pretendemos analizar las causas que inciden
en que estudiantes al realizar un cálculo aritmético sencillo de sustracción
no utilicen estrategias de cálculo mental y opten por desarrollar
algún tipo de algoritmo escrito. Se aplicarán seis diagnósticos a
estudiantes entre 7 a 12 años y a partir de ellos se determinarán
las diversas estrategias utilizadas y cuáles son las principales dificultades
que generan en los estudiantes este tipo de cálculos. Al finalizar
la investigación se pretende contar con información efectiva que permita
diseñar, organizar y secuenciar estrategias de cálculo mental para
articular el aprendizaje de esta habilidad.


\section{DESENVOLVIMENTO DA COMPETÊNCIA EM ANÁLISE DIDÁTICA DA MATEMÁTICA:
UMA PROPOSTA DE ESTUDO PARA A FORMAÇÃO INICIAL DE PROFESSORES – BRASIL}

\begin{datos}

Adriana Breda, Valderez Marina do Rosário Lima, Marcos Villela Pereira.

PUCRS ,

Brasil ,

adriana.breda@gmail.com; valderez.lima@pucrs.br; 

marcos.villela@pucrs.br

\end{datos}

Este trabalho tem como objetivo apresentar uma proposta de pesquisa,
em fase inicial, que preocupa-se em investigar de que maneira acontece
o desenvolvimento da competência em análise didática dos futuros professores
de matemática de uma universidade privada da cidade de Porto Alegre,
RS, Brasil. Para isso, criar-se-á um ciclo em que se realizará um
experimento de ensino de tal forma que os futuros professores possam
valorar seus próprios processos de instrução realizados e efetivar
sua melhora. 


\section{LA IDEA DE POBLACIÓN EN EL DESARROLLO DE UN CURSO DE ESTADÍSTICA
UNIVERSITARIA}

\begin{datos}

José Armando Albert, Ma. Guadalupe Tobías.

Tecnológico de Monterrey, 

México,

albert@itesm.mx; mgtl@itesm.mx

\end{datos}

En esta investigación se muestran algunas de las ideas principales
que los estudiantes presentan en el desarrollo de un curso de probabilidad
y estadística sobre la noción de población. Aunque pareciera una noción
simple y sin dificultad alguna, de hecho la investigación reporta
problemas sobre la comprensión y complejidad conceptual en los distintos
niveles del programa de estudio tales como población en estadística
descriptiva, y población en distribuciones de probabilidad y población
en la inferencia estadística. Esta investigación reporta una exploración
de este concepto en estudiantes universitarios.

\setcounter{section}{248}


\section{RESOLUCIÓN DE PROBLEMAS USANDO DIVERSAS APLICACIONES DE SOFTWARE}

\begin{datos}

Jorge Ávila Soria.

Universidad de Sonora,

México,

javilas9@gmail.com

\end{datos}

Desde la introducción de impresos, pasando por lápiz, papel y pizarrón,
incluyendo herramientas como reglas, compás, tablas de cálculo y calculadoras
elementales; la tecnología siempre ha influido en las maneras de enseñar
y aprender matemáticas en todos los niveles. Hoy es ineludible la
inclusión de la computadora como herramienta indispensable para la
generalización del planteamiento y resolución de problemas que permitan
a los estudiantes conocer las formas modernas de resolver problemas,
por medio de la modelación, el planteamiento algorítmico, la simulación
o el prototipo. Software de geometría dinámica, hojas de cálculo y
calculadoras científicas es usado para problemas de Álgebra.


\section{LAS EXPLICACIONES DE LOS PROFESORES DE PRIMARIA EN SITUACIÓN ESCOLAR.
EL CASO DE LAS FRACCIONES}

\begin{datos}

Paul Sánchez T., Evelia Reséndiz B., Ramón J. Llanos P.

Universidad Autónoma de Tamaulipas, 

México,

erbalderas@uat.edu.mx; rjardiel@uat.edu.mx 

\end{datos}

La presente investigación en proceso pretende abordar el estudio sobre
la enseñanza de las fracciones. Una de las maneras de tener acceso
a la información sobre cómo se introduce y desarrolla el tema de las
fracciones consiste en analizar el discurso del profesor, pero también
el de la interacción social que se realiza en el aula escolar. El
objetivo principal de nuestro trabajo pretende analizar las maneras
cómo se introduce y desarrolla las fracciones en situación de enseñanza
en tercer grado de primaria. La metodología es de corte cualitativo,
su utilizarán los registros etnográficos y grabaciones en audio.


\section{PRÁCTICAS DE ENSEÑANZA DE LAS MATEMÁTICAS A FINALES DEL SIGLO XIX}

\begin{datos}

Gustavo Adolfo Parra León.

Grupo Historia de la Práctica Pedagógica,

Colombia ,

gustav863@gmail.com

\end{datos}

El presente trabajo se ocupa de describir algunas de las prácticas
de enseñanza de las matemáticas en las escuelas primarias de Cundinamarca
a finales del siglo XIX, cuyos rasgos principales estuvieron centrados
en el orden, la utilidad y la ejercitación. Estos rasgos no sólo permiten
caracterizar estas prácticas, también constituyen una clave de lectura
en perspectiva histórica de la enseñanza de las matemáticas, en tanto
remiten a prácticas y discusiones que emergieron en los albores de
la Modernidad y que aún bajo otros parámetros, hoy tienen vigencia.


\section{ESTRATEGIAS MATEMÁTICAS A PARTIR DE REDES DE APRENDIZAJE}

\begin{datos}

Claudia Flores Estrada, Adriana Gómez Reyes.

CECyT 5; CECyT 13 –IPN,

México Claudia,

mo@gmail.com; orodelsilencio@yahoo.com.mx

\end{datos}

Como estrategia matemática se consideran las “Redes de Actividades
de Aprendizaje” las cuales están constituidas por Actividades de Aprendizaje
que permiten un mejor entendimiento en el estudiante de Nivel Medio
Superior. Esta red de actividades se vincula desde perspectivas diferentes
y se articulan de varias maneras para cumplir diversos objetivos didácticos.
En el diseño de la red de actividades se ha considerado como base
los resultados de investigación estudiados en el Seminario Repensar
las Matemáticas, específicamente se revisaron tres ciclos para conformar
dos redes de actividades de aprendizaje: 

\setcounter{section}{253}


\section{PERTINENCIA DE UNA PROPUESTA DE FORMACIÓN EN Y HACIA LA INVESTIGACIÓN
DE PROFESORES DE MATEMÁTICAS EN EJERCICIO, PARA EL DESARROLLO DE LA
CARACTERÍSTICA ESTRATEGIA }

\begin{datos}

William Andrey Suárez Moya, Liceth Katherin Beltrán Perdomo.

Universidad Distrital Francisco José de Caldas,

Colombia,

suarytos11@hotmail.com; lizbek320@hotmail.com 

\end{datos}

El proyecto “Formación en y hacia la investigación de profesores de
matemáticas en ejercicio” tiene como propósito diseñar una propuesta
de formación en investigación para profesores de educación básica
y media en Bogotá, caracterizando la investigación como elemento para
comprender la práctica; para ello se realiza un estudio identificando
diversas necesidades de formación en investigación, luego esto se
valida desarrollando un estudio teórico que conjugue lo analizado
con los resultados del primer estudio, denominando así unas características
de formación entre ellas estrategia, la cual brindará elementos que
sistematicen la experiencia de los profesoras al ser investigadores
de su propia práctica.


\section{UNA INTRODUCCION INSTRUMENTADA EN UN AMBIENTE CAS A LA TRANSFORMADA
DE LAPLACE}

\begin{datos}

Francisco Javier Cortés González .

CINVESTAV- IPN, 

México,

fjcortes@cinvestav.mx

\end{datos}

Desde un enfoque instrumental, en un ambiente CAS, nuestra investigación
está orientada hacia el desarrollo de esquemas de acción instrumentada
surgidos en la génesis instrumental. Las tareas involucran conceptos
relacionados con la integración impropia y el uso del parámetro, tienen
como objetivo identificar la relación funcional que se ajusta a la
gráfica de los puntos definidos por los valores del parámetro y el
valor de la integral impropia convergente. En una experiencia con
estudiantes de cursos iniciales observamos que utilizan el ambiente
CAS para identificar el tramo de función que corresponde a la Transformada
de Laplace de una función.


\section{CÁLCULO INFINITESIMAL Y OBSTÁCULOS EPISTEMOLÓGICOS: UNA MIRADA DESDE
LA TEORÍA DE LA ARGUMENTACIÓN}

\begin{datos}

Gloria Ines Neira Sanabria.

Universidad Distrital Francisco Jose de Caldas,

Bogotá - Colombia,

gneira@udistrital.edu.co; nicolauval@yahoo.es

\end{datos}

La argumentación como lenguaje de la filosofía de la matemática, de
la epistemología de la matemática, de la educación matemática, puede
ser una herramienta para abordar los obstáculos epistemológicos en
las rupturas de conocimiento al pasar de un nivel de conceptualización
a otro, y para analizar líneas discursivas en el nacimiento del cálculo
infinitesimal. Tenderé algunas relaciones entre los conceptos de obstáculo
epistemológico y de ruptura a la luz de la argumentación de Perelman,
y relacionaré algunos elementos de esta teoría con los ejemplos de
ruptura en simbolismo, en cuantificadores, en sentido y significado
de los signos matemáticos o lógicos.


\section{CURSO E-LEARNING COMO APOYO EN LA ENSEÑANZA DE UNA DISTRIBUCIÓN BINOMIAL
EN LA ASIGNATURA DE ESTADÍSTICA }

\begin{datos}

Miguel de Nazareth Pineda Becerril, Armando Aguilar Márquez, Juan
Carlos Axotla García, Frida María León Rodríguez, Omar García León.

mnazarethp@fesc.cuautitlan2.unam.mx; armandoa@servidor.unam.mx;

jc\_axotla@fesc.unam.mx

\end{datos}

Para facilitar el estudio del tema de Distribución Binomial en las
asignaturas de estadística que se imparten en la Facultad de Estudios
Superiores Cuautitlán, se desarrolló este tema con diferentes actividades
bajo la plataforma de Dokeos, Dentro de este trabajo se propone que,
aunado a la enseñanza de los profesores en el aula, los alumnos aborden
este tema mediante un curso E-learning, el cual contiene applets,
teoría, chat, videos, etc. En este curso E-learning se cuanta con
diferentes herramientas para que el alumno obtenga una mejor comprensión
del tema de distribución binomial.

\setcounter{section}{258}


\section{LA TRISECTRIZ DE LONGCHAMPS Y LA ESPIRAL DE ARQUÍMEDES EN LA TRISECCIÓN
DEL ÁNGULO}

\begin{datos}

Juan Andrés Avelino Ospina, José Jeancarlo Ortiz Eslava, Coautor,
Pablo Beltrán.

Gimnasio El Lago, 

Colombia,

piaospina@hotmail.com ; edma\_pabeltran162@pedagogica.edu.co

\end{datos}

La Trisectriz de Longchamps y La Espiral de Arquímedes son dos curvas
mecánicas que sirvieron en la historia de las matemáticas para la
solución de un problema clásico (La trisección del Ángulo). Por ello
este trabajo pretende hacer un estudio en la construcción de dichas
curvas mecánicas, para luego con la ayuda de ellas demostrar el problema
de la trisección del ángulo y al finalizar se hará un estudio analítico
de las curvas, con el fin de buscar la pertinencia en los pensamientos
matemáticos que puede generar dicho estudio. 


\section{LA DUPLICACIÓN DEL CUBO MEDIANTE CURVAS MECÁNICAS.}

\begin{datos}

Andrés David Caipa García, Miguel Eduardo Arcos Arévalo, Coautor,
Pablo Beltrán,

Gimnasio El Lago,

Colombia,

Claudia.garcia17@hotmail.com; edma\_pabeltran162@pedagogica.edu.co

\end{datos}

En el presente trabajo dos estudiantes de grado octavo (Media básica)
darán solución a la duplicación del cubo mediante geometría dinámica
y moderna, exponiendo a los participantes el método de solución y
las competencias matemáticas que desarrollaron, específicamente en
el pensamiento geométrico. Se pretende dar la reflexión de que los
conceptos matemáticos, pueden ser trabajados por cualquier tipo de
población estudiantil. Además de enriquecer el proyecto con los aportes
de los participantes.


\section{LA CUADRATRIZ DE DINOSTRATO}

\begin{datos}

Valentina Leal González, Darlene Karina Martínez Lara, Coautor, Pablo
Andrés Beltrán.

Gimnasio El Lago,

Colombia,

gloriagonza@hotmail.com ; edma\_pabeltran162@pedagogica.edu.co

\end{datos}

La Cuadratura de Dinostrato es una curva mecánica que sirvió en la
historia de las matemáticas para la solución de un problema clásico
(La cuadratura del círculo). Por ello este trabajo pretende hacer
un estudio en la construcción de dicha curva mecánica, para luego
con la ayuda de ellas demostrar el problema de la cuadratura del círculo
y al finalizar se hará un estudio analítico de la curva, con el fin
de buscar las ecuaciones cartesianas de las mismas. Este trabajo es
desarrollado y será presentado por dos estudiantes de grado Octavo
del Colegio Gimnasio el Lago, quienes reflexionaran las competencias
que desarrollaron en el presente proyecto. 


\section{ENSEÑANZA DE SUMA Y RESTA DE NÚMEROS NATURALES A NIÑOS CON SÍNDROME
DE DOWN}

\begin{datos}

Liliana Hernández Martínez.

Universidad Pedagógica y Tecnológica de Colombia,

Colombia,

lili\_girla@hotmail.com; lili.girla.hernandez@gmail.com 

\end{datos}

Este proyecto de investigación cualitativa tiene como objetivo diseñar,
aplicar y evaluar una estrategia para estudiantes con Síndrome de
Down (SD) de la Escuela Normal Superior Santiago de Tunja (Boyacá),
que permita un aprendizaje significativo de los conceptos de suma
y resta, facilitando la construcción del conocimiento matemático.
El marco referencial comprende la inclusión educativa, las reformas
curriculares que trae consigo y las características del SD, entre
otros. Éste se desarrollará según el método investigación-acción y
en sus etapas se utilizarán la observación y la entrevista como recolectores
de información, determinando así las características que puede tener
dicha estrategia, adecuada a las capacidades, necesidades e intereses
de los estudiantes. 


\section{ANÁLISIS CANÓNICO DE CORRESPONDENCIAS EN CALIDAD EDUCATIVA}

\begin{datos}

Katerine Tobio G., María Chávez M., Marínela Atencia S., Melba Vertel
M.

Universidad de Sucre, Colombia,

Katerinetobio05@hotmail.com; machame-@hotmail.com;

marinelasalcedo92@hotmail.com; melba.vertel@unisucre.edu.co

\end{datos}

El presente trabajo es un reporte que muestra el avance de nuestra
investigación en curso, donde describiremos las relaciones entre variables
socioeconómicas y la calidad educativa en los departamentos de la
Costa Norte Colombiana, utilizando la técnica “Análisis Canónico de
Correspondencia” (ACC), con el propósito de divulgar esta técnica
estadística poco frecuente en el ámbito de la educación y de buscar
aportes que permitan mejorar el proceso enseñanza-aprendizaje en dicha
región, contribuyendo a la resolución de la problemática referente
al rendimiento académico. Dicha investigación se realiza a partir
de los datos recolectados por el ICFES y el DANE. 


\section{TIC Y ENSEÑANZA DE LAS MATEMÁTICAS EN LA EDUCACIÓN SUPERIOR.}

\begin{datos}

Deninse Farias Javier Pérez María Urbano.

Universidad Simón Bolívar Sede del litoral,

Venezuela,

dfarias@usb.ve; perezj@usb.ve;

murbano@usb.ve 

\end{datos}

La matemática es una ciencia antigua de vital importancia, se originó
con la finalidad de resolver problemas cotidianos del hombre. Modificándose
su enseñanza a través del tiempo y de los recursos tecnológicos, siendo
esta la más buscada por las nuevas generaciones porque se encuentra
inmerso en nuestro día a día, produciendo en los estudiantes gran
motivación para los estudios, así sean considerados estos como duros,
rigurosos ò formales. La presente investigación es un trabajo documental,
que tiene como finalidad presentar las posibilidades didácticas que
ofrecen las tics en la enseñanza de las matemáticas a nivel superior. 


\section{PROPUESTA DIDÁCTICA EN LA OBTENCIÓN DE FÓRMULAS ALGEBRAICAS PARA
LAS PROGRESIONES ARITMÉTICAS Y GEOMÉTRICAS}

\begin{datos}

Evodio Jiménez Tapia,

Centro de Actualización del Magisterio,

Chilpancingo - México,

ifadmatematicas@gmail.com;cam\_matematicas@hotmail.es 

\end{datos}

Los alumnos de educación secundaria (que fluctúan entre los 12 a 15
años) tienen dificultades para expresar algebraicamente el comportamiento
de las progresiones aritméticas y geométricas, por lo anterior se
hace la propuesta didáctica en donde ellos mismos construyan sus fórmulas
a través de algunas actividades impresas que permitan armarlas y comprobarlas.


\section{PROCEDIMIENTOS MATEMÁTICOS EN UN ALGORITMO DE PROCESAMIENTO DE IMÁGENES
MÉDICAS. }

\begin{datos}

Yisel Reyes Cardoso, Maryelines Labrada Madrigal, Pedro Ernesto Salas
Oliva.

Universidad de las Ciencias Informáticas (UCI),

Cuba,

ycardoso@uci.cu; mlabrada@uci.cu; psalas@uci.cu

\end{datos}

La aplicación de modelos matemáticos al tratamiento digital de las
imágenes ha permitido grandes avances en muchísimos campos del conocimiento,
desde la compresión y edición de imágenes hasta las mediciones en
muchos tipos de estudios. Los modelos facilitan realzar determinadas
características que son de interés en la solución de un problema en
particular. La presente investigación tiene como objetivo el desarrollo
un algoritmo combinando las técnicas de filtrado y segmentación de
imágenes médicas con procedimientos matemáticos para medir con precisión
el grado estenosis coronaria en imágenes de coronariografía invasiva
sin usar el software del Angiógrafo.

Palabras clave: modelos matemáticos, mediciones, estenosis coronaria,
coronariografía.


\section{Artematicas. Integración de las artes en la enseñanza de las Matemáticas.}

\begin{datos}

Juan Manuel Zuluaga Arango$^{1}$, Franklin Eduardo Pérez Quintero$^{2}$.

$^{1}$U. Nacional, $^{1,2}$Secretaría de Educación de Medellín,
$^{2}$Corporación Universitaria Lasallista,

Colombia,

manolozuluaga2005@hotmail.com; franklinpromo@gmail.com

\end{datos}

En 2012 se diseñaron una serie de instrumentos y estrategias mediados
por las artes plásticas y escénicas, para que estudiantes Lasallistas
mejoraran sus resultados académicos en el área de Matemáticas. La
propuesta surgió bajo el deseo de mediar el aprendizaje de las matemáticas
con experiencias y estrategias poco convencionales. El ejercicio consistió
en pedirles a los jóvenes crear un personaje de historieta y abrir
un blog en la red, para que construyeran productos cuyo eje central
fuera la matemática. La idea buscaba crear espacios en los que ejercicios
artísticos-creativos fueran excusa para tener un encuentro extra-escolar
con las Matemáticas.


\section{EXPERIENCIA ENSEÑANZA-APRENDIZAJE, PRIMER AÑO PEDAGOGÍA EN MATEMÁTICAS}

\begin{datos}

Luisa Aburto Hageman, Roberto Johnson Herrera.

Pontificia Universidad Católica de Valparaíso,

Chile,

laburto@ucv.cl; rjohnson@ucv.cl

\end{datos}

La comunicación tiene por objetivo dar a conocer una experiencia sobre
la inducción a la Universidad de los estudiantes de la carrera de
pedagogía en matemáticas de la Pontificia Universidad Católica de
Valparaíso. Dicha experiencia consistió en la implementación del proyecto
“Un diseño de enseñanza - aprendizaje para estudiantes de primer año
de la carrera de Pedagogía en Matemáticas” , se esperaba mejorar los
rendimientos de los alumnos de primer año, la tasa de retención, tasa
de aprobación a partir de ciertas estrategias de acción.


\section{LA ESTADISTICA COMO HERRAMIENTA PARA LA CONSERVACIÓN DE ÁREAS NATURALES }

\begin{datos}

Alberto Salazar Barrios.

Centro de Investigación Científica y de Educación Superior de Ensenada
- Baja California,

México,

asalazar@cicese.edu.mx 

\end{datos}

Pensamiento relacionado con estadística. Superior En este trabajo
se analizó un conjunto de variables ambientales en una pradera de
pastos marinos, mediante el análisis de componentes principales el
cual se cimienta en la correlación y redundancia que existe entre
las variables para poder depurar los datos e identificar las variables
de más peso, para posteriormente utilizarlas en la elaboración de
un modelo matemático que describa correctamente el comportamiento
de la pradera. 


\section{NÚMEROS DECIMALES Y OPERACIONES UTILIZANDO LAS SITUACIONES PROBLEMA
Y EL MATERIAL CONCRETO.}

\begin{datos}

Carlos Alberto Diez Fonnegra, Óscar Leonardo Pantano Mogollón, John
Edison Castaño Giraldo, Daniel Andrés Fandiño Ríos, Juan Carlos Vega
Vega. 

Liceo Hermano Miguel La Salle,

Bogotá-Colombia,

carlosd@fundacionedp.org; leonardopantanom@gmail.com;

jecastanogi@gmail.com; daanfari23@gmail.com;

juan88cho@hotmail.com

\end{datos}

En esta propuesta de investigación se muestra una forma de nocionalizar
los números decimales a través de situaciones problema generatrices,
las cuales buscan crear la necesidad en el estudiante de utilizar
un conjunto numérico diferente al de los Naturales, basados en sucesos
históricos que dieron a un nuevo sistema de numeración. Posteriormente,
se expondrá el proceso de representación de cantidades decimales utilizando
material concreto, esto con el fin de dar solución a algunos de los
problemas epistemológicos asociados a este conjunto numérico. Finalmente,
se mostrará la construcción de las operaciones básicas entre números
decimales apoyados en el material concreto.

