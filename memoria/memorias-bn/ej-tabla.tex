\documentclass[12pt,a4paper]{article}
\usepackage[spanish]{babel}
\usepackage[utf8]{inputenc}
\usepackage[T1]{fontenc}
\usepackage[usenames,table]{xcolor}
\usepackage{multirow}
\usepackage{colortbl}
\usepackage{array, booktabs, tabularx,makecell}
\usepackage[most]{tcolorbox}
%\usepackage{tabularx}
%\usepackage{booktabs}
%\usepackage{environ}
\usepackage{times}
%\usepackage{tikz}
%\usetikzlibrary{shapes,calc}
%\usepackage{float}
\usepackage{fancyhdr}
\usepackage[absolute]{textpos}
%
%\usepackage{geometry}
\usepackage[paperheight=27.9cm,%
paperwidth=20.6cm,%
%centering,%
%textheight=26.9cm,
left=2.5cm,%
right=2.5cm,%
top=2.5cm,%
bottom=2cm,%
headheight=1cm,%
headsep=20pt,%
footskip=1cm,%
marginparsep=20pt,%
pdftex=false,%
letterpaper%
]{geometry}
\renewcommand{\rmdefault}{phv} % Arial
\renewcommand{\sfdefault}{phv} % Arial

\usepackage{textcomp} % necesario para el símbolo €


%\usepackage[pdftex]{graphicx}
\usepackage{caption,subcaption}
\setlength\textfloatsep{1.7\baselineskip plus 0.2\baselineskip minus 0.5\baselineskip}
\setlength\intextsep{2\baselineskip plus 0.2\baselineskip minus 0.2\baselineskip}

\usepackage{environ} %lo usaré para crear el environ de tabla colorida

\usepackage{tabularx}
\usepackage{booktabs} % Contiene el comando \midrule
\usepackage{colortbl}
\usepackage{tikz}
\usetikzlibrary{patterns}
\usetikzlibrary{backgrounds,fit,calc}
%% Define style of page number colour box
\newlength\pagenumwidth
\settowidth{\pagenumwidth}{99}
\tikzset{pagefooter/.style={
anchor=base,font=\sffamily\bfseries\small,
text=white,fill=ptctitle,text centered,
text depth=17mm,text width=\pagenumwidth}}

%% Concoct some colours of our own
\definecolor[named]{GreenTea}{HTML}{CAE8A2}
\definecolor[named]{MilkTea}{HTML}{C5A16F}
%%%%%%%%%%%%%%% Encabezado y pie de pagina
%%%%%%%%%%
%%% Re-define running headers on non-chapter pages
%%%%%%%%%%
\fancypagestyle{headings}{%
  \fancyhf{}   % Clear all headers and footers first
  %% Right headers on odd pages
  \fancyhead[RO]{%
    %% First draw the background rectangles
    \begin{tikzpicture}[remember picture,overlay]
    \fill[ptcbackground] (current page.north east) rectangle (current page.south west);
    \fill[white, rounded corners] ([xshift=-10mm,yshift=-20mm]current page.north east) rectangle ([xshift=15mm,yshift=17mm]current page.south west);
    \begin{pgfonlayer}{background}
    %          \path (-1.4cm,2.8cm) node (tl) {};
    %          \path (2.3cm, -8.4cm) node (br) {};
              \path[fill=brown!20] (current page.north west) rectangle (current page.south east);
          \end{pgfonlayer}
    \end{tikzpicture}
    %% Then the decorative line and the right mark
    \begin{tikzpicture}[xshift=-.75\baselineskip,yshift=.25\baselineskip,remember picture,    overlay,fill=ptctitle,draw=ptctitle]\fill circle(3pt);
    \draw[semithick](0,0) -- (current page.west |- 0,0);
        \end{tikzpicture} \sffamily\itshape\small\protect\nouppercase{Vig\'esima Octava Reuni\'on Latinoamericana de Matem\'aticas Educativa}
  }

  %% Left headers on even pages
  \fancyhead[LE]{%
    %% Background rectangles first
    \begin{tikzpicture}[remember picture,overlay]
     \fill[brown!20] (current page.north east) rectangle (current page.south west);
    \fill[ptcbackground] (current page.north east) rectangle (current page.south west);
    \fill[white, rounded corners] ([xshift=-15mm,yshift=-20mm]current page.north east) rectangle ([xshift=10mm,yshift=17mm]current page.south west);
           \end{tikzpicture}
    %% Then the right mark and the decorative line
    \sffamily\itshape\small\protect\nouppercase{Vig\'esima Octava Reuni\'on Latinoamericana de Matem\'aticas Educativa}\ 
    \begin{tikzpicture}[xshift=.5\baselineskip,yshift=.25\baselineskip,remember picture, overlay,fill=ptctitle,draw=ptctitle]
    \fill (0,0) circle (3pt); \draw[semithick](0,0) -- (current page.east |- 0,0 );
       \end{tikzpicture}
  }

  %% Right footers on odd pages and left footers on even pages,
  %% display the page number in a colour box
  \fancyfoot[RO,LE]{\tikz[baseline]\node[pagefooter]{\thepage};}
   \fancyfoot[CO,CE]{\tikz\node{\color{ptctitle}Barranquilla - Colombia};}
  \renewcommand{\headrulewidth}{0pt}
  \renewcommand{\footrulewidth}{0pt}
}
\usetikzlibrary{%
   decorations.fractals%
  ,decorations.pathmorphing%
  ,shadows%
}
\usetikzlibrary{decorations.pathmorphing}
\usetikzlibrary{decorations.text}
\usetikzlibrary{decorations.shapes} %para curvar texto
\usetikzlibrary{shapes,positioning} %para los flow charts
\usetikzlibrary{shapes,arrows,chains}
\pgfdeclarelayer{background}
\pgfdeclarelayer{foreground}
\pgfsetlayers{background,main,foreground}
\usetikzlibrary{intersections}


\newcommand{\helv}{\fontfamily{phv}\fontsize{8}{9}\selectfont}
\definecolor{ptcbackground}{RGB}{150,189,61}
\definecolor{ptctitle}{RGB}{37,92,0}
\colorlet{TablaBordeSuperior}{ptctitle!5!black} %\definecolor{TablaBordeSuperior}{cmyk}{0,0.51,1,0} % BurntOrange
\colorlet{TablaBordeInferior}{ptcbackground} %\definecolor{TablaBordeInferior}{cmyk}{0,0.75,1,0.24} % Bittersweet
\colorlet{TablaCentroSuperior}{ptctitle!10}
\colorlet{TablaCentroInferior}{ptctitle!20}
\colorlet{FuenteCabeceraTabla}{white}
\newcolumntype{M}[1]{>{\raggedright\arraybackslash}m{#1}}
\newcolumntype{L}[1]{>{\raggedleft\arraybackslash}m{#1}}
\newcolumntype{C}[1]{>{\centering\arraybackslash}m{#1}}
\newcommand{\fuentecabecera}[1]{\textcolor{FuenteCabeceraTabla}{\helv\textbf{#1}}}

\newcommand{\encabezadoUnaLineaSuperior}{0.8}
\newcommand{\encabezadoUnaLineaInferior}{0.8}
\newcommand{\espaciadoCabeceraCuerpoUnaLinea}{0.5em}
\newcommand{\encabezadoDosLineasSuperior}{1.3}
\newcommand{\encabezadoDosLineasInferior}{1.2}
%\newcommand{\espaciadoCabeceraCuerpoDosLineas}{1.5em} %hay que aumentar el valor de 0.5 a 1.5 puesto que la cabecera ocupa dos líneas
\newcommand{\espaciadoCabeceraCuerpoDosLineas}{2em}
\newcommand{\encabezadoTresLineasSuperior}{1.8}
\newcommand{\encabezadoTresLineasInferior}{1.8}
\newcommand{\espaciadoCabeceraCuerpoTresLineas}{0.5em}
\newcommand{\encabezadoSuperior}{\encabezadoUnaLineaSuperior} %por defecto una línea
\newcommand{\encabezadoInferior}{\encabezadoUnaLineaInferior} %por defecto una línea
\newcommand{\espaciadoCabeceraCuerpo}{\espaciadoCabeceraCuerpoUnaLinea} %por defecto el de una línea

\newcommand{\encabezadoUnaLinea}{
    \renewcommand{\encabezadoSuperior}{\encabezadoUnaLineaSuperior}
    \renewcommand{\encabezadoInferior}{\encabezadoUnaLineaInferior}
    \renewcommand{\espaciadoCabeceraCuerpo}{\espaciadoCabeceraCuerpoUnaLinea}}
\newcommand{\encabezadoDosLineas}{
    \renewcommand{\encabezadoSuperior}{\encabezadoDosLineasSuperior}
    \renewcommand{\encabezadoUnaLineaInferior}{\encabezadoDosLineasInferior}
    \renewcommand{\espaciadoCabeceraCuerpo}{\espaciadoCabeceraCuerpoDosLineas}}
\newcommand{\encabezadoTresLineas}{
    \renewcommand{\encabezadoSuperior}{\encabezadoTresLineasSuperior}
    \renewcommand{\encabezadoUnaLineaInferior}{\encabezadoTresLineasInferior}
    \renewcommand{\espaciadoCabeceraCuerpo}{\espaciadoCabeceraCuerpoTresLineas}}
%estos tres últimos comandos gestionan si el encabezado será de una línea, de dos o de tres

\newif\iflineaResaltado
\lineaResaltadofalse
\newcommand{\pieInferiorUno}{-5.2}
\newcommand{\pieInferiorDos}{3.45}
\newcommand{\resaltarPie}[2]{\lineaResaltadotrue\renewcommand{\pieInferiorUno}{#1}\renewcommand{\pieInferiorDos}{#2}}
%%%%%%%%%%%%%%%%%%%%%%%%%%%%%%%%%%%
\captionsetup[table]{labelformat = empty,labelfont=it,textfont={bf,it}}
\NewEnviron{tablacolorida}[2]{%
%  \vspace*{-2em}
    \begin{center}
        \captionsetup{type=table} %para que no me de error el posterior comando \captionof
         \captionof{table}{#1} \label{tab:#2} 
        \begin{tikzpicture}
            \node (tbl) {

            \BODY

            };
            \begin{pgfonlayer}{background}
                \draw[rounded corners,top color=TablaBordeSuperior,bottom color=TablaBordeInferior, draw=white] ($(tbl.north west)+(0.14,0)$)
                    rectangle ($(tbl.north east)-(0.13,\encabezadoSuperior)$); %el top color indica el color principal de la línea de cabecera %para tener encabezado de dos líneas (0.13,1.3)
                \draw[rounded corners,top color=white,bottom color=TablaBordeInferior,
                    middle color=TablaBordeSuperior,draw=white!20] ($(tbl.south west)
                    +(0.13,0.5)$) rectangle ($(tbl.south east)-(0.13,0)$);
               \draw[top color=TablaCentroSuperior,bottom color=TablaCentroInferior,draw=white] %el top color indica el color de inicio de las filas del medio, el "white" hace que el borde externo de las filas no se pinte
                    ($(tbl.north east)-(0.13,\encabezadoInferior)$) %con el parámetro 0.8 controlamos la altura de la línea de cabecera  %para tener encabezado de dos líneas (0.13,1.2)
                    rectangle ($(tbl.south west)+(0.13,0.2)$); %el 0.2 controla el grosor de la línea decorativa inferior

%                \iflineaResaltado
%                \draw[top color=TablaBordeSuperior,bottom color=TablaBordeSuperior, draw=white] ($(0,\pieInferiorUno)+(7.5,2)$)
%                    rectangle ($(-0.5,1)-(7,\pieInferiorDos)$); 
%                % ($(0,-4.8)+(7.5,2)$) ($(-0.5,1)-(7,3.1)$)
%                % ($(0,-5.2)+(7.5,2)$) ($(-0.5,1)-(7,3.45)$) %una fila más abajo
%                \fi
            \end{pgfonlayer}
        \end{tikzpicture}
       
    \end{center}
    %\vspace*{-1em}
}

\begin{document}

%\begin{tablacolorida}{Hitos del proyecto}{hitosProyecto}
%    \begin{tabularx}{1\textwidth}{m{0.10\linewidth} m{0.34\linewidth} m{0.48\linewidth}}
%        \arrayrulecolor{purple}
%        \centering\fuentecabecera{Hito} & \centering\fuentecabecera{Fecha} & \centering\arraybackslash\raisebox{-0.3em}{\fuentecabecera{Descripción}} \\[+\espaciadoCabeceraCuerpo]
%        \centering\textbf{H1} & \centering Mes 0 & \centering\arraybackslash Comienzo del proyecto \\ \midrule
%        \centering\textbf{H2} & \centering 2 meses y 3 semanas & \centering\arraybackslash Fin del estudio del arte \\ \midrule
%        \centering\textbf{H3}  & \centering 6 meses y 1 semana & \centering\arraybackslash Finalización de la plataforma de virtualización \\ \midrule
%        \centering\textbf{H4}  & \centering 9 meses & \centering\arraybackslash Finalización de lo concerniente a DTN \\ \midrule
%        \centering\textbf{H5}  & \centering 10 meses y 1 semana & \centering\arraybackslash Finalización de lo concerniente a CDN \\ \midrule
%        \centering\textbf{H6}  & \centering 12 meses & \centering\arraybackslash Fin del Proyecto Fin de Carrera (PFC) \\[0.5ex]
%    \end{tabularx}
%\end{tablacolorida}
\pagestyle{headings}
\renewcommand{\multirowsetup}{\centering}  
\resaltarPie{-9.7}{7.95}
\newcounter{paqueteTrabajo}
\newcounter{tarea}
\setcounter{paqueteTrabajo}{-1}
\newcommand{\filaRHpt}[7]{ \fuentecabecera{#1} & \centering \bfseries{#2} & \centering \bfseries{#3}   & \centering \bfseries{#4} & \centering \bfseries{#5}  & \centering \bfseries{#6} & \centering \bfseries{#7} \\ \midrule}
\newcommand{\filaRHt}[7]{#1  & \centering #2 & \centering #3   & \centering #4 & \centering #5   & \centering #6 & \centering #7   \\ \midrule}
\newcommand{\filaRHTOTAL}[7]{\centering \fuentecabecera{#1}   & \centering \fuentecabecera{#2} & \centering \fuentecabecera{#3 } & \centering \fuentecabecera{#4} & \centering \fuentecabecera{#5 }  & \centering \fuentecabecera{#6} & \centering \fuentecabecera{#7 }   \\}

\encabezadoUnaLinea
\noindent
\begin{tablacolorida}{\color{ptctitle}Horario General de RELME 28}{horario}
\helv

   \begin{tabularx}{1.06\textwidth}{!{\color{ptctitle}\vrule}c!{\color{ptctitle}\vrule}m{2cm}!{\color{ptctitle}\vrule}m{2cm}!{\color{ptctitle}\vrule}m{2cm}!{\color{ptctitle}\vrule}m{2.5cm}!{\color{ptctitle}\vrule}m{2cm}!{\color{ptctitle}\vrule}m{2cm}!{\color{ptctitle}\vrule}}
    \arrayrulecolor{ptctitle}
  \parbox[c][25pt][c]{2cm}{\small\bf\color{white}\centering HORA  }      & \small\bf\color{white}LUNES      &\multicolumn{2}{c|}{\small\bf\color{white}MARTES}   &\small\bf\color{white}  MIERCOLES     &  \small\bf\color{white}JUEVES   & \small\bf\color{white}VIERNES       \\
    
    %%%%%%%%%%%%%%%%%%%1
  \parbox[c][25pt][c]{2cm}{\centering 8:00-8:30}  & \multirow{6}{*}{Inaguraci\'on}   &  \multirowcell{4}{\parbox[c]{2cm}{\centering Cursos Cortos\\ \emph{44 Cursos}}}             & \multirow{4}{*}{} & \multirow{4}{*}{\parbox[c]{2cm}{\centering Cursos Cortos\\ \emph{44 Cursos}}}&\multirow{3}{*}{\parbox[c]{2cm}{\centering Talleres B\\ \emph{16 Salones}}}&\\\cline{1-1}
    %%%%%%%%%%%%%%%%%%%%%%%%%%%2
  \parbox[c][25pt][c]{2cm}{\centering  8:30-9:00} &  &  &   &     &     &  \\\cline{1-1} %%%%%%%%%%%%%%%%%%%%%%%%%%%%%3
 \parbox[c][25pt][c]{2cm}{\centering 9:00-9:30} & &      &   &    &    &                           \\\cline{1-1}\cline{6-6}
                          %%%%%%%%%%%%%%%%%%%%%%%%%%%%%%4
\parbox[c][25pt][c]{2cm}{\centering  9:30-10:00} &  &   &  &  & \multirowcell{3}{\parbox[c]{2cm}{\centering Talleres D\\ \emph{4 salones}\\ Reportes De \\ Investigación\\\emph{18 salones}}} &                         \\\cline{1-1}\cline{3-5}
                          %%%%%%%%%%%%%%%%%%%%%%%%%%%%%%%%%5
\parbox[c][25pt][c]{2cm}{\centering     10:00-10:30} &     &\multirowcell{3}{\parbox[c]{2cm}{\centering Talleres A\\ \emph{16 Salones}}}  &      &\multirowcell{3}{\parbox[c]{2cm}{\centering Mesa Redonda\\ \emph{salón AMKG}}\\[5pt]Talleres D\\ \emph{4 salones}\\ Reportes De \\ Investigación\\\emph{18 salones}}      &       &                                      \\\cline{1-1}\cline{6-6}
                          %%%%%%%%%%%%%%%%%%%%%%%%%%%%%%%%%%%%%%6
 \parbox[c][25pt][c]{2cm}{\centering   10:30-11:00 }&     &         &    &  &   & \\\cline{1-2}
    %%%%%%%%%%%%%%%%%%%%
 \parbox[c][25pt][c]{2cm}{\centering   11:00-11:30} & \multirowcell{3}{\parbox[c]{2cm}{\centering Conferencias Plenarias\\ \emph{1 Salón}}} &  &   &   & \multirowcell{3}{\parbox[c]{2cm}{\centering Talleres F\\ \emph{4 salones}\\ Reportes De \\ Investigación\\\emph{17 salones}}}   &  
                                               \\\cline{1-1}\cline{3-5}
                                               %%%%%%%%%%%%%%%%%%%%%%
\parbox[c][25pt][c]{2cm}{\centering  11:30-12:00} &  & \multirowcell{2}{\parbox[c]{2cm}{\centering  Comunicaciones Breves\\ \emph{19 Salones}}}  &   & \multirowcell{2}{\parbox[c]{2cm}{\centering  Conferencias \\ Especiales \\\emph{15 salones}\\Comunicaciones Breves\\ \emph{35 Salones}}}    &    &       \\\cline{1-1}
% &&&&&&                              \\
 % &&&&&&                              \\
             %%%%%%%%%%%%%%%%%%%%%%%%%%%%%%
 \parbox[c][35pt][c]{2cm}{\centering 12:00-12:30 } &  &   &  &  &  &                       \\\cline{1-7}
                          %%%%%%%%%%%%%%%%%%
 \parbox[c][25pt][c]{2cm}{\centering 12:30-14:00} & \multicolumn{6}{c|}{Almuerzo}                                \\ \cline{1-7}
                          %%%%%%%%%%%%%%%%%%%%%%%%
 %%%%%%%%%%%%%%%%%%%%%%%%%%%%%%Almuerzo%%%%%%%%%%%%%%%%%%%%%%%%%%%%%%
  \parbox[c][25pt][c]{2cm}{\centering 14:00-14:30}& \multirowcell{3}{\parbox[c]{2cm}{\centering Especiales\\ \emph{15 Salones}}}    & \multirowcell{3}{\parbox[c]{2cm}{\centering Talleres C\\ \emph{4 Salones}}}  & \multirowcell{6}{\parbox[c]{2cm}{\centering Posters}}  &\multirowcell{3}{\parbox[c]{2cm}{\centering Talleres B\\ \emph{16 Salones }}}  & \multirowcell{3}{\parbox[c]{2cm}{\centering Mesa Redonda\\ \emph{Salón AMKG}\\Reportes de\\ Investigación\\ \emph{19 Salones}}} & \\\cline{1-1}

 \parbox[c][25pt][c]{2cm}{\centering 14:30-15:00} &  &  &   &  &   &                           \\\cline{1-1}%%%%%%%%%%%%%%%%5%%%%%
   \parbox[c][25pt][c]{2cm}{\centering 15:00-15:30} &  & &    &   &     &   \\  \cline{1-3} \cline{5-6}
                  %%%%%%%%%%%%%%%%%%%%%%%%%%%%%%%%%%%%%
  \parbox[c][25pt][c]{2cm}{\centering 15:30-16:00} &  & \multirowcell{3}{\parbox[c]{2cm}{\centering Talleres F\\ \emph{4 Salones}\\Reportes de\\ Investigación\\ \emph{18 Salones}}}  &    &\multirowcell{3}{\parbox[c]{2cm}{\centering Posters}}    & \multirowcell{2}{\parbox[c]{2cm}{\centering Conferencias\\ Especiales \emph{15 Salones }}}     &   \\  \cline{1-1} 
  %%%%%%%%%%%%%%%%%%%%%%%%%%%%%%%%%%%
  \parbox[c][25pt][c]{2cm}{\centering 16:00-16:30} &  &  &   &  &   &                           \\\cline{1-2}\cline{6-6}
  %%%%%%%%%%%%%%%%%%%%%%%%%%%%%%%%%%%%%%%%%%%
  \parbox[c][25pt][c]{2cm}{\centering 16:30-17:00} &\multirowcell{3}{\parbox[c]{2cm}{\centering Mesa Redonda\\ \emph{Salón AMKG}\\Grupos de\\ Trabajo y \\ Discusión\\ \emph{6 Salones}\\ Talleres C\\ \emph{4 Salones}}}  &  &   &  &   &                           \\\cline{1-1}\cline{3-3}\cline{5-5}
  %%%%%%%%%%%%%%%%%%%%%%%%%%%%%%%%%%%%%
  \parbox[c][27pt][c]{2cm}{\centering 17:00-17:30} &  &  &   &\multirowcell{3}{\parbox[c]{2cm}{\centering Talleres E\\  \emph{3 Salones} y \\ Sala de\\ Informática\\ Reportes de\\ Investigación\\\emph{18 Salones}}}  &   &                           \\\cline{1-1}\cline{3-3}
  %%%%%%%%%%%%%%%%%%%%%%%%%%%%%%%%%%%%%%%%%%%%%
  \parbox[c][25pt][c]{2cm}{\centering 17:30-18:00} &  &  \multirowcell{3}{\parbox[c]{2cm}{\centering Talleres E\\  \emph{3 Salones} y \\ Sala de\\ Informática}} & \multirowcell{4}{\parbox[c]{2cm}{\centering Reportes de\\ Investigación\\ \emph{19 Salones} }}  &  &   &                           \\\cline{1-2}
  %%%%%%%%%%%%%%%%%%%%%%%%%%%%%%%%%%%
    \parbox[c][25pt][c]{2cm}{\centering 18:00-18:30} &\multirowcell{3}{\parbox[c]{2cm}{\centering Talleres A\\  \emph{16 Salones} }}  &  &   &  &   &                           \\\cline{1-1}\cline{5-5}
    %%%%%%%%%%%%%%%%%%%%%%%%%%%%%
     \parbox[c][25pt][c]{2cm}{\centering 18:30-19:00} &  &  &   &  &   &                           \\\cline{1-1}\cline{3-3}
     %%%%%%%%%%%%%%%%%%%%%%%%%%%%%%%%%%%%%%%%%%%%%%%
     %%%%%%%%%%%%%%%%%%%%%%%%%%%%%
     \parbox[c][25pt][c]{2cm}{\centering 19:00-19:30} &  &  &   &\multirowcell{2}{\parbox[c]{2cm}{\centering Asamblea CLAME\\  \emph{ Salón AMKG} }}   &  &                           \\\cline{1-4}
     %%%%%%%%%%%%%%%%%%%%%%%%%%%%%%%%%%%%%%%%%%%%%%
      \parbox[c][25pt][c]{2cm}{\centering 19:30-20:00} &  &  &   &  &   &                           \\\cline{1-7}
    
    \end{tabularx}
\end{tablacolorida}
\encabezadoUnaLinea
%%%%%%%%%%%%%%%%%%%%%%%%%%%%%%%%%%%%%%%%5


\end{document}