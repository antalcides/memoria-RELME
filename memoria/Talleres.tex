
\pagestyle{headings}
\nocite{*}
\fontsize{7}{8}\selectfont
%\setlength{\baselineskip}{5pt}
\pagecolor{white} 

\onecolumn
\chapter{Talleres } 
\renewcommand\thesection{TA\ \nplpadding{3}\numprint{\arabic{section}}} 
\setcounter{section}{0}
\chaptertoc
\twocolumn
\balance


\setcounter{section}{1}


\section{TÉCNICA PARA EL DISEÑO DE PROBLEMAS ADITIVOS INNOVADORES EN EDUCACIÓN
PRIMARIA}

\begin{datos}

JOSÉ ANTONIO MOSCOSO CANABAL, WILBERT SARAO PÉREZ.

ESCUELA NORMAL URBANA, UNIVERSIDAD PEDAGÓGICA NACIONAL UNIDAD 271,

MÉXICO,

mocaja6109@hotmail.com; matematicologico@hotmail.com

\end{datos}

El taller ofrece a los profesores de educación primaria en servicio
y a los estudiantes de magisterio la oportunidad de aprender a manejar
las variables: semánticas, sintácticas, de contexto y magnitudes de
números para diseñar problemas aditivos innovadores, novedosos y diversos
que permitan a los alumnos de educación primaria potencializar sus
esquemas de pensamiento y actualizar sus “cajitas de herramientas”
de que disponen para resolver situaciones problemáticas, y enfrentar
con mejor preparación académica los desafíos de la matemática en educación
secundaria. Los docentes que participan en el taller perciben que
son cada vez más competentes para diseñar problemas aditivos.


\section{EXPLORACIÓN DE LAS FUNCIONES POLINOMIALES A TRAVÉS DEL EMULADOR DE
LA CALCULADORA VOYAGE 200 PARA DISPOSITIVOS MOBILES CON SISTEMA ANDROID
Y COMPUTADORAS PERSONALES CON SISTEMA OPERATIVO WINDOWS}

\begin{datos}

Carlos Garrido, Renaldo Girón, Juan Orlando López.

Universidad de San Carlos de Guatemala,

Guatemala,

notascg@gmail.com; ingrga@gmail.com;

juanorlando@gmail.com

\end{datos}

Una de las calculadoras más completas para cálculos matemáticos que
operan en forma simbólica lo constituye la Voyage 200, ahora disponible
su emulador para Android y Windows. Uno de los temas más importantes
en matemáticas lo son las funciones polinomiales. A través del presente
taller los participantes explorarán las funciones polinomiales al
mismo tiempo que estarán inmersos en el ambiente de la calculadora
Voyage 200. Así que el taller se constituirá en una forma enriquecedora
de combinar matemática y tecnología.


\section{LA RESOLUCIÓN DE PROBLEMAS MATEMÁTICOS UNA ESTRATEGIA EN EL AULA
DE CLASE }

\begin{datos}

Thais Arreaza, Irving Valencia.

Universidad Central de Venezuela,

Venezuela,

tarreaza@gmail.com; irving.valencia@gmail.com

\end{datos}

La resolución de problemas ha jugado un papel fundamental en el desarrollo
de la matemática. Hoy día se reconoce ampliamente la importancia de
la misma como una estrategia didáctica eficaz. Es necesaria la toma
de conciencia sobre la aplicación en las aulas de clase. Este taller
constará de dos partes: en la primera, los facilitadores hablarán
sobre la evolución de la resolución de problemas, algunos modelos
utilizados y los factores que intervienen en ella; en la segunda parte,
los participantes resolverán problemas que les permitan repasar conocimientos
matemáticos adquiridos, activar procesos de pensamientos y aplicar
algunos de los modelos discutidos. 


\section{APRENDIZAJE GESTÁLTICO DE CÁLCULO INTEGRAL }

\begin{datos}

Ma. del Carmen Barrios Cuevas, Hilda Yolanda Guillén Lara.

CBENEQ, CBTis 118, CETis 105,

Querétaro - México,

barrioscuevas@prodigy.net.mx; hyguillen@msn.com

\end{datos}

Taller en bachillerato con 20 actividades lúdicas, emotivas, personales,
históricas, ecológicas, bio-energéticas, gestálticas, en completo
bienestar físico, mental y social. Donde el autoconocimiento es aplicable
al aprendizaje constructivo del cálculo y la enfermedad es el esfuerzo
que hace la naturaleza para sanar al hombre sin congruencia emocional.
Se evoluciona como persona respetando a la naturaleza, a los otros
y a sí mismo, se busca el arraigo y responsabilidad, desarrollan nuevas
conexiones neuronales, se hace uso de las funciones cerebrales superiores
en el aprendizaje cooperativo, la enseñanza es centrada en el estudiante
con intención de que matematice modelando el conocimiento. 


\section{CONSTRUCCIÓN DE PARALELOGRAMOS CON GEOGEBRA}

\begin{datos}

Leonela M. Rubio U., Angela K. Cervantes M., Juan L. Prieto G.

Grupo TEM ( Tecnologías en la Educación Matemática), Centro de Estudios
Matemáticos y Físicos (CEMAFI), Universidad del Zulia (LUZ),

Venezuela,

leonela.rubio@aprenderenred.com.ve; angela.cervantes@aprenderenred.com.ve;

juan.prieto@aprenderenred.com.ve 

\end{datos}

El propósito del taller es propiciar un ambiente de reflexión sobre
los paralelogramos, sus elementos, propiedades y relaciones, mediante
la resolución de tareas de construcción con GeoGebra. Las tareas de
construcción representan un medio para desarrollar conocimiento geométrico
por parte de los docentes, quienes elaboran procedimientos y esquemas
de acción que se apoyan en un saber integrado con las herramientas
del GeoGebra, y se transforman en instrumentos para resolver estas
situaciones. El taller se desarrolla en tres etapas, que incluyen:
diagnóstico, reflexión sobre la teoría geométrica en relación a procesos
de construcción y puesta en común de estos procesos. 


\section{LO CONCEPTUAL, INSTRUMENTAL Y DIDÁCTICO EN LA RESOLUCIÓN DE TAREAS
DE FUNCIONES CON GEOGEBRA }

\begin{datos}

Leonard A. Sánchez V, Juan Luis Prieto González.

Universidad Francisco de Miranda, Universidad del Zulia, 

Grupo TEM (Tecnologías en la Educación Matemática), CEMAFI,

Venezuela, 

leonardsanchez@gmail.com; juan.prieto@aprenderenred.com.ve

\end{datos}

Dada la naturaleza compleja del conocimiento necesario para integrar
tecnologías en el aula se propone un taller de formación para profesores
de Matemática centrado en la enseñanza de las funciones con GeoGebra,
en el cual los profesores desarrollen conocimiento conceptual, instrumental
y didáctico. En una primera etapa las tareas se abordan en ambiente
de lápiz y papel. Seguidamente, los facilitadores abordarán las mismas
tareas utilizando un recurso con GeoGebra, por último, se hará una
discusión de las potencialidades didácticas del recurso para abordarla.
Esta secuencia creará una discusión sobre el conocimiento matemático,
didáctico e instrumental necesario para enseñar funciones con tecnología. 

\setcounter{section}{10}


\section{TÉCNICAS DE INTEGRACIÓN ANALÍTICA MAS DE VEINTE MANERAS DE RESOLVER
UNA INTEGRAL DEFINIDA}

\begin{datos}

Joaquín Padovani, Erika Sabrina Estévez.

Universidad Interamericana de Puerto Rico,

Recinto de San Germán, Johns Hopkins University,

Baltimore - EE.UU. 

padovani1@hotmail.com; eesteve1@jhu.edu

\end{datos}

Las Técnicas Analíticas de Integración, constituyen el árbol temático
en el que se sustenta este Taller teórico-práctico. La semilla de
este fecundo viene siendo, la oportunidad que se le brindará a cada
una de los participantes, de revisar de una manera meticulosa, diversos
aspectos del tema. Entre estos se encuentran, las propiedades fundamentales
de la integral, el método de Hermite, el método de integración de
Leibniz-Woods- Feynman, y, naturalmente, las clásicas técnicas de
integración analítica. Los participantes dispondrán de amplias ocasiones
para familiarizarse con diversos artificios analíticos de integración,
empleados en las aulas universitarias, por exitosos educadores, en
las retropróximas cuatro décadas.


\section{ACTIVIDADES PARA EL ESTUDIO DE LA CIRCUNFERECIA UTLIZANDO TECNOLOGÍA
Y BASADAS EN EL MODELO DE VAN HIELE}

\begin{datos}

Héctor Osorio Abrego. 

Universidad Autónoma de Chiriquí,

Panamá,

hosorioa@cwpanama.net

\end{datos}

El taller consta de un conjunto de actividades de experimentación
didáctica que tienen por objeto proveer al docente participante de
ideas para desarrollar clases activas en torno al concepto de circunferencia,
donde el estudiante mediante observación, exploración, experimentación
y formulación de conjeturas pueda llegar a conocer propiedades y relaciones
entre los elementos de dicho objeto matemático, por sí mismo y con
guía del docente. Las actividades han sido diseñadas teniendo como
fundamento el modelo de razonamiento de Van Hiele y para su desarrollo
se requiere disponer de computadoras y del programa GeoGebra.


\section{NIVEL DE CONFIANZA: ¿EVOLUCIONAN LAS CONCEPCIONES? }

\begin{datos}

Ingrith Álvarez, Luisa Andrade, Felipe Fernández, Isabel Quintero.

Universidad Pedagógica Nacional,

Colombia,

ialvarez@pedagogica.edu.co; landrade@pedagogica.edu.co;

fjfernandez@pedagogica.edu.co; dma\_dquintero472@pedagogica.edu.co

\end{datos}

Producto de investigaciones, en la literatura se documentan diversas
interpretaciones erróneas en relación con el concepto de nivel de
confianza asociado a un intervalo de confianza, por lo que nos interesa
la puesta en práctica de un experimento de enseñanza con dos objetivos
primordiales, en primera instancia, reconocer cuáles son las interpretaciones
erradas que se evidencian en los estudiantes, y en segundo lugar,
propiciar a través del cuestionamiento de concepciones y de la simulación
reiterada de la construcción de intervalos, que dichas interpretaciones
evolucionen hacia una conceptualización frecuentista.


\section{VISUALIZACIÓN DE LAS FUNCIONES LINEAL Y CUADRÁTICA EMPLEANDO GEOGEBRA.
PROPUESTAS DE ENSEÑANZA}

\begin{datos}

Juan Antonio González Macías, Jesús Grajeda Rosas, Rosa María Farfán
Márquez.

Cinvestav-IPN, 

México,

jgonzalezm@cinvestav.mx;iq\_jesusgr@hotmail.com;

rfarfan@cinvestav.mx

\end{datos}

El tratamiento tradicional de las funciones polinomiales considera
parámetros de variación que resultan ambiguos al no ser observables
directamente. En el taller que presentamos se muestran propuestas
de enseñanza que van encaminadas a que el alumno visualice, manipule
y emplee sus propias herramientas en el entorno del software (GeoGebra),
tomando como referencia fenómenos físicos cotidianos. Partiendo de
la hipótesis que para acceder al pensamiento y lenguaje variacional
se requiere, entre otras cosas, del manejo de un universo de formas
gráficas extensas y ricas en significados por parte del que aprende.


\section{\uppercase{ Diseño de Situaciones de Aprendizaje para el Desarrollo
de la Competencia Matemática en alumnos de nivel medio superior.}}

\begin{datos}

Sergio Dávila Espinosa, Jonathan E. Martínez Medina.

Universidad Autónoma de San Luis Potosí, Facultad de Ciencias,

México,

sergio.davila@uaslp.mx; jonathan\_martinez@alumnos.uaslp.edu.mx

\end{datos}

Las evaluaciones estandarizadas que de la competencia matemática se
realizan a escala internacional como el proyecto PISA de la OCDE,
pueden provocar en los docentes una sensación de desánimo o culposa
cuando se analizan los magros resultados obtenidos por países como
México y Colombia. Sin embargo, poco se explora la utilidad de dicho
proyecto como impulsor de mejoras sustantivas en la enseñanza de la
matemática a través del estudio de sus marcos de referencia. 

Este taller busca compartir una metodología de diseño de situaciones
de aprendizaje mediante la focalización de los aprendizajes esperados,
el diseño de un marco hipotético pero realista que exija del alumno
la movilización de los componentes de competencia asociados a los
aprendizajes esperados, así como la elección de un producto tangible
que al tiempo que exija de los alumnos el desarrollo de aquello que
se pretende, permita al profesor su evaluación formativa. 


\section{O USO DO GEOGEBRA COMO FERRAMENTA NO ESTUDO DAS CÔNICAS}

\begin{datos}

Angelica Francisca de Araujo.

Universidade Federal do Oeste do Pará,

Brasil,

angelica.araujo@ufopa.edu.br

\end{datos}

As inovações com o uso do computador no campo educacional, fez com
que o mesmo passasse a ser uma ferramenta de complementação e aperfeiçoamento
na qualidade do ensino, criando novos ambientes de aprendizagem. O
Geogebra é um software livre, possui uma interface simples, oferece
dinamismo em seus recursos e é um ótimo facilitador das visualizações
geométricas que os professores tanto buscam. O objetivo principal
desse trabalho foi usar o Geogebra como facilitador no entendimento
das cônicas durante a disciplina de geometria analítica no curso de
licenciatura integrada em matemática e física da universidade federal
do oeste do Pará.


\section{RESOLVIENDO PROBLEMAS DE CÁLCULO DIFERENCIAL E INTEGRAL. ESTRATEGIAS
Y HERRAMIENTAS}

\begin{datos}

Giovanni Ruiz Faúndez, Andrea Silvina Seoane, Liliana Milevicich,
Alejandro Lois.

Universidad Tecnológica Nacional - Facultad Regional General Pacheco,

Argentina,

gruizfaundez@gmail.com, seoane\_andrea@yahoo.com.ar;

liliana\_milevicich@yahoo.com.ar; alelois@hotmail.com

\end{datos}

La propuesta del presente taller está dirigida a docentes que enseñan
Cálculo Diferencial e Integral en una o en dos variables en los últimos
años de escolaridad secundaria, en los institutos de formación docente
o en la universidad. El propósito es: 
\begin{description}
\item [{a)}] Trabajar con los participantes en la resolución de problemas
de cálculo diferencial e integral integrando las diferentes representaciones
(gráfica, numérica y algebraica). 
\item [{b)}] Analizar grupalmente la incorporación de los procesos de conjeturación,
experimentación, simulación y verificación, en la resolución de problemas
con herramientas informáticas. 
\item [{c)}] Comparar diferentes estrategias de resolución utilizadas por
diferentes grupos.
\end{description}

\section{INTRODUCCIÓN DE LAS APLICACIONES EN LAS CLASES DE ÁLGEBRA LINEAL}

\begin{datos}

$^{1}$Anelys Vargas Ricardo, $^{1}$Olga Lidia Pérez González, $^{2}$Ángela
Martín, $^{1}$Sandy Díaz Ramos.

$^{1}$Universidad de las Ciencias Informáticas , $^{1}$Universidad
de Camagüey, $^{2}$Universidad Autónoma de Santo Domingo, $^{1}$Universidad
de las Ciencias Informáticas,

Cuba$^{1}$, República Dominicana$^{2}$, 

\end{datos}

En los cursos de Álgebra Lineal, entre los alumnos, surgen desmotivación,
altos niveles de deserción y baja calidad en los resultados docentes
que obtienen ya que ellos no comprenden para qué les sirve esta materia.
Mostrar a sus estudiantes las aplicaciones del Álgebra Lineal desde
el comienzo de la clase es un aspecto poco abordado por algunos docentes.
En consecuencia, el objetivo de este taller es instruir a los docentes
sobre cómo introducir las aplicaciones del Álgebra Lineal en las clases
abordando problemas reales puede permitir la comprensión de los conceptos
más abstractos y aumentar la motivación de los estudiantes.

\setcounter{section}{19}


\section{RESOLUCIÓN DE PROBLEMAS Y DISEÑO DE ACTIVIDADES}

\begin{datos}

Víctor Manuel Pérez Torres, Marco Antonio Santillán Vázquez.

CCH, UNAM,

México,

victorpt07@gmail.com; santillanmarco11@gmail.com

\end{datos}

En este taller, enfocado a la formación de profesores de nivel medio
superior, aborda el diseño de actividad en el aula en función de plantear
problemas, resolverlos y analizar los procesos de resolución involucrados.
Se tratan temas centrales de resolución de problemas como: heurísticas,
estrategias meta-cognitivas, sistemas de creencias y orientaciones
del estándar resolución de problemas de la NCTM. Desde los problemas
se introducen las trayectorias hipotéticas de aprendizaje como modelo
del diseño de la actividad en el aula. 


\section{LA TÉCNICA DE LOS PALIGLOBOS EN LA CONSTRUCCIÓN DE POLIEDROS}

\begin{datos}

Myrian Luz Ricaldi Echevarria, Isabel Zoraida Torres Céspedes.

Colegio de la Inmaculada - Jesuitas, 

Colegio Peruano Norteamericano Abraham Lincoln,

Perú,

myrianluz@hotmail.com; isabeltz50@hotmail.com

\end{datos}

El taller propone la construcción de la estrella icosaédrica como
una composición de los 5 sólidos platónicos, empleando para ello paliglobos
e hilo de pescar. En el desarrollo de la actividad se analizarán las
relaciones y propiedades que se establecen entre los elementos de
los poliedros, así como la existencia de patrones matemáticos. Esta
actividad se propone para ser aplicada con estudiantes de tercero,
cuarto o quinto grado de educación secundaria. El marco teórico que
justifica la propuesta corresponde a la teoría de las situaciones
didácticas. 


\section{CONSTRUCIONES AUREAS EMPLEANDO EL CABRI}

\begin{datos}

Mariela Lilibeth Herrera.

Universidad de Carabobo,

Venezuela ,

marielalilibeth@gmail.com

\end{datos}

Este taller tiene como propósito enfatizar el uso del software como
herramienta para desarrollar estrategias visuales y de dibujo que
conllevan al progreso verbal y lógico del participante, con la finalidad
de propiciar la formulación de conjeturas y la necesidad de realizar
demostraciones formales. La propuesta se fundamentada en el desarrollo
de habilidades asociadas a los niveles de razonamiento de Van Hiele,
y como estrategias instruccionales se incluyen las fases de información,
de orientación guiada, de explicitación, y de orientación libre.


\section{LOS OCTÁNGULARES ALGUNAS ACTIVIDADES MATEMÁTICAS}

\begin{datos}

Diana Isabel Quintero Suica, Ingrith Álvarez Alfonso.

Universidad Pedagógica Nacional ,

Colombia ,

diana.isasui@gmail.com; alvarez.ingrith@gmail.com

\end{datos}

En el ámbito de la Educación Matemática en Colombia, los Lineamientos
Curriculares de Matemáticas y los Estándares Básicos de Competencias
en Matemáticas, encaminan la formación de personas matemáticamente
competentes a través del desarrollo de procesos generales inherentes
a la actividad matemática, entre ellos el proceso de modelación el
cual se pretende abordar, en este caso, a partir de la construcción
de números figurados tales como los octángulares, recreando caminos
para llegar a su expresión simbólica, pasando por la visualización
e identificación de patrones numéricos, la formulación y validación
de conjeturas y posiblemente la generalización de las mismas. 


\section{DISEÑO Y CREACIÓN DE HERRAMIENTAS DIDÁCTICAS PARA LA ENSEÑANZA DE
LA GEOMETRÍA, EN MEDIA GENERAL}

\begin{datos}

Yuraima Lilibeth Ramírez Rondón, Thais Arreaza de Castro.

Universidad Pedagógica Experimental Libertador, Instituto Pedagógico
de Caracas,

Venezuela,

yura2572@gmail.com; tarreaza@gmail.com 

\end{datos}

La matemática, y en particular la geometría, nos permiten conocer,
comprender y transformar la realidad que nos rodea, tanto de la naturaleza
como de la sociedad. Es por ello que se hace necesario que el docente
cuente con una serie de herramientas didácticas que mejoren el proceso
de su enseñanza y aprendizaje. Con este taller se pretende que los
participantes diseñen recursos didácticos que les permitan analizar
y exponer conceptos, propiedades y otros tipos de contenidos propios
de la geometría en la Educación Media General.


\section{EL RECUBRIMIENTO DE SUPERFICIES, UNA ALTERNATIVA PARA EL ESTUDIO
DE RELACIONES DE COVARIACIÓN }

\begin{datos}

Javier Triana, Ninfa Navarro José Luis Orozco Tróchez, Alfonso Palomá,
Maryory López, Vasken Stepanian Bassili, Edgar Alberto Guacaneme Suárez.

Colegio Champagnat de Bogotá,

Colombia,

javiertriana@colegiochampagnat.edu.co; joseluisorozco@colegiochampagnat.edu.co;

ninfanavarro@colegiochampagnat.edu.co; alfonsopaloma@colegiochampagnat.edu.co;

maryorylopez@colegiochampagnat.edu.co; vaskenstepanian@colegiochampagnat.edu;

guacaneme@pedagogica.edu.co

\end{datos}

En este taller se analiza una estrategia para abordar con estudiantes
de Básica Secundaria, el estudio de relaciones de covariación, al
realizar una tarea de recubrimiento de una superficie rectangular
con fichas cuadradas. Una de las variables didácticas consideradas
en la tarea es el conjunto numérico; así, inicialmente las longitudes
de los lados tanto de la superficie como de las fichas son números
enteros positivos luego se plantea la situación problema con números
racionales positivos incluido el estudio de tener recubrir una superficie
con una ficha de mayor tamaño que la superficie a recubrir y, finalmente,
cuando se incorporan valores irracionales positivos, se intenta modelar
la situación utilizando como recurso informático el programa Geogebra.


\section{REFLEXIÓN SOBRE CONOCIMIENTOS DIDACTICOS - MATEMÁTICOS EMERGENTES
DE TAREAS FORMATIVAS}

\begin{datos}

Autor: Patricia M. Konic, Juan D. Godino, Walter F. Castro, Mauro
Rivas.

Universidad Nacional de Río Cuarto, Universidad de Granada. España,
Universidad de Antioquia, Universidad de los Andes

Argentina, Colombia , Venezuela,

pkonic@gmail.com; jgodino@ugr.es;

wfcastro82@gmail.com; rmauro@ula.ve

\end{datos}

Se propone un taller con el propósito de debatir conocimientos específicos
para la enseñanza de la matemática en la formación de futuros profesores
de primaria. Se trata de un espacio para la vivencia, reflexión y
conceptualización. Se plantearán situaciones/problemas para promover
la discusión sobre el conocimiento didáctico-matemático puesto en
juego, sus limitaciones, bondades y posibles reformulaciones. En tal
sentido se propondrán tres problemas, en los que se activan diversos
objetos matemáticos, se evalúan procesos de significación y se identifican
posibles conflictos semióticos, ello en relación tanto con algunos
aspectos de la teoría de números como con el desarrollo del álgebra
temprana. 


\section{SITUACIÓN DIDÁCTICA PARA LA ENSEÑANZA DEL ALGEBRA, A TRAVÉS DE JUEGOS
DESDE SEXTO AÑO BÁSICO A SEGUNDO AÑO MEDIO.}

\begin{datos}

Yohana Swears Pozo.

Universidad Iberoamericana de Ciencias y Tecnología,

Chile

\end{datos}

El Álgebra se concibe como una poderosa herramienta para expresar
resultados. Analizando la información respecto al proceso de enseñanza
y aprendizaje del álgebra inicial, podemos decir que existe un grave
problema por parte de los estudiantes cómo también de los profesores
en el álgebra. Los estudiantes aprenden sin una significación de lo
aprendido, por otro lado existe una problemática por parte de los
profesores realizan actividades en que él estudiante se ve enfrentado
a una mecanización por parte de ellos .Este taller propone enseñanza
de álgebra a través del diseño de una situación didáctica con la utilización
de juegos algebraicos. 


\section{SIMULACIÓN DE DINÁMICAS DE INTERACCIÓN DE GRUPOS DE ORGANISMOS, UN
ACERCAMIENTO DESDE LA COMPLEJIDAD}

\begin{datos}

Jesús Enrique Hernández Zavaleta{*}, Vicente Carrión Velázquez{*},
Jaime Arrieta Vera{*}{*}.

CINVESTAV{*}, Universidad Autónoma de Guereero{*}{*}, 

México,

jherza@gmail.com; vcarrionv@cinvestav.mx;

jaime.arrieta@gmail.com

\end{datos}

La mirada de los fenómenos, a través del paradigma de sistemas complejos,
obliga a la emergencia o modificación de técnicas que permitan su
entendimiento e intervencion. En el taller se pretende que los asistentes
simulen dinámicas de interacción de un grupo de organismos utilizando
el Software NetLogo. Al manipular los parámetros de un modelo referente
a la dinámica de interacción de grupos de organismos se discuten nociones,
intrínsecas en la dinámica, como autoorganización, caos,emergencia
de patrones, estabilidad y la noción de descentralización. Éstas son
promotoras del principio de estabilidad del cambio, constructo que
empieza a estudiarse en la socioepistemología. 


\section{PLANOS, CILINDROS E QUÁDRICAS – UM ENFOQUE NO TRAÇADO DE GRÁFICOS
COM EXPLORAÇÃO DAS SEÇÕES TRANSVERSAIS NO WINPLOT}

\begin{datos}

Janine Freitas Mota Universidade Estadual de Montes Claros, João Bosco
Laudares, Dimas Felipe de Miranda, Saulo Furletti.

Pontifícia Universidade Católica de Minas Gerais – PUC-Minas , Instituto
Federal de Minas Gerais – IFMG , Fundação Educacional Montes Claros
Pontifícia, Universidade Católica de Minas Gerais – PUC-Minas

janinemota@gmail.com; dimasfm48@yahoo.com.br;

saulofurletti@yahoo.com.br; jblaudares@terra.com.br

\end{datos}

Este artigo apresenta resultados de uma Pesquisa de Mestrado que teve
como produto a edição de um Livro, que apresenta uma metodologia diferenciada
para o estudo das superfícies tridimensionais Planos, Cilindros e
Quádricas com a utilização do software Winplot. O estudante apresenta
dificuldades na visualização geométrica de figuras tridimensionais.
A metodologia utilizada contemplou os parâmetros da sequência didática,
com exploração da habilidade de visualização, utilizando seções transversais
e curvas de níveis, com melhor visualização. O livro editado tem ampla
ilustração dos traçados das figuras, a partir de seções transversais,
sem oferecer a figura pronta e possibilita uma aprendizagem mais significativa.


\section{LA COORDINACIÓN DE LAS PRINCIPALES TEORÍAS DE EDUCACIÓN MATEMÁTICA
EN EL AULA DE MATEMÁTICAS.}

\begin{datos}

Zenón Eulogio Morales Martínez. Pontificia Universidad Católica del
Perú – PUCP, Instituto de Investigación en Enseñanza de las Matemáticas
– IREM-PUCP, Institución Educativa Particular Agroestudio,

Perú.

\end{datos}

Analizaremos la influencia de la coordinación las principales Teorías
de la Educación Matemática sobre los procesos de enseñanza y aprendizaje
de las Matemáticas. Entre estas: la Teoría del Aprendizaje Significativo,
Ausubel (1983), la Teoría de la Transposición Didáctica, Chevallard
(1998) y la Teoría de las Representaciones Semióticas, Duval (1995)
y cómo influye la coordinación entre ellas. Realizaremos actividades
sobre temas de álgebra y de geometría. Se espera que la coordinación
de estas teorías permita el éxito en el aprendizaje, que se verá reflejado
cuando los alumnos hayan logrado desarrollar las principales capacidades
matemáticas del nivel escolar. 


\section{GEOGEBRA COMO HERRAMIENTA DE APOYO VISUAL EN LA SOLUCIÓN DE PROBLEMAS
DE MODELACIÓN EN MATEMÁTICA ESCOLAR}

\begin{datos}

Francisco Javier Córdoba Gómez, Elkin Alberto Castrillón Jiménez,
Carlos Alberto Rojas Hincapié.

Instituto Tecnológico Metropolitano,

Colombia,

fjcordob@yahoo.es; franciscocordoba@itm.edu.co; 

elkincastrillon@itm.edu.co; carojas72@gmail.com

\end{datos}

En el aprendizaje y la enseñanza de las matemáticas se presenta a
veces una dificultad para modelar matemáticamente determinados problemas
o situaciones puesto que a veces no se dispone de un medio adecuado
que permita una correcta representación y visualización del problema
mismo. Esta representación se puede convertir en un primer paso para
llegar a la formulación matemática del problema y su solución. GeoGebra
se presenta entonces como una poderosa herramienta que permite ese
primer acercamiento visual al problema. En este taller se pretende
entonces abordar algunos problemas matemáticos desde la visualización.


\section{NÚMEROS, SUCESIONES, ITERACIÓN DE FUNCIONES, FRACTALES Y OTROS CONCEPTOS
DEL CÁLCULO DIFERENCIAL E INTEGRAL.}

\begin{datos}

Luis Manuel Hernández Gallardo. 

Facultad de Ciencias de la UNAM,

México,

lmhg@fciencias.unam.mx

\end{datos}

En este taller se plantearán y analizarán problemas que con ciertas
peculiariedades interesantes cada uno de ellos, nos llevan en forma
casi inmediata a algunos de los conceptos e ideas fundamentales del
Cálculo Diferencial e Integral de una variable real. Con estos problemas
se busca introducir, generar y discutir en forma menos abstracta algunas
de las ideas y conceptos fundamentales del Cálculo, que cuando se
les presentan a los alumnos por primera vez, son difíciles de asimilar:
en este caso los conceptos de límite, derivada e integral indefinida
de una función. 


\section{EXPERIENCIAS GEOMÉTRICAS CON MOSAICOS}

\begin{datos}

Ana María Redolfi Gandulfo.

Universidade de Brasília - UnB,

Brasil,

gandulfo@uol.com.br

\end{datos}

En este taller estudiamos los mosaicos, sus elementos, clasificaciones
y propiedades, aplicando conceptos geométricos básicos y las simetrías
de las figuras planas. Caracterizamos y construimos los mosaicos unicelulares,
lado a lado, periódicos, regulares, semirregulares, irregulares y
los mosaicos no-periódicos, destacando sus propiedades. También analizamos
las obras de M. C. Escher y los mosaicos en arte y arquitectura, identificando
sus elementos y composiciones. Los temas son tratados mediante la
realización de experiencias, de observaciones y de resolución de problemas.
Todos los recursos didácticos necesarios y las soluciones de las actividades
serán colocados a disposición de los participantes.


\section{SI SE USA GEOMETRÍA DINÁMICA ENTONCES SE COMPRENDE LAS PROPOSICIONES
CONDICIONALES }

\begin{datos}

Nabil Ortegón Domínguez, Guillermo Salas Rodríguez, Carmen Samper.

Universidad Pedagógica Nacional,

Colombia,

nabilortegon@hotmail.com; gsalas@pedagogica.edu.co;

csamper@pedagogica.edu.co

\end{datos}

A través de una serie de tareas para desarrollar con un software de
geometría dinámica, buscamos propiciar la comprensión de lo que es
y lo que expresa una condicional en matemáticas. Por medio de problemas
propuestos, en los cuales se debe formular una conjetura como resultado
de la exploración realizada y la determinación de invariantes, se
busca que los participantes del taller comprendan que las condiciones
establecidas en el antecedente son suficientes para concluir el consecuente
y que el consecuente es necesariamente resultado de las condiciones
que se reportan en el antecedente.

\setcounter{section}{46}


\section{CONSTRUCCIÓN DE ESTRATEGIAS PARA LA ENSEÑANZA DE LA GEOMETRÍA Y ESTRUCTURACIÓN
DE SECUENCIAS DIDÁCTICAS.}

\begin{datos}

Martínez Carmen Ysabel, Meriño Victor Hugo, Vílchez Báez Ángel.

Universidad Nacional Experimental Rafael María Baralt, Universidad
del Zulia,

Venezuela,

\end{datos}

Saber cómo debe enseñarse el Contenido objeto de estudio es uno de
los elementos que constituyen el Conocimiento Didáctico del Contenido.
Este taller busca presentar elementos de geometría, como saber por
enseñar, asociándolos a programas oficiales de educación básica y
media, y considerando los factores ambientales y sociales del contexto,
se construyen estrategias de enseñanza adecuadas a la situación. Se
estructura la secuencia didáctica y se desarrolla una de las actividades
planificadas. Este encuentro brinda a los docentes la posibilidad
de intercambiar opiniones, visualizar posibilidades y oportunidades
de enseñar geometría y develar su dominio de las matemáticas. 


\section{MIRADAS PARA TRABAJAR LA COMPLEJIDAD DEL AULA MATEMÁTICA }

\begin{datos}

Jorge Ávila Contreras, Leonora Díaz Moreno; Oswaldo Martínez Padrón.

Universidad Católica Silva Henríquez, Universidad de Valparaíso, Universidad
Pedagógica Experimental Libertador,

Chile, Venezuela,

\end{datos}

Formación de profesores; Medio Superior y Superior; Empírico

Resumen Mediante el presente taller se busca compartir y discutir
con los asistentes cómo es que nos adentramos a conocer y trabajar
aspectos humanos que transversalizan el aula matemática, tales como
actitudes, creencias, emociones. Entre otros se explicarán e ilustrarán
el uso de Bitácoras de Reflexión Estudiantil y el cómo propiciar la
instalación de una práctica de consultar y explorar cómo se sienten
y aprenden los estudiantes cuando se involucran en actividades matemáticas
en clases. Se espera contribuir a afinar la acción pedagógica matemática
en el aula a fin de favorecer la generación de mayores y mejores aprendizajes. 


\section{\uppercase{ El rol de los gestos en el pensamiento algebraico, multiplicativo
y aditivo}}

\begin{datos}

John Gómez Triana, Anderson Javier Mojica Vargas, Oscar Leonardo Pantano
Mogollón, Rodolfo Vergel Causado.

Secretaría de Educación Distrital , Universidad Distrital Francisco
José de Caldas , Universidad Pedagógica Nacional , Universidad Distrital
Francisco José de Caldas,

Bogotá-Colombia,

johngomezt@gmail.com; javiermojicav@hotmail.com;

leonardopantanom@gmail.com; rodolfovergel@gmail.com

\end{datos}

Se propone un reporte de investigación que se enmarca en la perspectiva
semiótica cultural de la enseñanza y aprendizaje de las matemáticas
propuesta en la Teoría Cultural de la Objetivación. El objetivo es
reportar una evidencia de cómo el sentido de la indeterminancia, la
analiticidad y la expresión simbólica se hacen evidentes en la actividad
matemática de los estudiantes cuando resuelven tareas en contextos
algebraicos. Para tal fin se muestra un análisis multimodal del pensamiento
y de la actividad reconociendo la importancia de los recursos cognitivos,
físicos y perceptuales que los estudiantes utilizan cuando trabajan
con ideas matemáticas. 


\section{TALLER DE RESOLUCIÓN DE PROBLEMAS DE MATEMATICA CON IDEAS DEL PENSAMIENTO
CRITICO PARA PROFESORES DE EDUCACION MEDIA }

\begin{datos}

Claudia Vargas.

Universidad del Bío-Bío, 

Chile,

cvargas@ubiobio.cl

\end{datos}

El cursillo de dos sesiones será acerca de la resolución de problemas
como activadora del pensamiento matemático el cual puede ser fomentado
en los alumnos de educación secundaria. Los temas a desarrollar serán:
Qué es la resolución de problemas y por qué trabajarla en la educación
matemática. Dos Metodologías de Resolución de Problemas de Matemática:
Las semejanzas del método de Polya y las ideas para el desarrollo
del Pensamiento Crítico. Principales dificultades en la resolución
de problemas de matemática.

En ambas sesiones se trabajará con ejemplos de problemas de matemática
que los profesores participantes resolverán en grupo y también de
manera individual.

\setcounter{section}{55}


\section{EPIINFO. DISEÑO DE CUESTIONARIOS PARA EL ANÁLISIS DE DATOS.}

\begin{datos}

Elisa Mendoza González.

Universidad de Panamá, 

República de Panamá,

emendoza2729@gmail.com

\end{datos}

En el proceso de enseñanza aprendizaje es importante contar con herramientas
tecnológicas accesibles que permitan completar el proceso metodológico.
En los cursos de metodología de investigación, estadística descriptiva
o bioestadística, el programa Epiinfo es un programa útil y de fácil
aplicación. El taller se realiza de manera práctica en tres fases,
la primera, corresponderá al diseño del formulario, se distinguen
los tipos de campos: numéricos, textos, fechas, horas, lógicas; en
la segunda fase, realizan la captura de datos, y en la tercera, Analizar,
para generar Estadísticas descriptivas, Tablas, Frecuencias, Gráficas,
otros. El programa puede utilizarse en cualquier campo del saber.


\section{ELEMENTOS DE WX-MÁXIMA EN EL AULA DE MATEMÁTICAS }

\begin{datos}

Lucero Antolínez Quijano, Yetza Ximena Díaz Pinzón.

Universidad Pedagógica Nacional, Universidad de Boyacá,

Colombia,

luceroaq@gmail.com; yetzad@gmail.com

\end{datos}

En “Principios y estándares para la educación matemática” publicación
de Nacional Council of Teachers of Mathematics, el principio de la
tec­nología, se considera fundamental, influyente y enriquecedor,
en la enseñanza y aprendizaje de las matemáticas, indicando que la
existencia, versatilidad y potencia de la tecnolo­gía llaman a reexaminar
qué matemáticas deberían aprender los alumnos y cómo hacerlo mejor.
Este trabajo mostrará algunas características y bondades de WX- Máxima,
un sistema de álgebra computacional de código libre, que facilita
manipular expresiones simbólicas y numéricas, y cuyo entorno gráfico
ofrece representaciones en dos y tres dimensiones con resultados de
alta precisión. 


\section{INTRODUCCIÓN DE LAS APLICACIONES EN LAS CLASES DE ÁLGEBRA LINEAL}

\begin{datos}

Anelys Vargas Ricardo, Olga Lidia Pérez González, Ángela Martín, Sandy
Díaz Ramos.

$^{1}$Universidad de las Ciencias Informáticas, $^{1}$Universidad
de Camagüey , $^{2}$Universidad Autónoma de Santo Domingo, $^{1}$Universidad
de las Ciencias Informáticas,

$^{1}$Cuba, $^{2}$República Dominicana, 

anelys@uci.cu; olga.perez@reduc.edu.cu;

m.angela24@gmail.com; sdiaz@uci.cu 

\end{datos}

En los cursos de Álgebra Lineal, entre los alumnos, surgen desmotivación,
altos niveles de deserción y baja calidad en los resultados docentes
que obtienen ya que ellos no comprenden para qué les sirve esta materia.
Mostrar a sus estudiantes las aplicaciones del Álgebra Lineal desde
el comienzo de la clase es un aspecto poco abordado por algunos docentes.
En consecuencia, el objetivo de este taller es instruir a los docentes
sobre cómo introducir las aplicaciones del Álgebra Lineal en las clases
abordando problemas reales puede permitir la comprensión de los conceptos
más abstractos y aumentar la motivación de los estudiantes.

\setcounter{section}{61}


\section{SIMULANDO PRÁCTICAS SOCIALES CON GEOGEBRA. EL CASO DE UNA GRANJA
DE CAMARONES, PARA DESARROLLAR COMPETENCIAS EN LOS ESTUDIANTES DE
NIVEL MEDIO SUPERIOR}

\begin{datos}

Noé Camacho Calderón, Silvino Bailón Cortez, Margarito Gódinez de
Dios, Santiago Ramiro Velázquez Bustamante.

Colegio de Bachilleres del Estado de Guerrero - El Coacoyul - Zihuatanejo
Gro.

México,

noeilyn\_21@hotmail.com; margodios@yahoo.com.mx;

sramiro@prodigy.net.mx

\end{datos}

El desarrollo de competencias en los estudiantes depende principalmente
de dos factores, la sociedad y los profesores, el profesor tiene una
ardua labor como actor intelectual en el proceso enseñanza aprendizaje
de los conceptos aprendidos dentro del aula en general. Sin embargo,
dentro del ámbito educativo como tal se carece de ofertas de capacitado,
en la práctica se constata que los profesores carecen de herramientas
didácticas para desarrollar las competencias que la RIEMS propone,
el objetivo del taller propuesto es precisamente en este sentido,
capacitar a estudiantes y profesores en la modelación y simulación
de practicas sociales, en este caso la granja de camarones por medio
del GeoGebra y de esta manera constrir algunas herramientas. 

\setcounter{section}{63}


\section{PRODUCCIÓN DE RECURSOS DIDÁCTICOS PARA ENSEÑAR MATEMÁTICA EN LA ESCUELA
SECUNDARIA}

\begin{datos}

Gladys Brunini$^{1,2,3}$, Virginia Ciccioli$^{1,4}$, Eliana Dominguez$^{1,5,6}$,
Natalia Sgreccia$^{1,7}$.

$^{1}$FCEIA-UNR, $^{2}$EESOPI 2037, $^{3}$EETP 656, $^{4}$ENS
33, $^{5}$EESOPA 8240, $^{6}$EESOPI 8270, $^{7}$CONICET,

Argentina,

gladys232003@hotmail.com; vickyc\_03@hotmail.com;

elianadominguez7@hotmail.com; nataliasgreccia@hotmail.com

\end{datos}

Este taller tiene como propósito presentar algunos recursos didácticos
(tangram chino, geoplano cuadrangular y circular, ludo matemático
y dominó matemático), que fueron elaborados por los docentes del curso,
para enseñar variados contenidos matemáticos del secundario básico.
Se analizarán ventajas y desventajas en su diseño, se compartirán
los instructivos de elaboración, se construirán algunos y se identificarán
habilidades matemáticas que su uso involucra, a través de secuencias
didácticas especialmente diseñadas. Esto aporta a reconocer el potencial
didáctico de los recursos y a propiciar una utilización fundamentada
de los mismos a partir de situaciones áulicas concretas que se recrearán
en conjunto. 


\section{RESOLUCIÓN DE PROBLEMAS A TRAVÉS DE UNA EXPLORACIÓN DIGITAL}

\begin{datos}

Eduardo Basurto Hidalgo, Eduardo Mancera Martínez Benemérita.

Escuela Nacional de Maestros, v

México,

basurtomat@hotmail.com; mancera.eduardo@gmail.com

\end{datos}

El uso de tecnologías digitales a favor de la enseñanza de las matemáticas
ofrece hoy en día una amplia gama de dispositivos y software, así
como una diversidad de modos de uso de la misma, el presente taller
pretende mostrar a los docentes e investigadores interesados en este
tipo de herramientas, una alternativa metodológica diseñada no sólo
en el ámbito ejecutor de algoritmos, sino también con prestaciones
que permiten realizar orquestaciones didácticas que ofrezcan la posibilidad
de explorar problemas desde diversas perspectivas que aporten sentido
a la actividad matemática y ayuden en la creación de significados
de los objetos matemáticos.


\section{ANÁLISIS EPISTÉMICO DE LAS LECCIONES DE FRACCIONES EN LOS LIBROS
DE TEXTOS DE SEXTO GRADO}

\begin{datos}

Johanna Franzone, Delisa Bencomo.

Universidad Nacional Experimental de Guayana,

Venezuela ,

jfranzone0113@gmail.com;dbencomo@uneg.edu.ve

\end{datos}

Con este taller se busca que el participante conozca las bondades
de las herramientas teóricas-metodológicas del Enfoque Ontosemiótico
de la cognición y de la instrucción matemática (EOS) propuesto por
Godino y Colaboradores, para la reconstrucción del significado de
fracciones presente en las lecciones del los libros de texto de matemática. 


\section{\uppercase{ Solución de problemas con fundamentos algebraicos y
apoyo de software matemático }}

\begin{datos}

José Isaac Sánchez Guerra, Carlos Oropeza Legorreta, Carlos Oropeza
Ugalde.

Facultad de Estudios Superiores Cuautitlán - UNAM., CBT Gabriel V.
Alcocer,

México,

joejade@hotmail.com; coropeza96@hotmail.com;

lambo.r.gini@hotmail.com. 

\end{datos}

En el presente taller se propone el uso de software matemático como
instrumento verificador de los resultados que se obtienen a partir
del análisis de problemas relativos al álgebra con otras asignaturas
en la formación de ingenieros, impulsando a la vez el tránsito entre
registros de representación. El estudio que se propone es de naturaleza
cualitativa, pues en él se tiene la intención de encontrar rasgos
de corte analítico y geométrico que permitan la identificación de
cada una de las categorías en estudio. Se proponen actividades centradas
en la resolución de problemas y se analizan algunas de las posibles
soluciones.


\section{TALLER: MAS ALLÁ DE LOS “TACHES” Y “PALOMITAS”}

\begin{datos}

Asela Carlón Monroy, Sergio Cruz Contreras,

UNAM - FES-ACATLÁN, 

MÉXICO,

correoaselasergio@gmail.com; asela.carlon@gmail.com

\end{datos}

La interpretación en el proceso comunicativo, que tiene lugar en una
clase de matemáticas, es fundamental. Recordando elementos teóricos
provenientes de esquemas conceptuales generales como son el trabajo
de Piaget, Vygotsky, Ausubel, y de formulaciones específicas de la
Educación Matemática, como los aportes de los van Hiele, Schoenfeld,
Lesh, Hiebert y Carpenter, así como nociones generales de hermenéutica,
se analizan e interpretan producciones de estudiantes de bachillerato,
con el propósito de poner de manifiesto el papel que tienen los conocimientos
teóricos cuando el profesor intenta interpretar las producciones de
los estudiantes, en particular los errores, en la clase de matemáticas.


\section{SIMULAR DESDE GRÁFICAS DIFERENCIALES: EL ROL DE LA GESTUALIDAD.}

\begin{datos}

Eduardo Carrasco Henríquez, Juan Felipe Flores Robles.

Universidad Autónoma de Guerrero, Universidad Austral, 

México, Chile Juan,

F10res@hotmail.com; ecarrasc@gmail.com

\end{datos}

La socioepistemológico señala que las actividades intencionadas desde
una práctica social son centrales para el logro de entendimientos.
Así la simulación, como práctica imbricada con la modelación que se
incorpora en los curriculum escolares, Al ser ejercida tiene como
recurso central a la gestualidad en la actividad de conocer. Gestualidad
que asociada a una idea conforma, al actuar, un espacio gestual matemático.
El taller propone una discusión sobre las prácticas de simulación
y los recursos estudiantiles al ejercerla, en particular la gestualidad,
explorando las estrategias y herramientas de indagación para evidenciar
el rol de este recurso en la actividad estudiantil.


\section{\uppercase{ Usos de las gráficas y análisis de la variación en la
derivada. Una visión Socioepistemologíca}}

\begin{datos}

Mario Adrián Caballero Pérez, Claudio Enrique Opazo Arrellano.

Cinvestav, 

México D.F. 

mcaballero@cinvestav.mx; opazoferrari\_claudio@hotmail.com 

\end{datos}

Desde una postura Socioepistemológica, el discurso matemático escolar
ha soslayado la naturaleza variacional del Cálculo, relegando su estudio
a un aprendizaje basado en la memorización de formulas y algoritmos.
Nuestro taller propone dar una mirada diferente, resaltando este carácter
variacional mediante el análisis de los diferentes tipos de variaciones
y usos que las gráficas expresan. El taller se encuentra dirigido
tanto a profesores que busquen formas alternativas de encarar el estudio
de las gráficas, como a investigadores en formación que deseen ahondar
en una corriente teórica relativo al estudio de las gráficas y la
variación.


\section{MATEMÁTICAS FINANCIERAS UTILIZANDO EXCEL}

\begin{datos}

Julio Moisés Sánchez Barrera.

Facultad de Estudios Superiores Cuautitlán UNAM,

México,

juliomoisessb@yahoo.com.mx

\end{datos}

Capacitación para el trabajo, Superior y Tecnología avanzada.

Este taller pretende que el asistente conozca y familiarice con los
conceptos y modelos matemáticos financieros más comunes haciendo uso
del software Excel, el asistente podrá constatar como con el uso de
las nuevas tecnologías el cálculo es más preciso y rápido no importando
la cantidad de periodos utilizados, denotando que la base de las matemáticas
financieras esta en las progresiones aritméticas y geométricas. 

Por su aplicación estas matemáticas financieras deben de ser conocidas
por todo estudiante en su formación profesional. En este taller se
desarrollará la transposición didáctica del saber sabio al saber enseñado
aplicado a la vida real.


\section{\uppercase{ Del discurso a la práctica: construyendo equidad de
género en el aula de Matemáticas}}

\begin{datos}

Claudia Rodríguez Muñoz.

Centro de investigación y de Estudios Avanzados del IPN,

México,

Claurom65@yahoo.com

\end{datos}

El taller se formula a partir de los resultados de una investigación
longitudinal sobre las estudiantes de educación básica y la matemática
escolar en México. La literatura actual da cuenta de las diferencias
entre hombres y mujeres al estudiar matemáticas en distintos niveles
educativos y en diversos ambientes socioculturales. Pasar de la investigación
a la implantación de acciones afirmativas permitirá que las y los
docentes participantes construyan herramientas necesarias para que,
desde un enfoque de equidad entre los géneros, sugieran y generen
formas alternativas y creativas para eliminar el sexismo y los estereotipos
de género en el aula de matemáticas.


\section{UN ACERCAMIENTO AL CONCEPTO DE FUNCIÓN A TRAVÉS DE LA MANIPULACIÓN
DE OBJETOS GEOMÉTRICOS, DONDE SE PRESENTAN PATRONES FUNCIONALES DE
DEPENDENCIA Y DE GENERALIZACIÓN, UTILIZANDO UN SOFTWARE DE GEOMETRÍA
DINÁMICA”}

\begin{datos}

JOSE FRANCISCO PUERTO MONTERROZA.

Institución Educativa Docentes de Turbaco,

Turbaco-Bolívar - Colombia,

jopuermon@gmail.com

\end{datos}

Con este taller se pretende mostrar como con el uso de un software
se puede favorecer un acercamiento significativo al concepto de función
a través de actividades funcionales y de generalización y una movilidad
por los diferentes sistemas de representación (verbal, tabular, gráfico
y algebraico). Se establece una relación funcional de dependencia
entre una variable inicial (segmento) y una variable final (tamaño,
perímetro o área) de un cuadrado, posibilitándose así la visualización
y el reconocimiento de patrones de variación y cambio entre magnitudes.
Se hace un registro tabular (relación de dependencia cuantitativa)
y un registro gráfico (relación de dependencia cualitativa) lo que
facilitara la identificación y caracterización de un modelo funcional
y su correspondiente expresión algebraica. 


\section{DISEÑO DE UN PIZARRON DIGITAL PARA LA CLASE DE MATEMÁTICAS}

\begin{datos}

José Lorenzo Sánchez Alavez.

Centro de Investigación y Estudios Avanzados del IPN,

México,

klorenzk@ciencias.unam.mx

\end{datos}

En este taller se diseñará una superficie interactiva con el potencial
didáctico de un pizarrón digital, que permita al docente generar situaciones
interactivas de aprendizaje bajo un enfoque metodológico de resolución
de problemas. La propuesta se fundamenta en el enfoque sociocultural
de la acción mediada, a partir de las ideas desarrolladas por Vygotsky,
Brunner, Luis Moreno-Armella y James Wertsch, entre otros; y se conjuga
con los recursos tecnológicos al alcance inmediato del docente. Finalmente,
se mostrarán diseños didácticos, usando distintos softwares disponibles
en la red, como Geogebra, Poly 32, Sankoré, BuilAR y TI Nspire CX.
cutidos. 

