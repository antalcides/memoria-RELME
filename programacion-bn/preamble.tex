\usepackage{marco}
\usepackage{lettrine}
%\usepackage{keyval}% http://ctan.org/pkg/keyval
%\usepackage{environ}% http://ctan.org/pkg/environ
\usetikzlibrary{calc,trees,positioning,arrows,chains,shapes.geometric,
decorations.pathreplacing,decorations.pathmorphing,shapes,%
matrix,shapes.symbols,plotmarks,decorations.markings,shadows}
\usepackage{theoremref} %refereciar teoremas Por ejemplo: \thlabel{foobar}
\usepackage{numprint}
\usepackage{marginnote}
\usepackage{fancybox}
\usepackage{hhline}
\usepackage{multirow} 
\usepackage{colortbl}%
\usepackage{pifont} 
\usepackage{eurosym} %para el Euro
%\usepackage{xltxtra}%para logos de la familia TeX
%\usepackage{mathspec}
\usepackage{metalogo}
%\usepackage{tabular}
%\usepackage{amsmath}
\usepackage{lipsum-es}
%\usepackage{xparse}% para declarar comandos
\usepackage{amssymb}
%\usepackage[thref,thmmarks,framed, amsthm]{ntheorem}
\usepackage{comment}
\usepackage[explicit]{titlesec}
\usepackage{emptypage}%pagina en blanco al final de capitulo
%%%%%%%%%%%%%%%%%%%%%%%%%%%%%%
%\documentclass[openany,svgnames,x11names]{book}
\usepackage{titletoc}
\usepackage{fancyhdr}
\usepackage{pagecolor}
\usepackage[spanish]{layout}
\usepackage{ucs}%codificacion vieja
\usepackage[utf8x]{inputenc}
%\usepackage[latin1]{inputenc}
%\usepackage{showframe}% traza layout en cada pagina
%%%======================================revisar
\usepackage{pifont}
\usetikzlibrary{shapes,snakes,positioning}
\pgfdeclarelayer{background}
\pgfdeclarelayer{foreground}
\pgfsetlayers{background,main,foreground}
\usepackage{wallpaper}
\usetikzlibrary{calc}
\usetikzlibrary{arrows}
\usepackage{epstopdf}
\usepackage{floatflt}
\usepackage{pgfplots}
\usepackage{setspace}
\usepackage{amsmath,amssymb}
\usepackage{float}
%\usepackage{booktabs}
\usepackage{courier}
\usepackage{units}
\usepackage{url}
\usepackage{float}
\usepackage{mathpazo}
\usepackage{amsfonts}
\usepackage{fancyvrb}
\usepackage{enumerate}
\usepackage{ifthen}
\usepackage{cancel}
\usepackage{layout}
\usepackage{footnote}
\usepackage{etex}%si pgfplot tiene problemas con new dim
\usepackage[frame,letter,cam]{crop}
\usepackage{microtype,soul,filecontents}
\usepackage{bbding}
\usepackage{lettrine,caption,multicol}
\usepackage{soul}
\usepackage{palatino}
\usepackage{calligra}
\usepackage[T1]{fontenc}
\usepackage[listings,theorems]{tcolorbox}
\usepackage{filecontents,ragged2e}
\usepackage{floatflt}
\usepackage[makeindex]{imakeidx}
\usepackage{lmodern}
\usepackage{etoolbox}
\usepackage{tabularx}
% \usepackage{flushend} % balance de las columnas con twocolumn
%\usepackage{minitoc}
\usepackage{etoc}
%%%%%%%%%%%%%%%%%%%%%%%%%%%%%
\usepackage[paperheight=27.9cm,%
paperwidth=20.6cm,%
%centering,%
%textheight=26.9cm,
left=2.5cm,%
right=2.5cm,%
top=2.5cm,%
bottom=2cm,%
headheight=1cm,%
headsep=20pt,%
footskip=1cm,%
marginparsep=20pt,%
pdftex=false,%
letterpaper%
]{geometry}
%\usepackage[paperheight=25cm,%
%paperwidth=17cm,%
%centering,%
%left=1.5cm,%
%right=2cm,%
%top=2.5cm,%
%bottom=1.5cm,%
%headheight=0.5cm,%
%headsep=10pt,%
%%footskip=1cm,%
%marginparsep=20pt,
%margin=2cm,
%pdftex=false
%]{geometry}
%\usepackage[frame,center,letter,pdflatex]{crop}
%\usepackage{amsthm}
%\usepackage[framed, amsthm]{ntheorem}
\usepackage{listings}
\definecolor{lightgrey}{rgb}{0.9,0.9,0.9}
\definecolor{darkgreen}{rgb}{0,0.6,0}
\usepackage{fourier-orns}
% % % % % % % % % % % % % % % % % % % % % % % % % % %
% % % %preamble de framed % % %
\usepackage{fixltx2e}
%\usepackage{etex}
\usepackage{lmodern}
\usepackage{textcomp}
\usepackage{array}
\usepackage{booktabs}
\usepackage{microtype}
\newcommand*{\mail}[1]{\href{mailto:#1}{\texttt{#1}}}
\newcommand*{\pkg}[1]{\textsf{#1}}
\newcommand*{\cs}[1]{\texttt{\textbackslash#1}}
\makeatletter
\newcommand*{\cmd}[1]{\cs{\expandafter\@gobble\string#1}}
\makeatother
\newcommand*{\env}[1]{\texttt{#1}}
\newcommand*{\opt}[1]{\texttt{#1}}
\newcommand*{\meta}[1]{\textlangle\textsl{#1}\textrangle}
\newcommand*{\marg}[1]{\texttt{\{}\meta{#1}\texttt{\}}}

% % % % % % % % % % % % % % % % %framed
\def\indexname{\'Indice}
\def\contentsname{ CONTENIDO}
\def\listfigurename{Tabla de figuras}
\def\bibname{Bibliograf\'{\i}a}
\def\tablename{Tabla}
\def\proofname{Demostraci\'on}
\def\appendixname{Ap\'endice}
\def\chaptername{ Cap\'{\i}tulo}
\def\figurename{Figura}
%%%%%%%%%%%%%%%%%%%%%%%%%%%%%%%%% definicion de colores ================)
\definecolor{est1}{RGB}{0,177,235}
\definecolor{est2}{RGB}{0,119,158}
\definecolor{est3}{RGB}{235,137,0}
\definecolor{est4}{RGB}{158,66,0}
\definecolor{est5}{RGB}{20,20,20}
\definecolor{est6}{RGB}{235,235,235}
\definecolor{naranja1}{rgb}{1,0.5,0}
\definecolor{naranja2}{RGB}{255,127,0}
\definecolor{naranja3}{cmyk}{0,0.5,1,0}
\definecolor{naranja4}{HTML}{FF7F00}
%rgb
\definecolor{rojo}{rgb}{1,0,0}
\definecolor{verde}{rgb}{0,1,0}
\definecolor{azul}{rgb}{0,0,1}
%cmyk
\definecolor{blanco}{cmyk}{0,0,0,0}
\definecolor{cian}{cmyk}{1,0,0,0}
\definecolor{magenta}{cmyk}{0,1,0,0}
\definecolor{amarillo}{cmyk}{0,0,1,0}
\definecolor{negro}{cmyk}{0,0,0,1}
\definecolor{theblue}{rgb}{0.02,0.04,0.48}
\definecolor{thered}{rgb}{0.65,0.04,0.07}
\definecolor{thegreen}{rgb}{0.06,0.44,0.08}
\definecolor{thegrey}{gray}{0.5}
\definecolor{theshade}{gray}{0.94}
\definecolor{theframe}{gray}{0.75}
\definecolor{burl}{rgb}{0.27,0.22,0.20}
\definecolor{caper}{rgb}{0.36,0.46,0.23}
\definecolor{rhodo}{rgb}{0.58,0.63,0.45}
\definecolor{wood}{rgb}{0.61,0.51,0.43}
\definecolor{mesh}{rgb}{0.97,0.93,0.81}
\definecolor{wood}{rgb}{0.61,0.51,0.43}
\definecolor{warningColor}{named}{Red3}
\definecolor{doc}{RGB}{0,60,110}
\definecolor{boxheadcol}{gray}{.6}
\definecolor{boxcol}{gray}{.9}
\definecolor[named]{PowderBlue}{HTML}{B0E0E6}
\definecolor[named]{MidnightBlue}{HTML}{191970}
\definecolor{bl}{rgb}{0,0.2,0.8}
\definecolor{shcolor}{HTML}{FDEDD0}
\definecolor[named]{GreenTea}{HTML}{CAE8A2}
\definecolor[named]{MilkTea}{HTML}{C5A16F}
\definecolor[named]{SaddleBrown}{HTML}{8B4513}
\definecolor{FrameColor}{rgb}{0.25,0.25,1.0}
\definecolor{TitleColor}{rgb}{1.0,1.0,1.0}
\definecolor{TFFrameColor}{HTML}{CAE8A2}
\definecolor{TFTitleColor}{HTML}{C5A16F}
\definecolor{secnum}{RGB}{13,151,225}
%\definecolor{ptcbackground}{RGB}{150,189,61}
%\definecolor{ptctitle}{RGB}{37,92,0}
\definecolor{ptcbackground}{gray}{0.90}
\definecolor{ptctitle}{gray}{0.60}
\definecolor{shadecolor}{RGB}{212,237,252}
\definecolor{visgreen}{rgb}{0.733, 0.776, 0}
\definecolor{myBGcolor}{HTML}{F6F0D6}
\definecolor[named]{PowderBlue}{HTML}{B0E0E6}
\definecolor[named]{MidnightBlue}{HTML}{191970}
\definecolor{mybrown}{RGB}{128,64,0}
\definecolor{lightgrey}{rgb}{0.9,0.9,0.9}
\definecolor{darkgreen}{rgb}{0,0.6,0}
\definecolor{Tan}{cmyk}{0.14,0.42,0.56,0}
%%%%%%%%%%%%%%%%
%%%%% Definicion de listing===========
\usepackage{caption}
\DeclareCaptionFont{white}{\color{white}}
\DeclareCaptionFormat{listing}{\colorbox{gray}{\parbox{\dimexpr\textwidth-2\fboxsep\relax}{C\'odigo \thesection .\ 
\thesource\ #3}}}
\captionsetup[source]{format=listing,labelfont=white,textfont=white, singlelinecheck=false, margin=0pt, font={bf,footnotesize}}
\newcounter{source}[section]
\lstnewenvironment{source}[2][]
{\refstepcounter{source}
\captionsetup{options=source}
\lstset{%
basicstyle=\tiny\ttfamily\bf,language={[LaTeX]TeX},caption=#1,label=#2,  
numbersep=5mm, numbers=left, numberstyle=\tiny, % number style
breaklines=true,framexleftmargin=10mm, xleftmargin=10mm,
backgroundcolor=\color{ptcbackground!60},frameround=fttt,escapeinside=??,
rulecolor=\color{ptctitle},
morekeywords={% Give key words here                                         % keywords
    maketitle},
keywordstyle=\color[rgb]{0,0,1},                    % keywords
        commentstyle=\color[rgb]{0.133,0.545,0.133},    % comments
        stringstyle=\color[rgb]{0.627,0.126,0.941}  % strings
%columns=fullflexible   
}
        }
{}
%%%%================= definicion del capitulo ======================)
 \newcommand*\chapterlabel{}
\titleformat{\chapter}
  {\gdef\chapterlabel{}
   \normalfont\sffamily\Huge\bfseries\scshape}
  {\gdef\chapterlabel{\thechapter\ }}{0pt}
  {\begin{tikzpicture}[remember picture,overlay]
    \node[yshift=-3cm] at (current page.north west)
      {\begin{tikzpicture}[remember picture, overlay]
        \draw[fill=ptcbackground!60,draw=ptcbackground!60] (0,0) rectangle
          (\paperwidth,3cm);
          \draw[ultra thick,fill=ptctitle,draw=ptctitle](0,0) -- (current page.east |- 0,0 );
          \draw [ptctitle,fill=ptctitle, ultra thick] (0.5,0) circle [radius=0.1];
         \draw[ptctitle,fill=ptctitle, ultra thick] (21,0) circle [radius=0.1];
        \node[anchor=east,xshift=.9\paperwidth,rectangle,
              rounded corners=20pt,inner sep=11pt,
              fill=ptctitle,draw=ptctitle]
              {\color{white} \chapterlabel\protect#1};
%               \draw[fill=green] (current page.north west) rectangle (current page.south east);
       \end{tikzpicture}        
      }; 
      \begin{pgfonlayer}{background}
%          \path (-1.4cm,2.8cm) node (tl) {};
%          \path (2.3cm, -8.4cm) node (br) {};
          \path (current page.north west) rectangle (current page.south east);
      \end{pgfonlayer}
   \end{tikzpicture}  
     \vspace{20pt}
  }
\titlespacing*{\chapter}{0pt}{50pt}{-60pt}

%%%======================================== TOC
\preto{\frontmatter}{\pagecolor{ptcbackground}}{}{}
\preto{\mainmatter}{\pagecolor{white}}{}{}
\preto{\backmatter}{\pagestyle{empty}\pagecolor{myBGcolor}}{}{}{}
%\patchcmd{\backmatter}{\pagecolor{myBGcolor}\pagestyle{empty}}{}{}{}
%\preto{\tableofcontents}{\begin{snugshade*}}{}{}
%\appto{\tableofcontents}{\end{snugshade*}}{}{}
%\patchcmd{\tableofcontents}{\contentsname}{\color{ptctitle}\contentsname}{}{}


    %%%%%%%%%%%%%%%%%%%%%%%%%%%%%%%%%%
 \setcounter{tocdepth}{0}
    \titlecontents{subsection}
  [5.8em]{\scriptsize\sffamily}
  {\color{ptctitle}\contentslabel{3.5em}\normalcolor}{}
  {\titlerule*[1000pc]{.}\contentspage\hspace*{-5.8em}\\\hspace*{-5.8em}\vspace*{2pt}%
    \color{ptctitle}\rule{\dimexpr\textwidth-15.5pt\relax}{1pt}}

    %%%%%%%%%%%%%%%%%%%%%%%%%%%%%%%
    \titlecontents{section}
  [4em]{\scriptsize\sffamily}
  {\color{ptctitle}\contentslabel{3.5em}\normalcolor}{}
  {\titlerule*[1000pc]{.}\contentspage\hspace*{-2em}\\\hspace*{-3em}\vspace*{2pt}%
    \color{ptctitle}\rule{\dimexpr\textwidth-20pt\relax}{1pt}}

\titlecontents{lsection}
  [5.8em]{\sffamily}
  {\color{ptctitle}\contentslabel{3.5em}\normalcolor}{}
  {\titlerule*[1000pc]{.}\contentspage\\\hspace*{-5.8em}\vspace*{2pt}%
    \color{ptctitle}\rule{\dimexpr\textwidth-15.5pt\relax}{1pt}}

\makeatletter
\renewcommand*\l@chapter[2]{%
\thispagestyle{empty}
  \ifnum \c@tocdepth >\m@ne
    \addpenalty{-\@highpenalty}%
    \vskip 1.0em \@plus\p@
    \setlength\@tempdima{1.5em}%
    \begingroup
      \parindent \z@ \rightskip \@pnumwidth
      \parfillskip -\@pnumwidth
      \leavevmode
      \advance\leftskip\@tempdima
      \hskip -\leftskip
      \colorbox{ptctitle}{\strut%
        \makebox[\dimexpr\textwidth-2\fboxsep-7pt\relax][l]{%
          \color{white}\bfseries\sffamily\protect#1%
          \nobreak\hfill\nobreak\hb@xt@\@pnumwidth{\hss #2}}}\par\smallskip
      \penalty\@highpenalty
    \endgroup
  \fi}
\makeatother
%%% crear toc por capitulo con etoc 
\newcommand*\chaptertoc{% 
  \setcounter{tocdepth}{2}% 
  \etocsettocstyle{\subsection*{\subtoc}}{}% 
  {\footnotesize \localtableofcontents }
} 
\def\subtoc{\colorbox{ptctitle}{
\renewcommand{\baselinestretch}{1}
     \parbox[t]{\dimexpr\textwidth-2\fboxsep\relax}{%
    \strut\color{white}\bfseries\sffamily \makebox[7em]{%
Contenido      }\hfill Eventos \hfill P\'agina}
}}
%%% crear toc por capitulo con titlesec
\newcommand\PartialToC{%
\startcontents[chapters]%
\begin{mdframed}[backgroundcolor=ptcbackground,hidealllines=true]
\printcontents[chapters]{l}{1}{\colorbox{ptctitle}{%
  \parbox[t]{\dimexpr\textwidth-2\fboxsep\relax}{%
    \strut\color{white}\bfseries\sffamily \makebox[5em]{%
Contenido      }\hfill Cap\'{i}tulo~\thechapter\hfill P\'agina}}\vskip5pt}
\end{mdframed}%
}
% Define partial toc for part pages
%% Set the uniform width of the colour box
%% displaying the page number in footer
%% to the width of "99"
\newlength\pagenumwidth
\settowidth{\pagenumwidth}{99}

%% Define style of page number colour box
\tikzset{pagefooter/.style={
anchor=base,font=\sffamily\bfseries\small,
text=white,fill=ptctitle,text centered,
text depth=17mm,text width=\pagenumwidth}}

%% Concoct some colours of our own
\definecolor[named]{GreenTea}{HTML}{CAE8A2}
\definecolor[named]{MilkTea}{HTML}{C5A16F}
%%%%%%%%%%%%%%% Encabezado y pie de pagina
%%%%%%%%%%
%%% Re-define running headers on non-chapter pages
%%%%%%%%%%
\fancypagestyle{headings}{%
  \fancyhf{}   % Clear all headers and footers first
  %% Right headers on odd pages
  \fancyhead[RO]{%
    %% First draw the background rectangles
    \begin{tikzpicture}[remember picture,overlay]
    \fill[ptcbackground] (current page.north east) rectangle (current page.south west);
    \fill[white, rounded corners] ([xshift=-10mm,yshift=-20mm]current page.north east) rectangle ([xshift=15mm,yshift=17mm]current page.south west);
    \begin{pgfonlayer}{background}
    %          \path (-1.4cm,2.8cm) node (tl) {};
    %          \path (2.3cm, -8.4cm) node (br) {};
              \path[fill=brown!20] (current page.north west) rectangle (current page.south east);
          \end{pgfonlayer}
    \end{tikzpicture}
    %% Then the decorative line and the right mark
    \begin{tikzpicture}[xshift=-.75\baselineskip,yshift=.25\baselineskip,remember picture,    overlay,fill=ptctitle,draw=ptctitle]\fill circle(3pt);
    \draw[semithick](0,0) -- (current page.west |- 0,0);
        \end{tikzpicture} \sffamily\itshape\small\protect\nouppercase{Vig\'esima Octava Reuni\'on Latinoamericana de Matem\'atica Educativa}
  }

  %% Left headers on even pages
  \fancyhead[LE]{%
    %% Background rectangles first
    \begin{tikzpicture}[remember picture,overlay]
     \fill[brown!20] (current page.north east) rectangle (current page.south west);
    \fill[ptcbackground] (current page.north east) rectangle (current page.south west);
    \fill[white, rounded corners] ([xshift=-15mm,yshift=-20mm]current page.north east) rectangle ([xshift=10mm,yshift=17mm]current page.south west);
           \end{tikzpicture}
    %% Then the right mark and the decorative line
    \color{ptctitle}\sffamily\itshape\small\protect\nouppercase{Vig\'esima Octava Reuni\'on Latinoamericana de Matem\'atica Educativa}\ 
    \begin{tikzpicture}[xshift=.5\baselineskip,yshift=.25\baselineskip,remember picture, overlay,fill=ptctitle,draw=ptctitle]
    \fill (0,0) circle (3pt); \draw[semithick](0,0) -- (current page.east |- 0,0 );
       \end{tikzpicture}
  }

  %% Right footers on odd pages and left footers on even pages,
  %% display the page number in a colour box
  \fancyfoot[RO,LE]{\tikz[baseline]\node[pagefooter]{\thepage};}
   \fancyfoot[CO,CE]{\tikz\node{\color{ptctitle}Barranquilla - Colombia};}
  \renewcommand{\headrulewidth}{0pt}
  \renewcommand{\footrulewidth}{0pt}
}

%%%%%%%%%%
%%% Re-define running headers on chapter pages
%%%%%%%%%%
\fancypagestyle{plain}{%
  %% Clear all headers and footers
  \fancyhf{}
  %% Right footers on odd pages and left footers on even pages,
  %% display the page number in a colour box
  \fancyfoot[RO,LE]{\tikz[baseline]\node[pagefooter]{\thepage};}
 
  
    %% First draw the background rectangles
     \fancyhead[LE]{\begin{tikzpicture}[remember picture,overlay]
    \fill[ptcbackground] (current page.north east) rectangle (current page.south west);
    \fill[white, rounded corners] ([xshift=-10mm,yshift=-20mm]current page.north east) rectangle ([xshift=15mm,yshift=17mm]current page.south west);
    \begin{pgfonlayer}{background}
    %          \path (-1.4cm,2.8cm) node (tl) {};
    %          \path (2.3cm, -8.4cm) node (br) {};
              \path[fill=ptcbackground] (current page.north west) rectangle (current page.south east);
          \end{pgfonlayer}
    \end{tikzpicture}
    \sffamily\itshape\small\protect\nouppercase{\rightmark}
  }
    \fancyhead[RO]{%
    %% First draw the background rectangles
    \begin{tikzpicture}[remember picture,overlay]
    \fill[ptcbackground] (current page.north east) rectangle (current page.south west);
    \fill[white, rounded corners] ([xshift=-10mm,yshift=-20mm]current page.north east) rectangle ([xshift=15mm,yshift=17mm]current page.south west);
    \begin{pgfonlayer}{background}
    %          \path (-1.4cm,2.8cm) node (tl) {};
    %          \path (2.3cm, -8.4cm) node (br) {};
              \path[fill=ptcbackground] (current page.north west) rectangle (current page.south east);
          \end{pgfonlayer}
    \end{tikzpicture}
    \sffamily\itshape\small\protect\nouppercase{\rightmark}
  }
  \renewcommand{\headrulewidth}{0pt}
  \renewcommand{\footrulewidth}{0pt}
}
%%%%%%%%%%%%%%%%%%%%%%%%%%%%%%% empty page

%%%%%% def de seccion

\newcommand\titlebar{%
\tikz[baseline,trim left=0cm,trim right=3cm] {
    \node [
        fill=ptctitle!90,
        text = white,
        anchor= base east,
        minimum height=3.5ex] (a) at (3cm,0) {
        \textbf{\thesection}
    };
}%
}
\titleformat{\section}{\normalfont\footnotesize\sf}{\titlebar}{0.25cm}{\textcolor{ptctitle}{#1}}      %% Change color if needed and remove \sf.
\titlespacing*{\section}{-2cm}{3.5ex plus 1ex minus .2ex}{2.3ex plus .2ex}
\renewcommand*{\thesection}{\arabic{section}}
% \usetikzlibrary{shapes.symbols,shadows,calc}
%% the tikz picture that will be used for the title formatting
%% \SecTitle{<signal direction>}{<node anchor>}{<node horiz, shift>}{<node x position>}{#5}
%% the fifth argument will be used by \titleformat to write the section title using #1
%\newcommand\SecTitle[5]{%
%\begin{tikzpicture}[overlay,every node/.style={signal, draw, text=white, signal to=nowhere}]
%  \node[ptctitle,fill, signal to=#1, inner sep=1em, drop shadow,
%    text=white,font=\large\sffamily,anchor=#2,
%    xshift=\the\dimexpr-\marginparwidth-\marginparsep-#3\relax] 
%    at (#4,0) {#5};
%\end{tikzpicture}%
%}
%
%\titleformat{name=\section,page=even}
%{\normalfont}{}{12pt}
%{\SecTitle{east}{west}{12pt}{5cm}{\thesection\ #1}}[\addvspace{20pt}]
%
%\titleformat{name=\section,page=odd}
%{\normalfont\sffamily}{}{0em}
%{\SecTitle{west}{east}{12pt}{\paperwidth}{#1\  \thesection}}[\addvspace{20pt}]
  %%%%%8============ Final de tabla de contenido ===========================)
% % % % % % % % % % % %bibname url
\usepackage{url}

%% Define a new 'leo' style for the package that will use a smaller font.
\makeatletter
\def\url@leostyle{%
  \@ifundefined{selectfont}{\def\UrlFont{\sf}}{\def\UrlFont{\small\ttfamily}}}
\makeatother
%% Now actually use the newly defined style.
\urlstyle{leo}
% % % % % % % % % % % % % % % %
\usepackage{bodegraph}

\usetikzlibrary{intersections}
\usetikzlibrary{calc}
\usetikzlibrary{positioning}
% Define the layers to draw the diagram
\pgfdeclarelayer{background}
\pgfdeclarelayer{foreground}
\pgfsetlayers{background,main,foreground}


% % % % % % % % % % % % % % % % % %
\newenvironment{lista}{
\begin{itemize}
 \renewcommand{\labelitemi}{{
 \colorbox{wood!70!black}{\color{white}{\ding{42}}}
 }}
}{\end{itemize}}
\newenvironment{figura}[3]{\begin{figure}[H]
\centering
                               #1
                              \caption{#2}
                              \label{#3}
                              \end{figure}
}{ \vskip 5pt }
\newcommand{\nota}{\colorbox{teal!20!white}{\color{black}{Nota:}}\ }
\newcommand{\dem}{\colorbox{teal!20!white}{\color{black}{Demostraci\'on:}}\ }
\newcommand{\notacion}{\colorbox{teal!20!white}{\color{black}{Notaci\'on:}}\ }
\newcommand{\solucion}{\colorbox{teal!20!white}{\color{black}{Soluci\'on:}}\ }
\def\texto{Sean $f$ y $g$ dos funciones y sean $\alpha$ y $\beta$ dos n\'umeros reales. 
Entonces se verifican las siguientes propiedades:

 \[1.\quad \int (f(x)+g(x))\,dx = \int f(x)\,dx + \int g(x)\,dx  \]
 \[2.\quad \int \alpha f(x)\, dx =\alpha \int f(x)\,dx \]

 Estas dos propiedades se pueden englobar en una:
 \[ \int (\alpha f(x)+\beta g(x)) \, dx = \alpha\int f(x)\,dx+\beta\int
   g(x)\,dx \]
{\bf Ejemplo}:

 \[ \int (2x-3x^2)\, dx = 2\int x\,dx -3\int x^2\, dx \]}
 \def\Web#1{\href{#1}{%
     \tikz \node[fill=myBGcolor](0,0) {#1};%
   }}
   \def\Item{\colorbox{wood!70!black}{\color{white}{\ding{42}}}}
   \def\web#1{\Item\ \href{http://ctan.org/pkg/#1}{\textbf{#1.}}\par\vspace{10pt}}
   % % % % % %listing
   \lstset{%
   basicstyle=\small\ttfamily\bf,language={[LaTeX]TeX}, numbersep=5mm, numbers=left, numberstyle=\tiny, % number style
   breaklines=true,framexleftmargin=10mm, xleftmargin=10mm,
   backgroundcolor=\color{ptcbackground!60},frameround=fttt,escapeinside=??,
   rulecolor=\color{ptctitle},
   morekeywords={% Give key words here                                         % keywords
       maketitle},
   keywordstyle=\color[rgb]{0,0,1},                    % keywords
           commentstyle=\color[rgb]{0.133,0.545,0.133},    % comments
           stringstyle=\color[rgb]{0.627,0.126,0.941}  % strings
   %columns=fullflexible   
   }
   % % % % % %
%%%%%% Definicion de caja %%%%%%%%%%%%%%%%%%%%%
%  \newboxedtheorem[title= Teorema. \thesection.\thecaja ,labelbox= ,boxcolor=MilkTea,background = ptcbackground!60,titleboxcolor=black,titleboxcolor=MilkTea,titlebackground=ptctitle]{caja}{Teorema}
%  % % % % % % % %
%  \newboxedtheorem[title=Lemma.\ \thecaja ,labelbox= ,boxcolor=MilkTea,background = ptcbackground!60,titleboxcolor=black,titleboxcolor=MilkTea,titlebackground=ptctitle]{cajo}{Teorema}
  % % % % % % % % % % % % % %
  \nboxedtheorem[boxcolor=MilkTea,background = ptcbackground!60,titleboxcolor=black,titleboxcolor=MilkTea,titlebackground=ptctitle]{ncaja}{Postulado}
  % % % % % % % % % % % % % % %
 \tipptheorem[tipplogo=interrogacion,boxheadcol=MidnightBlue,boxcol=PowderBlue]{notas}{Nota}
 % % % % % % % % % % % %
 \notatheorem[tipplogo=pregunta,boxheadcol=MidnightBlue,boxcol=PowderBlue]{obs}{Observaci\'on}
 %%%%%%%%%%%%%%
 \frametheorem[]{ejemplo}{Ejemplo}
 %%%%%%%%%%%%%%
 \beamertheorem[]{beamercaja}{Estilo Beamer}
 %%%%%%%%%%%%%%
 \framedtheorem[]{frameth}{Fancy}
 %%%%%%%%%%%%%%%%%%%
 \xcolortheorem[background=mybrown!5 ,titlebackground=mybrown!40!black ,titleboxcolor=mybrown!40!black ,boxcolor=mybrown!40!black]{geo}{Ejemplo}
 %%%%%%%%%%%%%%
%  \warningtheorem[textcol=black, boxheadcol=gray!80, boxcol=ptctitle, tipplogo=icon-tipp, texttcolor=black ,labeltext=, size=0.8\textwidth, iconline=red ]{xcolorth}{Fancy}
  %%%%%%%%%%tcolorbox%%%
  \newcounter{postulado}
\newenvironment{postulado}[2]{\vskip 5pt
\refstepcounter{postulado}
    \begin{tcolorbox}[colback=mybrown!5,colframe=mybrown!40!black,title=Postulado.\thechapter.\thepostulado  \ \bf{#2}]
 #1
\end{tcolorbox}\index{Postulado!#2}            }{

                \vskip 5pt
 }
  %%%%%%%%%%%%%%%
  %%%<
\newcommand{\cdefault}[4][named]{\begin{tikzpicture}
\fill[#2,draw=negro] (0,0) rectangle ++(2,1);
\node[below] at (1,0) {#2};
\node[below=4mm] at (1,0) {\tiny #3 \{#4\}};
\node[below=6mm] at (1,0) {\tiny #1};
\end{tikzpicture}}
%%%%%%%%%%%%%>
%  \lstnewenvironment{javacode}[2]
%{\singlespacing\lstset{language=java, label=#1, caption=#2}}
%{}
%%%%%%%%%%%%%%%cambio de margen
\newenvironment{changemargin}[5]
{
\begin{list}{}
{
\global\setlength{\textheight}{#1}%
  \global\setlength{\textwidth}{#2}
\setlength{\topsep}{0pt}
\setlength{\evensidemargin}{0pt}%
\setlength{\oddsidemargin}{0pt}
\setlength{\leftmargin}{#3}%
\setlength{\rightmargin}{#4}%
\setlength{\listparindent}{\parindent}%
\setlength{\itemindent}{\parindent}%
\setlength{\parsep}{\parskip}%
\hoffset #5
}
\item[]
}
{\end{list}}
%%%%%%%%%cambiamargen%%%%%%%%%%%%%%%
\newenvironment{cambiamargen}[5]
{
\begin{list}{}
{
\global\setlength{\textheight}{#1}%
 \setlength{\topmargin}{#2}
\setlength{\evensidemargin}{0pt}%
\setlength{\oddsidemargin}{0pt}
\setlength{\leftmargin}{-}%
\setlength{\rightmargin}{#4}%
\setlength{\listparindent}{\parindent}%
\setlength{\itemindent}{\parindent}%
\setlength{\parsep}{\parskip}%
\hoffset #5
}
\item[]
}
{\end{list}}
\newenvironment{dems}[1]{ \dem
\it #1  }{\hfill$\square$\vspace*{5pt}}
\newenvironment{datos}[1]{\fontsize{6}{7}\selectfont\bf 
\linespread{1.5}\selectfont
 \raggedleft 
 #1}{}
% \newcommand{\salon}[2][302H]{\colorbox{ptctitle}{ \color{white} \bf Sal\'on\ #1 \ \bfseries\sffamily #2:}}
%%%%%%%%%%%%%%%%%%%%%%%%%%%
\newcommand{\salonpp}[2][302H]{\vspace*{10pt}\bf \textcolor{ptctitle}{\uppercase{#1}} \ \bfseries\sffamily \textcolor{ptctitle}{#2}}
\newcommand{\salonp}[2][302H]{\vspace*{10pt}\bf \textcolor{ptctitle}{\uppercase{#1}} \ \bfseries\sffamily \textcolor{ptctitle}{#2:}}
\newcommand{\salon}[2][302H]{ \vspace*{10pt} \bf \textcolor{ptctitle}{\uppercase{Sal\'on\ #1}} \ \bfseries\sffamily \textcolor{ptctitle}{#2:}}
\newcommand{\salonn}[2][302H]{ \vspace*{10pt} \bf \textcolor{ptctitle}{\uppercase{Sal\'on\ #1}} \ \bfseries\sffamily \textcolor{ptctitle}{#2}}
 \newcommand{\hora}[1]{\vspace*{5pt} \begin{center}
 \colorbox{ptctitle}{
\parbox[t]{0.8\textwidth}{  \color{white}\bfseries\sffamily \normalsize  \centering #1 }
}\vspace*{5pt}\end{center}}
%%%%%%%%%%%%%%%%%%%%%%%%%%%%%%%%%%%%%%%%%%%%%%
%  \newcommand{\hora}[1]{\vspace*{5pt} \begin{center}
% \parbox[t]{0.8\textwidth}{  \bfseries\sffamily \normalsize  \centering \textcolor{ptctitle}{#1 }
%}\vspace*{5pt}\end{center}}
\newcommand{\act}[1]{\begin{mdframed}[align
= center,backgroundcolor=ptctitle,hidealllines=true]
\parbox[t]{\dimexpr\textwidth-2\fboxsep\relax}{  \color{white}\bfseries\sffamily\normalsize \centering \uppercase{#1} }
\end{mdframed}
\vspace*{2pt}}
  %%%%%%%%%%%%%%%%%%%%%%%%%%%%% Empieza el documento