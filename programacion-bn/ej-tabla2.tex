\documentclass[12pt,a4paper]{article}
\usepackage[spanish]{babel}
\usepackage[utf8]{inputenc}
\usepackage[T1]{fontenc}
\usepackage[usenames,table]{xcolor}
\usepackage{multirow}
\usepackage{colortbl}
\usepackage{array, booktabs, tabularx,makecell}
\usepackage[most]{tcolorbox}
%\usepackage{tabularx}
%\usepackage{booktabs}
%\usepackage{environ}
\usepackage{times}
%\usepackage{tikz}
%\usetikzlibrary{shapes,calc}
%\usepackage{float}
\usepackage{fancyhdr}
\usepackage[absolute]{textpos}
%
%\usepackage{geometry}
\usepackage[paperheight=27.9cm,%
paperwidth=20.6cm,%
%centering,%
%textheight=26.9cm,
left=2.5cm,%
right=2.5cm,%
top=2.5cm,%
bottom=2cm,%
headheight=1cm,%
headsep=20pt,%
footskip=1cm,%
marginparsep=20pt,%
pdftex=false,%
letterpaper%
]{geometry}
\renewcommand{\rmdefault}{phv} % Arial
\renewcommand{\sfdefault}{phv} % Arial

\usepackage{textcomp} % necesario para el símbolo €


%\usepackage[pdftex]{graphicx}
\usepackage{caption,subcaption}
\setlength\textfloatsep{1.7\baselineskip plus 0.2\baselineskip minus 0.5\baselineskip}
\setlength\intextsep{2\baselineskip plus 0.2\baselineskip minus 0.2\baselineskip}

\usepackage{environ} %lo usaré para crear el environ de tabla colorida

\usepackage{tabularx}
\usepackage{booktabs} % Contiene el comando \midrule
\usepackage{colortbl}
\usepackage{tikz}
\usetikzlibrary{patterns}
\usetikzlibrary{backgrounds,fit,calc}
%% Define style of page number colour box
\newlength\pagenumwidth
\settowidth{\pagenumwidth}{99}
\tikzset{pagefooter/.style={
anchor=base,font=\sffamily\bfseries\small,
text=white,fill=ptctitle,text centered,
text depth=17mm,text width=\pagenumwidth}}

%% Concoct some colours of our own
\definecolor[named]{GreenTea}{HTML}{CAE8A2}
\definecolor[named]{MilkTea}{HTML}{C5A16F}
%%%%%%%%%%%%%%% Encabezado y pie de pagina
%%%%%%%%%%
%%% Re-define running headers on non-chapter pages
%%%%%%%%%%
\fancypagestyle{headings}{%
  \fancyhf{}   % Clear all headers and footers first
  %% Right headers on odd pages
  \fancyhead[RO]{%
    %% First draw the background rectangles
    \begin{tikzpicture}[remember picture,overlay]
    \fill[ptcbackground] (current page.north east) rectangle (current page.south west);
    \fill[white, rounded corners] ([xshift=-10mm,yshift=-20mm]current page.north east) rectangle ([xshift=15mm,yshift=17mm]current page.south west);
    \begin{pgfonlayer}{background}
    %          \path (-1.4cm,2.8cm) node (tl) {};
    %          \path (2.3cm, -8.4cm) node (br) {};
              \path[fill=brown!20] (current page.north west) rectangle (current page.south east);
          \end{pgfonlayer}
    \end{tikzpicture}
    %% Then the decorative line and the right mark
    \begin{tikzpicture}[xshift=-.75\baselineskip,yshift=.25\baselineskip,remember picture,    overlay,fill=ptctitle,draw=ptctitle]\fill circle(3pt);
    \draw[semithick](0,0) -- (current page.west |- 0,0);
        \end{tikzpicture} \sffamily\itshape\small\protect\nouppercase{Vig\'esima Octava Reuni\'on Latinoamericana de Matem\'aticas Educativa}
  }

  %% Left headers on even pages
  \fancyhead[LE]{%
    %% Background rectangles first
    \begin{tikzpicture}[remember picture,overlay]
     \fill[brown!20] (current page.north east) rectangle (current page.south west);
    \fill[ptcbackground] (current page.north east) rectangle (current page.south west);
    \fill[white, rounded corners] ([xshift=-15mm,yshift=-20mm]current page.north east) rectangle ([xshift=10mm,yshift=17mm]current page.south west);
           \end{tikzpicture}
    %% Then the right mark and the decorative line
    \sffamily\itshape\small\protect\nouppercase{Vig\'esima Octava Reuni\'on Latinoamericana de Matem\'aticas Educativa}\ 
    \begin{tikzpicture}[xshift=.5\baselineskip,yshift=.25\baselineskip,remember picture, overlay,fill=ptctitle,draw=ptctitle]
    \fill (0,0) circle (3pt); \draw[semithick](0,0) -- (current page.east |- 0,0 );
       \end{tikzpicture}
  }

  %% Right footers on odd pages and left footers on even pages,
  %% display the page number in a colour box
  \fancyfoot[RO,LE]{\tikz[baseline]\node[pagefooter]{\thepage};}
   \fancyfoot[CO,CE]{\tikz\node{\color{ptctitle}Barranquilla - Colombia};}
  \renewcommand{\headrulewidth}{0pt}
  \renewcommand{\footrulewidth}{0pt}
}
\usetikzlibrary{%
   decorations.fractals%
  ,decorations.pathmorphing%
  ,shadows%
}
\usetikzlibrary{decorations.pathmorphing}
\usetikzlibrary{decorations.text}
\usetikzlibrary{decorations.shapes} %para curvar texto
\usetikzlibrary{shapes,positioning} %para los flow charts
\usetikzlibrary{shapes,arrows,chains}
\pgfdeclarelayer{background}
\pgfdeclarelayer{foreground}
\pgfsetlayers{background,main,foreground}
\usetikzlibrary{intersections}


\newcommand{\helv}{\fontfamily{phv}\fontsize{8}{9}\selectfont}
\definecolor{ptcbackground}{RGB}{150,189,61}
\definecolor{ptctitle}{RGB}{37,92,0}
\colorlet{TablaBordeSuperior}{ptctitle!5!black} %\definecolor{TablaBordeSuperior}{cmyk}{0,0.51,1,0} % BurntOrange
\colorlet{TablaBordeInferior}{ptcbackground} %\definecolor{TablaBordeInferior}{cmyk}{0,0.75,1,0.24} % Bittersweet
\colorlet{TablaCentroSuperior}{ptctitle!10}
\colorlet{TablaCentroInferior}{ptctitle!20}
\colorlet{FuenteCabeceraTabla}{white}
\newcolumntype{M}[1]{>{\raggedright\arraybackslash}m{#1}}
\newcolumntype{L}[1]{>{\raggedleft\arraybackslash}m{#1}}
\newcolumntype{C}[1]{>{\centering\arraybackslash}m{#1}}
\newcommand{\fuentecabecera}[1]{\textcolor{FuenteCabeceraTabla}{\helv\textbf{#1}}}

\newcommand{\encabezadoUnaLineaSuperior}{0.8}
\newcommand{\encabezadoUnaLineaInferior}{0.8}
\newcommand{\espaciadoCabeceraCuerpoUnaLinea}{0.5em}
\newcommand{\encabezadoDosLineasSuperior}{1.3}
\newcommand{\encabezadoDosLineasInferior}{1.2}
%\newcommand{\espaciadoCabeceraCuerpoDosLineas}{1.5em} %hay que aumentar el valor de 0.5 a 1.5 puesto que la cabecera ocupa dos líneas
\newcommand{\espaciadoCabeceraCuerpoDosLineas}{2em}
\newcommand{\encabezadoTresLineasSuperior}{1.8}
\newcommand{\encabezadoTresLineasInferior}{1.8}
\newcommand{\espaciadoCabeceraCuerpoTresLineas}{0.5em}
\newcommand{\encabezadoSuperior}{\encabezadoUnaLineaSuperior} %por defecto una línea
\newcommand{\encabezadoInferior}{\encabezadoUnaLineaInferior} %por defecto una línea
\newcommand{\espaciadoCabeceraCuerpo}{\espaciadoCabeceraCuerpoUnaLinea} %por defecto el de una línea

\newcommand{\encabezadoUnaLinea}{
    \renewcommand{\encabezadoSuperior}{\encabezadoUnaLineaSuperior}
    \renewcommand{\encabezadoInferior}{\encabezadoUnaLineaInferior}
    \renewcommand{\espaciadoCabeceraCuerpo}{\espaciadoCabeceraCuerpoUnaLinea}}
\newcommand{\encabezadoDosLineas}{
    \renewcommand{\encabezadoSuperior}{\encabezadoDosLineasSuperior}
    \renewcommand{\encabezadoUnaLineaInferior}{\encabezadoDosLineasInferior}
    \renewcommand{\espaciadoCabeceraCuerpo}{\espaciadoCabeceraCuerpoDosLineas}}
\newcommand{\encabezadoTresLineas}{
    \renewcommand{\encabezadoSuperior}{\encabezadoTresLineasSuperior}
    \renewcommand{\encabezadoUnaLineaInferior}{\encabezadoTresLineasInferior}
    \renewcommand{\espaciadoCabeceraCuerpo}{\espaciadoCabeceraCuerpoTresLineas}}
%estos tres últimos comandos gestionan si el encabezado será de una línea, de dos o de tres

\newif\iflineaResaltado
\lineaResaltadofalse
\newcommand{\pieInferiorUno}{-5.2}
\newcommand{\pieInferiorDos}{3.45}
\newcommand{\resaltarPie}[2]{\lineaResaltadotrue\renewcommand{\pieInferiorUno}{#1}\renewcommand{\pieInferiorDos}{#2}}
%%%%%%%%%%%%%%%%%%%%%%%%%%%%%%%%%%%
\captionsetup[table]{labelformat = empty,labelfont=it,textfont={bf,it}}
\NewEnviron{tablacolorida}[2]{%
%  \vspace*{-2em}
    \begin{center}
        \captionsetup{type=table} %para que no me de error el posterior comando \captionof
         \captionof{table}{#1} \label{tab:#2} 
        \begin{tikzpicture}
            \node (tbl) {

            \BODY

            };
            \begin{pgfonlayer}{background}
                \draw[rounded corners,top color=TablaBordeSuperior,bottom color=TablaBordeInferior, draw=white] ($(tbl.north west)+(0.14,0)$)
                    rectangle ($(tbl.north east)-(0.13,\encabezadoSuperior)$); %el top color indica el color principal de la línea de cabecera %para tener encabezado de dos líneas (0.13,1.3)
                \draw[rounded corners,top color=white,bottom color=TablaBordeInferior,
                    middle color=TablaBordeSuperior,draw=white!20] ($(tbl.south west)
                    +(0.13,0.5)$) rectangle ($(tbl.south east)-(0.13,0)$);
               \draw[top color=TablaCentroSuperior,bottom color=TablaCentroInferior,draw=white] %el top color indica el color de inicio de las filas del medio, el "white" hace que el borde externo de las filas no se pinte
                    ($(tbl.north east)-(0.13,\encabezadoInferior)$) %con el parámetro 0.8 controlamos la altura de la línea de cabecera  %para tener encabezado de dos líneas (0.13,1.2)
                    rectangle ($(tbl.south west)+(0.13,0.2)$); %el 0.2 controla el grosor de la línea decorativa inferior

%                \iflineaResaltado
%                \draw[top color=TablaBordeSuperior,bottom color=TablaBordeSuperior, draw=white] ($(0,\pieInferiorUno)+(7.5,2)$)
%                    rectangle ($(-0.5,1)-(7,\pieInferiorDos)$); 
%                % ($(0,-4.8)+(7.5,2)$) ($(-0.5,1)-(7,3.1)$)
%                % ($(0,-5.2)+(7.5,2)$) ($(-0.5,1)-(7,3.45)$) %una fila más abajo
%                \fi
            \end{pgfonlayer}
        \end{tikzpicture}
       
    \end{center}
    %\vspace*{-1em}
}

\begin{document}

%\begin{tablacolorida}{Hitos del proyecto}{hitosProyecto}
%    \begin{tabularx}{1\textwidth}{m{0.10\linewidth} m{0.34\linewidth} m{0.48\linewidth}}
%        \arrayrulecolor{purple}
%        \centering\fuentecabecera{Hito} & \centering\fuentecabecera{Fecha} & \centering\arraybackslash\raisebox{-0.3em}{\fuentecabecera{Descripción}} \\[+\espaciadoCabeceraCuerpo]
%        \centering\textbf{H1} & \centering Mes 0 & \centering\arraybackslash Comienzo del proyecto \\ \midrule
%        \centering\textbf{H2} & \centering 2 meses y 3 semanas & \centering\arraybackslash Fin del estudio del arte \\ \midrule
%        \centering\textbf{H3}  & \centering 6 meses y 1 semana & \centering\arraybackslash Finalización de la plataforma de virtualización \\ \midrule
%        \centering\textbf{H4}  & \centering 9 meses & \centering\arraybackslash Finalización de lo concerniente a DTN \\ \midrule
%        \centering\textbf{H5}  & \centering 10 meses y 1 semana & \centering\arraybackslash Finalización de lo concerniente a CDN \\ \midrule
%        \centering\textbf{H6}  & \centering 12 meses & \centering\arraybackslash Fin del Proyecto Fin de Carrera (PFC) \\[0.5ex]
%    \end{tabularx}
%\end{tablacolorida}
\pagestyle{headings}
\renewcommand{\multirowsetup}{\centering}  
\resaltarPie{-9.7}{7.95}
\newcounter{paqueteTrabajo}
\newcounter{tarea}
\setcounter{paqueteTrabajo}{-1}
\newcommand{\filaRHpt}[7]{ \fuentecabecera{#1} & \centering \bfseries{#2} & \centering \bfseries{#3}   & \centering \bfseries{#4} & \centering \bfseries{#5}  & \centering \bfseries{#6} & \centering \bfseries{#7} \\ \midrule}
\newcommand{\filaRHt}[7]{#1  & \centering #2 & \centering #3   & \centering #4 & \centering #5   & \centering #6 & \centering #7   \\ \midrule}
\newcommand{\filaRHTOTAL}[7]{\centering \fuentecabecera{#1}   & \centering \fuentecabecera{#2} & \centering \fuentecabecera{#3 } & \centering \fuentecabecera{#4} & \centering \fuentecabecera{#5 }  & \centering \fuentecabecera{#6} & \centering \fuentecabecera{#7 }   \\}

\encabezadoUnaLinea
\noindent
\begin{tablacolorida}{\color{ptctitle}Horario General de RELME 28}{horario}
\helv

   \begin{tabularx}{1.06\textwidth}{!{\color{ptctitle}\vrule}c!{\color{ptctitle}\vrule}m{2cm}!{\color{ptctitle}\vrule}m{2cm}!{\color{ptctitle}\vrule}m{2cm}!{\color{ptctitle}\vrule}m{2.5cm}!{\color{ptctitle}\vrule}m{2cm}!{\color{ptctitle}\vrule}}
    \arrayrulecolor{ptctitle}
 \textbf{\scriptsize Domingo} & \textbf{\scriptsize Lunes} & \textbf{\scriptsize Martes} & \textbf{\scriptsize Miércoles} & \textbf{\scriptsize Jueves} & \textbf{\scriptsize Viernes}\tabularnewline
\hline 
\multirow{10}{4cm}{\textbf{\scriptsize Inscripciones definitivas de los participantes:
Desde las 9:00 hasta las 4:00 pm }} & \multirow{1}{3.2cm}{\textbf{\scriptsize Acto de apertura \hspace{1cm}8:30 am - 10:30
am}} & \multicolumn{1}{>{\centering}p{3cm}||}{\textbf{\scriptsize }%
\parbox[c]{1\columnwidth}{%
\textbf{\scriptsize Cursos cortos ( 1º sesión )}{\scriptsize \par}

\textbf{\scriptsize 8:00 am - 10:00 am }%
}} & \multicolumn{1}{>{\centering}p{3cm}||}{\textbf{\scriptsize }%
\parbox[c]{1\columnwidth}{%
\textbf{\scriptsize Cursos cortos ( 2º sesión )}{\scriptsize \par}

\textbf{\scriptsize 8:00 am - 10:00 am}%
}} & \multicolumn{1}{>{\centering}p{3cm}||}{\textbf{\scriptsize }%
\parbox[t]{1\columnwidth}{%
\textbf{\scriptsize Grupos de discusión}{\scriptsize \par}

\textbf{\scriptsize 8:30 am - 9:30 am}%
}} & \multirow{10}{3cm}{}\tabularnewline
\cline{2-5} 
 & \textbf{\scriptsize Conferencia inaugural\hspace{1cm} 10: 00 am -
11:30 am} & \textbf{\scriptsize }%
\parbox[c]{1\columnwidth}{%
\textbf{\scriptsize Conferencias especiales }{\scriptsize \par}

\textbf{\scriptsize 10:15 am - 11:15 am}%
} & \textbf{\scriptsize }%
\parbox[t]{1\columnwidth}{%
\textbf{\scriptsize Conferencias especiales }{\scriptsize \par}

\textbf{\scriptsize 10:15 am - 11:15 am}%
} & \textbf{\scriptsize }%
\parbox[t]{1\columnwidth}{%
\textbf{\scriptsize Reportes de investigación}{\scriptsize \par}

\textbf{\scriptsize 9:45 am - 11:15 am}%
} & \tabularnewline
\cline{2-5} 
 & \textbf{\scriptsize Acto cultural}{\scriptsize \par}

\textbf{\scriptsize 11:30 am - 102:30 pm} & \textbf{\scriptsize }%
\parbox[t]{1\columnwidth}{%
\textbf{\scriptsize Comunicaciones breves}{\scriptsize \par}

\textbf{\scriptsize 11:30 am - 12:30 pm}%
} & \textbf{\scriptsize }%
\parbox[t]{1\columnwidth}{%
\textbf{\scriptsize Comunicaciones breves}{\scriptsize \par}

\textbf{\scriptsize 11:30 am - 12:30 pm}%
} & \textbf{\scriptsize }%
\parbox[t]{1\columnwidth}{%
\textbf{\scriptsize Conferencias especiales }{\scriptsize \par}

\textbf{\scriptsize 11:30 am - 12:30 pm}%
} & \tabularnewline
\cline{2-5} 
 & \multicolumn{4}{c|}{\textbf{\scriptsize }%
\parbox[b][1\totalheight][s]{12cm}{%
\begin{center}
\textbf{Almuerzo}
\par\end{center}

\begin{center}
\textbf{12:30 pm - 2:00 pm}
\par\end{center}%
}} & \tabularnewline
\cline{2-5} 
 & \textbf{\scriptsize }%
\parbox[t]{1\columnwidth}{%
\textbf{\scriptsize Talleres grupo A }{\scriptsize \par}

\textbf{\scriptsize ( 1º sesión )}{\scriptsize \par}

\textbf{\scriptsize 2:00 pm - 3:30 pm}%
} & \textbf{\scriptsize }%
\parbox[t]{1\columnwidth}{%
\textbf{\scriptsize Talleres grupo A }{\scriptsize \par}

\textbf{\scriptsize ( 2º sesión )}{\scriptsize \par}

\textbf{\scriptsize 2:00 pm - 3:30 pm}%
} & \textbf{\scriptsize }%
\parbox[t]{1\columnwidth}{%
\textbf{\scriptsize Talleres grupo B }{\scriptsize \par}

\textbf{\scriptsize ( 1º sesión )}{\scriptsize \par}

\textbf{\scriptsize 2:00 pm - 3:30 pm}%
} & \textbf{\scriptsize }%
\parbox[t]{1\columnwidth}{%
\textbf{\scriptsize Talleres grupo B }{\scriptsize \par}

\textbf{\scriptsize ( 2º sesión )}{\scriptsize \par}

\textbf{\scriptsize 2:00 pm - 3:30 pm}%
} & \tabularnewline
\cline{2-5} 
 & \textbf{\scriptsize }%
\parbox[t]{1\columnwidth}{%
\textbf{\scriptsize Comunicaciones breves}{\scriptsize \par}

\textbf{\scriptsize 3:45 pm - 4:45 pm}%
} & \textbf{\scriptsize }%
\parbox[t]{1\columnwidth}{%
\textbf{\scriptsize Reportes de investigación}{\scriptsize \par}

\textbf{\scriptsize 83:30 pm - 5:00 pm}%
} & \textbf{\scriptsize }%
\parbox[t]{1\columnwidth}{%
\textbf{\scriptsize Exposición de posters y fotografías}{\scriptsize \par}

\textbf{\scriptsize 3:30 pm - 5:00 pm}%
} & \textbf{\scriptsize }%
\parbox[t]{3cm}{%
\textbf{\scriptsize Exposición de posters}{\scriptsize \par}

\textbf{\scriptsize y fotografías}{\scriptsize \par}

\textbf{\scriptsize 3:45 pm - 5:15 pm}%
} & \tabularnewline
\cline{2-5} 
 & \textbf{\scriptsize }%
\parbox[t]{1\columnwidth}{%
\textbf{\scriptsize Reportes de investigación}{\scriptsize \par}

\textbf{\scriptsize 5:00 pm - 6:30 pm}%
} & \textbf{\scriptsize }%
\parbox[t]{1\columnwidth}{%
\textbf{\scriptsize Mesas redondas}{\scriptsize \par}

\textbf{\scriptsize Grupos de discusión}{\scriptsize \par}

\textbf{\scriptsize 5:00 pm - 6:30 pm}%
} & \textbf{\scriptsize }%
\parbox[t]{1\columnwidth}{%
\textbf{\scriptsize Reportes de investigación}{\scriptsize \par}

\textbf{\scriptsize 5:00 pm - 6:30 pm}%
} & \textbf{\scriptsize }%
\parbox[t]{1\columnwidth}{%
\textbf{\scriptsize Acto de apertura}{\scriptsize \par}

\textbf{\scriptsize 8:30 am - 10:30 am}%
} & \tabularnewline
\cline{2-5} 
 &  &  & \textbf{\scriptsize }%
\parbox[t]{1\columnwidth}{%
\textbf{\scriptsize Reunión de delegaciones}{\scriptsize \par}

\textbf{\scriptsize 6:30 pm - 7:30 apm}%
} & \textbf{\scriptsize }%
\parbox[t]{1\columnwidth}{%
\textbf{\scriptsize Conferencia de clausura}{\scriptsize \par}

\textbf{\scriptsize 5:30 pm - 6:30 pm}%
} & \tabularnewline
\cline{2-5} 
 & \multirow{2}{3.2cm}{} & \multirow{2}{3cm}{} & \multirow{2}{3cm}{\textbf{\scriptsize Asamblea de CLAME\hspace{1cm}8:00 pm - 9:00 pm}} & \textbf{\scriptsize }%
\parbox[t]{1\columnwidth}{%
\textbf{\scriptsize Premiación de posters }{\scriptsize \par}

\textbf{\scriptsize y fotografías}{\scriptsize \par}

\textbf{\scriptsize Entrega de sede}{\scriptsize \par}

\textbf{\scriptsize 6:30 pm - 7:30 pm}%
} & \tabularnewline
\cline{5-5} 
 &  &  &  & \textbf{\scriptsize Acto cultural y lunch de clausura}{\scriptsize \par}

\textbf{\scriptsize 7:30 pm - 9:30 pm} & \tabularnewline
\cline{1-5} 
\end{tabular}
\end{tablacolorida}
\encabezadoUnaLinea
%%%%%%%%%%%%%%%%%%%%%%%%%%%%%%%%%%%%%%%%5


\end{document}